\chapter{Gruppen}
In diesem Kapitel und dem n\"{a}chsten Kapitel untersuchen wir algebraische Strukturen wie
\href{http://de.wikipedia.org/wiki/Gruppe_(Mathematik)}{\emph{Gruppen}}, 
\href{http://de.wikipedia.org/wiki/Ring_(Algebra)}{\emph{Ringe}}
 und 
\href{http://en.wikipedia.org/wiki/Field_(mathematics)}{\emph{K\"{o}rper}}, 
wobei wir in diesem Kapitel mit den Gruppen beginnen.
Die Theorie der Gruppen ist urspr\"{u}nglich aus dem Bestreben entstanden, Formeln f\"{u}r die
Nullstellen von beliebigen Polynomen zu finden.  Sp\"{a}ter hat die Gruppentheorie auch in
anderen Gebieten der Mathematik, Physik und Informatik zahlreiche Anwendungen gefunden.
Wir werden die Definition einer Gruppe sp\"{a}ter bei der Definition eines Vektorraums ben\"{o}tigen, 
denn ein Vektorraum ist eine kommutative Gruppe, f\"{u}r die zus\"{a}tzlich eine Skalar-Multiplikation
definiert ist.

\section{Die Definition der Gruppe}
\begin{Definition}[Gruppe]
Ein Tripel $\langle G, e, \circ \rangle$ hei\3t \colorbox{gold}{\emph{Gruppe}} falls folgendes gilt:
\begin{enumerate}
\item $G$ ist eine Menge.
\item $e$ ist ein Element der Menge $G$.
\item $\circ$ ist eine bin\"{a}re Funktion auf der Menge $G$, es gilt also
      \\[0.2cm]
      \hspace*{1.3cm}
      $\circ : G \times G \rightarrow G$.
      \\[0.2cm]
      Wir schreiben den Funktions-Wert $\circ(x,y)$ als $x \circ y$ und benutzen
      $\circ$ also als Infix-Operator.
\item Es gilt
      \\
      \hspace*{1.3cm}
      $e \circ x = x$ \quad f\"{u}r alle $x \in G$,
      \\[0.2cm]
      das Element $e$ ist also bez\"{u}glich der Operation $\circ$ ein \colorbox{gold}{\emph{links-neutrales}}
      Element.
\item F\"{u}r alle $x \in G$ gibt es ein $y \in G$, so dass
      \\[0.2cm]
      \hspace*{1.3cm}
      $y \circ x = e$
      \\[0.2cm]
      gilt.  Wir sagen, dass $y$ ein zu $x$ bez\"{u}glich der Operation $\circ$
      \colorbox{gold}{\emph{links-inverses}} Element ist.
\item Es gilt das folgende \colorbox{gold}{\emph{Assoziativ-Gesetz}}:
      \\[0.2cm]
      \hspace*{1.3cm}
      $(x \circ y) \circ z = x \circ (y \circ z)$ \quad f\"{u}r alle $x,y,z \in G$.
\end{enumerate}
Falls zus\"{a}tzlich das \colorbox{gold}{\emph{Kommutativ-Gesetz}}
\\[0.2cm]
\hspace*{1.3cm}
$\forall x, y \in G: x \circ y = y \circ x$
\\[0.2cm]
gilt, dann sagen wir, dass $\langle G, e, \circ \rangle$ eine \colorbox{gold}{\emph{kommutative Gruppe}} ist.
\end{Definition}

\examples
Bevor wir S\"{a}tze \"{u}ber Gruppen beweisen, pr\"{a}sentieren wir zun\"{a}chst einige Beispiele, anhand
derer klar wird, worum es bei Gruppen \"{u}berhaupt geht.
\begin{enumerate}
\item $\langle \mathbb{Z}, 0, + \rangle$ ist eine kommutative Gruppe, denn es gilt:
      \begin{enumerate}
      \item $0 + x = x$ \quad f\"{u}r alle $x \in \mathbb{Z}$.
      \item $-x + x = 0$ \quad f\"{u}r alle $x \in \mathbb{Z}$,
        
            und damit ist die Zahl $-x$ das \emph{Links-Inverse} der Zahl $x$ bez\"{u}glich
            der Addition.
      \item $(x + y) + z = x + (y + z)$ \quad f\"{u}r alle $x,y,z \in \mathbb{Z}$.
      \item $x + y = y + x$ \quad f\"{u}r alle $x,y \in \mathbb{Z}$.
      \end{enumerate}
      Dieses Beispiel zeigt, dass der Begriff der Gruppe versucht, die Eigenschaften der
      Addition auf den ganzen Zahlen zu verallgemeinern.
\item Definieren wir $\mathbb{Q}_+$ als die Menge der positiven rationalen Zahlen, also als
      \\[0.2cm]
      \hspace*{1.3cm}
      $\mathbb{Q}_+ := \{ q \in \mathbb{Q} \mid q > 0 \}$
      \\[0.2cm]
      und bezeichnen wir mit
      \\[0.2cm]
      \hspace*{1.3cm}
      $\cdot : \mathbb{Q}_+ \times \mathbb{Q}_+ \rightarrow \mathbb{Q}_+$
      \\[0.2cm]
      die Multiplikation, so ist die Struktur $\langle \mathbb{Q}, 1, \cdot \rangle$ eine
      kommutative Gruppe, denn es gilt: 
      \begin{enumerate}
      \item $1 \cdot q = q$ \quad f\"{u}r alle $q \in \mathbb{Q}_+$.
      \item $\frac{1}{q} \cdot q = 1$ \quad f\"{u}r alle $q \in \mathbb{Q}_+$,
        
            und damit ist die Zahl $\frac{1}{q}$ das \emph{Links-Inverse} der Zahl $x$ bez\"{u}glich der
            Multiplikation.
      \item $(x \cdot y) \cdot z = x \cdot (y \cdot z)$ \quad f\"{u}r alle $x,y,z \in \mathbb{Q}_+$.
      \item $x \cdot y = y \cdot x$ \quad f\"{u}r alle $x,y \in \mathbb{Q}_+$.
      \end{enumerate}
\item In den letzten beiden Beispielen war die der Gruppe zu Grunde liegende Menge $G$
      jedesmal unendlich.  Das dies keineswegs immer so ist, zeigt das n\"{a}chste Beispiel.

      Wir definieren die Menge $G$ als
      \\[0.2cm]
      \hspace*{1.3cm}
      $G := \{ e, a \}$
      \\[0.2cm]
      und definieren nun auf der Menge $G$ eine Verkn\"{u}fung
      \\[0.2cm]
      \hspace*{1.3cm}
      $\circ: G \times G \rightarrow G$
      \\[0.2cm]
      indem wir definieren:
      \\[0.2cm]
      \hspace*{1.3cm}
      $
      \begin{array}[t]{lcl}
        e \circ e := e, & \quad & e \circ a := a, \\
        a \circ e := a, & \quad & a \circ a := e. 
      \end{array}
      $
      \\[0.2cm]
      Dann ist $\langle G, e, \circ \rangle$ eine kommutative Gruppe, denn offenbar gilt f\"{u}r alle $x \in G$, dass 
      $e \circ x = x$ ist und wir finden auch f\"{u}r jedes der beiden Elemente ein links-inverses Element:
      Das Links-Inverse zu $e$ ist $e$ und das Links-Inverse zu $a$ ist $a$.  Es bleibt das
      Assoziativ-Gesetz nachzuweisen.  Dazu m\"{u}ssen wir die Gleichung
      \\[0.2cm]
      \hspace*{1.3cm}
      $(x \circ y) \circ z = x \circ (y \circ z)$
      \\[0.2cm]
      f\"{u}r alle Werte $x,y,z \in G$ pr\"{u}fen.  Es gibt insgesamt 8 F\"{a}lle:
      \begin{enumerate}
      \item $(e \circ e) \circ e = e \circ e = e$ und $e \circ (e \circ e) = e \circ e = e$. $\checkmark$
      \item $(e \circ e) \circ a = e \circ a = a$ und $e \circ (e \circ a) = e \circ a = a$. $\checkmark$
      \item $(e \circ a) \circ e = a \circ e = a$ und $e \circ (a \circ e) = e \circ a = a$. $\checkmark$
      \item $(e \circ a) \circ a = a \circ a = e$ und $e \circ (a \circ a) = e \circ e = e$. $\checkmark$
      \item $(a \circ e) \circ e = a \circ e = a$ und $a \circ (e \circ e) = a \circ e = a$. $\checkmark$
      \item $(a \circ e) \circ a = a \circ a = e$ und $a \circ (e \circ a) = a \circ a = e$. $\checkmark$
      \item $(a \circ a) \circ e = e \circ e = e$ und $a \circ (a \circ e) = a \circ a = e$. $\checkmark$
      \item $(a \circ a) \circ a = e \circ a = a$ und $a \circ (a \circ a) = a \circ e = a$. $\checkmark$
      \end{enumerate}
      Die Tatsache, dass die Verkn\"{u}fung $\circ$ kommutativ ist, folgt unmittelbar aus der Definition.
      Wir werden uns im Kapitel zur Zahlentheorie noch n\"{a}her mit endlichen Gruppen besch\"{a}ftigen.
      \eox
\end{enumerate}

Bevor wir weitere Beispiele von Gruppen pr\"{a}sentieren, beweisen wir einige S\"{a}tze, die unmittelbar aus der
Definition der Gruppen folgen.

\begin{Satz}[Links-Inverses ist auch Rechts-Inverses] \lb
  Ist $\langle G, e, \circ \rangle$ eine Gruppe, ist $a \in G$ ein beliebiges Element aus
  $G$ und ist $b$ ein Links-Inverses zu $a$, gilt also
  \\
  \hspace*{1.3cm}
  $b \circ a = e$,
  \\[0.2cm]
  dann ist $b$ auch ein Rechts-Inverses zu $a$, es gilt folglich
  \\[0.2cm]
  \hspace*{1.3cm}
  $a \circ b = e$.  
\end{Satz}

\proof
Zun\"{a}chst bemerken wir, dass das Element $b$ ebenfalls ein Links-Inverses haben muss.  Es gibt also ein 
$c \in G$, so dass
\\[0.2cm]
\hspace*{1.3cm}
$c \circ b = e$
\\[0.2cm]
gilt.  Nun haben wir die folgende Kette von Gleichungen:
\\[0.2cm]
\hspace*{1.3cm}
$
\begin{array}[t]{lcll}
  a \circ b & = & e \circ (a \circ b)           & \mbox{denn $e$ ist links-neutral,} \\[0.2cm]
            & = & (c \circ b) \circ (a \circ b) 
                & \mbox{denn $c$ ist links-invers zu $b$, also gilt $c \circ b = e$,}  \\[0.2cm]
            & = & c \circ \bigl(b \circ (a \circ b)\bigr) 
                & \mbox{Assoziativ-Gesetz}  \\[0.2cm]
            & = & c \circ \bigl((b \circ a) \circ b\bigr) 
                & \mbox{Assoziativ-Gesetz}  \\[0.2cm]
            & = & c \circ \bigl(e \circ b\bigr) 
                & \mbox{denn $b$ ist links-invers zu $a$, also gilt $b \circ a = e$,}  \\[0.2cm]
            & = & c \circ b 
                & \mbox{denn $e$ ist links-neutral}  \\[0.2cm]
            & = & e 
                & \mbox{denn $c$ ist links-invers zu $b$.}  
\end{array}
$
\\[0.2cm]
Insgesamt haben wir also $a \circ b = e$ bewiesen. \qed

\remark
Da jedes zu einem Element $a$ links-inverse Element $b$ auch rechts-invers ist, sprechen wir im Folgenden
immer nur noch von einem \colorbox{gold}{\emph{inversen Element}} und lassen den Zusatz
``\emph{links}'' bzw.~``\emph{rechts}'' weg.  
\eox

\begin{Satz}[Links-neutrales Element ist auch rechts-neutrales Element] \lb
  Ist $\langle G, e, \circ \rangle$ eine Gruppe, so gilt
  \\[0.2cm]
  \hspace*{1.3cm}
  $a \circ e = a$ \quad f\"{u}r alle $a \in G$.
\end{Satz}

\proof
Es sei $a \in G$ beliebig und $b$ ein zu $a$ inverse Element.  Dann haben wir die folgende Kette von
Gleichungen:
\\[0.2cm]
\hspace*{1.3cm}
$
\begin{array}[b]{lcll}
  a \circ e & = & a \circ (b \circ a) 
                & \mbox{denn $b$ ist invers zu $a$,} \\[0.2cm]
            & = & (a \circ b) \circ a
                & \mbox{Assoziativ-Gesetz} \\[0.2cm]
            & = & e \circ a
                & \mbox{denn $b$ ist invers zu $a$,} \\[0.2cm]
            & = & a 
                & \mbox{denn $e$ ist links-neutral.}
\end{array}
$
\qed
\pagebreak

\remark
Da das links-neutrale Element $e$ einer Gruppe $\langle G, e, \circ \rangle$ auch rechts-neutral ist,
sprechen wir im Folgenden
immer nur noch von einem \colorbox{gold}{\emph{neutralen Element}} und lassen den Zusatz
``\emph{links}'' bzw.~``\emph{rechts}'' weg. 
\eox

\begin{Satz}[Eindeutigkeit des neutralen Elements] \lb
  Ist $\langle G, e, \circ \rangle$ eine Gruppe und ist $f \in G$ ein weiteres  Element, so dass
  \\[0.2cm]
  \hspace*{1.3cm}
  $f \circ x = x$ \quad f\"{u}r alle $x \in G$ gilt,
  \\[0.2cm]
  so folgt schon $f = e$.
\end{Satz}

\proof
Wir haben die folgende Kette von Gleichungen:
\\[0.2cm]
\hspace*{1.3cm}
$
\begin{array}[t]{lcll}
  f & = & f \circ e & \mbox{denn $e$ ist neutrales Element und damit auch rechts-neutral,} \\[0.2cm]
    & = & e         & \mbox{denn $f \circ x = x$ f\"{u}r alle $x \in G$, also auch f\"{u}r $x = e$.} 
\end{array}
$
\\[0.2cm]
Also gilt $f = e$ gezeigt. \qed

\remark
Der letzte Satz zeigt, dass das neutrale Element eindeutig bestimmt ist.  Wir sprechen daher in Zukunft
immer von \underline{\color{red}{\emph{dem}}} neutralen Element anstatt von \emph{einem} neutralen Element.
\eox

\begin{Satz}[Eindeutigkeit des inversen Elements] \lb
  Ist $\langle G, e, \circ \rangle$ eine Gruppe, ist $a \in G$ und sind $b,c$ beide invers zu $a$, so folgt
  $b = c$:
  \\[0.2cm]
  \hspace*{1.3cm}
  $b \circ a = e \;\wedge\; c \circ a = e \;\rightarrow\; b = c$.
\end{Satz}

\proof
Wir haben die folgende Kette von Gleichungen:
\\[0.2cm]
\hspace*{1.3cm}
$
\begin{array}[t]{lcll}
  c & = & c \circ e & \mbox{denn $e$ ist neutrales Element,} \\[0.2cm]
    & = & c \circ (a \circ b) & \mbox{denn $b$ ist invers zu $a$,} \\[0.2cm]
    & = & (c \circ a) \circ b & \mbox{Assoziativ-Gesetz} \\[0.2cm]
    & = & e \circ b & \mbox{denn $c$ ist invers zu $a$,} \\[0.2cm]
    & = & b & \mbox{denn $e$ ist neutrales Element.} \\[0.2cm]
\end{array}
$
\\[0.2cm]
Also ist $c = b$ gezeigt. \qed


\remark Der letzte Satz zeigt, dass in einer Gruppe $\langle G, e, \circ \rangle$ f\"{u}r ein
gegebenes Element $a$ das zugeh\"{o}rige inverse Element eindeutig bestimmt ist. 
Wir sprechen daher in Zukunft
immer von \underline{\color{red}{\emph{dem}}} inversen Element anstatt von \emph{einem} inversen Element.
Weiter k\"{o}nnen wir eine Funktion
\\[0.2cm]
\hspace*{1.3cm} $^{-1}: G \rightarrow G$
\\[0.2cm]
definieren, die f\"{u}r alle $a \in G$ das zu $a$ inverse Element berechnet: Es gilt also
\\[0.2cm]
\hspace*{1.3cm} $a^{-1} \circ a = e$ \quad und \quad $a \circ a^{-1} = e$ \quad f\"{u}r alle $a \in G$.
\eox

\remark
Ist $\langle G, e, \circ \rangle$ eine Gruppe und sind die Operation $\circ$ und das neutrale Element 
$e$ aus dem Zusammenhang klar, so sprechen wir einfach von der Gruppe $G$, obwohl wir formal korrekt
eigentlich von der Gruppe $\langle G, e, \circ \rangle$ sprechen m\"{u}ssten. 
\eox

\begin{Satz}[$(a \circ b)^{-1} = b^{-1} \circ a^{-1}$] \lb
  Ist $\langle G, e, \circ \rangle$ eine Gruppe und bezeichnen wir das zu $x$ inverse Element mit
  $x^{-1}$, so gilt
  \\[0.2cm]
  \hspace*{1.3cm}
  $(a \circ b)^{-1} = b^{-1} \circ a^{-1}$ \quad f\"{u}r alle $a,b \in G$.
  \\[0.2cm]
  \textbf{\color{red}{Beachten}} Sie, dass die Reihenfolge der Variablen $a$ und $b$ auf der linken Seite dieser
  Gleichung gerade anders herum ist als auf der rechten Seite!
\end{Satz}

\proof
Wir haben
\\[0.2cm]
\hspace*{1.3cm}
$
\begin{array}[t]{lcll}
  (b^{-1} \circ a^{-1}) \circ (a \circ b) & = & b^{-1} \circ \bigl(a^{-1} \circ (a \circ b)\bigr) 
                                              & \mbox{Assoziativ-Gesetz} \\[0.2cm]
                                          & = & b^{-1} \circ \bigl((a^{-1} \circ a) \circ b\bigr) 
                                              & \mbox{Assoziativ-Gesetz} \\[0.2cm]
                                          & = & b^{-1} \circ (e \circ b) \\[0.2cm]
                                          & = & b^{-1} \circ b \\[0.2cm]
                                          & = & e \\[0.2cm]
\end{array}
$
\\[0.2cm]
Also gilt $(b^{-1} \circ a^{-1}) \circ (a \circ b) = e$ und damit ist gezeigt,  dass das Element
$(b^{-1} \circ a^{-1})$ zu $a \circ b$ invers ist.  Da das inverse Element eindeutig bestimmt ist, folgt
\\[0.2cm]
\hspace*{1.3cm}
$(a \circ b)^{-1} = b^{-1} \circ a^{-1}$. \qed


\begin{Satz}[$(a^{-1})^{-1} = a$]
  Ist $\langle G, e, \circ \rangle$ eine Gruppe und bezeichnen wir das zu $x$ inverse Element mit
  $x^{-1}$, so gilt
  \\[0.2cm]
  \hspace*{1.3cm}
  $(a^{-1})^{-1} = a$ \quad f\"{u}r alle $a \in G$.
  \\[0.2cm]
  Das inverse Element des zu $a$ inversen Elements ist also wieder $a$.
\end{Satz}

\proof
Wir haben
\\[0.2cm]
\hspace*{1.3cm}
$
\begin{array}[t]{lcll}
  (a^{-1})^{-1} & = & (a^{-1})^{-1} \circ e
                    & \mbox{$e$ ist auch rechts-neutral}                      \\[0.2cm]
                & = & (a^{-1})^{-1} \circ (a^{-1} \circ a)
                    & \mbox{denn $a^{-1} \circ a = e$}                        \\[0.2cm]
                & = & \bigl((a^{-1})^{-1} \circ a^{-1}\bigr) \circ a
                    & \mbox{Assoziativ-Gesetz}                                \\[0.2cm]
                & = & e \circ a
                    & \mbox{denn $(a^{-1})^{-1}$ ist das Inverse zu $a^{-1}$} \\[0.2cm]
                & = & a
\end{array}
$
\\[0.2cm]
Also gilt $(a^{-1})^{-1} = a$. \qed


\begin{Definition}[Halb-Gruppe]
  Eine Paar $\langle G, \circ \rangle$ ist eine \emph{Halb-Gruppe}, falls gilt:
  \begin{enumerate}
  \item $G$ ist eine Menge,
  \item $\circ$ ist eine bin\"{a}re Funktion auf $G$, es gilt also
        \\[0.2cm]
        \hspace*{1.3cm}
        $\circ: G \times G \rightarrow G$.
        \\[0.2cm]
        Genau wie bei Gruppen schreiben wir $\circ$ als Infix-Operator.
  \item F\"{u}r den Operator $\circ$ gilt das Assoziativ-Gesetz
        \\[0.2cm]
        \hspace*{1.3cm}
        $(x \circ y) \circ z = x \circ (y \circ z)$.
  \end{enumerate}
  Ist der Operator $\circ$ aus dem Zusammenhang klar, so sagen wir oft auch,
  dass $G$ eine Halb-Gruppe ist.
\end{Definition}

\examples
\begin{enumerate}
\item Das Paar $\langle \mathbb{N}, + \rangle$ ist eine Halb-Gruppe.
\item Das Paar $\langle \mathbb{Z}, \cdot \rangle$ ist eine Halb-Gruppe.
\end{enumerate}

\noindent
Falls $G$ eine Gruppe ist, so lassen sich die Gleichungen
\\[0.2cm]
\hspace*{1.3cm}
$a \circ x = b$ \quad und \quad $y \circ a = b$
\\[0.2cm]
f\"{u}r alle $a,b \in G$ l\"{o}sen: Durch Einsetzen verifzieren Sie sofort, dass $x := a^{-1} \circ b$
eine L\"{o}sung der ersten Gleichung ist, w\"{a}hrend $y := b \circ a^{-1}$ die zweite Gleichung l\"{o}st.
Interessant ist nun, dass auch die Umkehrung dieser Aussage richtig ist, denn es gilt der folgende
Satz. 

\begin{Satz} \label{satz:solvable}
Ist $\langle G, \circ \rangle$ eine Halb-Gruppe, in der f\"{u}r alle Werte $a,b \in G$ die beiden Gleichungen 
\\[0.2cm]
\hspace*{1.3cm}
$a \circ x = b$ \quad und \quad $y \circ a = b$
\\[0.2cm]  
f\"{u}r die Variablen $x$ und $y$ eine L\"{o}sung in $G$ haben, dann
gibt es ein neutrales Element $e \in G$, so 
dass $\langle G, e, \circ \rangle$  eine Gruppe ist.
\end{Satz}

\proof
Es sei $b$ ein beliebiges Element von $G$.  Nach Voraussetzung hat die Gleichung
\\[0.2cm]
\hspace*{1.3cm}
$x \circ b = b$
\\[0.2cm]
eine L\"{o}sung, die wir mit $e$ bezeichnen.  F\"{u}r dieses $e$ gilt also
\\[0.2cm]
\hspace*{1.3cm}
$e \circ b = b$.
\\[0.2cm]
Es sei nun $a$ ein weiteres beliebiges Element von $G$.  Dann hat die Gleichung
\\[0.2cm]
\hspace*{1.3cm}
$b \circ y = a$
\\[0.2cm]
nach Voraussetzung ebenfalls eine L\"{o}sung, die wir mit $c$ bezeichnen.  Es gilt dann
\\[0.2cm]
\hspace*{1.3cm}
$b \circ c = a$.
\\[0.2cm]
Dann haben wir folgende Gleichungs-Kette
\\[0.2cm]
\hspace*{1.3cm}
$
\begin{array}[t]{lcll}
  e \circ a & = & e \circ (b \circ c)
                & \mbox{wegen $b \circ c = a$}  \\[0.2cm]
            & = & (e \circ b) \circ c
                & \mbox{Assoziativ-Gesetz}  \\[0.2cm]
            & = & b \circ c
                & \mbox{wegen $e \circ b = b$}  \\[0.2cm]
            & = & a
                & \mbox{wegen $b \circ c = a$.}  \\[0.2cm]
\end{array}
$
\\[0.2cm]
Wir haben also insgesamt f\"{u}r jedes $a \in G$ gezeigt, dass $e \circ a = a$ ist und damit ist
$e$ ein links-neutrales Element bez\"{u}glich der Operation $\circ$.  Nach Voraussetzung hat nun 
die Gleichung
\\[0.2cm]
\hspace*{1.3cm}
$x \circ a = e$
\\[0.2cm]
f\"{u}r jedes $a$ eine L\"{o}sung, nennen wir diese $d$.  Dann gilt
\\[0.2cm]
\hspace*{1.3cm}
$d \circ a = e$
\\[0.2cm]
und wir sehen, dass zu jedem $a \in G$ ein links-inverses Element existiert.
Da das Assoziativ-Gesetz ebenfalls g\"{u}ltig ist, denn $\langle G, \circ \rangle$ ist eine Halb-Gruppe,
ist $\langle G, e, \circ \rangle$ eine Gruppe.   \qed

\remark Es sei $\langle G, e, \circ \rangle$ eine Gruppe, weiter sei $a,b,c \in G$ und es gelte
\\[0.2cm]
\hspace*{1.3cm} $a \circ c = b \circ c$.
\\[0.2cm]
Multiplizieren wir diese Gleichung auf beiden Seiten mit $c^{-1}$, so sehen wir, dass dann $a = b$ gelten
muss.  \"{A}hnlich folgt aus
\\[0.2cm]
\hspace*{1.3cm} $c \circ a = c \circ b$
\\[0.2cm]
die Gleichung $a = b$.  In einer Gruppe gelten also die beiden folgenden K\"{u}rzungs-Regeln:
\\[0.2cm]
\hspace*{1.3cm} 
$a \circ c = b \circ c \rightarrow a = b$ \quad und \quad
$c \circ a = c \circ b \rightarrow a = b$.
\\[0.2cm]
Interessant ist nun die Beobachtung, dass im Falle einer endlichen Halb-Gruppe 
$\langle G, \circ \rangle$ aus der G\"{u}ltigkeit der  
K\"{u}rzungs-Regeln geschlossen werden kann, dass $G$ eine Gruppe ist.
Um dies zu sehen, brauchen wir drei Definitionen und einen Satz.
\eox

\begin{Definition}[injektiv]
  Eine Funktion $f: M \rightarrow N$ ist \colorbox{gold}{\emph{injektiv}} genau dann, wenn
  \\[0.2cm]
  \hspace*{1.3cm}
  $f(x) = f(y) \rightarrow x = y$ \quad f\"{u}r alle $x,y\in M$ 
  \\[0.2cm]
  gilt.  Diese Forderung ist logisch \"{a}quivalent zu der Formel
  \\[0.2cm]
  \hspace*{1.3cm}
  $x \not= y \rightarrow f(x) \not= f(y)$,
  \\[0.2cm]
  verschiedene Argumente werden also auf verschiedene Werte abgebildet.
\eox
\end{Definition}

\begin{Definition}[surjektiv]
  Eine Funktion $f: M \rightarrow N$ ist \colorbox{gold}{\emph{surjektiv}} genau dann, wenn
  \\[0.2cm]
  \hspace*{1.3cm}
  $\forall y \in N: \exists x \in M: f(x) = y$ 
  \\[0.2cm]
  gilt.  Jedes Element $y$ aus $N$ tritt also als Funktionswert auf.
\eox
\end{Definition}

\begin{Definition}[bijektiv] \lb
  Eine Funktion $f: M \rightarrow N$ ist \colorbox{gold}{\emph{bijektiv}} genau dann, wenn
 $f$ sowohl injektiv als auch surjektiv ist.  
\eox
\end{Definition}

\begin{Satz} \label{satz:injektiv_folgt_surjektiv}
  Es sei $M$ eine endliche nicht-leere Menge und die Funktion 
\\[0.2cm]
\hspace*{1.3cm}
$f:M \rightarrow M$
\\[0.2cm]
sei injektiv.  Dann ist $f$ auch surjektiv.
\eox
\end{Satz}

\proof
Da $M$ endlich ist, k\"{o}nnen wir $M$ in der Form
\\[0.2cm]
\hspace*{1.3cm}
$M = \{ x_1, \cdots, x_n \}$
\\[0.2cm]
schreiben.  Wir f\"{u}hren den Beweis nun indirekt und nehmen an, dass $f$ nicht surjektiv ist.  Dann
existiert also ein $y \in M$, so dass
\\[0.2cm]
\hspace*{1.3cm}
$\forall x \in M: f(x) \not = y$
\\[0.2cm]
gilt.  Weiter definieren wir
\\[0.2cm]
\hspace*{1.3cm}
$f(M) = \{f(x_1), \cdots, f(x_n) \}$.
\\[0.2cm]
Die Menge $f(M)$ bezeichnen wir als das \emph{Bild} von $M$ unter $f$.
Da $f$ injektiv ist, wissen wir, dass aus  $x_i \not= x_j$ folgt, dass $f(x_i) \not= f(x_j)$ gilt.
Damit sind alle Elemente in $f(M)$ verschieden und wir haben
\\[0.2cm]
\hspace*{1.3cm}
$\textsl{card}\bigl(f(M)\bigr) = \textsl{card}\bigl(\{f(x_1), \cdots, f(x_n) \}\bigr) = n$.
\\[0.2cm]
Andererseits gilt $f(M) \subseteq M$ und aus $y \not\in f(M)$ folgt dann, dass
\\[0.2cm]
\hspace*{1.3cm}
$\textsl{card}\bigl(f(M)\bigr) < n$ gelten muss. ${\color{red}{\lightning}}$  \qed


\begin{Satz}
  Es sei $M$ eine endliche nicht-leere Menge und die Funktion 
  \\[0.2cm]
  \hspace*{1.3cm}
  $f:M \rightarrow M$
  \\[0.2cm]
  sei surjektiv.  Dann ist $f$ auch injektiv.
\end{Satz}

\proof
Es sei $M = \{x_1, \cdots, x_n \}$.  Wir f\"{u}hren den Beweis indirekt und nehmen an, dass $f$ nicht injektiv
ist.  Dann gibt es $x_i, x_j \in M$, so dass einerseits $x_i \not= x_j$, aber andererseits $f(x_i) =
f(x_j)$ gilt.  
Damit enth\"{a}lt dann aber die Menge
\\[0.2cm]
\hspace*{1.3cm}
$\{ f(x_1), \cdots, f(x_n) \}$
\\[0.2cm]
weniger als $n$ Elemente, denn mindestens die beiden Elemente $f(x_i)$ und $f(x_j)$ sind ja gleich.
Dann kann $f$ aber nicht mehr surjektiv sein, denn dazu m\"{u}sste 
\\[0.2cm]
\hspace*{1.3cm}
$\{ f(x_1), \cdots, f(x_n) \} = M$
\\[0.2cm] 
gelten.  Dieser Widerspruch zeigt, dass $f$ injektiv sein muss.
\qed

\remark
Die letzten beiden S\"{a}tze k\"{o}nnen wir zusammenfassen indem wir sagen, dass eine auf einer endlichen
Menge definierte Funktion $f:M \rightarrow M$ genau dann injektiv ist, wenn $f$ surjektiv ist.
\eox

\begin{Satz}
  Es sei $\langle G, \circ \rangle$ eine endliche Halb-Gruppe, in der die beiden K\"{u}rzungs-Regeln
  \\[0.2cm]
  \hspace*{1.3cm}
  $a \circ c = b \circ c \rightarrow a = b$ \quad und \quad
  $c \circ a = c \circ b \rightarrow a = b$
  \\[0.2cm]
  f\"{u}r alle $a,b,c \in G$ gelten.  Dann ist $G$ bereits eine Gruppe.
\end{Satz}

\proof Wir beweisen die Behauptung in dem wir zeigen, dass f\"{u}r alle $a,b \in G$ die beiden Gleichungen
\\[0.2cm]
\hspace*{1.3cm} $a \circ x = b$ \quad und \quad $y \circ a = b$
\\[0.2cm]
eine L\"{o}sung $x$ bzw. $z$ haben, denn dann folgt die Behauptung aus Satz \ref{satz:solvable}.  Zun\"{a}chst definieren wir f\"{u}r
jedes $a \in G$ eine Funktion
\\[0.2cm]
\hspace*{1.3cm} $f_a: G \rightarrow G$ \quad durch \quad $f_a(x) := a \circ x$.
\\[0.2cm]
Diese Funktionen $f_a(x)$ sind alle injektiv, denn aus
\\[0.2cm]
\hspace*{1.3cm} $f_a(x) = f_a(y)$
\\[0.2cm]
folgt nach Definition der Funktion $f_a$ zun\"{a}chst
\\[0.2cm]
\hspace*{1.3cm} $a \circ x = a \circ y$
\\[0.2cm]
und aus der G\"{u}ltigkeit der ersten K\"{u}rzungs-Regel folgt nun $x = y$.  Nach dem letzten Satz ist $f_a$ dann
auch surjektiv.  Es gibt also zu jedem $b \in G$ ein $x \in G$ mit
\\[0.2cm]
\hspace*{1.3cm} $f_a(x) = b$ \quad beziehungsweise \quad $a \circ x = b$.
\\[0.2cm]
Damit haben wir gesehen, dass f\"{u}r beliebige $a,b \in G$ die Gleichung $a \circ x = b$ immer eine L\"{o}sung
$x$ hat.  Genauso l\"{a}sst sich  zeigen, dass f\"{u}r beliebige $a,b \in G$ die Gleichung
\\[0.2cm]
\hspace*{1.3cm}
$y \circ a = b$
\\[0.2cm]
eine L\"{o}sung $y$ hat.  Dann ist $G$ nach Satz \ref{satz:solvable} eine Gruppe. \qed

\exercise
F\"{u}r eine Primzahl $p$ definieren wir
\\[0.2cm]
\hspace*{1.3cm}
$\mathbb{Z}_p := \{ 0, 1, 2, \cdots, p - 1 \}$
\\[0.2cm]
als die Menge aller nicht-negativen ganzen Zahlen, die kleiner als $p$ sind.  Weiter
definieren wir f\"{u}r alle $k \in \{1,\cdots,p-1\}$ eine Funktion
\\[0.2cm]
\hspace*{1.3cm}
$f_k: \mathbb{Z}_p \rightarrow \mathbb{Z}_p$ \quad durch \quad $f_k(n) = (k \cdot n) \;\texttt{\symbol{37}}\;p$.
\\[0.2cm]
Zeigen Sie, dass die Funktionen $f_k$ f\"{u}r alle $k \in \{1, \cdots, p-1\}$ bijektiv sind.

\hint
Sie d\"{u}rfen die folgenden beiden Fakten aus der elementaren Zahlentheorie ohne Beweis verwenden:
\begin{enumerate}
\item $(a-b) \;\texttt{\symbol{37}}\;p = 0 \;\leftrightarrow\; a\;\texttt{\symbol{37}}\;p = b \;\texttt{\symbol{37}}\;p$
      \quad f\"{u}r alle $a,b,p \in \mathbb{Z}$.
\item F\"{u}r jede Primzahl $p$ und f\"{u}r beliebige Zahlen $a,b \in \mathbb{Z}$ gilt
      \\[0.2cm]
      \hspace*{1.3cm}
      $(a \cdot b) \;\texttt{\symbol{37}}\;p = 0 \;\rightarrow\; a \;\texttt{\symbol{37}}\;p = 0 \;\vee\; b \;\texttt{\symbol{37}}\;p = 0$.
      \eox
\end{enumerate}

\solution
Wir zeigen, dass die Funktionen $f_k$ f\"{u}r alle $k \in \{1, \cdots, p-1\}$ injektiv sind.  Da die
Menge $\mathbb{Z}_p$ endlich ist, sind die Funktionen $f_k$ dann nach Satz \ref{satz:injektiv_folgt_surjektiv} 
auch surjektiv und damit bijektiv.  Um die Injektivit\"{a}t der Funktion $f_k$ zu zeigen, nehmen wir
an, dass
\\[0.2cm]
\hspace*{1.3cm}
$f_k(x) = f_k(y)$ \quad f\"{u}r $x,y \in \mathbb{Z}_p$ gilt.
\\[0.2cm]
Wir haben dann zu zeigen, dass daraus \colorbox{green}{$x = y$} folgt.  Nach Definition der Funktion $f_k$ folgt aus
$f_k(x) = f_k(y)$ die Gleichung
\\[0.2cm]
\hspace*{1.3cm}
$(k \cdot x) \;\texttt{\symbol{37}}\;p = (k \cdot y) \;\texttt{\symbol{37}}\;p$.
\\[0.2cm]
Nach dem ersten Hinweis ist dies \"{a}quivalent zu 
\\[0.2cm]
\hspace*{1.3cm}
$(k \cdot x - k \cdot y) \;\texttt{\symbol{37}}\;p = 0$.
\\[0.2cm]
Klammern wir hier noch $k$ aus, so erhalten wir
\\[0.2cm]
\hspace*{1.3cm}
$k \cdot (x - y) \;\texttt{\symbol{37}}\;p = 0$.
\\[0.2cm]
Nach dem zweiten Hinweis folgt nun
\\[0.2cm]
\hspace*{1.3cm}
$k \;\texttt{\symbol{37}}\;p = 0 \;\vee\; (x - y) \;\texttt{\symbol{37}}\;p = 0$.
\\[0.2cm]
Da $k < p$ und $k \not= 0$ ist, wissen wir, dass $k \;\texttt{\symbol{37}}\;p \not= 0$ ist.  Also haben wir
\\[0.2cm]
\hspace*{1.3cm}
$(x - y) \;\texttt{\symbol{37}}\;p = 0$,
\\[0.2cm]
was wieder nach dem 1.~Hinweis zu
\\[0.2cm]
\hspace*{1.3cm}
$x \;\texttt{\symbol{37}}\;p = y \;\texttt{\symbol{37}}\;p$
\\[0.2cm]
\"{a}quivalent ist.  Da nun $x$ und $y$ beide nicht-negativ und kleiner als $p$ sind, gilt
\\[0.2cm]
\hspace*{1.3cm}
$x \;\texttt{\symbol{37}}\;p = x$ \quad und $y \;\texttt{\symbol{37}}\;p = y$.
\\[0.2cm]
Also haben wir die zu zeigende Gleichung \colorbox{green}{$x = y$} 
hergeleitet und folglich die Injektivit\"{a}t von $f_k$ nachgewiesen.  \qeds


\section{Die Permutations-Gruppe $\mathcal{S}_n$}
Bisher waren alle Gruppen, die wir kennengelernt haben, kommutativ.  Das \"{a}ndert sich jetzt, denn wir
werden gleich eine Gruppe kennen lernen, die nicht kommutativ ist.  Zun\"{a}chst definieren wir f\"{u}r alle
positiven nat\"{u}rlichen Zahlen $n \in \mathbb{N}$ die Menge $\mathbb{Z}_n^+$ als die Menge aller nat\"{u}rlichen Zahlen
von $1$ bis $n$:
\\[0.2cm]
\hspace*{1.3cm}
$\mathbb{Z}_n^+ := \{ i \in \mathbb{N} \mid 1 \leq i \wedge i \leq n \}$.
\\[0.2cm]
Eine Relation $R \subseteq \mathbb{Z}_n^+ \times \mathbb{Z}_n^+$ hei\3t eine \emph{Permutation} genau dann, wenn $R$ auf $\mathbb{Z}_n^+$
als bijektive Funktion aufgefasst werden kann und dass ist genau dann der Fall, wenn folgendes gilt:
\begin{enumerate}
\item Die Relation $R$ ist links-total auf $\mathbb{Z}_n^+$:
      \\[0.2cm]
      \hspace*{1.3cm}
      $\forall x \in \mathbb{Z}_n^+: \exists y \in \mathbb{Z}_n^+: \pair(x,y) \in R$.
\item Die Relation $R$ ist rechts-total auf $\mathbb{Z}_n^+$:
      \\[0.2cm]
      \hspace*{1.3cm}
      $\forall y \in \mathbb{Z}_n^+: \exists x \in \mathbb{Z}_n^+: \pair(x,y) \in R$.
\item Die Relation $R$ ist rechts-eindeutig:
      \\[0.2cm]
      \hspace*{1.3cm}
      $\forall x, y_1, y_2 \in \mathbb{Z}_n^+:\bigl(\pair(x,y_1) \in R \wedge \pair(x,y_2) \in R \rightarrow y_1 = y_2\bigr)$.
\end{enumerate}
Aus der ersten und der dritten Forderung folgt, dass die Relation $R$ als Funktion
\\[0.2cm]
\hspace*{1.3cm} $R : \mathbb{Z}_n^+ \rightarrow \mathbb{Z}_n^+$
\\[0.2cm]
aufgefasst werden kann.  Aus der zweiten Forderung folgt, dass diese Funktion surjektiv ist.  Da die
Menge $\mathbb{Z}_n^+$ endlich ist, ist die Funktion $R$ damit auch injektiv.
Wir definieren nun $\mathcal{S}_n$ als die Menge aller
Permutationen auf der Menge $\mathbb{Z}_n^+$:
\\[0.2cm]
\hspace*{1.3cm}
$\mathcal{S}_n := \{ R \subseteq \mathbb{Z}_n^+ \times \mathbb{Z}_n^+ \mid \mbox{$R$ ist Permutation auf $\mathbb{Z}_n^+$} \}$.
\\[0.2cm]
Weiter definieren wir die identische Permutation $\mathtt{id}_n$ auf $\mathbb{Z}_n^+$ als
\\[0.2cm]
\hspace*{1.3cm}
$\mathtt{id}_n := \{ \pair(x,x) \mid x \in \mathbb{Z}_n^+ \}$.
\\[0.2cm]
Wir erinnern an die Definition des relationalen Produkts, es gilt:
\\[0.2cm]
\hspace*{1.3cm}
$R_1 \circ R_2 := \bigl\{ \pair(x,z) \mid \exists y \in \mathbb{Z}_n^+:\bigl( \pair(x,y) \in R_1 \wedge \pair(y,z) \in R_2\bigr) \bigl\}$.
\\[0.2cm]
Die entscheidende Beobachtung ist nun, dass $R_1 \circ R_2$ eine Permutation ist, wenn $R_1$ und $R_2$
bereits Permutationen sind.

\exercise
Beweisen Sie
\\[0.2cm]
\hspace*{1.3cm}
$\forall R_1, R_2 \in \mathcal{S}_n: R_1 \circ R_2 \in \mathcal{S}_n$.
\exend


\remark
Wir hatten fr\"{u}her bereits gezeigt, dass f\"{u}r das relationale Produkt das Assoziativ-Gesetz gilt und wir
haben ebenfalls gesehen, dass f\"{u}r die identische Permutation $\mathtt{id}_n$ die Beziehung
\\[0.2cm]
\hspace*{1.3cm}
$\mathtt{id}_n \circ R = R$ \quad f\"{u}r alle $R \in \mathcal{S}_n$
\\[0.2cm]
gilt.  Weiter sehen wir: Ist $R \in \mathcal{S}_n$, so haben wir
\\[0.2cm]
\hspace*{1.3cm}
$
\begin{array}[t]{cll}
   & R^{-1} \circ R \\[0.2cm]
 = & \{ \pair(x,z) \in \mathbb{Z}_n^+\times \mathbb{Z}_n^+ \mid \exists y: 
        (\pair(x,y) \in R^{-1} \wedge \pair(y,z) \in R )
     \} \\[0.2cm]
 = & \{ \pair(x,z) \in \mathbb{Z}_n^+ \times \mathbb{Z}_n^+ \mid 
        \exists y: (\pair(y,x) \in R \wedge \pair(y,z) \in R )
     \} 
   & \\[0.2cm]
 = & \{ \pair(x,z) \in \mathbb{Z}_n^+ \times \mathbb{Z}_n^+ \mid x = z \}
    & \mbox{Begr\"{u}ndung siehe unten}    \\[0.2cm]
 = & \mathtt{id}_n.
\end{array}
$
\\[0.2cm]
Hier haben wir in der vorletzten Zeile die \"{A}quivalenz
\\[0.2cm]
\hspace*{1.3cm}
$\exists y: (\pair(y,x) \in R \wedge \pair(y,z) \in R) \leftrightarrow x = z$
\\[0.2cm]
benutzt.  Hier folgt die Richtung von links nach rechts aus der Tatsache, dass $R$ rechts-eindeutig
ist, w\"{a}hrend die Richtung von rechts nach links daraus folgt, dass $R$ rechts-total ist.
Folglich ist f\"{u}r eine Permutation $R$ der Ausdruck $R^{-1}$ tats\"{a}chlich das Inverse bez\"{u}glich des
relationalen Produkts $\circ$.  Damit ist klar, dass die 
Struktur $\langle \mathcal{S}_n, \mathtt{id}_n, \circ \rangle$ eine Gruppe ist.
Diese Gruppe tr\"{a}gt den Namen \emph{Permutations-Gruppe}.
\eox
\vspace*{-0.2cm}

\exercise
Zeigen Sie, dass $\mathcal{S}_3$ keine kommutative Gruppe ist.  Schreiben Sie dazu eine \textsl{SetlX}-Programm,
dass zun\"{a}chst die Menge $\mathcal{S}_3$ berechnet und anschlie\3end \"{u}berpr\"{u}ft, ob in dieser Menge das
Kommutativ-Gesetz gilt.
\exend

\section{Untergruppen und Faktor-Gruppen}
\begin{Definition}[Untergruppe]
  Es sei $\langle G, e, \circ \rangle$ eine Gruppe und es sei $U \subseteq G$.  Dann ist $U$ eine 
  \emph{Untergruppe} von $G$, geschrieben $U \leq G$, falls folgendes gilt:
  \begin{enumerate}
  \item $\forall x, y \in U: x \circ y \in U$,

        die Menge $U$ ist also unter der Operation $\circ$ abgeschlossen.
  \item $e \in U$,

        das neutrale Element der Gruppe $G$ ist also auch ein Element der Menge $U$.
  \item Bezeichnen wir das zu $x \in G$ bez\"{u}glich der Operation $\circ$ inverse Element mit $x^{-1}$, 
        so gilt
        \\[0.2cm]
        \hspace*{1.3cm}
        $\forall x \in U: x^{-1} \in U$,
        \\[0.2cm]
        die Menge $U$ ist also unter der Operation $\cdot^{-1}: x \mapsto x^{-1}$ abgeschlossen.
        \eox
  \end{enumerate}
\end{Definition}

\remark
Falls $U$ eine Untergruppe der Gruppe $\langle G, e, \circ \rangle$ ist, dann ist 
$\langle U, e, \circ_{|U} \rangle$
offenbar eine Gruppe.  Hierbei bezeichnet $\circ_{|U}$ die Einschr\"{a}nkung der Funktion $\circ$ auf $U$, es
gilt also
\\[0.2cm]
\hspace*{1.3cm}
$\circ_{|U}: U \times U \rightarrow U$ \quad mit $\circ_{|U}(x,y) := \circ(x,y)$ f\"{u}r alle $x,y \in U$.
\eox

\examples
\begin{enumerate}
\item In der Gruppe $\langle \mathbb{Z}, 0, + \rangle$ ist die Menge
      \\[0.2cm]
      \hspace*{1.3cm}
      $2 \mathbb{Z} := \{ 2 \cdot x \mid x \in \mathbb{Z} \}$
      \\[0.2cm]
      der geraden Zahlen eine Untergruppe, denn wir haben:
      \begin{enumerate}
      \item Die Addition zweier gerader Zahlen liefert wieder eine gerade Zahl:
            \\[0.2cm]
            \hspace*{1.3cm}
            $2 \cdot x + 2 \cdot y = 2 \cdot (x + y) \in 2 \mathbb{Z}$.
      \item $0 \in 2 \mathbb{Z}$, denn $0 = 2 \cdot 0 \in \mathbb{Z}$.
      \item Das bez\"{u}glich der Addition inverse Element einer geraden Zahl ist offenbar wieder gerade,
            denn es gilt
            \\[0.2cm]
            \hspace*{1.3cm}
            $- (2 \cdot x) = 2 \cdot (-x) \in 2 \mathbb{Z}$.
      \end{enumerate}
\item Das letzte Beispiel l\"{a}sst sich verallgemeinern: Ist $k \in \mathbb{N}$ und definieren wir
      \\[0.2cm]
      \hspace*{1.3cm}
      $k \mathbb{Z} := \{ k \cdot x \mid x \in \mathbb{Z} \}$
      \\[0.2cm]
      als die Menge der Vielfachen von $k$, so l\"{a}sst sich genau wie in dem letzten Beispiel zeigen, dass
      die Menge $k\mathbb{Z}$ eine Untergruppe der Gruppe $\langle \mathbb{Z}, 0, + \rangle$ ist.
\item Wir definieren die Menge $G$ als
      \\[0.2cm]
      \hspace*{1.3cm}
      $G := \{ e, a, b, c \}$
      \\[0.2cm]
      und definieren auf der Menge $G$ eine Funktion $\circ: G \times G \rightarrow G$ durch
      die folgende Verkn\"{u}fungs-Tafel:
      \\[0.2cm]
      \hspace*{1.3cm}
      \begin{tabular}[t]{l|llll}
      $\circ$ & $e$ & $a$ & $b$ & $c$ \\
      \hline
          $e$ & $e$ & $a$ & $b$ & $c$ \\
          $a$ & $a$ & $e$ & $c$ & $b$ \\
          $b$ & $b$ & $c$ & $e$ & $a$ \\
          $c$ & $c$ & $b$ & $a$ & $e$ \\
      \end{tabular}
      \\[0.2cm]
      Wollen wir zu gegeben $x,y \in G$ den Wert $x \circ y$ mit Hilfe dieser Tabelle finden, so k\"{o}nnen
      wir den Wert $x \circ y$ in der Zeile, die mit $x$ beschriftet ist und der Spalte, die mit $y$
      beschriftet ist, finden.  Beispielsweise gilt $a \circ b = c$.  Es l\"{a}sst sich zeigen, dass
      $\langle G, e, \circ \rangle$ eine Gruppe ist.  Definieren wir die Mengen
      \\[0.2cm]
      \hspace*{1.3cm} $U := \{ e, a \}$, \quad $V := \{ e, b \}$, \quad und \quad $W := \{ e, c \}$, 
      \\[0.2cm]
      so k\"{o}nnen Sie leicht nachrechen, dass 
      $U \leq G$, $V \leq G$ und $W \leq G$ gilt.
      \eox
\end{enumerate}

\exercise
Berechnen Sie alle Untergruppen der Gruppe $\langle \mathcal{S}_3, \mathtt{id}_3, \circ \rangle$.
\exend
\vspace*{0.3cm}

Untergruppen sind interessant, weil sich mit ihrer Hilfe unter bestimmten Umst\"{a}nden 
neue Gruppen bilden lassen, sogenannte \emph{Faktor-Gruppen}.

\begin{Definition}[Faktor-Gruppe]
  Es sei $\langle G, 0, + \rangle$ eine kommutative Gruppe und $U \leq G$.
  Dann definieren wir f\"{u}r jedes $a \in G$ die Menge $a{\color{blue}{+}}U$ als
  \\[0.2cm]
  \hspace*{1.3cm}
  $a {\color{blue}{+}} U := \{ a + x \mid x \in U \}$.
  \\[0.2cm]
  Wir bezeichnen die Mengen $a {\color{blue}{+}} U$ als \emph{\colorbox{gold}{Nebenklassen} von $G$ bez\"{u}glich $U$}.
  Nun definieren wir die Menge $G/U$ (gelesen: $G$ modulo $U$) als
  \\[0.2cm]
  \hspace*{1.3cm}
  $G/U := \bigl\{ a {\color{blue}{+}} U \mid a \in G \bigr\}$.
  \\[0.2cm]
  $G/U$ ist also die Menge der Nebenklassen von $G$ bez\"{u}glich $U$.
  Weiter definieren wir ein Operation ${\color{red}{+}}: G/U \times G/U \rightarrow G/U$ durch
  \\[0.2cm]
  \hspace*{1.3cm}
  $(a {\color{blue}{+}} U) {\color{red}{+}} (b {\color{blue}{+}} U) := (a + b) {\color{blue}{+}} U$.
\end{Definition}

\remark
Zun\"{a}chst ist gar nicht klar, dass die Definition
\\[0.2cm]
\hspace*{1.3cm}
  $(a {\color{blue}{+}} U) {\color{red}{+}} (b {\color{blue}{+}} U) := (a + b) {\color{blue}{+}} U$
\\[0.2cm]
\"{u}berhaupt Sinn macht.  Wir m\"{u}ssen zeigen, dass f\"{u}r alle $a_1,a_2, b_1, b_2 \in G$
\\[0.2cm]
\hspace*{1.3cm}
$a_1 {\color{blue}{+}} U = a_2 {\color{blue}{+}} U \;\wedge\; b_1 {\color{blue}{+}} U = b_2 {\color{blue}{+}} U \;\Rightarrow\; (a_1 + b_1) {\color{blue}{+}} U = (a_2 + b_2) {\color{blue}{+}} U$
\\[0.2cm]
gilt, denn sonst ist die Operation ${\color{red}{+}}$ auf den Nebenklassen von $G$ bez\"{u}glich $U$ nicht eindeutig
definiert.  Um diesen Nachweis f\"{u}hren zu k\"{o}nnen, zeigen wir zun\"{a}chst einen Hilfssatz, der uns dar\"{u}ber
Aufschluss gibt, wann zwei Nebenklassen $a {\color{blue}{+}} U$ und $b {\color{blue}{+}} U$ gleich sind.
\eox

\begin{Lemma}
  Es sei $\langle G, 0, + \rangle$ eine kommutative Gruppe und $U \leq G$.
  Weiter seien $a,b \in G$.  Dann gilt:
  \\[0.2cm]
  \hspace*{1.3cm}
  $a {\color{blue}{+}} U = b {\color{blue}{+}} U$ \quad g.d.w. \quad $a - b \in U$.
\end{Lemma}

\proof
Wir zerlegen den Beweis in zwei Teile:
\begin{enumerate}
\item[``$\Rightarrow$'']:
  Gelte $a {\color{blue}{+}} U = b {\color{blue}{+}} U$.  Wegen $0 \in U$ haben wir
  \\[0.2cm]
  \hspace*{1.3cm}
  $a = a + 0 \in a {\color{blue}{+}} U$
  \\[0.2cm]
  und wegen der Voraussetzung $a {\color{blue}{+}} U = b {\color{blue}{+}} U$ folgt daraus
  \\[0.2cm]
  \hspace*{1.3cm}
  $a \in b {\color{blue}{+}} U$.
  \\[0.2cm]
  Also gibt es ein $u \in U$, so dass
  \\[0.2cm]
  \hspace*{1.3cm}
  $a = b + u$
  \\[0.2cm]
  gilt.  Daraus folgt $a - b = u$ und weil $u \in U$ ist, haben wir also
  \\[0.2cm]
  \hspace*{1.3cm}
  $a - b \in U$.  $\checkmark$
\item[``$\Leftarrow$'']: Gelte nun $a - b \in U$.  Weil $U$ eine Untergruppe ist und Untergruppen zu jedem
  Element auch das Inverse enthalten, gilt dann auch $-(a -b) \in U$, also $b - a \in U$.  Wir zeigen
  nun, dass sowohl
  \\[0.2cm]
  \hspace*{1.3cm} 
  $a {\color{blue}{+}} U \subseteq b {\color{blue}{+}} U$ \quad als auch \quad
  $b {\color{blue}{+}} U \subseteq a {\color{blue}{+}} U$ 
  \\[0.2cm]
  gilt.
  \begin{enumerate}
  \item Sei $x \in a {\color{blue}{+}} U$.  Dann gibt es ein $u \in U$, so dass
        \\[0.2cm]
        \hspace*{1.3cm}
        $x = a + u$
        \\[0.2cm]
        gilt.  Daraus folgt
        \\[0.2cm]
        \hspace*{1.3cm}
        $x = b + \bigl((a - b) + u\bigr)$.
        \\[0.2cm]
        Nun ist aber nach Voraussetzung $a - b \in U$ und da auch $u \in U$ ist, folgt damit, dass auch
        \\[0.2cm]
        \hspace*{1.3cm}
        $v := (a - b) + u \in U$
        \\[0.2cm]
        ist, denn die Untergruppe ist bez\"{u}glich der Addition abgeschlossen.  Damit haben wir
        \\[0.2cm]
        \hspace*{1.3cm}
        $x = b + v$ mit $v \in U$
        \\[0.2cm]
        und nach Definition von $b {\color{blue}{+}} U$ folgt dann $x \in b {\color{blue}{+}} U$.
  \item Sei nun $x \in b {\color{blue}{+}} U$.  Dann gibt es ein $u \in U$, so dass
        \\[0.2cm]
        \hspace*{1.3cm}
        $x = b + u$
        \\[0.2cm]
        gilt.  Durch elementare Umformung sehen wir, dass
        \\[0.2cm]
        \hspace*{1.3cm}
        $x = a + \bigl((b - a) + u\bigr)$
        \\[0.2cm]
        gilt. Nun ist aber, wie oben gezeigt, $b - a \in U$ und da auch $u \in U$ ist, folgt damit, dass auch
        \\[0.2cm]
        \hspace*{1.3cm}
        $v := (b - a) + u \in U$
        \\[0.2cm]
        ist.  Damit haben wir
        \\[0.2cm]
        \hspace*{1.3cm}
        $x = a + v$ mit $v \in U$
        \\[0.2cm]
        und nach Definition von $a {\color{blue}{+}} U$ folgt nun $x \in a {\color{blue}{+}} U$. \qed
  \end{enumerate}
\end{enumerate}

\exercise
Es sei $\langle G, 0, + \rangle$ eine kommutative Gruppe und $U \leq G$ sei eine
Untergruppe von $G$.   Wir definieren auf der Menge $G$ eine Relation $\approx_U$ wie folgt:
\\[0.2cm]
\hspace*{1.3cm}
$x \approx_U y \df x - y \in U$.
\\[0.2cm]
Zeigen Sie, dass $\approx_U$ eine \"{A}quivalenz-Relation auf $G$ ist.
\exend

\begin{Lemma}
  Es sei $\langle G, 0, + \rangle$ eine kommutative Gruppe und $U \leq G$.
  Weiter seien $a,b \in G$.  Dann ist 
  \\[0.2cm]
  \hspace*{1.3cm}
  $(a {\color{blue}{+}} U) {\color{red}{+}} (b {\color{blue}{+}} U) := (a + b) {\color{blue}{+}} U$.
  \\[0.2cm]
  wohldefiniert.
\end{Lemma}

\proof
Wir haben zu zeigen, dass f\"{u}r alle $a_1,a_2,b_1,b_2 \in G$ die Formel
\\[0.2cm]
\hspace*{1.3cm}
$a_1 {\color{blue}{+}} U = a_2 {\color{blue}{+}} U \;\wedge\; b_1 {\color{blue}{+}} U = b_2 {\color{blue}{+}} U \;\Rightarrow\; (a_1 + b_1) {\color{blue}{+}} U = (a_2 + b_2) {\color{blue}{+}} U$
\\[0.2cm]
gilt.  Sei also $a_1 {\color{blue}{+}} U = a_2 {\color{blue}{+}} U$ und $b_1 {\color{blue}{+}} U = b_2 {\color{blue}{+}} U$ vorausgesetzt.  
Zu zeigen ist dann
\\[0.2cm]
\hspace*{1.3cm}
$(a_1 + b_1) {\color{blue}{+}} U = (a_2 + b_2) {\color{blue}{+}} U$.
\\[0.2cm]
Aus $a_1 {\color{blue}{+}} U = a_2 {\color{blue}{+}} U$ folgt nach dem letzten Lemma $a_1 - a_2 \in U$  und aus
$b_1 {\color{blue}{+}} U = b_2 {\color{blue}{+}} U$ folgt $b_1 - b_2 \in U$.  Da $U$ unter der Operation $+$ abgeschlossen ist, folgt
\\[0.2cm]
\hspace*{1.3cm}
$(a_1 - a_2) + (b_1 - b_2) \in U$
\\[0.2cm]
und das ist \"{a}quivalent zu
\\[0.2cm]
\hspace*{1.3cm}
$(a_1 + b_1) - (a_2 + b_2) \in U$.
\\[0.2cm]
Aus der R\"{u}ckrichtung des letzten  Lemmas folgt nun
\\[0.2cm]
\hspace*{1.3cm}
$(a_1 + b_1) {\color{blue}{+}}  U = (a_2 + b_2) {\color{blue}{+}} U$.
\\[0.2cm]
Damit ist gezeigt, dass die Addition auf den Nebenklassen von $U$ wohldefiniert ist. 
\qed

\begin{Satz}
 Es sei $\langle G, 0, + \rangle$ eine kommutative Gruppe und $U \leq G$.
 Dann ist $\langle G/U, 0 {\color{blue}{+}} U, {\color{red}{+}} \rangle$ mit der oben definierten Addition von Nebenklassen eine kommutative Gruppe.
\end{Satz}

\proof
Der Beweis zerf\"{a}llt in drei Teile.
\begin{enumerate}
\item $0 {\color{blue}{+}} U$ ist das links-neutrale Element, denn wir haben
      \\[0.2cm]
      \hspace*{1.3cm}
      $(0 {\color{blue}{+}} U) {\color{red}{+}} (a {\color{blue}{+}} U) = (0 + a) {\color{blue}{+}} U = a {\color{blue}{+}} U$ \quad f\"{u}r alle $a \in G$.
\item $-a {\color{blue}{+}} U$ ist das links-inverse Element zu $a {\color{blue}{+}} U$, denn wir haben
      \\[0.2cm]
      \hspace*{1.3cm}
      $(-a {\color{blue}{+}} U) {\color{red}{+}} (a {\color{blue}{+}} U) = (-a + a) {\color{blue}{+}} U = 0 {\color{blue}{+}} U$ \quad f\"{u}r alle $a \in G$.
\item Es gilt das Assoziativ-Gesetz, denn
      \\[0.2cm]
      \hspace*{1.3cm}
      $
      \begin{array}[t]{cl}
        & \bigl((a {\color{blue}{+}} U) {\color{red}{+}} (b {\color{blue}{+}} U)\bigr) {\color{red}{+}} (c {\color{blue}{+}} U)   \\
      = & \bigl((a + b) {\color{blue}{+}} U\bigr) {\color{red}{+}} (c {\color{blue}{+}} U)         \\
      = & \bigl((a + b) + c\bigr) {\color{blue}{+}} U               \\
      = & \bigl(a + (b + c)\bigr) {\color{blue}{+}} U               \\
      = & (a {\color{blue}{+}} U) {\color{red}{+}} \bigl((b + c) {\color{blue}{+}} U\bigr)         \\
      = & (a {\color{blue}{+}} U) {\color{red}{+}} \bigl((b {\color{blue}{+}} U) {\color{red}{+}} (c {\color{blue}{+}} U)\bigr).  \\
      \end{array}
      $
\item Es gilt das Kommutativ-Gesetz, denn
      \\[0.2cm]
      \hspace*{1.3cm}
      $
      \begin{array}[t]{cll}
        & (a {\color{blue}{+}} U) {\color{red}{+}} (b {\color{blue}{+}} U) \\
      = & (a + b) {\color{blue}{+}} U       \\
      = & (b + a) {\color{blue}{+}} U & \mbox{denn $G$ ist eine kommutative Gruppe} \\
      = & (b {\color{blue}{+}} U) {\color{red}{+}} (a {\color{blue}{+}} U). & \hspace*{\fill} \Box
      \end{array}
      $
\end{enumerate}

\example
Wir haben fr\"{u}her bereits gesehen, dass die Mengen 
\\[0.2cm]
\hspace*{1.3cm}
$k\mathbb{Z} := \bigr\{ k \cdot x \mid x \in \mathbb{Z} \bigr\}$
\\[0.2cm]
Untergruppen der Gruppe $\langle \mathbb{Z}, 0, + \rangle$ sind.  Der letzte Satz zeigt nun, dass die Menge
\\[0.2cm]
\hspace*{1.3cm}
$\mathbb{Z}_k := \mathbb{Z}/(k\mathbb{Z}) = \bigl\{ l {\color{blue}{+}} k\mathbb{Z} \mid l \in \mathbb{Z} \}$
\\[0.2cm]
zusammen mit der durch
\\[0.2cm]
\hspace*{1.3cm}
$(l_1 {\color{blue}{+}} k\mathbb{Z}) {\color{red}{+}} (l_2 {\color{blue}{+}} k\mathbb{Z}) = (l_1 + l_2) {\color{blue}{+}} k\mathbb{Z}$
\\[0.2cm]
definierten Addition eine Gruppe ist, deren neutrales Element die Menge $0 {\color{blue}{+}} k\mathbb{Z} = k\mathbb{Z}$ ist.
Es gilt
\\[0.2cm]
\hspace*{1.3cm}
$l_1 {\color{blue}{+}} k\mathbb{Z} = l_2 {\color{blue}{+}} k\mathbb{Z}$
\\[0.2cm]
genau dann, wenn
\\[0.2cm]
\hspace*{1.3cm}
$l_1 - l_2 \in k\mathbb{Z}$
\\[0.2cm]
ist, und dass ist genau dann der Fall, wenn $l_1 - l_2$ ein Vielfaches von $k$ ist, wenn also
$l_1 \approx_k l_2$ gilt.  Wie wir bereits fr\"{u}her gezeigt haben, ist dies genau dann der Fall, wenn
\\[0.2cm]
\hspace*{1.3cm}
$l_1 \modulo k = l_2 \modulo k$
\\[0.2cm]
ist.  Damit sehen wir, dass die Menge $\mathbb{Z}/(k\mathbb{Z})$ aus genau $k$ verschiedenen Nebenklassen besteht, denn
es gilt
\\[0.2cm]
\hspace*{1.3cm}
$\mathbb{Z}_k = \bigl\{ l {\color{blue}{+}} k\mathbb{Z} \mid l \in \{0, \cdots, k - 1\} \bigr\}$.
\vspace*{0.2cm}

\exercise
Es sei $\langle G, e, \circ \rangle$ eine endliche kommutative Gruppe und es gelte $U \leq G$.
Zeigen Sie, dass dann $\mathtt{card}(U)$ ein Teiler von $\mathtt{card}(G)$ ist. 

\hint
Zeigen Sie, dass alle Nebenklassen von $G$ bez\"{u}glich der Untergruppe $U$ dieselbe
Kardinalit\"{a}t haben.
\eoxs

\solution
Wir zeigen zun\"{a}chst, dass alle Nebenklassen von $G$ dieselbe Anzahl von Elementen haben.  Dazu
definieren wir f\"{u}r jedes $a \in G$ eine Funktion
\\[0.2cm]
\hspace*{1.3cm}
$f_a:U \rightarrow a + U$
\\[0.2cm]
durch die Festlegung
\\[0.2cm]
\hspace*{1.3cm}
$f_a(x) := a + x$.
\\[0.2cm]
Die Funktion $f_a$ ist f\"{u}r beliebiges $a \in G$ injektiv, denn aus
\\[0.2cm]
\hspace*{1.3cm}
$f_a(x) = f_a(y)$
\\[0.2cm]
folgt nach Definition der Funktion $f_a$ sofort
\\[0.2cm]
\hspace*{1.3cm}
$a + x = a + y$.
\\[0.2cm]
Subtrahieren wir auf beiden Seiten dieser Gleichung $a$, so erhalten wir die Gleichung
\\[0.2cm]
\hspace*{1.3cm}
$x = y$
\\[0.2cm]
und damit ist die Funktion $f_a$ injektiv.  Die Funktion $f_a$ ist auch surjektiv, denn wenn 
$y \in a + U$ ist, dann gibt es nach Definition von $a + U$ ein $x \in U$ mit $y = a + x$, also
$y = f_a(x)$.  Damit ist die Funktion $f_a$ bijektiv und folglich haben die Mengen $U$ und $a+U$
dieselbe Anzahl von Elementen:
\\[0.2cm]
\hspace*{1.3cm}
$\textsl{card}(U) = \textsl{card}(a + U)$.
\\[0.2cm]
Wir haben schon in einer fr\"{u}heren Aufgabe gezeigt, dass die durch
\\[0.2cm]
\hspace*{1.3cm}
$x \approx_U y \stackrel{\mbox{\scriptsize def}}{\Longleftrightarrow}\; x -y \in U$
\\[0.2cm]
definierte Relation eine \"{a}quivalenz-Relation ist und wir wissen aus der Mengenlehre, dass die
\"{a}quivalenz-Klassen, die von dieser \"{a}quivalenz-Relation erzeugt werden, eine Partition der Menge $G$
bilden.  Damit gibt es also eine Menge
\\[0.2cm]
\hspace*{1.3cm}
$\{g_1,\cdots, g_k\}$ \quad mit $g_i \in G$ f\"{u}r alle $i=1,\cdots,k$
\\[0.2cm]
so dass die Menge $\bigl\{ [g_1]_{\approx_U}, \cdots, [g_k]_{\approx_U}\bigr\}$ eine Partition von
$G$ ist.  Damit haben wir:
\begin{enumerate}
\item $[g_1]_{\approx_U} \cup \cdots \cup [g_k]_{\approx_U} = G$ \quad und
\item $[g_i]_{\approx_U} \cap [g_j]_{\approx_U} = \emptyset$ \quad falls $i \not= j$.
\end{enumerate}
Nun m\"{u}ssen wir nur noch bemerken, dass die Nebenklassen von $G$ bez\"{u}glich $U$ mit den
\"{a}quivalenz-Klassen identisch sind, die von der \"{a}quivalenz-Relation $\approx_U$ erzeugt werden, denn
es gilt
\\[0.2cm]
\hspace*{1.3cm}
$[g]_{\approx_U} = g + U$,
\\[0.2cm] 
was wir wir folgt einsehen k\"{o}nnen:
\\[0.2cm]
\hspace*{1.3cm}
$
\begin{array}{cl}
                & x \in [g]_{\approx_U}  \\[0.2cm]
\Leftrightarrow & x \approx_U g        \\[0.2cm]
\Leftrightarrow & x - g \in U          \\[0.2cm] 
\Leftrightarrow & x \in g + U.
\end{array}
$
\\[0.2cm]
Damit ist dann klar, dass
\\[0.2cm]
\hspace*{1.3cm}
$
\begin{array}{lcl}
  \textsl{card}(G) & = & \textsl{card}\bigl([g_1]_{\approx_U}\bigr) + \cdots + \textsl{card}\bigl([g_k]_{\approx_U}\bigr) 
                         \\[0.2cm]
                   & = & \textsl{card}(g_1 + U) + \cdots + \textsl{card}(g_k + U) \\[0.2cm]
                   & = & \textsl{card}(U) + \cdots + \textsl{card}(U) \\[0.2cm]
                   & = & k \cdot \textsl{card}(U)
\end{array}
$
\\[0.2cm]
gilt und damit ist $\textsl{card}(U)$ ein Teiler von $\textsl{card}(G)$.  \qed

\exercise
Es seien $\langle G_1, e_1, \circ\rangle$ und $\langle G_2, e_2, *\rangle$ kommutative Gruppen.
Eine Abbildung
\\[0.2cm]
\hspace*{1.3cm}
 $f: G_1 \rightarrow G_2$ 
\\[0.2cm]
ist ein \emph{Gruppen-Homomorphismus} genau dann, wenn 
\\[0.2cm]
\hspace*{1.3cm}
$f(x \circ y) = f(x) * f(y)$ \quad f.a. $x,y \in G_1$
\\[0.2cm]
gilt.  L\"{o}sen Sie die folgenden Teilaufgaben:
\begin{enumerate}
\item Zeigen Sie, dass die Menge
      \\[0.2cm]
      \hspace*{1.3cm}
      $f^{-1}(e_2) := \{ x \in G_1 \mid f(x) = e_2 \}$
      \\[0.2cm]
      eine Untergruppe von $G_1$ ist.
\item Es sei $U \leq G_1$.  Zeigen Sie, dass dann auch 
      \\[0.2cm]
      \hspace*{1.3cm}
      $f(U) := \{ f(x) \mid x \in U \}$
      \\[0.2cm]
      eine Untergruppe von $G_2$ ist.
      \exend
\end{enumerate} 

\section{Monoide}
Zum Schluss dieses Kapitels definieren wir den Begriff eines \emph{Monoiden}.  Vereinfacht gesagt sind
Monoide solche Halbgruppen, in denen es ein neutrales Element gibt.  Die formale Definition folgt.


\begin{Definition}[Monoid]
  Ein Tripel $\langle G, e, \circ \rangle$ ist ein \emph{Monoid} genau dann, wenn Folgendes gilt:
  \begin{enumerate}
  \item $G$ ist eine Menge,
  \item $e \in G$,
  \item $\circ: G \times G \rightarrow G$ und es gelten die folgenden beiden Gesetze:
  \begin{enumerate}
  \item Es gilt das Assoziativ-Gesetz
        \\[0.2cm]
        \hspace*{1.3cm}
        $(a \circ b) \circ c = a \circ (b \circ c)$ \quad f\"{u}r alle $a,b,c \in G$.
  \item $e$ ist ein neutrales Element, es gilt also
        \\[0.2cm]
        \hspace*{1.3cm}
        $e \circ a = a \circ e = a$ \quad     f\"{u}r alle $a \in G$. \eox
  \end{enumerate}
  \end{enumerate}
\end{Definition}

\examples
\begin{enumerate}
\item Die Struktur $\langle \mathbb{Z}, 1, \cdot \rangle$ ist ein Monoid, denn f\"{u}r die Multiplikation
      ganzer Zahlen gilt das Assoziativ-Gesetz und die Zahl $1$ ist bez\"{u}glich der Multiplikation ein
      neutrales Element.
\item Es sei $\Sigma$ ein Alphabet, also ein Menge von Buchstaben und $\Sigma^*$ sei die Menge aller
      W\"{o}rter, die aus Buchstaben von $\Sigma$ gebildet werden k\"{o}nnen.  In der Informatik bezeichnen
      wir diese W\"{o}rter als \emph{Strings}.  Wir definieren f\"{u}r Strings eine Multiplikation
      \\[0.2cm]
      \hspace*{1.3cm}
      $\cdot: \Sigma^* \times \Sigma^* \rightarrow \Sigma^*$
      \\[0.2cm]
      als \emph{Konkatenation}, f\"{u}r zwei Strings $x$ und $y$ setzen wir also
      \\[0.2cm]
      \hspace*{1.3cm}
      $x \cdot y := xy$,
      \\[0.2cm]
      wobei mit $xy$ der String gemeint ist, denn wir durch Hintereinanderschreiben der Buchstaben
      aus $x$ und $y$ erhalten.  Ist beispielsweise $x = \texttt{\symbol{34}ab\symbol{34}}$ und $y = \texttt{\symbol{34}dc\symbol{34}}$,
      so gilt 
      \\[0.2cm]
      \hspace*{1.3cm}
      $x \cdot y = \texttt{\symbol{34}abdc\symbol{34}}$.
      \\[0.2cm]
      Weiter definieren wie $\varepsilon$ als den leeren String, wir setzen also
      \\[0.2cm]
      \hspace*{1.3cm}
      $\varepsilon = \texttt{\symbol{34}\symbol{34}}$.
      \\[0.2cm]
      Dann ist $\varepsilon$ ein neutrales Element bez\"{u}glich der Konkatenation von Strings.  Da
      au\3erdem f\"{u}r die Konkatenation von Strings das Assoziativ-Gesetz gilt, ist $\langle \Sigma^*, \varepsilon, \cdot \rangle$ ein
      Monoid. 
      \eox
\end{enumerate}


%%% Local Variables: 
%%% mode: latex
%%% TeX-master: "lineare-algebra"
%%% End: 
