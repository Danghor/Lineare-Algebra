\chapter{Mathematische Grundlagen} 
Die erste Informatik-Vorlesung legt die Grundlagen, die für das weitere Studium
der Informatik benötigt werden.  Bei diesen Grundlagen handelt es sich im wesentlichen
um die \emph{mathematische Logik} und die \emph{Mengenlehre}.  
Als Anwendung dieser Grundlagen werden zwei Programmiersprachen vorgestellt:
\begin{enumerate}
\item \textsc{Setl} (\underline{set} \underline{l}anguage) ist eine mengenbasierte
      Programmiersprache, in der dem Programmierer 
      die Operationen der Mengenlehre zur Verfügung gestellt werden.
\item \textsl{Prolog} (\underline{pro}gramming in \underline{log}ic) basiert
      auf prädikatenlogischen Konzepten.
\end{enumerate}
Da insbesondere die mathematische Logik und die
Mengenlehre sehr abstrakte Gebiete sind, bereiten sie erfahrungsgemäß vielen Studenten
Schwierigkeiten.  Der Umgang mit dieser, auf den ersten Blick trocken anmutenden Materie fällt
auch deshalb schwer, weil zunächst noch gar nicht klar ist, wozu Logik und Mengenlehre in der
Informatik überhaupt benötigt werden.  Aus diesem Grunde möchte ich an den Anfang dieser
Vorlesung eine Motivation stellen. Diese Motivation soll Ihnen zeigen, dass es zwingend
notwendig ist, den Entwurf von Software und Hardware auf eine solide wissenschaftliche
Grundlage zu stellen. 

\section{Motivation und Überblick}
Wir beginnen mit der Feststellung, dass informationstechnische Systeme 
(im Folgenden kurz als IT-Systeme bezeichnet) zu den komplexesten Systemen gehören, die
die Menschheit je entwickelt hat.  Das lässt sich schon an dem Aufwand erkennen,
der bei der Erstellung von IT-Systemen anfällt.  So sind im Bereich der Telekommunikations-Industrie
IT-Projekte, bei denen mehr als 1000 Entwickler über mehrere Jahre zusammenarbeiten,
die Regel.  Es ist offensichtlich, dass ein Scheitern solcher Projekte mit enormen
 Kosten verbunden ist.  Einige Beispiele mögen dies verdeutlichen.
\begin{enumerate}
\item Am 9.~Juni 1996 stürzte die Rakete Ariane 5 auf ihrem Jungfernflug ab.
      Ursache war ein Kette von Software-Fehlern:  Ein Sensor im Navigations-System
      der Ariane 5 misst die horizontale Neigung und speichert diese zunächst als Gleitkomma-Zahl
      mit einer Genauigkeit von 64 Bit ab.  Später wird dieser Wert dann in eine 
      16 Bit Festkomma-Zahl konvertiert.
      Bei dieser Konvertierung trat ein Überlauf ein, da die zu konvertierende Zahl
      zu groß war, um als 16 Bit Festkomma-Zahl dargestellt werden zu können.
      In der Folge gab das Navigations-System auf dem Datenbus, der dieses System mit
      der Steuerungs-Einheit verbindet, eine Fehlermeldung aus.
       Die Daten dieser Fehlermeldung wurden von der Steuerungs-Einheit als Flugdaten 
      interpretiert.  Die Steuer-Einheit leitete daraufhin eine Korrektur des
      Fluges ein, die dazu führte, dass die Rakete auseinander brach und die 
      automatische Selbstzerstörung eingeleitet werden musste.
      Die Rakete war mit 4 Satelliten beladen. Der wirtschaftliche Schaden, der durch den Verlust dieser
      Satelliten entstanden ist, lag bei mehreren 100 Millionen Dollar.
      
      Ein vollständiger Bericht über die Ursache des Absturzes des Ariane 5 findet sich
      im Internet unter der Adresse \\[0.1cm]
      \hspace*{1.3cm} \texttt{http://www.ima.umn.edu/\symbol{126}arnold/disasters/ariane5rep.html}
\item Die Therac 25 ist ein medizinisches Bestrahlungs-Gerät, das durch 
      Software kontrolliert wird.  Durch  Fehler in dieser Software erhielten 1985
      mindestens 6 Patienten eine Überdosis an Strahlung.  Drei dieser Patienten sind an den Folgen dieser 
      Überdosierung gestorben. 

      Einen detailierten Bericht über diese Unfälle finden Sie unter \\[0.1cm]
      \hspace*{1.3cm} \texttt{http://courses.cs.vt.edu/\symbol{126}cs3604/lib/Therac\_25/Therac\_1.html}      
\item Im ersten Golfkrieg konnte eine irakische \textsl{Scud} Rakete von dem \textsl{Patriot} Flugabwehrsystem
      aufgrund eines Programmier-Fehlers in der Kontrollsoftware des Flugabwehrsystems
      nicht abgefangen werden.  28 Soldaten verloren dadurch ihr Leben, 100 weitere wurden
      verletzt. \\[0.1cm]
      \hspace*{1.3cm} \texttt{http://www.ima.umn.edu/\symbol{126}arnold/disasters/patriot.html}
\item Im Internet finden Sie unter \\[0.1cm]
      \hspace*{1.3cm}
      \href{http://www.computerworld.com/article/2515483/enterprise-applications/epic-failures--11-infamous-software-bugs.html}{\texttt{http://www.computerworld.com/\\
              article/2515483/enterprise-applications/epic-failures--11-infamous-software-bugs.html}}
      \\[0.1cm]
      eine Auflistung von schweren Unfällen, die auf Software-Fehler zurückgeführt werden konnten.
\end{enumerate}
Diese Beispiele zeigen, dass bei der Konstruktion von IT-Systemen mit großer Sorgfalt
und Präzision gearbeitet werden sollte.  Die Erstellung von IT-Systemen muss auf einer 
wissenschaftlich fundierten Basis erfolgen, denn nur dann ist es möglich, die Korrekheit
solcher Systeme zu \emph{verifizieren}, also mathematisch zu beweisen.
Diese oben geforderte wissenschaftliche Basis für die Entwicklung von IT-Systemen ist die Informatik, 
und diese hat ihre Wurzeln sowohl in der Mengenlehre als auch in der mathematischen
Logik.  Diese beiden Gebiete werden uns daher im ersten Semester
des Informatik-Studiums beschäftigen.

Sowohl die Mengenlehre als auch die Logik haben unmittelbare praktische Anwendungen in der
Informatik.
\begin{enumerate}
\item Die Mengenlehre und die damit verbundene Theorie der Relationen bietet die Grundlage
      der Theorie der relationalen Datenbanken.  Außerdem basiert die Programmier-Sprache \textsc{Setl}
      auf der Mengenlehre.
\item Die Programmier-Sprache \textsl{Prolog}, die vorwiegend im Bereich der KI
      (Künstliche Intelligenz) eingesetzt wird, setzt Konzepte der mathematischen
      Logik um.  
\end{enumerate}
Neben den unmittelbaren Anwendungen von Logik und Mengenlehre hat die Beschäftigung mit
diesen beiden Gebiete aber noch eine andere, sehr wichtige Funktion:
Ohne die Einführung geeigneter Abstraktionen sind komplexe Systeme nicht beherrschbar.
Kein Mensch ist in der Lage, alle Details eines Software-Systems, dass aus mehreren
$100\,000$ Programm-Zeilen besteht, zu verstehen.   Die einzige Chance um ein solches
System zu beherrschen besteht in der Einführung geeigneter Abstraktionen.
Daher gehört ein überdurchschnittliches Abstraktionsvermögen zu den wichtigsten Werkzeugen
eines Informatikers.  Die Beschäfigung mit Logik und Mengenlehre trainiert gerade dieses
abstrakte Denkvermögen. 

% Schließlich gibt es für Sie noch einen sehr gewichtigen Grund, sich intensiv mit Logik und
% Mengenlehre zu beschäftigen, den ich Ihnen nicht verschweigen möchte:  Es handelt sich
% dabei um die Klausur am Ende des ersten Semesters!  

Aus meiner Erfahrung weiß ich, dass einige der Studenten sich unter dem Thema Informatik etwas Anderes
vorgestellt haben als die Diskussion abstrakter Konzepte.  Für diese Studenten ist die Beherrschung
einer Programmiersprache und einer dazugehörigen Programmierumgebung das Wesentliche der Informatik.
Natürlich ist die Beherrschung einer Programmiersprache für einen Informatiker unabdingbar.  Sie
sollten sich allerdings darüber im klaren sein, dass das damit verbundene Wissen sehr vergänglich
ist, denn niemand kann heute sagen, in welcher Programmiersprache in 10 Jahren programmiert werden wird.
Im Gegensatz dazu sind die mathematischen Grundlagen der Informatik wesentlich beständiger.


\paragraph{Überblick über den Inhalt der Vorlesung:} 
Im Rest dieses Kapitels werden wir zunächst den Begriff der 
\emph{prädikatenlogischen Formel} auf einer informellen Ebene einführen.  Auf dieser Ebene
werden wir prädikatenlogische Formeln zunächst nur als Abkürzungen vorstellen:
Die Sprache der Prädikaten-Logik bietet uns einen Weg, komplizierte
Zusammenhänge prägnanter und kürzer darzustellen, als dies mit den Mitteln der natürlichen
Sprache möglich ist.  Um allerdings die Bedeutung prädikatenlogischer Formeln 
mathematisch präzisieren zu können, benötigen wir einige Grundbegriffe aus der
Mengenlehre, mit der wir uns im Rest des ersten Kapitels beschäftigen.

Die Begriffs-Bildungen der Mengenlehre sind nicht sehr kompliziert, dafür aber um so
abstrakter.  Um diese Begriffs-Bildungen konkreter werden zu lassen und darüber hinaus den
Studenten ein Gefühl für die Nützlichkeit der Mengenlehre zu geben, stellen wir im zweiten
Kapitel die Sprache \textsc{Setl2} vor.  Dies ist eine Programmier-Sprache, die auf der
Mengenlehre aufgebaut ist.  Neben den klassischen Datentypen wie Zahlen und Strings gibt
es hier als Datentypen zusätzlich Mengen.  Dadurch ist es in \textsc{Setl2} möglich,
Algorithmen in der Sprache der Mengenlehre zu formulieren.  Solche Algorithmen sind zwar
meistens nicht so effizient wie Implementierungen in einer klassischen
Programmier-Sprache, aber dafür in der Regel wesentlich klarer (und damit schneller zu
implementieren) als beispielsweise ein entsprechendes \textsl{C}-Programm.  
Zusätzlich hat
\textsc{Setl2} eine konzeptuelle Ähnlichkeit mit Datenbank-Abfrage-Sprachen wie
beispielsweise \textsl{SQL}, so dass sich eine Vertrautheit mit den Konzepten dieser
Sprache auch später noch als nützlich erweist.

In dem dritten Kapitel widmen wir uns der \emph{Aussagen-Logik}.  Diese kann als ein Teil der
Prädikaten-Logik aufgefasst werden. Die Handhabung aussagenlogischer Formeln ist einfacher als die
Handhabung prädikatenlogischer Formeln.  Daher bietet sich die Aussagen-Logik gewissermassen als
Trainings-Objekt an um mit den Methoden der Logik vertraut zu werden.  Die Aussagen-Logik hat
gegenüber der Prädikaten-Logik noch einen weiteren Vorteil: Sie ist \emph{entscheidbar}, d.h.~wir
können ein Programm schreiben, dass als Eingabe eine aussagenlogische Formel verarbeitet und welches
dann entscheidet, ob diese Formel gültig ist.  Ein solches Programm existiert für beliebige Formeln
der Prädikaten-Logik nicht.  Darüber hinaus gibt es in der Praxis eine Reihe von Problemen, die
bereits mit Hilfe der Aussagenlogik gelöst werden können.  Beispielsweise lässt sich die Frage nach der
Korrektheit kombinatorischer digitaler Schaltungen auf die Entscheidbarkeit einer aussagenlogischen
Formel zurückführen.

Im vierten Kapitel behandeln wir die Prädikatenlogik und analysieren den Begriff
des prädikatenlogischen Beweises mit Hilfe eines \emph{Kalküls}.  Ein
\emph{Kalkül} ist dabei ein formales Verfahren, einen mathematischen Beweis zu führen.
Ein solches Verfahren lässt sich programmieren.  Der im vierten Kaptel vorgestellte
\emph{Resolutions-Kalkül]} bildet  die Grundlage für die
Programmier-Sprache \textsl{Prolog}, deren Grundzüge wir im fünften Kapitel
skizzieren.  

Zum Schluss möchte ich hier noch ein Paar Worte zum Gebrauch von neuer und alter
Rechtschreibung und der Verwendung von Spell-Checkern in diesem Skript sagen.
Dieses Skript wurde unter Verwendung strenger marktwirtschaftlicher Kriterien
erstellt.  Im Klartext heißt das: Zeit ist Geld und als Dozent an der DHBW hat man
weder das eine noch das andere.  Daher ist es sehr wichtig zu wissen, wo eine
zusätzliche Investition von Zeit noch einen für die Studenten nützlichen Effekt
bringt und wo dies nicht der Fall ist.  Ich habe mich an aktuellen
Forschungs-Ergebnissen zum Nutzen der Rechtschreibung orientiert. Diese zeigen,
dass es nicht wichtig ist, in welcher Reihenfolge die Bcushatebn in eniem Wrot
setehn, das eniizge was wihtcig ist, ist dass der esrte und der ltzete Bcusthabe
an der rcihitgen Psoiiton sthet. Der Rset knan ein ttolaer Böldisnn sien,
trtodzem knan man ihn onhe Porbelme lseen. Das ist so, wiel wir nciht jdeen
Buhctsaben eniezln lseen, snoedrn das Wrot als gseatmes.  Wie sie sheen, ist das
tastcähilch der Flal. $\displaystyle\smiley$




%%% Local Variables: 
%%% mode: latex
%%% TeX-master: "lineare-algebra"
%%% End: 
