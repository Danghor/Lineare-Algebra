\chapter{Vektor-R\"{a}ume}
In diesem Kapitel werden wir zun\"{a}chst den f\"{u}r den Rest dieser Vorlesung grundlegenden
Begriff des \href{https://de.wikipedia.org/wiki/Vektorraum}{Vektor-Raums}
einf\"{u}hren.  Die Theorie der Vektor-R\"{a}ume bildet die Grundlage f\"{u}r unsere sp\"{a}tere
Behandlung \href{https://de.wikipedia.org/wiki/Lineares_Gleichungssystem}{linearer Gleichungs-Systeme}. 
Au\3erdem ben\"{o}tigen wir Vektor-R\"{a}ume bei der L\"{o}sung von
\href{https://en.wikipedia.org/wiki/Recurrence_relation}{Rekurrenz-Gleichungen}, die wir im letzten  
Kapitel dieses Skriptes diskutieren.  Daneben gibt es zahlreiche weitere Anwendungen von Vektor-R\"{a}umen in der
Informatik.  Diese alle aufzulisten w\"{u}rde Ihnen jetzt wenig helfen, wir beginnen statt dessen mit
der Definition. 

\renewcommand{\labelenumi}{\arabic{enumi}.}
\renewcommand{\labelenumii}{(\alph{enumii})}

\section{Definition und Beispiele}
\begin{Definition}[Vektor-Raum]
Ein Paar $\mathcal{V} = \bigl\langle \langle V, \vec{0}, + \rangle, \cdot \bigr\rangle$ ist ein
\colorbox{yellow}{\emph{$\mathbb{K}$-Vektor-Raum}} falls gilt:
\begin{enumerate}
\item $\mathbb{K}$ ist ein K\"{o}rper.  

      In allen Beispielen, die uns in dieser Vorlesung begegnen werden,
      ist $\mathbb{K}$ entweder der K\"{o}rper der reellen Zahlen $\mathbb{R}$ oder der K\"{o}rper der komplexen Zahlen $\mathbb{C}$. 
\item $\langle V, \vec{0}, + \rangle$ ist eine kommutative Gruppe.
\item $\cdot: \mathbb{K} \times V \rightarrow V$ ist eine Abbildung, die jeder Zahl $\lambda \in \mathbb{K}$ und jedem 
      $\vec{x} \in V$ ein Element $\lambda \cdot \vec{x} \in V$ zuordnet. 
      Diese Funktion  wird als \colorbox{yellow}{\emph{Skalar-Multiplikation}} bezeichnet und
      \"{u}blicherweise in Infix-Notation geschrieben. 

      Die Skalar-Multiplikation muss au\3erdem den folgenden Gesetzen gen\"{u}gen:
      \begin{enumerate}
      \item $(\alpha \cdot \beta)  \cdot \vec{x} =  \alpha \cdot (\beta \cdot \vec{x})$ \quad f.a.  $\alpha,\beta \in \mathbb{K}$,  $\vec{x} \in V$.

            Beachten Sie, dass der Operator ``$\cdot$'', der hier in dem Ausdruck $(\alpha \cdot \beta)$ auftritt, 
            die Multiplikation in dem K\"{o}rper $\mathbb{K}$ bezeichnet, w\"{a}hrend alle anderen Auftreten des Operators ``$\cdot$'' 
            die Skalar-Multiplikation bezeichnen.  

            Dieses Gesetz wird als das \colorbox{yellow}{\emph{Assoziativ-Gesetz}} bezeichnet.  Es dr\"{u}ckt
            aus, dass die Multiplikation in dem K\"{o}rper $\mathbb{K}$ mit der Skalar-Multiplikation
            vertr\"{a}glich ist.
      \item $(\alpha + \beta) \cdot \vec{x} = \alpha \cdot \vec{x} + \beta \cdot \vec{x}$ \quad f.a.  $\alpha,\beta \in \mathbb{K}$,  $\vec{x} \in V$.

            Beachten Sie, dass der Operator ``$+$'', der  hier in dem Ausdruck $(\alpha + \beta)$
            auftritt, die Addition in dem K\"{o}rper $\mathbb{K}$ bezeichnet, w\"{a}hrend der
            Operator ``$+$'' in dem Ausdruck auf der rechten Seite dieser Gleichung die 
            Addition in der Gruppe $\langle V, \vec{0}, + \rangle$ bezeichnet.  
      \item $\alpha \cdot (\vec{x} + \vec{y}) = \alpha \cdot \vec{x} + \alpha \cdot \vec{y}$ \quad f.a. $\alpha \in \mathbb{K}$,  $\vec{x}, \vec{y} \in V$.

            Die letzten beiden Gesetze werden als \colorbox{yellow}{\emph{Distributiv-Gesetze}}
            bezeichnet.  Sie zeigen, inwiefern die Skalar-Multiplikation mit der Addition im Vektor-Raum
            $\mathcal{V}$ und der Addition im K\"{o}rper $\mathbb{K}$ vertr\"{a}glich ist.
      \item $1 \cdot \vec{x} = \vec{x}$ \quad f.a. $\vec{x} \in V$.  

            Dieses Gesetz dr\"{u}ckt aus, dass das neutrale Element der Multiplikation des K\"{o}rpers
            $\mathbb{K}$ auch bez\"{u}glich der Skalar-Multiplikation ein neutrales Element ist. 
      \end{enumerate}
\end{enumerate} 
Ist $\mathcal{V} = \bigl\langle \langle V, \vec{0}, + \rangle, \cdot \bigr\rangle$  ein $\mathbb{K}$-Vektor-Raum, so bezeichnen wir die Elemente der Menge
$V$ als \colorbox{yellow}{\emph{Vektoren}}, w\"{a}hrend die Elemente aus dem K\"{o}rper $\mathbb{K}$ 
\colorbox{yellow}{\emph{Skalare}} genannt werden.  Den K\"{o}rper $\mathbb{K}$ nennen wir den
\colorbox{yellow}{\emph{Skalaren-K\"{o}rper}} des Vektor-Raums $\mathcal{V}$.  
\eoxs
\end{Definition}

\remark
Es ist in der Literatur \"{u}blich, Vektoren von Skalaren durch Fettdruck zu unterscheiden.  
Da aber an der Tafel ein Fettdruck kaum m\"{o}glich ist, habe ich mich dazu entschlosen, 
Vektoren durch Pfeilen zu kennzeichnen.
\eoxs


\example
F\"{u}r eine Menge $K$ haben wir die Menge $K^n$ als die Menge aller Listen der L\"{a}nge $n$ definiert,
deren Elemente aus der Menge $K$ stammen.  Ist nun $\mathbb{K} = \langle K, 0, 1, +, \cdot\rangle$
ein K\"{o}rper, so definieren wir zun\"{a}chst
\\[0.2cm]
\hspace*{1.3cm}
$\vec{0} := \underbrace{[0, \cdots, 0]}_n$ .
\\[0.2cm]
Anschlie�end definieren wir eine Addition ``${\color{blue}{+}}$'' auf $K^n$ komponentenweise durch
\\[0.2cm]
\hspace*{1.3cm}
$[x_1, \cdots, x_n] {\color{blue}{+}} [y_1, \cdots, y_n] := [x_1 + y_1, \cdots, x_n + y_n]$,
\\[0.2cm]
so k\"{o}nnen Sie nachrechnen, dass mit diesen Definitionen das Tripel
\\[0.2cm]
\hspace*{1.3cm}
$\langle \mathbb{K}^n, \vec{0}, +\rangle$
\\[0.2cm]
eine kommutative Gruppe ist.  Definieren wir weiter die Skalar-Multiplikation
${\color{blue}\cdot}$ durch
\\[0.2cm]
\hspace*{1.3cm}
$\alpha {\color{blue}{\cdot}} [x_1, \cdots, x_n] := [\alpha \cdot x_1, \cdots, \alpha \cdot x_n]$,
\\[0.2cm]
wobei in den Ausdr\"{u}cken $\alpha \cdot x_i$ der Operator ``$\cdot$'' die Multiplikation in dem K\"{o}rper $\mathbb{K}$ bezeichnet,
dann l\"{a}sst sich mit etwas Rechenaufwand einsehen, dass das Paar
\\[0.2cm]
\hspace*{1.3cm}
$\mathbb{K}^n := \bigl\langle \langle K^n, \vec{0}, {\color{blue}{+}} \rangle, {\color{blue}{\cdot}} \bigr\rangle$ 
\\[0.2cm]
ein $\mathbb{K}$-Vektor-Raum ist.  Der K\"{u}rze halber werden wir in Zukunft einfach von $\mathbb{K}^n$ als Vektor-Raum reden, wobei wir dann in Wahrheit das obige Paar meinen. 
\eox

\exercise
Beweisen Sie, dass die oben definierte Struktur $\mathbb{K}^n$ f\"{u}r einen beliebigen K\"{o}rper
$\mathbb{K}$ ein $\mathbb{K}$-Vektor-Raum ist. \eox 


Der Begriff des Vektor-Raums versucht, die algebraische Struktur der Menge $\mathbb{K}^n$
axiomatisch zu erfassen.  Das ist deswegen n\"{u}tzlich, weil es neben dem Vektor-Raum 
$\mathbb{K}^n$ noch viele andere Beispiele gibt, welche dieselbe algebraische
Struktur wie der Vektor-Raum $\mathbb{K}^n$ aufweisen.  


\example
Die Menge $\mathbb{R}^{\mathbb{R}}$ ist die Menge der Funktionen der Form
\\[0.2cm]
\hspace*{1.3cm}
 $f: \mathbb{R} \rightarrow \mathbb{R}$,
\\[0.2cm]
also die Menge aller Funktionen von $\mathbb{R}$ nach $\mathbb{R}$.
Definieren wir die Addition zweier Funktionen punktweise, definieren wir f\"{u}r $f,g \in \mathbb{R}^{\mathbb{R}}$ also die
Funktion $f+g$ indem wir
\\[0.2cm]
\hspace*{1.3cm}
$(f+g)(x) := f(x) + g(x)$  \quad f.a. $x \in \mathbb{R}$
\\[0.2cm]
setzen, so ist die so definierte Funktion $f + g$ wieder eine Funktion von
$\mathbb{R}$ nach $\mathbb{R}$.  F\"{u}r ein $\alpha \in \mathbb{R}$ und $f
\in\mathbb{R}^{\mathbb{R}}$ definieren wir die Funktion
$\alpha {\color{blue}{\cdot}} f$ als
\\[0.2cm]
\hspace*{1.3cm}
$(\alpha {\color{blue}{\cdot}} f)(x) := \alpha \cdot f(x)$ \quad f.a. $x \in \mathbb{R}$.
\\[0.2cm]
Dann ist auch  $\alpha {\color{blue}{\cdot}} f$ eine Funktion von $\mathbb{R}$ nach $\mathbb{R}$.  
Schlie\3lich definieren wir eine Funktion
 $\vec{0}:\mathbb{R} \rightarrow \mathbb{R}$, indem wir
\\[0.2cm]
\hspace*{1.3cm}
$\vec{0}(x) := 0$ \quad f.a. $x \in \mathbb{R}$ 
\\[0.2cm]
setzen.  Offensichtlich  gilt $\vec{0} \in \mathbb{R}^{\mathbb{R}}$.  Nun k\"{o}nnen Sie
in einer zwar l\"{a}nglichen, aber geradlinigen Rechnung nachpr\"{u}fen, dass die so definierte Struktur
\\[0.2cm]
\hspace*{1.3cm}
$\bigl\langle \langle \mathbb{R}^{\mathbb{R}}, \vec{0}, + \rangle, \cdot \bigr\rangle$
\\[0.2cm]
ein Vektor-Raum ist.  Das folgt letzlich daraus, dass das Assoziativ-Gesetz und das Distributiv-Gesetz f\"{u}r reelle Zahlen
gilt.
 \eoxs

\example
Es sei $\mathbb{K} = \langle K, 0,1,+,\cdot\rangle$ ein K\"{o}rper.  Dann definieren wir $K^\mathbb{N}$ als den Raum aller
Folgen mit Elementen aus $K$.  Definieren wir f\"{u}r zwei
Folgen
$\bigl(x_n\bigr)_{n\in\mathbb{N}}$ und $\bigl(y_n\bigr)_{n\in\mathbb{N}}$ die Summe durch
\\[0.2cm]
\hspace*{1.3cm}
$\bigl(x_n\bigr)_{n\in\mathbb{N}} {\color{blue}{+}} \bigl(y_n\bigr)_{n\in\mathbb{N}} := \bigl(x_n + y_n\bigr)_{n\in\mathbb{N}}$ 
\\[0.2cm]
und die Skalar-Multiplikation durch
\\[0.2cm]
\hspace*{1.3cm}
$\alpha {\color{blue}{\cdot}} \bigl(x_n\bigr)_{n\in\mathbb{N}} := \bigl(\alpha \cdot x_n \bigr)_{n\in\mathbb{N}}$
\\[0.2cm]
und definieren wir $\vec{0}$ als die Folge $(0)_{n\in\mathbb{N}}$, also als die Folge,
deren s\"{a}mtliche Glieder den Wert $0$ haben, so l\"{a}sst sich zeigen, dass die Struktur
 $\bigl\langle\langle K^\mathbb{N}, \vec{0}, {\color{blue}{+}}\rangle, {\color{blue}{\cdot}}\rangle$  ein Vektor-Raum ist.
\eox


\exercise
Es sei $\bigl\langle \langle V, 0, +\rangle, \cdot\bigr\rangle$ ein $\mathbb{K}$-Vektor-Raum.  Beweisen Sie:
\renewcommand{\labelenumi}{(\alph{enumi})}
\begin{enumerate}
\item $0 \cdot \vec{x} = \vec{0}$ \quad f.a. $\vec{x} \in V$,
\item $\forall \alpha \in \mathbb{K}: \forall \vec{x} \in V: \bigl(\alpha \cdot \vec{x} = \vec{0} \rightarrow \alpha = 0 \vee \vec{x} = 0\bigr)$,
\item $(-1) \cdot \vec{x} = -\vec{x}$ \quad f.a. $\vec{x} \in V$,

      wobei hier mit $-\vec{x}$ das additive Inverse von $x$ in der Gruppe $\langle V, \vec{0}, +\rangle$
      bezeichnet wird. 
      \eoxs
\end{enumerate}
\renewcommand{\labelenumi}{\arabic{enumi}.}

\section{Basis und Dimension}
In diesem Abschnitt f\"{u}hren wir den f\"{u}r die Theorie der Vektor-R\"{a}ume zentralen Begriff
der \colorbox{yellow}{\emph{Dimension}}
ein.  Dazu definieren wir zun\"{a}chst, was wir unter einem \colorbox{yellow}{\emph{Erzeugenden-System}}
verstehen und wann eine Menge von Vektoren \colorbox{yellow}{\emph{linear unabh\"{a}ngig}} ist.

\begin{Definition}[Linear-Kombination] \lb
Es sei $\mathcal{V} = \bigl\langle \langle V, \vec{0}, + \rangle, \cdot \bigr\rangle$ ein $\mathbb{K}$-Vektor-Raum.  Ein Vektor $\vec{x} \in V$ ist eine
\colorbox{yellow}{\emph{Linear-Kombination}} der Vektoren $\vec{y}_1, \cdots, \vec{y}_n \in V$ genau dann, wenn es Skalare
$\alpha_1, \cdots, \alpha_n \in \mathbb{K}$ gibt, so dass die Gleichung
\\[0.2cm]
\hspace*{1.3cm}
$\vec{x} = \alpha_1 \cdot \vec{y}_1 + \cdots + \alpha_n \cdot \vec{y}_n$
\\[0.2cm]
gilt.  
\colorbox{red}{Zus\"{a}tzlich m\"{u}ssen die Vektoren $\vec{y}_i$ alle paarweise verschieden sein.}
Die obige Gleichung werden wir in Zukunft der K\"{u}rze halber gelegentlich auch in der Form
\\[0.2cm]
\hspace*{1.3cm}
$\vec{x} = \sum\limits_{i=1}^n \alpha_i \cdot \vec{y}_i$
\\[0.2cm]
schreiben.
\eoxs
\end{Definition}

\example
Definieren wir
\\[0.2cm]
\hspace*{1.3cm}
$\vec{y}_1 := [1, 0, 2]$, \quad  $\vec{y}_2 := [1, 2, 0]$, \quad  $\vec{y}_3 := [0, -1, 3]$  \quad und \quad  $\vec{x} := [3, 1, 11]$,
\\[0.2cm]
so ist $\vec{x}$ eine Linear-Kombination der Vektoren $\vec{y}_1$, $\vec{y}_2$, $\vec{y}_3$, denn es gilt
\\[0.2cm]
\hspace*{1.3cm}
$\vec{x} = 1 \cdot \vec{y}_1 + 2 \cdot \vec{y}_2 + 3 \cdot \vec{y}_3$.  \eoxs

\begin{Definition}[linear unabh\"{a}ngig] \lb
Es sei $\mathcal{V} = \bigl\langle \langle V, \vec{0}, + \rangle, \cdot \bigr\rangle$ ein $\mathbb{K}$-Vektor-Raum.  
Eine Menge $B \subseteq V$ ist \colorbox{yellow}{\emph{linear unabh\"{a}ngig}} genau dann, wenn 
f\"{u}r jede endliche Teilmenge
\\[0.2cm]
\hspace*{1.3cm}
$\bigl\{ \vec{x}_1, \cdots, \vec{x}_n \bigr\} \subseteq B$,
\\[0.2cm]
\colorbox{red}{bei der die Vektoren $\vec{x}_i$ paarweise verschieden sind,} die Formel
\\[0.2cm]
\hspace*{1.3cm}
$\sum\limits_{i=1}^n \alpha_i \cdot \vec{x}_i = \vec{0} \;\Rightarrow\; \forall i \in \{1,\cdots,n\}: \alpha_i = 0$
\\[0.2cm]
gilt.  Mit anderen Worten:  Der Nullvektor $\vec{0}$ l\"{a}sst sich nur als die sogenannte
\colorbox{yellow}{\emph{triviale Linear-Kombination}} aus Vektoren der Menge $B$ darstellen.  Demgegen\"{u}ber hei\3t eine
Menge $B \subseteq V$ \colorbox{yellow}{\emph{linear abh\"{a}ngig}} genau dann, wenn $B$ nicht linear unabh\"{a}ngig ist.  In
diesem Fall gibt es dann Vektoren $\vec{x}_1$, $\cdots$, $\vec{x}_n \in B$, die paarweise
verschieden sind, sowie Skalare $\alpha_1,\cdots,\alpha_n \in \mathbb{K}$, so dass einerseits
\\[0.2cm]
\hspace*{1.3cm}
$\sum\limits_{i=1}^n \alpha_i \cdot \vec{x}_i = \vec{0}$, \quad aber andererseits auch \quad
$\exists i \in \{1,\cdots,n\}: \alpha_i \,\not=\, 0$
\\[0.2cm]
gilt.  
\eoxs
\end{Definition}

\example
Definieren wir
\\[0.2cm]
\hspace*{1.3cm}
$\vec{y}_1 := [1, 0, 2]$, \quad  $\vec{y}_2 := [1, 2, 0]$ \quad und \quad $\vec{y}_3 := [0, -1, 3]$,
\\[0.2cm]
so ist die Menge $B := \{ \vec{y}_1, \vec{y}_2, \vec{y}_3 \}$
linear unabh\"{a}ngig.  Zum Beweis dieser Behauptung nehmen wir zun\"{a}chst an, dass es $\alpha_1$,
$\alpha_2$ und $\alpha_3$, gibt, so dass
\\[0.2cm]
\hspace*{1.3cm}
$\vec{0} = \alpha_1 \cdot [1,0,2] + \alpha_2 \cdot [1,2,0] + \alpha_3 \cdot [0,-1,3]$
\\[0.2cm]
gilt.  Ersetzen wir $\vec{0}$ durch die Liste $[0,0,0]$ und rechnen die rechte Seite
dieser Gleichung aus, so erhalten wir die Gleichung
\\[0.2cm]
\hspace*{1.3cm}
$[0,0,0] = [\alpha_1  + \alpha_2, 2 \cdot \alpha_2 - \alpha_3, 2 \cdot \alpha_1 + 3 \cdot \alpha_3]$.
\\[0.2cm]
Die drei Komponenten der Vektoren auf der linken und der rechten Seite dieser Gleichung m\"{u}ssen
gleich sein.  Damit m\"{u}ssen die drei Gleichungen
\\[0.2cm]
\hspace*{1.3cm}
$0 = \alpha_1 + \alpha_2$, \quad $0 = 2 \cdot \alpha_2 - \alpha_3$ \quad und \quad $0 = 2 \cdot \alpha_1 + 3 \cdot \alpha_3$
\\[0.2cm]
gelten.  Aus der zweiten Gleichung folgt nun $\alpha_3 = 2 \cdot \alpha_2$, w\"{a}hrend aus der ersten
Gleichung $\alpha_1 = - \alpha_2$ folgt.  Ersetzen wir nun in der letzten Gleichung 
$\alpha_1$ durch $-\alpha_2$ und $\alpha_3$ durch $2 \cdot \alpha_2$, so erhalten wir die neue Gleichung
\\[0.2cm]
\hspace*{1.3cm}
$0 = 2 \cdot (- \alpha_2) + 3 \cdot 2 \cdot \alpha_2$,
\\[0.2cm]
die wir auch als $0 = 4 \cdot \alpha_2$ schreiben k\"{o}nnen.  Daraus folgt aber sofort $\alpha_2 = 0$
und das impliziert dann auch $\alpha_1 = 0$ und $\alpha_3 = 0$.  Damit haben wir gezeigt, dass die drei
Vektoren $\vec{y}_1$, $\vec{y}_2$, $\vec{y}_3$ sich nur trivial zu dem Null-Vektor
$\vec{0}$ kombinieren lassen.  Folglich ist die Menge $B$ linear unabh\"{a}ngig.
\eox

\begin{Definition}[Erzeugenden-System]
  Es sei $\mathcal{V} = \bigl\langle \langle V, \vec{0}, + \rangle, \cdot \bigr\rangle$ ein $\mathbb{K}$-Vektor-Raum 
  und $B \subseteq V$.  Die Teilmenge $B$ ist ein
  \colorbox{yellow}{\emph{Erzeugenden-System}} des Vektor-Raums $\mathcal{V}$ genau dann, wenn sich jeder Vektor 
  $\vec{x} \in V$ als Linear-Kombination von Vektoren aus $B$ schreiben l\"{a}sst. 
  Als Formel schreibt sich dies wie folgt:
  \\[0.2cm]
  \hspace*{1.3cm}
  $\forall \vec{x} \in V: \exists n \in \mathbb{N}: \exists  \vec{y}_1, \cdots,
  \vec{y}_n \in B: \exists \alpha_1, \cdots, \alpha_n \in \mathbb{K}: 
  \vec{x} = \sum\limits_{i=1}^n \alpha_i \cdot \vec{y}_i
  $. \eoxs
\end{Definition}

F\"{u}r jeden Vektor-Raum $\mathcal{V} = \bigl\langle \langle V, \vec{0}, + \rangle, \cdot \bigr\rangle$ 
gibt es ein triviales Erzeugenden-System, denn nat\"{u}rlich ist die gesamte 
Menge $V$ ein Erzeugenden-System von $\mathcal{V}$.  Um das einzusehen, setzen wir f\"{u}r einen gegebenen Vektor
$\vec{x} \in V$ in der obigen Definition $n:=1$, $\vec{y}_1 := \vec{x}$ und $\alpha_1 := 1$
und haben dann trivialerweise
\\[0.2cm]
\hspace*{1.3cm}
$\vec{x} = 1 \cdot \vec{x} = 1 \cdot \vec{y}_1$,
\\[0.2cm]
womit $\vec{x}$ als Linear-Kombination von Elementen der Menge $V$ dargestellt ist.  Aber
nat\"{u}rlich ist ein solches Erzeugenden-System nicht sonderlich interessant.  Interessanter sind
Erzeugenden-Systeme, die m\"{o}glichst wenige Elemente haben, die also bez\"{u}glich der Anzahl
der Elemente \emph{minimal} sind.  Dies f\"{u}hrt zu der folgenden zentralen Definition.

\begin{Definition}[Basis]
  Es sei $\mathcal{V} = \bigl\langle \langle V, \vec{0}, + \rangle, \cdot \bigr\rangle$ ein $\mathbb{K}$-Vektor-Raum 
  und $B \subseteq V$.  Die Teilmenge $B$ ist eine
  \colorbox{yellow}{\emph{Basis}} von $V$, wenn das Folgende gilt:
  \begin{enumerate}
  \item $B$ ist ein Erzeugenden-System von $V$ und
  \item $B$ ist linear unabh\"{a}ngig.  \eoxs
  \end{enumerate}
\end{Definition}

\noindent
Der n\"{a}chste Satz zeigt, dass eine Basis eine \colorbox{red}{maximale} Menge linear unabh\"{a}ngiger Vektoren ist.

\begin{Satz}
  Es sei $\mathcal{V} = \bigl\langle \langle V, \vec{0}, + \rangle, \cdot \bigr\rangle$ 
  ein $\mathbb{K}$-Vektor-Raum und $B$ sei eine Basis von $\mathcal{V}$.  Ist $\vec{x} \in V \backslash B$,
  so ist die Menge $B \cup \{ \vec{x} \}$ linear abh\"{a}ngig.
\end{Satz}

\proof
Da $B$ eine Basis ist, ist $B$ insbesondere auch ein Erzeugenden-System von $\mathcal{V}$.  Damit gibt es ein
$n \in \mathbb{N}$ und Vektoren $\vec{y}_1,\cdots,\vec{y}_n \in B$ sowie Skalare $\alpha_1, \cdots,\alpha_n \in \mathbb{K}$,
so dass
\\[0.2cm]
\hspace*{1.3cm}
$\vec{x} = \alpha_1 \cdot \vec{y}_1 + \cdots + \alpha_n \cdot \vec{y}_n$
\\[0.2cm]
gilt.  O.B.d.A. k\"{o}nnen wir hier davon ausgehen, dass die Vektoren $\vec{y}_i$ paarweise
verschieden sind, denn wenn f\"{u}r $i\not= j$ die Gleichung $\vec{y}_i = \vec{y}_j$
gelten sollte, so k\"{o}nnen wir den Vektor $\vec{y}_j$ in der obigen Summe fallen lassen,
indem wir $\alpha_i$ durch $\alpha_i + \alpha_j$ ersetzen.

Stellen wir die obige Gleichung f\"{u}r $\vec{x}$ zu der Gleichung
\\[0.2cm]
\hspace*{1.3cm}
$\vec{0} = (-1) \cdot \vec{x} + \alpha_1 \cdot \vec{y}_1 + \cdots + \alpha_n \cdot \vec{y}_n$
\\[0.2cm]
um, so haben wir eine nicht-triviale Linear-Kombination des Null-Vektors aus Vektoren der Menge 
$B \cup \{ \vec{x} \}$ gefunden.  Dies zeigt, dass die Menge $B \cup \{ \vec{x} \}$ linear abh\"{a}ngig
ist. \qeds

\exercise
\"{U}berlegen Sie, an welcher Stelle die Voraussetzung $\vec{x} \not\in B$ in dem obigen Beweis benutzt wird!
\eox

\noindent
Der letzte Satz l\"{a}sst sich in dem folgenden Sinne umkehren.

\begin{Satz}
  Es sei $\mathcal{V} = \bigl\langle \langle V, \vec{0}, + \rangle, \cdot \bigr\rangle$ ein $\mathbb{K}$-Vektor-Raum 
  und $B \subseteq V$.  Falls $B$ eine \emph{maximale} linear 
  unabh\"{a}ngige Teilmenge von $V$ ist, falls also gilt:
  \begin{enumerate}
  \item $B$ ist linear unabh\"{a}ngig und
  \item f\"{u}r alle Vektoren $\vec{x} \in V \backslash B$ ist die Menge $B \cup \{ \vec{x} \}$
        linear abh\"{a}ngig,
  \end{enumerate}
  dann ist $B$ schon eine Basis von $V$.
\end{Satz}

\proof
Wir m\"{u}ssen nur noch zeigen, dass $B$ ein Erzeugenden-System von $\mathcal{V}$ ist.  Dazu ist nachzuweisen,
dass sich jeder Vektor $\vec{x} \in V$ als Linear-Kombination von Vektoren aus $B$ schreiben
l\"{a}sst.  Wir unterscheiden zwei F\"{a}lle:
\begin{enumerate}
\item $\vec{x} \in B$.

      In diesem Fall setzen wir $n := 1$, $\vec{y}_1 := \vec{x}$ und $\alpha_1 := 1$.  Damit
      gilt offenbar
      \\[0.2cm]
      \hspace*{1.3cm}
      $\vec{x} = \alpha_1 \cdot \vec{y}_1$
      \\[0.2cm]
      und wir haben die gesuchte Linear-Kombination gefunden.
\item $\vec{x} \not\in B$.

      Nach Voraussetzung ist die Menge $B \cup \{ \vec{x} \}$ linear abh\"{a}ngig.  Damit l\"{a}sst sich
      der Null-Vektor als nicht-triviale Linear-Kombination von Vektoren aus $B \cup \{\vec{x}\}$
      schreiben.  Nun gibt es zwei M\"{o}glichkeiten:
      \begin{enumerate}
      \item Der Vektor $\vec{x}$ wird bei dieser Linear-Kombination gar nicht ben\"{o}tigt.
            Dann h\"{a}tten wir aber eine nicht-triviale Linear-Kombination des Null-Vektors aus
            Vektoren der Menge $B$.  Da die Menge $B$ nach Voraussetzung linear unabh\"{a}ngig ist,
            kann dieser Fall nicht eintreten.
      \item Der Vektor $\vec{x}$ tritt in der nicht-trivialen Linear-Kombination des
            Null-Vektors auf.  Es gibt dann ein $n \in \mathbb{N}$, sowie Vektoren $\vec{y}_1,
            \cdots, \vec{y}_n$ und Skalare $\alpha_1, \cdots, \alpha_n, \alpha_{n+1}$,
            so dass 
            \\[0.2cm]
            \hspace*{1.3cm}
            $\vec{0} = \alpha_1 \cdot \vec{y}_1 + \cdots + \alpha_n \cdot \vec{y}_n + \alpha_{n+1} \cdot \vec{x}$
            \\[0.2cm]
            gilt.  Hierbei muss $\alpha_{n+1} \not=0$ gelten, denn sonst w\"{u}rde $\vec{x}$ in der
            Linear-Kombination gar nicht ben\"{o}tigt und diesen Fall haben wir ja bereits
            ausgeschlossen.  Damit 
            k\"{o}nnen wir die obige Gleichung zu
            \\[0.2cm]
            \hspace*{1.3cm}
            $\vec{x} = -\bruch{\alpha_1}{\alpha_{n+1}} \cdot \vec{y}_1 - \cdots - \bruch{\alpha_n}{\alpha_{n+1}} \cdot \vec{y}_n$
            \\[0.2cm] 
            umstellen.  Diese Gleichung zeigt, dass $\vec{x}$ sich als Linear-Kombination von Vektoren aus
            $B$ schreiben l\"{a}sst und das war zu zeigen. 
      \end{enumerate}
\end{enumerate}
Insgesamt haben wir jetzt gezeigt, dass sich jeder Vektor $\vec{x}\in V$ als Linear-Kombination
von Vektoren aus $B$ schreiben l\"{a}sst und damit ist $B$ ein Erzeugenden-System von $V$.  \qed

Die letzten beiden S\"{a}tze lassen sich dahingehen zusammenfassen, dass eine Menge
$B \subseteq V$ genau dann eine Basis von $V$ ist, wenn $B$ eine maximale linear unabh\"{a}ngige
Teilmenge von $V$ ist.  Die n\"{a}chsten beide S\"{a}tze zeigen, dass sich eine Basis auch als minimales
Erzeugenden-System charakterisieren l\"{a}sst.

\begin{Satz}
  Es sei $\mathcal{V} = \bigl\langle \langle V, \vec{0}, + \rangle, \cdot \bigr\rangle$ ein
  $\mathbb{K}$-Vektor-Raum und $B$ sei eine Basis von $\mathcal{V}$.  Ist $\vec{x} \in B$,
  so ist die Menge $B \backslash \{ \vec{x} \}$ kein Erzeugenden-System von $\mathcal{V}$.
\end{Satz}

\proof
Wir f\"{u}hren den Beweis indirekt und nehmen an, dass die Menge $B \backslash \{ \vec{x} \}$ ein 
Erzeugenden-System von $\mathcal{V}$ ist.  Dann m\"{u}sste sich insbesondere auch der Vektor $\vec{x}$ als
Linear-Kombination von Vektoren aus  $B \backslash \{ \vec{x} \}$ schreiben lassen.  Es g\"{a}be dann
also ein $n \in \mathbb{N}$ sowie Vektoren  $\vec{y}_1, \cdots, \vec{y}_n \in B\backslash \{\vec{x}\}$ und Skalare 
$\alpha_1, \cdots, \alpha_n \in \mathbb{K}$, so dass
\\[0.2cm]
\hspace*{1.3cm}
$\vec{x} = \alpha_1 \cdot \vec{y_1} + \cdots + \alpha_n \cdot \vec{y}_n$
\\[0.2cm]
gelten w\"{u}rde.  Diese Gleichung k\"{o}nnen wir zu
\\[0.2cm]
\hspace*{1.3cm}
$\vec{0} =  \alpha_1 \cdot \vec{y_1} + \cdots + \alpha_n \cdot \vec{y}_n + (-1) \cdot \vec{x}$
\\[0.2cm]
umstellen.  Da die Vektoren $\vec{y}_1, \cdots, \vec{y}_n, \vec{x}$ verschiedene Vektoren aus $B$ sind,
h\"{a}tten wir damit eine nicht-triviale Linear-Kombination des Null-Vektors gefunden, was der Tatsache
widerspricht, dass die Menge $B$ als Basis insbesondere linear unabh\"{a}ngig ist. \qed

Der letzte Satz zeigt, dass eine Basis ein \emph{minimales Erzeugenden-System} des Vektor-Raums $\mathcal{V}$
ist.  Wie wir jetzt sehen werden, l\"{a}sst sich dieser Satz auch umkehren.

\begin{Satz}
  Es sei $\mathcal{V} = \bigl\langle \langle V, \vec{0}, + \rangle, \cdot \bigr\rangle$ ein $\mathbb{K}$-Vektor-Raum 
  und $B \subseteq V$.  Falls $B$ eine \emph{minimales Erzeugenden-System}
  von $\mathcal{V}$ ist, falls also gilt:
  \begin{enumerate}
  \item $B$ ist ein Erzeugenden-System von $\mathcal{V}$ und
  \item f\"{u}r alle Vektoren $\vec{x} \in B$ ist die Menge $B \backslash \{ \vec{x} \}$
        kein Erzeugenden-System von $\mathcal{V}$,
  \end{enumerate}
  dann ist $B$ schon eine Basis von $\mathcal{V}$.
\end{Satz}

\proof
Da die Menge $B$ nach Voraussetzung bereits ein Erzeugenden-System von $\mathcal{V}$ ist, m\"{u}ssen wir
lediglich zeigen, dass $B$ linear unabh\"{a}ngig ist.  Wir f\"{u}hren auch diesen Beweis als
Widerspruchsbeweis und nehmen an, dass $B$ linear abh\"{a}ngig w\"{a}re.  Mit dieser Annahme finden wir eine
nicht-triviale Linear-Kombination des Null-Vektors mit Hilfe von Vektoren aus $B$, wir finden also
ein $n \in \mathbb{N}$ sowie paarweise verschiedene Vektoren $\vec{y}_1, \cdots, \vec{y}_n \in B$ und Skalare
$\alpha_1, \cdots, \alpha_n \in \mathbb{K}$, so dass
\\[0.2cm]
\hspace*{1.3cm}
$\vec{0} =  \alpha_1 \cdot \vec{y}_1 + \cdots + \alpha_n \cdot \vec{y}_n$
\\[0.2cm]
gilt, wobei wenigstens einer der Skalare $\alpha_i$ von $0$ verschieden ist.  Sei also $\alpha_i \not= 0$. 
Dann k\"{o}nnen wir die obige Gleichung zu 
\\[0.2cm]
\hspace*{1.3cm}
$\ds(-\alpha_i) \cdot \vec{y_i} = \sum\limits_{j=1 \atop j\not=i}^n \alpha_j \cdot \vec{y}_j$
\\[0.2cm]
umstellen.  Da $\alpha_i \not= 0$ ist, k\"{o}nnen wir durch $\alpha_i$ teilen und finden
\\[0.2cm]
\hspace*{1.3cm}
$\ds \vec{y_i} = \sum\limits_{j=1 \atop j\not=i}^n \bruch{-\alpha_j}{\alpha_i} \cdot \vec{y}_j$.
\\[0.2cm]
Die letzte Gleichung zeigt, dass sich $\vec{y}_i$ als Linear-Kombination der Vektoren
$\vec{y}_1, \cdots, \vec{y}_{i-1}$,  $\vec{y}_{i+1}, \cdots, \vec{y}_n$ schreiben l\"{a}sst.  Wir werden jetzt zeigen, dass damit dann auch die Menge
$B \backslash \{ \vec{y_i} \}$ ein Erzeugenden-System von $V$ ist.  Sei dazu $\vec{x}$ ein
beliebiger Vektor aus $V$.  Da $B$ ein Erzeugenden-System von $\mathcal{V}$ ist, gibt es zun\"{a}chst 
ein $m \in\mathbb{N}$ sowie Vektoren $\vec{z}_1, \cdots, \vec{z}_m \in B$ und Skalare $\beta_1, \cdots,
\beta_m \in \mathbb{K}$, so dass
\\[0.2cm]
\hspace*{1.3cm}
$\ds \vec{x} = \sum\limits_{j=1}^m \beta_j \cdot \vec{z}_j$
\\[0.2cm]
gilt.   Falls alle Vektoren $\vec{z}_j$ von $\vec{y}_i$ verschieden sind,  ist nichts mehr zu
zeigen, denn dann haben wir $\vec{x}$ bereits als Linear-Kombination von Vektoren der Menge $B \backslash \{\vec{y_i}\}$ 
geschrieben.  Sollte allerdings eines der $\vec{z}_j$, sagen wir
$\vec{z}_k$, mit $\vec{y_i}$ identisch sein, dann k\"{o}nnen wir die obige Gleichung wie folgt
umschreiben:
\\[0.2cm]
\hspace*{1.3cm}
$\ds\vec{x} = \sum\limits_{j=1 \atop j \not=k}^m \beta_j \cdot \vec{z}_j + 
              \beta_k \cdot \sum\limits_{j=1 \atop j \not= i} \bruch{-\alpha_j}{\alpha_i} \cdot \vec{y}_j
$
\\[0.2cm]
Auch in diesem Fall haben wir also $\vec{x}$ als Linear-Kombination von Vektoren der Menge 
$B \backslash \{ \vec{y}_i \}$ schreiben k\"{o}nnen.  Dies zeigt, dass die Menge
$B \backslash \{ \vec{y}_i \}$ bereits ein Erzeugenden-System von $\mathcal{V}$ ist und steht im
Widerspruch dazu, dass $B$ ein minimales Erzeugenden-System ist.  Dieser Widerspruch zeigt, dass die
Annahme, dass $B$ linear abh\"{a}ngig ist, falsch sein muss. \qeds

\exercise
Es sei $\mathcal{V} = \bigl\langle \langle V, \vec{0}, + \rangle, \cdot \bigr\rangle$ ein $\mathbb{K}$-Vektor-Raum und es gelte $B := \{ \vec{x}_1, \cdots, \vec{x}_n \} \subseteq V$.
Zeigen Sie:  $B$ ist genau dann eine Basis von $\mathcal{V}$, wenn sich jeder Vektor $\vec{y} \in V$ in
eindeutiger Weise als Linear-Kombination der Vektoren aus $B$ schreiben l\"{a}sst.
\eox

\begin{Lemma}[Basis-Austausch-Lemma]
  Es gelte:
  \begin{enumerate}
  \item $\mathcal{V} = \bigl\langle \langle V, \vec{0}, + \rangle, \cdot \bigr\rangle$ ist ein $\mathbb{K}$-Vektor-Raum, 
  \item $B$ eine Basis von $\mathcal{V}$,
  \item $U \subseteq B$,
  \item $\vec{x} \in V\backslash B$ und 
  \item $U \cup \{ \vec{x} \}$ ist linear unabh\"{a}ngig.
  \end{enumerate}
  Dann gibt es einen Vektor $\vec{y} \in B \backslash U$, so dass die Menge
  $\bigl(B \backslash \{ \vec{y} \}\bigr) \cup \{ \vec{x} \}$ wieder eine Basis von $\mathcal{V}$ ist:
  \\[0.2cm]
  \hspace*{1.3cm}
  $\exists \vec{y} \in B \backslash U: \bigl(B \backslash \{ \vec{y} \}\bigr) \cup \{ \vec{x} \}$ Basis von $\mathcal{V}$.
  \\[0.2cm]
  Wir k\"{o}nnen also den Vektor $\vec{x}$ so gegen einen Vektor $\vec{y}$ aus der Basis $B$
  austauschen, dass die Vektoren aus der Menge $U$ weiterhin Bestandteil der Basis sind.
\end{Lemma}

\proof
Da $B$ eine Basis ist, l\"{a}sst sich $\vec{x}$ als Linear-Kombination von Vektoren aus $B$
darstellen.  Es gibt also ein $n \in \mathbb{N}$, Skalare $\alpha_1, \cdots, \alpha_n \in \mathbb{K}$ 
und Vektoren $\vec{y}_1, \cdots, \vec{y}_n\in B$, so dass 
\\[0.2cm]
\hspace*{1.3cm}
$\vec{x} = \sum\limits_{i=1}^n \alpha_i \cdot \vec{y}_i$ 
\\[0.2cm]
gilt.  Da die Menge $U \cup \{ \vec{x} \}$ linear unabh\"{a}ngig ist, ist $\vec{x}$ sicher von $\vec{0}$
verschieden und damit k\"{o}nnen in der Gleichung f\"{u}r $\vec{x}$ 
nicht alle $\alpha_i$ den Wert $0$ haben.  O.B.d.A.~k\"{o}nnen wir sogar fordern,
dass alle $\alpha_i \not= 0$ sind, denn falls $\alpha_i = 0$ w\"{a}re, w\"{u}rden wir den Term $\alpha_i \cdot \vec{y}_i$
in der obigen Summe einfach weglassen.
Aus der Tatsache, dass die Menge $U \cup \{ \vec{x} \}$ linear unabh\"{a}ngig ist, folgt au\3erdem, dass in der obigen Darstellung
nicht alle Vektoren $\vec{y}_i$ Elemente der Menge $U$ sind, denn wir k\"{o}nnen die obige Gleichung zu
\\[0.2cm]
\hspace*{1.3cm}
$\ds\vec{0} = \sum\limits_{i=1}^n \alpha_i \cdot \vec{y}_i + (-1) \cdot \vec{x}$ 
\\[0.2cm]
umstellen und wenn nun alle $\vec{y}_i \in U$ w\"{a}ren, dann w\"{u}rde das der linearen Unabh\"{a}ngigkeit von
$U \cup \{\vec{x} \}$ widersprechen.  Es gibt also ein $k \in \{1,\cdots,n\}$, so dass $\vec{y}_k \in B \backslash U$ ist.  Wir definieren
\\[0.2cm]
\hspace*{1.3cm}
$\vec{y} := \vec{y}_k$, \quad woraus bereits $\vec{y} \in B \backslash U$ folgt.
\\[0.2cm]
Au\3erdem bemerken wir, dass $\vec{y}_i \in B \backslash \{\vec{y}\}$ ist f\"{u}r alle
$i \in \{1,\cdots,n\} \backslash \{k\}$, denn die Vektoren $\vec{y}_i$ sind f\"{u}r alle $i\in \{1,\cdots,n\}$ paarweise verschieden.
Wir stellen die obige Gleichung f\"{u}r $\vec{x}$ wie folgt um:
\\[0.2cm]
\hspace*{1.3cm}
$\ds\vec{x} = \sum\limits_{i=1 \atop i \not= k}^n \alpha_i \cdot \vec{y}_i +  \alpha_k \cdot \vec{y}$.
\\[0.2cm]
Diese Gleichung l\"{o}sen wir nach $\vec{y}$ auf und erhalten
\\[0.2cm]
\hspace*{1.3cm}
$\ds\vec{y} = \bruch{1}{\alpha_k} \cdot \vec{x} - \sum\limits_{i=1 \atop i \not= k}^n
\bruch{\alpha_i}{\alpha_k} \cdot \vec{y}_i$. \hspace*{\fill} $(*)$
\\[0.2cm]
Wir m\"{u}ssen nun zeigen, dass die Menge 
\\[0.2cm]
\hspace*{1.3cm}
 $(B \backslash \{ \vec{y} \}) \cup \{ \vec{x} \}$
\\[0.2cm]
eine Basis von $V$ ist.  Dazu sind zwei Dinge nachzuweisen.
\begin{enumerate}
\item Als erstes zeigen wir, dass  $(B \backslash \{ \vec{y} \}) \cup \{ \vec{x} \}$ ein Erzeugenden-System von $\mathcal{V}$ ist.

      Dazu betrachten wir einen beliebigen Vektor $\vec{z} \in V$.  Da $B$ ein Erzeugenden-System
      von $\mathcal{V}$ ist, l\"{a}sst sich $\vec{z}$ als Linear-Kombination von Vektoren aus $B$ darstellen.  Es gibt
      also ein $m \in \mathbb{N}$, Skalare $\beta_1, \cdots, \beta_m \in \mathbb{K}$ und 
      Vektoren $\vec{u}_1, \cdots, \vec{u}_m \in B$, so dass
      \\[0.2cm]
      \hspace*{1.3cm}
      $\ds\vec{z} = \sum\limits_{j=1}^m \beta_j \cdot \vec{u}_j$
      \\[0.2cm]
      gilt.   Falls nun alle $\vec{u}_j$ von dem Vektor $\vec{y}$ verschieden sind, 
      dann haben wir $\vec{z}$ bereits als Linear-Kombination von Vektoren der Menge
      $B \backslash \{ \vec{y} \} \cup \{ \vec{x} \}$ dargestellt und der Beweis ist
      abgeschlossen.  Andernfalls gilt $\vec{u}_l = \vec{y}$ f\"{u}r ein $l \in \{1,\cdots,m\}$.
      In diesem Fall haben wir
      \\[0.2cm]
      \hspace*{1.3cm}
      $\ds\vec{z} = \sum\limits_{j=1 \atop j \not= l}^m \beta_j \cdot \vec{u}_j + \beta_l \cdot \vec{y}$.
      \\[0.2cm]
      Hier setzen wir f\"{u}r $\vec{y}$ den Wert ein, den wir in der Gleichung $(*)$ oben gefunden
      haben. Das liefert
      \\[0.2cm]
      \hspace*{1.3cm}
      $\ds\vec{z} = \sum\limits_{j=1 \atop j \not= l}^m \beta_j \cdot \vec{u}_j +
      \bruch{\beta_l}{\alpha_k} \cdot \vec{x} - \sum\limits_{i=1 \atop i \not= k}^n
      \bruch{\beta_l \cdot \alpha_i}{\alpha_k} \cdot \vec{y}_i$.
      \\[0.2cm]
      In dieser Darstellung sind nun alle der beteiligten Vektoren Elemente der Menge
      $(B \backslash \{ \vec{y} \}) \cup \{ \vec{x} \}$.  Damit haben wir also gezeigt, dass der
      Vektor $\vec{z}$ als Linear-Kombination dieser Menge dargestellt werden kann.
      Da $\vec{z}$ ein beliebiger Vektor aus $V$ war, ist damit gezeigt, dass die Menge
      $(B \backslash \{ \vec{y} \}) \cup \{ \vec{x} \}$ ein Erzeugenden-System von $\mathcal{V}$ ist.
\item Als n\"{a}chstes ist nachzuweisen, dass die Menge  $(B \backslash \{ \vec{y} \}) \cup \{ \vec{x} \}$
      linear unabh\"{a}ngig ist.  Dazu nehmen wir an, dass wir eine Linear-Kombination von Vektoren aus
      der Menge $(B \backslash \{ \vec{y} \}) \cup \{ \vec{x} \}$ haben, die den Null-Vektor ergibt.
      Wir haben dann also ein $m \in \mathbb{N}$, Skalare $\gamma_1, \cdots, \gamma_m, \delta \in \mathbb{K}$,
      sowie paarweise verschiedene Vektoren $\vec{v}_1, \cdots, \vec{v}_m \in B \backslash \{ \vec{y} \}$, so dass
      \\[0.2cm]
      \hspace*{1.3cm}
      $\ds\vec{0} = \sum\limits_{j=1}^m \gamma_j \cdot \vec{v}_j + \delta \cdot \vec{x}$
      \\[0.2cm]
      gilt.  Wir m\"{u}ssen nun zeigen, dass $\gamma_j = 0$ f\"{u}r alle $j \in \{1,\cdots,n\}$ gilt und dass au\3erdem 
      $\delta = 0$ ist.  Wir f\"{u}hren diesen Nachweis durch eine Fallunterscheidung.
      \begin{enumerate}
      \item $\delta = 0$:  Dann haben wir
            \\[0.2cm]
            \hspace*{1.3cm}
            $\ds\vec{0} = \sum\limits_{j=1}^m \gamma_j \cdot \vec{v}_j$
            \\[0.2cm]
            und da die Vektoren $\vec{v}_j$ Elemente der linear unabh\"{a}ngigen Menge $B$ sind, folgt 
            $\gamma_j = 0$ f\"{u}r alle $j \in \{1,\cdots,m\}$.   
      \item $\delta \not= 0$:  Jetzt setzen wir f\"{u}r $\vec{x}$ in der Darstellung des Null-Vektors den Wert 
            \\[0.2cm]
            \hspace*{1.3cm}
            $\ds\vec{x} = \sum\limits_{j=1 \atop j \not= k}^n \alpha_j \cdot \vec{y}_j +  \alpha_k \cdot \vec{y}$
            \\[0.2cm]
            ein und erhalten
            \\[0.2cm]
            \hspace*{1.3cm}
            $\ds\vec{0} = \sum\limits_{j=1}^m \gamma_j \cdot \vec{v}_j + \sum\limits_{i=1 \atop i \not= k}^n \delta\cdot \alpha_i \cdot \vec{y}_i + \delta\cdot \alpha_k \cdot \vec{y}$.
            \\[0.2cm]
            Die Vektoren $\vec{v}_j$ sowie die Vektoren $\vec{y}_i$ mit $i \not= k$ sind Elemente der Menge
            $B \backslash \{ \vec{y} \}$.  Da $\vec{y} = \vec{y}_k \in B$ ist,  haben wir dann insgesamt eine Linear-Kombination des Null-Vektors aus
            Elementen der Menge $B$.  Da die Menge $B$ linear unabh\"{a}ngig ist, folgt zun\"{a}chst, dass der Koeffizient
            $\delta\cdot \alpha_k = 0$  ist.  Wegen $\alpha_k \not= 0$ folgt daraus $\delta = 0$.  Das widerspricht aber der Annahme $\delta = 0$
            und damit sehen wir, dass dieser Fall gar nicht eintreten kann.
            \qed
      \end{enumerate}
\end{enumerate}

Das gerade bewiesene Lemma zeigt uns, dass wir in einer Basis einen beliebigen von dem Nullvektor
verschiedenen Vektor $\vec{x}$ 
gegen einen Vektor der Basis austauschen k\"{o}nnen.  Der n\"{a}chste Satz verallgemeinert diesen Tatbestand
und zeigt, dass wir 
in einer Basis $B$ eine linear unabh\"{a}ngige Menge $U$ gegen eine Teilmenge $W$ von $B$ austauschen
k\"{o}nnen, welche dieselbe Anzahl an Elementen hat wie $U$.

\begin{Satz}[Basis-Austausch-Satz]
  Es gelte:
  \begin{enumerate}
  \item $\mathcal{V} = \bigl\langle \langle V, \vec{0}, + \rangle, \cdot \bigr\rangle$ ist ein $\mathbb{K}$-Vektor-Raum,
  \item $B$ ist eine Basis von $\mathcal{V}$,
  \item $U \subseteq V$ ist linear unabh\"{a}ngig,
  \item $U$ ist endlich. 
  \end{enumerate}
  Dann gibt es eine Teilmenge $W \subseteq B$, so dass gilt:
  \begin{enumerate}
  \item $\bigl(B \backslash W\bigr) \cup U$ ist eine Basis von V,
  \item $\textsl{card}(W) = \textsl{card}(U)$.
  \end{enumerate}
  Wir k\"{o}nnen also eine gegebene linear unabh\"{a}ngige Menge $U$ in eine gegeben Basis einbinden,
  indem wir eine Teilmenge $W$ von $B$ durch $U$ ersetzten.  Diese Teilmenge enth\"{a}lt die selbe
  Anzahl von Elementen wir die Menge $U$.
\end{Satz}

\proof
Die Idee bei diesem Beweis besteht darin, dass wir die Elemente der Menge $U$ mit Hilfe
des Basis-Austausch-Lemmas sukzessive gegen Elemente der Menge $B$ austauschen.  F\"{u}r jedes
Element $\vec{x}$ aus der Menge $U$ finden wir dabei ein $\vec{y}$ aus der Menge
$B$, so dass wir in der Basis $B$ das Element $\vec{y}$ durch $\vec{x}$ ersetzen k\"{o}nnen.
Um diese Argumentation formal wasserdicht zu machen, definieren wir $n :=
\textsl{card}(U)$ und f\"{u}hren 
den exakten Beweis durch Induktion nach $n$. 
\begin{enumerate}
\item[I.A.:] $n = 0$. 

             Dann gilt offenbar $U = \{\}$ und wir k\"{o}nnen $W := \{\}$ definieren.  Wegen
             \\[0.2cm]
             \hspace*{1.3cm}
             $\bigl(B \backslash W\bigr) \cup U = \bigl(B \backslash \{\}\bigr) \cup \{\} = B$ \quad und \quad
             $\textsl{card}(W) = \textsl{card}(\{\}) = \textsl{card}(U)$
             \\[0.2cm]
             ist dann nichts mehr zu zeigen, denn $B$ ist nach Voraussetzung eine Basis von $V$.
\item[I.S.:] $n \mapsto n+1$. 

             Es gilt jetzt $\textsl{card}(U) = n + 1$. Da $U$ damit nicht leer ist, gibt es einen Vektor $\vec{x} \in U$. Wir definieren 
             $U' := U \backslash \{ \vec{x} \}$.  Damit folgt $\textsl{card}(U') = n$.  Nach Induktions-Voraussetzung
             finden wir daher eine Menge $W' \subseteq B$, so dass einerseits
             \\[0.2cm]
             \hspace*{1.3cm}
             $\textsl{card}(W')= \textsl{card}(U') = n$ 
             \\[0.2cm]
             und andererseits die Menge $\bigl(B \backslash W'\bigr) \cup U'$  eine Basis von $V$ ist.  Wir wenden nun das Basis-Austausch-Lemma
             auf die Basis $\bigl(B \backslash W'\bigr) \cup U'$, den Vektor $\vec{x}$ und die Menge $U'$ an.  Damit finden wir 
             ein $\vec{y} \in B \backslash W'$, so dass
             \\[0.2cm]
             \hspace*{1.3cm}
             $\bigl(\bigl((B \backslash W') \cup U'\bigr)  \backslash \{ \vec{y} \}\bigr) \cup \{ \vec{x} \}$
             \\[0.2cm]
             eine Basis von $\mathcal{V}$ ist.  Wir definieren 
             \\[0.2cm]
             \hspace*{1.3cm}
             $W := W' \cup \{ \vec{y} \}$ \quad und erinnern daran, dass \quad
             $U = U' \cup \{ \vec{x} \}$ 
             \\[0.2cm]
             gilt.  Damit haben wir 
             \\[0.2cm]
             \hspace*{1.3cm}
             $\bigl(\bigl((B \backslash W') \cup U'\bigr)  \backslash \{ \vec{y} \}\bigr) \cup \{ \vec{x} \} = (B \backslash W) \cup U$.
             \\[0.2cm]
             Au\3erdem gilt
             \\[0.2cm]
             \hspace*{1.3cm}
             $
             \begin{array}[t]{lcl}
               \textsl{card}(W) & = & \textsl{card}(W' \cup \{\vec{y}\}) \\
                                & = & \textsl{card}(W') + 1        \\
                                & = & \textsl{card}(U') + 1        \\
                                & = & \textsl{card}(U' \cup \{ \vec{x} \}) \\
                                & = & \textsl{card}(U).
             \end{array}
             $.
             \\[0.2cm]
             Damit ist der Beweis abgeschlossen.  \qeds
\end{enumerate}

\begin{Korollar}
  Ist $\mathcal{V} = \bigl\langle \langle V, \vec{0}, + \rangle, \cdot \bigr\rangle$ ein $\mathbb{K}$-Vektor-Raum, $B$ eine Basis von $\mathcal{V}$ und $U \subseteq V$ linear unabh\"{a}ngig,
  so gilt $\textsl{card}(U) \leq \textsl{card}(B)$.
\end{Korollar}

\proof
Der Basis-Austausch-Satz besagt, dass wir eine Teilmenge $W \subseteq B$ finden, so dass einerseits 
$\textsl{card}(W) = \textsl{card}(U)$ gilt und dass andererseits $(B\backslash W) \cup U$ eine Basis
von $B$ ist.  Aus $W \subseteq B$ folgt $\textsl{card}(W) \leq \textsl{card}(B)$ und 
wegen $\textsl{card}(W) = \textsl{card}(U)$ folgt die Behauptung. \qed

Haben wir zwei verschiedene Basen $B_1$ und $B_2$ eines Vektor-Raums $B$, so k\"{o}nnen wir mit dem letzten Korollar sowohl
$\textsl{card}(B_2) \leq \textsl{card}(B_1)$ als auch $\textsl{card}(B_1) \leq \textsl{card}(B_2)$ folgern.   Das zeigt,
dass zwei verschiedene Basen eines Vektor-Raums dieselbe Anzahl von Elementen haben. 
Ist diese Anzahl endlich und hat den Wert $n$, so bezeichnen wir sie als die \emph{Dimension} des Vektor-Raums $\mathcal{V}$ und definieren
\\[0.2cm]
\hspace*{1.3cm}
$\textsl{dim}(\mathcal{V}) := n$.
\pagebreak

\exercise
\renewcommand{\labelenumi}{(\alph{enumi})}
\begin{enumerate}
\item Zeigen Sie, dass der Vektor-Raum $\mathbb{K}^n$ die Dimension $n$ hat. 
\item Zeigen Sie, dass der Vektor-Raum $\mathbb{K}^\mathbb{N}$ keine endliche Basis hat.  \eoxs
\end{enumerate}
\renewcommand{\labelenumi}{\arabic{enumi}.}



\section{Untervektor-R\"{a}ume}
\remark
Um in den folgenden Abschnitten die Notation nicht zu schwerf\"{a}llig werden zu lassen, werden wir im Falle eines
$\mathbb{K}$-Vektor-Raums $\mathcal{V} = \bigl\langle \langle V, \vec{0}, + \rangle, \cdot \bigr\rangle$ 
nicht mehr zwischen der Struktur $\bigl\langle \langle V, \vec{0}, + \rangle, \cdot \bigr\rangle$
und der Menge $V$ unterscheiden und statt dessen die Menge $V$ als $\mathbb{K}$-Vektor-Raum bezeichnen.  Sie
sollten sich dar\"{u}ber im Klaren sein, dass wir dann eigentlich nicht die Menge $V$ meinen, sondern
die Struktur $\bigl\langle \langle V, \vec{0}, + \rangle, \cdot \bigr\rangle$.  Falls der K\"{o}rper
$\mathbb{K}$ f\"{u}r die Diskussion unwichtig  ist, sprechen wir auch k\"{u}rzer einfach von einem Vektor-Raum.
\eox

Ist $V$ ein Vektor-Raum und ist $U \subseteq V$, so dass $U$ f\"{u}r sich betrachtet ebenfalls ein
Vektor-Raum ist, so bezeichnen wir $U$ als
\emph{Untervektor-Raum} von $\mathcal{V}$.  Eine zu dem eben Gesagten \"{a}quivalente Definition folgt.

\begin{Definition}[Untervektor-Raum]
Es sei $V$ ein $\mathbb{K}$-Vektor-Raum und es gelte $U \subseteq V$.
Dann ist $U$ ein Untervektor-Raum von $V$ genau dann, 
wenn Folgendes gilt:
\begin{enumerate}
\item $U \not= \emptyset$,
\item $\forall \vec{x}, \vec{y} \in U: \vec{x} + \vec{y} \in U$ \quad und \quad
\item $\forall \alpha \in \mathbb{K}: \forall \vec{x} \in U: \alpha \cdot \vec{x} \in U$.  \eoxs
\end{enumerate}
\end{Definition}

Eine Teilmenge $U$ von $V$ ist also ein Untervektor-Raum von $V$, falls $U$ den Null-Vektor enth\"{a}lt und zus\"{a}tzlich
unter Addition und Skalar-Multiplikation abgeschlossen ist.  Es l\"{a}sst sich leicht zeigen, dass ein Untervektor-Raum $U$
auch selbst ein Vektor-Raum ist:  Die G\"{u}ltigkeit des Assoziativ-Gesetzes und der Distributiv-Gesetze folgt einfach aus
der Tatsache, dass diese Gesetze schon in $V$ gelten und damit erst recht in $U$, denn $U$ ist ja eine Teilmenge von
$V$. 

\remark
Statt der ersten Bedingung h\"{a}tten wir auch fordern k\"{o}nnen, dass $\vec{0} \in U$ gilt, denn wenn es
irgendeinen Vektor $\vec{x} \in U$ gibt,  dann ist wegen der Abgeschlossenheit von $U$ unter
Skalar-Multiplikation auch der Vektor
\\[0.2cm]
\hspace*{1.3cm}
$0 \cdot \vec{x} = 0 \in U$.  
\\[0.2cm]
Umgekehrt folgt aus $\vec{0} \in U$ nat\"{u}rlich sofort, dass $U$ nicht leer ist.  In der Praxis werden
wir den Nachweis, dass $U \not= \emptyset$ ist, immer dadurch f\"{u}hren, dass wir $\vec{0} \in U$
zeigen. 
\eoxs

\example
Es sei $V = \mathbb{R}^3$.  Ist weiter $\vec{z} = [ z_1, z_2, z_3 ] \in \mathbb{R}^3$ und definieren wir
\\[0.2cm]
\hspace*{1.3cm}
$U := \{ \alpha \cdot \vec{z} \mid \alpha \in \mathbb{R} \}$,
\\[0.2cm]
so ist $U$ ein Untervektor-Raum des $\mathbb{R}^3$.

\proof
Es sind drei Eigenschaften zu pr\"{u}fen:
\begin{enumerate}
\item Offenbar gilt $\vec{0} = 0 \cdot \vec{z} \in U$ und damit ist die erste Eigenschaft bereits gezeigt.
\item Seien $\vec{x}, \vec{y} \in U$.  Dann gibt es $\alpha, \beta \in \mathbb{R}$, so dass
      \\[0.2cm]
      \hspace*{1.3cm}
      $\vec{x} = \alpha \cdot \vec{z}$ \quad und \quad $\vec{y} = \beta \cdot \vec{z}$
      \\[0.2cm]
      gilt.  Daraus folgt unter Benutzung des Distributiv-Gesetzes
      \\[0.2cm]
      \hspace*{1.3cm}
      $\vec{x} + \vec{y} = \alpha \cdot \vec{z} + \beta \cdot \vec{z} = (\alpha + \beta) \cdot \vec{z} \in U$.
\item Sei nun $\alpha \in \mathbb{R}$ und $z \in U$.  Dann gibt es ein $\beta \in \mathbb{R}$, so dass
      $\vec{x} = \beta \cdot \vec{z}$ gilt.  Mit Hilfe des Assoziativ-Gesetzes schlie\3en wir nun wie folgt:
      \\[0.2cm]
      \hspace*{1.3cm}
      $\alpha \cdot \vec{x} = \alpha \cdot (\beta \cdot \vec{z}) = (\alpha \cdot \beta) \cdot \vec{z} \in U$.
      \qeds
\end{enumerate}

\remark
Geometrisch handelt es sich bei der Menge $U$ um eine Gerade, die durch den Nullpunkt geht.
Das n\"{a}chste Beispiel verallgemeinert das letzte Beispiel in dem nun nicht mehr ein einzelner Vektor
$\vec{z}$ den Raum $U$ erzeugt, sondern der Untervektor-Raum durch eine beliebige Menge $M$ von Vektoren
erzeugt wird.

\example
Es sei $V$ ein $\mathbb{K}$-Vektor-Raum und $M \subseteq V$ sei eine nicht-leere Menge von Vektoren.  Dann definieren wir die Menge
$\textsl{span}_\mathbb{K}(M)$ als die Menge aller endlichen Linear-Kombinationen von Vektoren aus $M$, wir setzen also
\\[0.2cm]
\hspace*{1.3cm}
$\textsl{span}_\mathbb{K}(M) := 
 \bigl\{ \alpha_1 \cdot \vec{x}_1 + \cdots + \alpha_n \cdot \vec{x}_n \mid n \in \mathbb{N} \wedge \forall i \in
 \{1,\cdots,n\}:\alpha_i \in \mathbb{K} \wedge \vec{x}_i \in M \bigr\}
$.
\\[0.2cm]
Dann ist die so definierte Menge $\textsl{span}_\mathbb{K}(M)$ ein Untervektor-Raum von $V$.

\proof
Es sind drei Eigenschaften zu pr\"{u}fen.
\begin{enumerate}
\item Da $M$ nicht leer ist, finden wir ein $\vec{v} \in M$.  Offenbar gilt dann
      \\[0.2cm]
      \hspace*{1.3cm}
      $\vec{0} = 0 \cdot \vec{v} \in \textsl{span}_\mathbb{K}(M)$.      
\item Es seien $\vec{x}, \vec{y} \in \textsl{span}_\mathbb{K}(M)$.  Dann gibt es $m,n \in \mathbb{N}$,
      sowie $\alpha_1, \cdots, \alpha_m, \beta_1, \cdots, \beta_n \in \mathbb{K}$ und
      $\vec{x}_1, \cdots, \vec{x}_m, \vec{y}_1, \cdots, \vec{y}_n \in M$, so dass 
      \\[0.2cm]
      \hspace*{1.3cm}
      $\vec{x} = \alpha_1 \cdot \vec{x}_1 + \cdots + \alpha_m \cdot \vec{x}_m$ \quad und \quad
      $\vec{y} = \beta_1 \cdot \vec{y}_1 + \cdots + \beta_n \cdot \vec{y}_n$ 
      \\[0.2cm]
      gilt. Damit haben wir
      \\[0.2cm]
      \hspace*{1.3cm}
      $
       \vec{x} + \vec{y} =  
          \alpha_1 \cdot \vec{x}_1 + \cdots + \alpha_m \cdot \vec{x}_m +
          \beta_1 \cdot \vec{y}_1 + \cdots + \beta_n \cdot \vec{y}_n \in \textsl{span}_\mathbb{K}(M)
      $
      \\[0.2cm]
      denn die Summe 
      $\alpha_1 \cdot \vec{x}_1 + \cdots + \alpha_m \cdot \vec{x}_m + \beta_1 \cdot \vec{y}_1 + \cdots + \beta_n \cdot \vec{y}_n$
      ist auch wieder eine Linear-Kombination von Vektoren aus $M$.
\item Nun sei $\vec{x} \in \textsl{span}_\mathbb{K}(M)$ und $\alpha \in \mathbb{K}$.
      Dann gibt es zun\"{a}chst ein $n \in \mathbb{N}$ sowie Skalare $\beta_1, \cdots, \beta_n$
      und Vektoren $\vec{x}_1, \cdots, \vec{x}_n \in M$, so dass
      \\[0.2cm]
      \hspace*{1.3cm}
      $\vec{x} = \beta_1 \cdot \vec{x}_1 + \cdots + \beta_m \cdot \vec{x}_m$
      \\[0.2cm]
      gilt.  Damit haben wir
      \\[0.2cm]
      \hspace*{1.3cm}
      $\alpha \cdot \vec{x} = \alpha \cdot (\beta_1 \cdot \vec{x}_1 + \cdots + \beta_m \cdot \vec{x}_m) =
       (\alpha \cdot \beta_1) \cdot \vec{x}_1 + \cdots + (\alpha \cdot \beta_m) \cdot \vec{x}_m \in
       \textsl{span}_\mathbb{K}(M)$.
      \qeds
\end{enumerate}


\example
Es sei $\mathbb{R}^{\mathbb{R}}$ die Menge der (bereits fr\"{u}her definierten) reellwertigen Funktionen
auf $\mathbb{R}$.
Weiter sei $c \in \mathbb{R}$ beliebig.  Definieren wir die Menge $N_c$ als
\\[0.2cm]
\hspace*{1.3cm}
$N_c := \{ f \in \mathbb{R}^{\mathbb{R}} \mid f(c) = 0 \}$,
\\[0.2cm]
also als die Menge der Funktionen $f$, die an der Stelle $c$ eine Nullstelle haben, dann ist
$N_c$ ein Untervektor-Raum von $\mathbb{R}^{\mathbb{R}}$.

\exercise
Beweisen Sie, dass $N_c$ ein Untervektor-Raum von $\mathbb{R}^{\mathbb{R}}$ ist.
\eox

\begin{Satz}
  Ist $V$ ein Vektor-Raum und sind $U_1$ und $U_2$ Untervektor-R\"{a}ume von $V$, so ist auch die Menge
  $U_1 \cap U_2$ ein Untervektor-Raum von $V$.
\end{Satz}

\proof
Wir haben drei Eigenschaften nachzuweisen.
\begin{enumerate}
\item Da $U_1$ und $U_2$ Untervektor-R\"{a}ume sind, gilt
      \\[0.2cm]
      \hspace*{1.3cm}
      $\vec{0} \in U_1$ \quad und \quad $\vec{0} \in U_2$, \quad woraus sofort \quad
      $\vec{0} \in U_1 \cap U_2$ \quad folgt.
\item Seien $\vec{x}, \vec{y} \in U_1 \cap U_2$.  Dann gilt nat\"{u}rlich
      \\[0.2cm]
      \hspace*{1.3cm}
      $\vec{x} \in U_1$, \quad 
      $\vec{x} \in U_2$, \quad 
      $\vec{y} \in U_1$, \quad und \quad
      $\vec{y} \in U_2$.
      \\[0.2cm] 
      Da $U_1$ und $U_2$ Untervektor-R\"{a}ume sind, folgt dann
      \\[0.2cm]
      \hspace*{1.3cm}
      $\vec{x} + \vec{y} \in U_1$ \quad und \quad $\vec{x} + \vec{y} \in U_2$,
      \\[0.2cm]
      also insgesamt
      \\[0.2cm]
      \hspace*{1.3cm}
      $\vec{x} + \vec{y} \in U_1 \cap U_2$.
\item Sei nun $\vec{x} \in U_1 \cap U_2$ und $\alpha \in \mathbb{K}$.  Daraus folgt sofort
      \\[0.2cm]
      \hspace*{1.3cm}
      $\vec{x} \in U_1$ \quad und \quad
      $\vec{x} \in U_2$.
      \\[0.2cm] 
      Da $U_1$ und $U_2$ Untervektor-R\"{a}ume sind, k\"{o}nnen wir folgern, dass
      \\[0.2cm]
      \hspace*{1.3cm}
      $\alpha \cdot \vec{x} \in U_1$ \quad und \quad $\alpha \cdot \vec{x} \in U_2$
      \\[0.2cm]
      gilt, so dass wir insgesamt
      \\[0.2cm]
      \hspace*{1.3cm}
      $\alpha \cdot \vec{x} \in U_1 \cap U_2$
      \\[0.2cm]
      haben.  \qeds
\end{enumerate}

\exercise
Es sei $V$ ein Vektor-Raum und $U_1$ und $U_2$ seien Untervektor-R\"{a}ume von $V$.  Beweisen oder widerlegen Sie, dass
dann auch die Menge $U_1 \cup U_2$ ein Untervektor-Raum von $V$ ist.
\eox


\section{Euklidische Vektor-R\"{a}ume}
\begin{Definition}[Skalar-Produkt, Euklidischer Vektor-Raum] \lb
  Es sei $V$ ein $\mathbb{R}$-Vektor-Raum.  Eine Abbildung 
  \\[0.2cm]
  \hspace*{1.3cm}
  $\langle \cdot \mid \cdot \rangle: V \times V \rightarrow \mathbb{R}$
  \\[0.2cm]
  ist ein \emph{Skalar-Produkt} genau dann, wenn die folgenden Bedingungen erf\"{u}llt sind:
  \begin{enumerate}
  \item Die Abbildung $\langle \cdot\mid \cdot \rangle$ ist \emph{bilinear}, dass hei\3t es gelten die
        folgenden Gleichungen:
        \begin{enumerate}
        \item $\langle \vec{x} + \vec{y}\mid \vec{z} \rangle = \langle \vec{x}\mid \vec{z} \rangle + \langle \vec{y}\mid \vec{z} \rangle$
              \quad f\"{u}r alle $\vec{x}, \vec{y}, \vec{z} \in V$.
        \item $\langle \vec{x}\mid \vec{y} + \vec{z} \rangle = \langle \vec{x}\mid \vec{y} \rangle + \langle \vec{x}\mid \vec{z} \rangle$
              \quad f\"{u}r alle $\vec{x}, \vec{y}, \vec{z} \in V$.
        \item $\langle \alpha \cdot \vec{x}\mid \vec{y} \rangle = \alpha \cdot \langle \vec{x}\mid \vec{y} \rangle$
              \quad f\"{u}r alle $\vec{x}, \vec{y} \in V$ und alle $\alpha \in \mathbb{R}$.
        \item $\langle \vec{x}\mid \alpha \cdot \vec{y} \rangle = \alpha \cdot \langle \vec{x}\mid \vec{y} \rangle$
              \quad f\"{u}r alle $\vec{x}, \vec{y} \in V$ und alle $\alpha \in \mathbb{R}$.
        \end{enumerate}
  \item Die Abbildung  $\langle \cdot\mid \cdot \rangle$ ist \emph{symmetrisch}, dass hei\3t es gilt
        \\[0.2cm]
        \hspace*{1.3cm}
        $\langle \vec{x}\mid \vec{y} \rangle = \langle \vec{y}\mid \vec{x} \rangle$ 
        \quad f\"{u}r alle $\vec{x}, \vec{y} \in V$.
  \item Die Abbildung   $\langle \cdot\mid \cdot \rangle$ ist \emph{positiv definit}, dass hei\3t es
        gilt:
        \begin{enumerate}
        \item $\langle \vec{x}\mid \vec{x} \rangle \geq 0$ \quad f\"{u}r alle $\vec{x} \in V$,
        \item $\langle \vec{x}\mid \vec{x} \rangle = 0 \leftrightarrow \vec{x} = \vec{0}$ \quad f\"{u}r alle $\vec{x} \in V$.
        \end{enumerate}
  \end{enumerate} 
  Falls auf einem Vektor-Raum $V$ ein Skalar-Produkt $\langle \cdot \mid \cdot \rangle$ definiert
  ist, so nennen wir $V$ einen \emph{euklidischen Vektor-Raum}.  \eoxs
\end{Definition}

\example
Definieren wir auf der Menge $\mathbb{R}^n$ 
\\[0.2cm]
\hspace*{1.3cm}
$\ds\bigl\langle [x_1, \cdots, x_n] \mid [y_1, \cdots, y_n] \bigr\rangle := \sum\limits_{i=1}^n x_i \cdot y_i$,
\\[0.2cm]
so ist die so definierte Abbildung $\langle \cdot \mid \cdot \rangle$ ein Skalar-Produkt und damit
ist der $\mathbb{R}^n$ dann ein euklidischer Vektor-Raum.  \eox

\exercise
Beweisen Sie, dass der $\mathbb{R}^n$ zusammen mit der oben gegebenen Definition des Skalar-Produkts
ein euklidischer Vektor-Raum ist.  \eox


Falls $V$ eine euklidischer Vektor-Raum ist, so l\"{a}sst sich f\"{u}r die Vektoren $\vec{x} \in V$ eine \emph{Norm}
einf\"{u}hren, die durch
\\[0.2cm]
\hspace*{1.3cm}
$\|\vec{x}\| := \sqrt{\langle \vec{x} \mid \vec{x} \rangle}$
\\[0.2cm]
definiert wird.  Diese Norm kann als die L\"{a}nge des Vektors $\vec{x}$ interpretiert werden, denn sie
stimmt im Falle des $\mathbb{R}^2$ und des $\mathbb{R}^3$ mit der L\"{a}nge des Vektors $\vec{x}$
\"{u}berein.  Zwischen der Norm und dem Skalar-Produkt gibt es eine wichtige Beziehung, die als die
Cauchy-Schwarzsche Ungleichung bezeichnet wird.

\begin{Satz}[Cauchy-Schwarzsche Ungleichung] \lb
  Es sei $V$ ein euklidischer Vektor-Raum mit dem Skalar-Produkt $\langle \cdot \mid \cdot \rangle$.
  Dann gilt
  \\[0.2cm]
  \hspace*{1.3cm}
  $\bigl|\langle \vec{x}, \vec{y} \rangle\bigr| \leq \| \vec{x} \| \cdot \|\vec{y}\|$ 
  \quad f\"{u}r alle $\vec{x}, \vec{y} \in V$.
\end{Satz}

\proof
F\"{u}r beliebiges $\alpha \in \mathbb{R}$ gilt
\\[0.2cm]
\hspace*{1.3cm}
$\langle \vec{x} - \alpha \cdot \vec{y} \mid \vec{x} - \alpha \cdot \vec{y} \rangle \geq 0$,
\\[0.2cm]
denn die Abbildung $\langle \cdot \mid \cdot \rangle$ ist positiv definit.
Nutzen wir nun die Bilinearit\"{a}t der Abbildung $\langle \cdot \mid \cdot \rangle$, so k\"{o}nnen wir
diese Ungleichung auch als
\\[0.2cm]
\hspace*{1.3cm}
$\langle \vec{x} \mid \vec{x} \rangle - \alpha \cdot \langle \vec{x}\mid \vec{y} \rangle  - 
\alpha \cdot \langle \vec{y} \mid \vec{x} \rangle + \alpha^2 \cdot \langle \vec{y} \mid \vec{y} \rangle \geq 0$
\\[0.2cm]
schreiben.  Nutzen wir nun die Symmetrie der Abbildung $\langle \cdot \mid \cdot \rangle$ aus,
so vereinfacht sich diese Ungleichung zu der Ungleichung
\\[0.2cm]
\hspace*{1.3cm}
$\langle \vec{x} \mid \vec{x} \rangle - 2 \cdot \alpha \cdot \langle \vec{x}\mid \vec{y} \rangle  
 + \alpha^2 \cdot \langle \vec{y} \mid \vec{y} \rangle \geq 0
$.
\\[0.2cm]
Diese Ungleichung gilt f\"{u}r alle $\alpha \in \mathbb{R}$ und damit auch, wenn wir
\\[0.2cm]
\hspace*{1.3cm}
$\ds\alpha := \frac{\langle \vec{x} \mid \vec{y} \rangle}{\langle \vec{y} \mid \vec{y} \rangle}$
\\[0.2cm]
definieren.  Setzen wir diesen Wert oben ein, so erhalten wir die Ungleichung
\\[0.2cm]
\hspace*{1.3cm}
$\ds\langle \vec{x} \mid \vec{x} \rangle - 
 2 \cdot \frac{\langle \vec{x} \mid \vec{y} \rangle}{\langle \vec{y} \mid \vec{y} \rangle} \cdot \langle \vec{x}\mid \vec{y} \rangle  
 +  \left(\frac{\langle \vec{x} \mid \vec{y} \rangle}{\langle \vec{y} \mid \vec{y} \rangle}\right)^2 \cdot \langle \vec{y} \mid \vec{y} \rangle \geq 0
$.
\\[0.2cm]
Multiplizieren wir diese Ungleichung mit $\langle \vec{y} \mid \vec{y} \rangle$, so erhalten wir die
Ungleichung 
\\[0.2cm]
\hspace*{1.3cm}
$\ds\langle \vec{x} \mid \vec{x} \rangle \cdot \langle \vec{y} \mid \vec{y} \rangle - 
 2 \cdot \langle \vec{x} \mid \vec{y} \rangle \cdot \langle \vec{x} \mid \vec{y} \rangle  
 +  \bigl(\langle \vec{x} \mid \vec{y} \rangle\bigr)^2  \geq 0
$,
\\[0.2cm]
die wir zu 
\\[0.2cm]
\hspace*{1.3cm}
$\ds\langle \vec{x} \mid \vec{x} \rangle \cdot \langle \vec{y} \mid \vec{y} \rangle \geq \bigl(\langle \vec{x} \mid \vec{y} \rangle\bigr)^2$
\\[0.2cm]
vereinfachen k\"{o}nnen.  Ziehen wir hier auf beiden Seiten die Wurzel und ber\"{u}cksichtigen die
Definition der Norm $\|\vec{x}\|$, so erhalten wir
\\[0.2cm]
\hspace*{1.3cm}
$\|\vec{x}\| \cdot \|\vec{y}\| \geq \bigl|\langle \vec{x} \mid \vec{y} \rangle\bigr|$
\\[0.2cm]
und das ist die Cauchy-Schwarzsche Ungleichung.  \qed

\remark
Die Cauchy-Schwarzsche Ungleichung erm\"{o}glicht es, den Winkel $\varphi$ zwischen zwei Vektoren $\vec{x}$ und $\vec{y}$
eines euklidischen Vektor-Raums zu definieren, denn wir k\"{o}nnen vereinbaren, dass
\\[0.2cm]
\hspace*{1.3cm}
$\ds\cos(\varphi) := \frac{\langle \vec{x} \mid \vec{y} \rangle}{\|\vec{x}\| \cdot \|\vec{y}\|}$
\\[0.2cm]
gilt.  Aus der Cauchy-Schwarzschen Ungleichung folgt dann, dass f\"{u}r den so definierten Wert
$\cos(\varphi)$ die Ungleichung
\\[0.2cm]
\hspace*{1.3cm}
$-1 \leq \cos(\varphi) \leq 1$
\\[0.2cm]
gilt und damit sicher gestellt ist, dass diese Gr\"{o}\3e tats\"{a}chlich als Kosinus eines Winkels interpretiert werden
kann.  
\eox

\exercise
Beweisen Sie, dass im $\mathbb{R}^2$ der Winkel, der zwischen zwei Vektoren $\vec{a}$ und $\vec{b}$
eingeschlossen wird,  der Gleichung
\\[0.2cm]
\hspace*{1.3cm}
$\ds\cos(\varphi) := \frac{\langle \vec{a} \mid \vec{b} \rangle}{\|\vec{a}\| \cdot \|\vec{b}\|}$
\\[0.2cm]
gen\"{u}gt.  Zur Vereinfachung k\"{o}nnen Sie sich auf den Fall beschr\"{a}nken, dass die 
beiden Vektoren $\vec{a}$ und $\vec{b}$ im ersten Quadranten liegen.
\eox

\begin{Satz}[Dreiecks-Ungleichung] \lb
   Es sei $V$ ein euklidischer Vektor-Raum.  Dann gilt
   \\[0.2cm]
   \hspace*{1.3cm}
   $\|\vec{x}+\vec{y}\| \leq \|\vec{x}\| + \|\vec{y}\|$ \quad f\"{u}r alle $\vec{x}, \vec{y} \in V$.  
   \\[0.2cm]
   Diese Ungleichung wird als Dreiecks-Ungleichung bezeichnet, denn Sie besagt,
   dass in einem Dreieck, bei dem zwei der Seiten durch die Vektoren $\vec{x}$ und $\vec{y}$ gegeben
   sind, die L\"{a}nge der dritten Seite, die durch den Vektor $\vec{x} + \vec{y}$ dargestellt wird, 
   kleiner als die Summe der L\"{a}ngen der beiden anderen Seiten ist.
\end{Satz}

\proof
Wir haben die die folgende Kette von Gleichungen und Ungleichungen:
\\[0.2cm]
\hspace*{1.3cm}
$
\begin{array}[t]{lcll}
  \|\vec{x} + \vec{y}\|^2 & = & \langle \vec{x} + \vec{y} \mid  \vec{x} + \vec{y} \rangle \\[0.2cm]
                          & = & \langle \vec{x} \mid \vec{x} \rangle + 
                                2 \cdot \langle \vec{x} \mid \vec{y} \rangle + 
                                \langle \vec{y} \mid \vec{y} \rangle \\[0.2cm]
                          & = & \|\vec{x}\|^2 + 2 \cdot \langle \vec{x} \mid \vec{y} \rangle + \|\vec{y}\|^2 \\[0.2cm]
                          & \leq & \|\vec{x}\|^2 + 2 \cdot |\langle \vec{x} \mid \vec{y} \rangle| + \|\vec{y}\|^2 \\[0.2cm]
                          & \leq & \|\vec{x}\|^2 + 2 \cdot \|\vec{x}\| \cdot \|\vec{y}\| + \|\vec{y}\|^2 \\[0.2cm]
                          & = & \bigl(\|\vec{x}\| + \|\vec{y}\|\bigr)^2 
\end{array}
$
\\[0.2cm]
Im vorletzten Schritt haben wir die Cauchy-Schwarzsche Ungleichung benutzt.  Insgesamt haben wir
gezeigt, dass
\\[0.2cm]
\hspace*{1.3cm}
$\|\vec{x} + \vec{y}\|^2 \leq \bigl(\|\vec{x}\| + \|\vec{y}\|\bigr)^2$
\\[0.2cm] 
gilt.  Ziehen wir auf beiden Seiten dieser Ungleichung die Wurzel, so erhalten wir die Behauptung.  
\qed

%%% Local Variables: 
%%% mode: latex
%%% TeX-master: "lineare-algebra"
%%% End: 
