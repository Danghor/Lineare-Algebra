\chapter{Einführung}
Das vorliegende Skript ist die Grundlage der Mathematik-Vorlesung des ersten Semesters.
Einige Kapitel und Abschnitte in diesem Skript sind mit einem ``$^*$'' versehen.  
Der dort vorgestellte Stoff ist für die Klausur nicht relevant.  Das Skript enthält diese Abschnitte
um eine der vielen Anwendungen der Mathematik, das
\href{https://de.wikipedia.org/wiki/RSA-Kryptosystem}{RSA-Kryptosystem}, präsentieren zu können.  

\section{Motivation}
Bevor wir uns mit dem eigentlichen Stoff dieser Vorlesung, der Mathematik, befassen, möchte ich
Ihnen aufzeigen, warum Sie als zukünftige Informatiker Mathematik benötigen.  
\begin{enumerate}
\item Historisch ist die Informatik als Teilgebiet der Mathematik entstanden.  Dies wird schon an
      dem Namen  ``\emph{Informatik}''  deutlich, den dieses Wort ist aus den beiden Wörtern
      ``\emph{Information}'' und ``\emph{Mathematik}'' gebildet worden ist, was  
      durch die Gleichung 
      \\[0.2cm]
      \hspace*{1.3cm}
      $\texttt{Informatik} = \texttt{Information} + \texttt{Mathematik}$
      \\[0.2cm]
      symbolisiert wird.  Aufgrund der Tatsache, dass die Informatik aus der Mathematik entstanden
      ist, bedient sich die Informatik an vielen Stellen
      mathematischer Sprech- und Denkweisen.  Um diese verstehen zu können, ist eine 
      Vertrautheit mit der formalen mathematischen Denkweise unabdingbar.
\item Mathematik schult das abstrakte Denken und genau das wird in der Informatik ebenfalls
      benötigt.  Ein komplexes Sofware-System, dass von hunderten von Programmierern über Jahre
      hinweg entwickelt wird, ist nur durch die Einführung geeigneter Abstraktionen beherrschbar.
      Die Fähigkeit, abstrakt denken zu können, ist genau das, was einen Mathematiker auszeichnet.
      Eine Möglichkeit,  diese Fähigkeit zu erwerben besteht darin, sich mit den abstrakten
      Gedankengebäuden, die in der Mathematik konstruiert werden, auseinander zu setzen.
\item Es gibt eine Vielzahl von mathematischen Methoden, die unmittelbar in der Informatik
      angewendet werden.  In dieser Vorlesung zeigen wir exemplarisch eine Reihe von Problemen, die
      ihren Ursprung in der Informatik haben, deren Lösung aber mathematische Methoden erfordert:
      \begin{enumerate}
      \item \href{https://en.wikipedia.org/wiki/Recurrence_relation}{\emph{Rekurrenz-Gleichungen}}
            sind Gleichungen, durch die Folgen von Zahlen definiert werden.  Beispielsweise können die
            \href{https://de.wikipedia.org/wiki/Fibonacci-Folge}{\emph{Fibonacci-Zahlen}} durch die
            Rekurrenz-Gleichung 
            \\[0.2cm]
            \hspace*{1.3cm}
            $a_{n+2} = a_{n+1} + a_n$  \quad und die Anfangs-Bedingungen $a_0 = 0$ und $a_1 = 1$
            \\[0.2cm]
            definiert werden.  Wir können mit der oberen Rekurrenz-Gleichung sukzessive die verschiedenen
            Werte der Folge $(a_n)_n$ berechnen und finden 
            \\[0.2cm]
            \hspace*{1.3cm}
            $a_0 = 0$, $a_1 = 1$, $a_2 = 1$, $a_3 = 2$, $a_4 = 3$, $a_5 = 5$, $a_6 = 8$, $a_7 = 13$, $\cdots$.
            \\[0.2cm]
            Wir werden später sehen, dass es eine geschlossene Formel zur Berechnung der Fiboncci-Zahlen
            gibt, es gilt
            \\[0.2cm]
            \hspace*{1.3cm}
            $\ds a_n = \frac{1}{\sqrt{5}} \cdot 
            \left( 
                  \biggl(\frac{1 + \sqrt{5}}{2}\biggr)^n - \biggl(\frac{1 - \sqrt{5}}{2}\biggr)^n 
            \right)
            $.
            \\[0.2cm]
            Sie werden im Laufe der ersten beiden Semester verschiedene Verfahren kennen lernen, mit
            denen sich für in der Praxis auftretende Rekurrenz-Gleichungen geschlossene Formeln
            finden lassen.  Solche Verfahren sind wichtig bei der Analyse der Komplexität von
            Algorithmen, denn die Berechnung der Laufzeit rekursiver Algorithmen führt auf
            Rekurrenz-Gleichungen.   
      \item Elementare Zahlentheorie bildet die Grundlage moderner kryptografischer Verfahren.
            Konkret beinhaltet dieses Skript eine Beschreibung des RSA-Algorithmus zur asymetrischen
            Verschlüsselung.  Allerdings werden wir das Kapitel zur Zahlen-Theorie, in dem der 
            RSA-Algorithmus beschrieben wird, aus Zeitgründen nicht besprechen können, es handelt
            sich bei dem Kapitel also nur um Zusatzstoff, welchen Sie sich bei Bedarf selbst aneignen
            können.  Ich habe das Kapitel hier deswegen eingefügt, damit Sie unmittelbar anhand
            eines konkreten Beispiels sehen können, welche Bedeutung die Mathematik in der
            Informatik hat.
      \end{enumerate}
      Die Liste der mathematischen Algorithmen, die in der Praxis eingesetzt werden, könnte leicht
      über mehrere Seiten fortgesetzt werden.  Natürlich können im Rahmen eines Bachelor-Studiums
      nicht alle mathematischen Verfahren, die in der Informatik eine Anwendung finden, auch
      tatsächlich diskutiert werden.  Das Ziel kann nur sein, Ihnen ausreichend mathematische
      Fähigkeiten zu vermitteln, so dass Sie später in der Lage sind, sich die
      mathematischen Verfahren, die sie  im Beruf tatsächlich benötigen, selbstständig anzueignen.
\item Mathematik schult das \emph{exakte Denken}.  Wie wichtig dieses ist, möchte ich mit den
      folgenden Beispielen verdeutlichen:  
      \begin{enumerate}
      \item Am 9.~Juni 1996 stürzte die Rakete Ariane 5 auf ihrem Jungfernflug ab.
            Ursache war ein Kette von Software-Fehlern:  Ein Sensor im Navigations-System
            der Ariane 5 misst die horizontale Neigung und speichert diese zunächst als Gleitkomma-Zahl
            mit einer Genauigkeit von 64 Bit ab.  Später wird dieser Wert dann in eine 
            16 Bit Festkomma-Zahl konvertiert.
            Bei dieser Konvertierung trat ein Überlauf ein, da die zu konvertierende Zahl
            zu groß war, um als 16 Bit Festkomma-Zahl dargestellt werden zu können.
            In der Folge gab das Navigations-System auf dem Datenbus, der dieses System mit
            der Steuerungs-Einheit verbindet, eine Fehlermeldung aus.
            Die Daten dieser Fehlermeldung wurden von der Steuerungs-Einheit als Flugdaten 
            interpretiert.  Die Steuer-Einheit leitete daraufhin eine Korrektur des
            Fluges ein, die dazu führte, dass die Rakete auseinander brach und die 
            automatische Selbstzerstörung eingeleitet werden musste.
            Die Rakete war mit 4 Satelliten beladen. Der wirtschaftliche Schaden, der durch den Verlust dieser
            Satelliten entstanden ist, lag bei mehreren 100 Millionen Dollar.
            
            Ein vollständiger Bericht über die Ursache des Absturzes des Ariane 5 findet sich
            im Internet unter der Adresse \\[0.2cm]
            \hspace*{1.3cm} 
            \href{http://www.ima.umn.edu/~arnold/disasters/ariane5rep.html}{\texttt{http://www.ima.umn.edu/\symbol{126}arnold/disasters/ariane5rep.html}}.
      \item Die Therac 25 ist ein medizinisches Bestrahlungs-Gerät, das durch 
            Software kontrolliert wird.  Durch  Fehler in dieser Software erhielten 1985
            mindestens 6 Patienten eine Überdosis an Strahlung.  Drei dieser Patienten sind an den
            Folgen dieser Überdosierung gestorben. 

            Einen detailierten Bericht über diese Unfälle finden Sie unter \\[0.2cm]
            \hspace*{1.3cm} 
            \href{http://courses.cs.vt.edu/~cs3604/lib/Therac_25/Therac\_1.html}{\texttt{http://courses.cs.vt.edu/\symbol{126}cs3604/lib/Therac\_25/Therac\_1.html}}.
      \item Im ersten Golfkrieg konnte eine irakische \textsl{Scud} Rakete von dem \textsl{Patriot}
            Flugabwehrsystem aufgrund eines Programmier-Fehlers in der Kontrollsoftware des Flugabwehrsystems
            nicht abgefangen werden.  28 Soldaten verloren dadurch ihr Leben, 100 weitere wurden
            verletzt. \\[0.2cm]
            \hspace*{1.3cm} 
            \href{http://www.ima.umn.edu/~arnold/disasters/patriot.html}{\texttt{http://www.ima.umn.edu/\symbol{126}arnold/disasters/patriot.html}}.
      \item Im Internet finden Sie auf der Seite der
            \href{https://de.wikipedia.org/wiki/Institute_of_Electrical_and_Electronics_Engineers}{\textsc{Ieee}}\footnote{
              Die Abkürzung \textsc{Ieee} steht für \emph{Institute of Electrical and Electronics Engineers}.}
            den Artikel   
            \\[0.2cm]
            \hspace*{1.3cm}
            \href{http://spectrum.ieee.org/static/the-staggering-impact-of-it-systems-gone-wrong}{Staggering Impact of IT Systems Gone Wrong}.
            \\[0.2cm]
            In diesem Artikel werden die Kosten, die durch fehlgeschlagene IT-Projekte entstanden sind, für eine Reihe von Projekten aufgelistet.
      \end{enumerate}
      Diese Beispiele zeigen, dass bei der Konstruktion von IT-Systemen mit großer Sorgfalt
      und Präzision gearbeitet werden sollte.  Die Erstellung von IT-Systemen muss auf einer 
      wissenschaftlich fundierten Basis erfolgen, denn nur dann ist es möglich, die Korrekheit
      solcher Systeme zu \emph{verifizieren}, also mathematisch zu beweisen.
      Diese oben geforderte wissenschaftliche Basis für die Entwicklung von IT-Systemen ist die Informatik, 
      und diese hat ihre Wurzeln sowohl in der Mengenlehre als auch in der mathematischen
      Logik.  Diese beiden Gebiete werden uns daher im ersten Semester
      des Informatik-Studiums beschäftigen.  Obwohl sowohl die Logik als auch die Mengenlehre zur
      Mathematik gehören, werden wir uns in dieser Mathematik-Vorlesung nur mit der Mengenlehre und
      der linearen Algebra beschäftigen.  Die Behandlung der Logik erfolgt dann im Rahmen der
      Informatik-Vorlesung. 
\end{enumerate}

% Schließlich gibt es für Sie noch einen sehr gewichtigen Grund, sich intensiv mit Logik und
% Mengenlehre zu beschäftigen, den ich Ihnen nicht verschweigen möchte:  Es handelt sich
% dabei um die Klausur am Ende des ersten Semesters!  

\section{Überblick}
Ich möchte Ihnen zum Abschluss dieser Einführung noch einen Überblick über all die Themen geben, die
ich im Rahmen der Vorlesung behandeln werde.  
\begin{enumerate}
\item Mathematische Formeln dienen der Abkürzung.  Sie werden aus den \emph{Junktoren} 
      \begin{enumerate}
      \item $\wedge$ (``\emph{und}''),
      \item $\vee$ (``\emph{oder}''),
      \item $\neg$ (``\emph{nicht}''),
      \item $\rightarrow$ (``\emph{wenn $\cdots$, dann}'') und
      \item $\leftrightarrow$ (``\emph{genau dann, wenn}'') 
      \end{enumerate}
      sowie den \emph{Quantoren}
      \begin{enumerate}
      \item $\forall$ (``\emph{für alle}'') und
      \item $\exists$ (``\emph{es gibt}'') 
      \end{enumerate}
      aufgebaut.
      Wir werden  Junktoren und Quantoren zunächst
      als reine Abkürzungen einführen.  Im Rahmen der Logik-Vorlesung werden wir die Bedeutung
      und Verwendung von Junktoren und Quantoren weiter untersuchen.
\item Mengenlehre

      Die Mengenlehre bildet die Grundlage der modernen Mathematik.  Die meisten Lehrbücher und
      Veröffentlichungen bedienen sich der Begriffsbildungen und Schreibweisen der Mengenlehre.
      Daher ist eine solide Grundlage an dieser Stelle für das weitere Studium unabdingbar.
\item Beweis-Prinzipien

      In der Informatik benötigen wir im wesentlichen vier Arten von Beweisen:
      \begin{enumerate}
      \item Ein \emph{direkter Beweis} folgert eine zu beweisende Aussage mit Hilfe elementarer 
            logischer Schlüsse und algebraischer Umformungen.  Diese Art von Beweisen kennen Sie
            bereits aus der Schule.
      \item Bei einem \emph{Beweis durch Fallunterscheidung} teilen wir den Beweis dadurch auf, dass
            wir alle in einer bestimmten Situation möglichen Fälle untersuchen und zeigen,
            dass die zu beweisende Aussage in jedem der Fälle wahr ist.  Beispielsweise können wir
            mit Hilfe einer Fallunterscheidung zeigen, dass die Zahl $n \cdot (n+1)$ für jede
            natürliche Zahl $n$ gerade ist.
      \item Ein \emph{indirekter Beweis} hat das Ziel zu zeigen, dass eine bestimmte Aussage $A$
            falsch ist.  Bei einem indirekten Beweis nehmen wir an, dass $A$ doch gilt und
            leiten aus dieser Annahme einen Widerspruch her.  Dieser Widerspruch zeigt uns dann,
            dass die Annahme $A$ nicht wahr sein kann.

            Beispielsweise werden wir mit Hilfe eines indirekten Beweises zeigen, dass 
            $\sqrt{2}$ keine rationale Zahl ist.  In der Informatik zeigen wir dann, dass das
            \emph{Halte-Problem} unlösbar ist:  Es ist nicht möglich, eine Funktion \texttt{stops}
            zu implementieren, so dass für eine gegebene einstellige Funktion $f$ und eine Eingabe $x$
            der Aufruf
            \\[0.2cm]
            \hspace*{1.3cm}
            $\texttt{stops}(f,x)$
            \\[0.2cm]
            genau dann als Ergebnis \texttt{true} zurück liefert, wenn der Aufruf der Funktion $f$
            mit der Eingabe $x$ terminiert.
      \item Ein induktiver Beweis hat das Ziel, eine Aussage für alle natürliche Zahlen zu beweisen.
            Beispielsweise werden wir zeigen, dass die Summenformel
            \\[0.2cm]
            \hspace*{1.3cm}
            $\ds\sum\limits_{i=1}^n i = \frac{1}{2} \cdot n \cdot (n+1)$ 
            \\[0.2cm]
            für alle natürlichen Zahlen $n \in \mathbb{N}$ gilt.  Summenformeln dieser Art treten
            beispielsweise bei der Berechnung der Rechenzeit von Programmen auf.
      \end{enumerate}
\item Grundlagen der Algebra

      Wir besprechen \emph{Gruppen}, \emph{Ringe} und \emph{Körper}.  Diese abstrakten Konzepte
      verallgemeinern  die Rechenregeln, die Sie von den reellen Zahlen kennen.  Sie bilden darüber
      hinaus die Grundlage für die lineare Algebra.
\item Zahlentheorie$^*$
  
      Dieses Skript enthält ein \underline{o}p\underline{tionales} Kapitel, in dem wir uns mit der 
      elementaren Zahlentheorie auseinandersetzen.  Die Zahlentheorie ist die Grundlage
      von vielen modernen Verschlüsselungs-Algorithmen.
\item Komplexe Zahlen

      Aus der Schule wissen Sie, dass die Gleichung
      \\[0.2cm]
      \hspace*{1.3cm}
      $x^2 = -1$
      \\[0.2cm]
      für $x \in \mathbb{R}$ keine Lösung hat.  Wir werden die Menge der reellen Zahlen $\mathbb{R}$
      zur Menge der komplexen Zahlen $\mathbb{C}$ erweitern und zeigen, dass jede quadratische
      Gleichung eine Lösung in der Menge der komplexen Zahlen hat.
\item Lineare Vektor-Räume

      Die Theorie der \emph{linearen Vektor-Räume} ist unter anderem die Grundlage für das Lösen von linearen
      Gleichungs-Systemen, linearen Rekurrenz-Gleichungen und linearen Differential-Gleichungen.
      Bevor wir uns also mit konkreten Algorithmen zur Lösung von Gleichungs-Systemen beschäftigen
      können, gilt es die Theorie der linearen Vektor-Räume zu verstehen.
\item Lineare Gleichungs-Systeme

      Lineare Gleichungs-Systeme treten sowohl in der Informatik als auch in vielen anderen Gebieten auf. 
      Wir zeigen, wie sich solche Gleichungs-Systeme algorithmisch lösen lassen.
\item Eigenwerte und Eigenvektoren

      Ist $A$ eine \emph{Matrix},  $\vec{x}$ ein Vektor, $\lambda$ eine Zahl und gilt darüber hinaus
      \\[0.2cm]
      \hspace*{1.3cm}
      $A \cdot \vec{x} = \lambda \cdot \vec{x}$
      \\[0.2cm]
      so nennen wir $\vec{x}$ ein Eigenvektor von der Matrix $A$ zum Eigenwert $\lambda$.
      
      Sie brauchen an dieser Stelle keine Angst haben: Im Laufe der Vorlesung werden den Begriff der
      \emph{Matrix} definieren  und die Frage, wie die Multiplikation $A \cdot \vec{x}$ der Matrix $A$ mit
      dem Vektor $\vec{x}$ definiert ist, wird ebenfalls noch geklärt.  Weiter werden wir sehen,
      wie Eigenvektoren berechnet werden können.
\item Rekurrenz-Gleichungen

      Die Analyse der Komplexität rekursiver Prozeduren führt auf Rekurrenz-Gleichungen.
      Wir werden Verfahren entwickeln, mit denen sich solche Rekurrenz-Gleichungen lösen lassen.
\end{enumerate}
\remark
Ich gehe davon aus,  dass das  Skript eine Reihe von Tippfehlern und auch anderen Fehlern enthalten
wird.  Ich möchte Sie darum bitten, mir solche Fehler per Email unter der Adresse
\\[0.2cm]
\hspace*{1.3cm}
\texttt{karl.stroetmann\symbol{64}dhbw-mannheim.de}
\\[0.2cm]
mitzuteilen.  Alternativ können Sie mir auch über
\href{https://github.com}{\texttt{https://github.com}} einen \texttt{pull}-Request schicken.

\section{Literaturhinweise}
Zum Schluss dieser Einführung möchte ich noch einige Hinweise auf die Literatur geben.  Dabei möchte
ich zwei Bücher besonders hervorheben:
\begin{enumerate}
\item Das Buch ``\emph{Set Theory and Related Topics}'' von Seymour Lipschutz \cite{lipschutz:1998} enthält den Stoff, der
      in diesem Skript in dem Kapitel über Mengenlehre abgehandelt wird.
\item Das Buch ``\emph{Linear Algebra}'' von Seymour Lipschutz und Marc Lipson \cite{lipschutz:2012}
      enthält den Stoff zur eigentlichen linearen Algebra.
\end{enumerate}
Beide Bücher enthalten eine große Anzahl von Aufgaben mit Lösungen, was gerade für den Anfänger
wichtig ist.  Darüber hinaus sind die Bücher
sehr preiswert.  

Vor etwa 30 Jahren habe ich selbst die lineare Algebra aus den Büchern von Gerd Fischer
\cite{fischer:2008} und Hans-Joachim Kowalsky \cite{kowalsky:2003} gelernt, an denen ich mich auch
jetzt wieder orientiert habe.  Zusätzlich habe ich in der Zwischenzeit das Buch 
``\emph{Linear Algebra Done Right}'' von Sheldon Axler \cite{axler:1997} gelesen, das sehr gut geschrieben ist
und einen alternativen Zugang zur linearen Algebra bietet, bei dem die Theorie der Determinanten
allerdings in den Hintergrund gerät.  Der fachkundige Leser wird bei der Lektüre dieses Skripts
unschwer Parallelen zu den oben zitierten Werken erkennen.   Demjenigen Leser, der sich mehr Wissen
aneignen möchte als das, was in dem engen zeitlichen Rahmen dieser Vorlesung vermittelt werden kann,
möchte ich auf die oben genannte Literatur verweisen, wobei mir persönlich die Darstellung des Buchs
von Sheldon Axler am besten gefällt. 


%%% Local Variables: 
%%% mode: latex
%%% TeX-master: "lineare-algebra"
%%% End: 
