\chapter{Ringe und K\"{o}rper}
In diesem Abschnitt behandeln wir 
\href{http://de.wikipedia.org/wiki/Ring_(Algebra)}{\emph{Ringe}} und 
\href{http://en.wikipedia.org/wiki/Field_(mathematics)}{\emph{K\"{o}rper}}.  Diese Begriffe werde ich gleich erkl\"{a}ren.
Im Folgenden m\"{o}chte ich einen kurzen \"{u}berblick \"{u}ber den Aufbau dieses Kapitels geben.  Da Sie Ringe und
K\"{o}rper  noch nicht kennen, wird dieser \"{u}berblick notwendigerweise informal und
unpr\"{a}zise sein.  Es geht mir hier nur darum, dass Sie eine, zun\"{a}chst sicher noch verschwommene, Vorstellung
von dem, was Sie in diesem Kapitel erwartet, bekommen.

Ringe sind Strukturen, in denen
sowohl eine Addition, eine Subtraktion und als auch eine Multiplikation vorhanden ist und au\3erdem f\"{u}r diese
Operationen ein Distributiv-Gesetz gilt.  Bez\"{u}glich der Addition muss die Struktur dabei eine kommutative
Gruppe sein. Ein typisches Beispiel f\"{u}r einen Ring ist die Struktur der ganzen
Zahlen. Ein Ring ist ein K\"{o}rper, wenn zus\"{a}tzlich auch noch eine Division m\"{o}glich ist.
Ein typisches Beispiel ist die Struktur der rationalen Zahlen.

Es gibt zwei wichtige Methoden um mit Hilfe eines Rings einen K\"{o}rper zu konstruieren.
Die erste Methode funktioniert in sogenanten Integrit\"{a}ts-Ringen, dass sind solche Ringe, in denen sich 
das neutrale Element der Addition, also die $0$, nicht als Produkt zweier von $0$ verschiedenener Elemente 
darstellen l\"{a}sst.  Dann l\"{a}sst sich n\"{a}mlich aus einem Integrit\"{a}ts-Ring, der bez\"{u}glich der
Multiplikation ein 
neutrales Element enth\"{a}lt, ein sogenannter Quotienten-K\"{o}rper erzeugen.  Die Konstruktion dieses K\"{o}rpers
verl\"{a}uft analog zu der Konstruktion der rationalen Zahlen aus den ganzen Zahlen.

Die zweite Methode funktioniert mit Hilfe sogenannter \emph{maximaler Ideale}.  Wir werden in Ringen zun\"{a}chst
\emph{Ideale} definieren.  Dabei sind Ideale das Analogon zu Untergruppen in der Gruppen-Theorie.
Anschlie\3end zeigen wir, wie sich mit Hilfe eines Ideals $I$ auf einem Ring $R$ eine Kongruenz-Relation
erzeugen l\"{a}sst.  Die Konstruktion ist dabei analog zur Konstruktion der
Faktor-Gruppe aus dem letzten Abschnitt.  F\"{u}r \emph{maximale Ideale} werden wir schlie\3lich zeigen,
dass der so erzeugte Faktor-Ring sogar ein K\"{o}rper ist.

\section{Definition und Beispiele}
\begin{Definition}[Ring]
Ein $4$-Tupel $\mathcal{R} = \langle R, 0, +, \cdot \rangle$ ist ein \emph{Ring}, falls gilt:
\begin{enumerate}
\item $\langle R, 0, + \rangle$ ist eine kommutative Gruppe,
\item $\cdot: R \times R \rightarrow R$ ist eine Funktion f\"{u}r welche die folgenden Gesetze gelten:
      \begin{enumerate}
      \item Assoziativ-Gesetz:  F\"{u}r alle $x, y,z \in R$ gilt
            \\[0.2cm]
            \hspace*{1.3cm}
            $(x \cdot y) \cdot z = x \cdot (y \cdot z)$.
      \item Distributiv-Gesetze: F\"{u}r alle $x, y,z \in R$ gilt
            \\[0.2cm]
            \hspace*{1.3cm} 
            $x \cdot (y + z) = x \cdot y + x \cdot z$ \quad und \quad
            $(x + y) \cdot z = x \cdot z + y \cdot z$.
      \end{enumerate}
\end{enumerate}
Die Operation ``$+$'' nennen wir die Addition auf dem Ring $\mathcal{R}$, die Operation ``$\cdot$'' bezeichnen 
wir als Multiplikation.
$\mathcal{R}$ ist ein \emph{kommutativer Ring} falls zus\"{a}tzlich f\"{u}r die Multiplikation
das Kommutativ-Gesetz
\\[0.2cm]
\hspace*{1.3cm}
$x \cdot y = y \cdot x$ \quad f\"{u}r alle $x,y \in R$
\\[0.2cm]
gilt.  Wir sagen, dass $R$ eine \emph{Eins} hat, wenn es f\"{u}r die Multiplikation ein Element $e$ gibt, so dass
\\[0.2cm]
\hspace*{1.3cm}
$e \cdot x = x \cdot e = x$ \quad f\"{u}r alle $x \in R$
\\[0.2cm]
gilt.  Dieses Element $e$ hei\3t dann die \emph{Eins} des Rings.  In diesem Fall schreiben wir
$\mathcal{R} = \langle R, 0, e, +, \cdot \rangle$, wir geben die Eins also explizit in der Struktur
an.
\eox
\end{Definition}

\remark
Bevor wir Beispiele betrachten, bemerken wir einige unmittelbare Konsequenzen der obigen Definition.
\begin{enumerate}
\item In jedem Ring $\mathcal{R} = \langle R, 0, +, \cdot \rangle$ gilt
      \\[0.2cm]
      \hspace*{1.3cm}
      $0 \cdot x = 0$,
      \\[0.2cm]
      denn wir haben die Gleichung
      \\[0.2cm]
      \hspace*{1.3cm}
      $
      \begin{array}[t]{lcll}
        0 \cdot x & = & (0 + 0) \cdot x 
                      & \mbox{denn $0 + 0 = 0$} \\[0.2cm]
                  & = & 0 \cdot x + 0 \cdot x
                      & \mbox{Distributiv-Gesetz}
      \end{array}
      $
      \\[0.2cm]
      Wenn wir nun auf beiden Seiten der eben gezeigten Gleichung 
      $0 \cdot x$ abziehen, dann erhalten wir die Gleichung
      \\[0.2cm]
      \hspace*{1.3cm}
      $0 = 0 \cdot x$.
      \\[0.2cm]
      Genauso gilt nat\"{u}rlich auch $x \cdot 0 = 0$. 
\item Bezeichnen wir das bez\"{u}glich der Addition $+$ zu einem Element $x$ inverse Element mit $-x$,
      so gilt 
      \\[0.2cm]
      \hspace*{1.3cm}
      $-(x \cdot y) = (-x) \cdot y = x \cdot (-y)$.
      \\[0.2cm]
      Wir zeigen die erste dieser beiden Gleichungen, die zweite l\"{a}sst sich analog nachweisen.
      Um zu zeigen, dass $-(x \cdot y) = (-x) \cdot y$ ist, reicht es nachzuweisen, dass
      \\[0.2cm]
      \hspace*{1.3cm}
      $x \cdot y + (-x) \cdot y = 0$
      \\[0.2cm]
      ist, denn das Inverse ist in einer Gruppe eindeutig bestimmt.  Die letzte Gleichung folgt aber sofort
      aus dem Distributiv-Gesetz, denn wir haben
      \\[0.2cm]
      \hspace*{1.3cm}
      $
      \begin{array}[t]{lcl}
      x \cdot y + (-x) \cdot y & = & (x + - x) \cdot y \\[0.2cm]
                               & = & 0 \cdot y \\[0.2cm]
                               & = & 0. \hspace*{\fill} \diamond
      \end{array}
      $
\end{enumerate}


\examples
\begin{enumerate}
\item Die Struktur $\langle \mathbb{Z}, 0, 1, +, \cdot \rangle$ ist ein kommutativer
      Ring mit  Eins.  
\item Die Struktur $\langle \mathbb{Q}, 0, 1, +, \cdot \rangle$ ist ebenfalls ein kommutativer Ring mit
      Eins.   \eox
\end{enumerate}

In dieser Vorlesung werden wir nur solche Ringe betrachten, die erstens kommutativ sind und
zweitens eine Eins haben. 
Ein wichtiger Spezialfall ist der Fall eines kommutativen Rings 
$\mathcal{R} = \langle R, 0, 1, +, \cdot \rangle$ mit Eins, 
f\"{u}r den die Struktur $\langle R \backslash \{0\}, 1, \cdot \rangle$
eine Gruppe ist.  Dieser Spezialfall liegt beispielsweise bei dem Ring  
$\langle \mathbb{Q}, 0, 1, +, \cdot \rangle$ vor.  In einem solchen Fall sprechen wir von einem \emph{K\"{o}rper}.  Die formale
Definition folgt.
\pagebreak


\begin{Definition}[K\"{o}rper]
Ein $5$-Tupel $\mathcal{K} = \langle K, 0, 1, +, \cdot \rangle$ ist ein \emph{K\"{o}rper}, falls gilt:
\begin{enumerate}
\item $\langle K, 0, + \rangle$ ist eine kommutative Gruppe,
\item $\langle K \backslash \{ 0 \}, 1, \cdot \rangle$ ist ebenfalls eine kommutative Gruppe,
\item Es gilt das Distributiv-Gesetz: F\"{u}r alle $x, y,z \in K$ haben wir
      \\[0.2cm]
      \hspace*{1.3cm} 
      $x \cdot (y + z) = x \cdot y + x \cdot z$.
\end{enumerate}
Wieder nennen wir die  Operation ``$+$'' die \emph{Addition}, w\"{a}hrend wir die Operation ``$\cdot$'' 
als die \emph{Multiplikation} bezeichnen. \eox
\end{Definition}

Unser Ziel ist es sp\"{a}ter, Ringe zu K\"{o}rpern zu erweitern.  Es gibt bestimmte Ringe, in
denen dies auf keinen Fall m\"{o}glich ist.  Betrachten wir als Beispiel den Ring
$\mathcal{R} := \langle \{ 0, 1, 2, 3 \}, 0, 1, +_4, \cdot_4 \rangle$, bei dem die Operationen
``$+_4$'' und ``$\cdot_4$'' wie folgt definiert sind:
\\[0.2cm]
\hspace*{1.3cm}
$x +_4 y := (x + y) \modulo 4$ \quad und \quad
$x \cdot_4 y := (x \cdot y) \modulo 4$.
\\[0.2cm]
Es l\"{a}sst sich zeigen, dass die so definierte Struktur $\mathcal{R}$ ein Ring ist.  In
diesem Ring gilt
\\[0.2cm]
\hspace*{1.3cm}
$2 \cdot_4 2 = 4 \modulo 4 = 0$.
\\[0.2cm]
Falls es uns gelingen w\"{u}rde, den Ring $\mathcal{R}$ so zu einem K\"{o}rper 
$\mathcal{K} = \langle K, 0, 1, +_4, \cdot_4 \rangle$ zu erweitern, dass 
$\{0,1,2,3\} \subseteq K$ gelten w\"{u}rde, so m\"{u}sste die Zahl 
$2$ in diesem K\"{o}rper ein Inverses $2^{-1}$ haben.  Wenn wir dann die Gleichung
\\[0.2cm]
\hspace*{1.3cm}
$2 \cdot_4 2 = 0$
\\[0.2cm]
auf beiden Seiten mit $2^{-1}$ multiplizieren w\"{u}rden, h\"{a}tten wir die Gleichung
\\[0.2cm]
\hspace*{1.3cm}
$2 = 0$
\\[0.2cm]
hergeleitet.  Dieser Widerspruch zeigt, dass sich der Ring $\mathcal{R}$ sicher nicht zu
einem K\"{o}rper erweitern l\"{a}sst.

\begin{Definition}[Integrit\"{a}ts-Ring]
  Ein Ring $\mathcal{R} = \langle R, 0, +, \cdot \rangle$ hei\3t \emph{nullteilerfrei}, wenn
  \\[0.2cm]
  \hspace*{1.3cm}
  $\forall a, b \in R: \bigl(a \cdot b = 0 \rightarrow a = 0 \vee b = 0\bigr)$
  \\[0.2cm]
  gilt, die Zahl $0$ l\"{a}sst sich in einem nullteilerfreien Ring also nur als triviales Produkt darstellen.
  Ein nullteilerfreier, kommutativer Ring, der eine Eins hat, wird als \emph{Integrit\"{a}ts-Ring}
  bezeichnet.
\eox
\end{Definition}

\remark
In einem nullteilerfreien Ring $\mathcal{R} = \langle R, 0, +, \cdot \rangle$ 
gilt die folgende \emph{Streichungs-Regel}:
\\[0.2cm]
\hspace*{1.3cm}
$\forall a,b,c \in R: \bigl(a \cdot c = b \cdot c \wedge c \not= 0 \rightarrow a = b)$.
\eox

\proof
Wir nehmen an, dass $c \not = 0$ ist und dass $a \cdot c = b \cdot c$ gilt.  Es ist dann $a = b$ zu zeigen. 
Wir formen die Voraussetzung  $a \cdot c = b \cdot c$ wie folgt um
\\[0.2cm]
\hspace*{1.3cm}
$
\begin{array}[t]{cll}
            & a \cdot c = b \cdot c      & \mid - b \cdot c \\[0.2cm]
\Rightarrow & a \cdot c - b \cdot c  = 0                    \\[0.2cm]
\Rightarrow & (a - b) \cdot c = 0                           \\[0.2cm]
\Rightarrow & (a - b) = 0 & \mbox{denn $\mathcal{R}$ ist nullteilerfreier und $c \not= 0$} \\[0.2cm]
\Rightarrow & a = b.
\end{array}
$
\\[0.2cm]
Damit ist der Beweis abgeschlossen. \qed
\vspace*{0.2cm}

Ist $R$ ein Ring und ist $\sim$ eine \"{a}quivalenz-Relationen auf $R$,
so l\"{a}sst sich auf dem Quotienten-Raum $R/\!\sim$ unter bestimmten Umst\"{a}nden ebenfalls eine Ring-Struktur
definieren.   Das funktioniert aber nur, wenn die Addition und die Multiplikation des Rings in gewisser
Weise mit der \"{a}quivalenz-Relationen \emph{vertr\"{a}glich} sind.  In diesem Fall nennen wir dann $\sim$ eine
\emph{Kongruenz-Relation} auf $R$.  Die formale Definition folgt.

\begin{Definition}[Kongruenz-Relation]
Es sei $\mathcal{R} = \langle R, 0, 1, +, \cdot \rangle$ ein kommutativer Ring mit Eins und $\sim\; \subseteq R \times R$ sei eine
\"{a}quivalenz-Relation auf $R$.   Wir nennen $\sim$ eine \emph{Kongruenz-Relation} falls zus\"{a}tzlich
die folgenden beiden Bedingungen erf\"{u}llt sind:
\begin{enumerate}
\item $\forall a_1, a_2, b_1, b_2 \in R: 
       \bigl(a_1 \sim a_2 \wedge b_1 \sim b_2 \rightarrow a_1 + b_1 \sim a_2 + b_2\bigr)
      $,

      die Relation $\sim$ ist also \emph{vertr\"{a}glich} mit der Addition auf $R$.
\item $\forall a_1, a_2, b_1, b_2 \in R: 
       \bigl(a_1 \sim a_2 \wedge b_1 \sim b_2 \rightarrow a_1 \cdot b_1 \sim a_2 \cdot b_2\bigr)
      $,

      die Relation $\sim$ ist also auch mit der Multiplikation auf $R$ \emph{vertr\"{a}glich}.
      \eox
\end{enumerate}
\end{Definition}
  
Falls $\sim$ eine Kongruenz-Relation auf einem Ring $\mathcal{R} = \langle R, 0, 1, +, \cdot \rangle$ ist,
lassen sich die Operationen ``$+$'' und ``$\cdot$'' auf die Menge $R/\!\sim$ der von $\sim$ erzeugten
\"{a}quivalenz-Klassen fortsetzen, denn in diesem Fall k\"{o}nnen wir f\"{u}r $a,b \in R$ definieren:
\\[0.2cm]
\hspace*{1.3cm}
$[a]_\sim + [b]_\sim := [ a + b ]_\sim$ \quad und \quad
$[a]_\sim \cdot [b]_\sim := [ a \cdot b ]_\sim$.
\\[0.2cm]
Um nachzuweisen, dass diese Definitionen tats\"{a}chlich Sinn machen, betrachten wir vier
 Elemente $[a_1]_\sim, [a_2]_\sim, [b_1]_\sim, [b_2]_\sim \in R/\!\sim$, f\"{u}r die
\\[0.2cm]
\hspace*{1.3cm}
$[a_1]_\sim = [a_2]_\sim$ \quad und \quad
$[b_1]_\sim = [b_2]_\sim$ 
\\[0.2cm]
gilt.  Wir m\"{u}ssen zeigen, dass dann auch 
\\[0.2cm]
\hspace*{1.3cm}
$[a_1 + b_1]_\sim = [a_2 + b_2]_\sim$
\\[0.2cm]
gilt.  An dieser Stelle erinnern wir daran, dass nach Satz \ref{satz:aequivalenz-klassen}
allgemein f\"{u}r eine beliebige \"{a}quivalenz-Relation $R$ auf einer Menge $M$ die Beziehung
\\[0.2cm]
\hspace*{1.3cm}
$[a]_R = [b]_R \;\leftrightarrow\; \langle a, b \rangle \in R$
\\[0.2cm]
gilt.  Damit folgt aus den Voraussetzungen $[a_1]_\sim = [a_2]_\sim$ und $[b_1]_\sim = [b_2]_\sim$, dass
\\[0.2cm]
\hspace*{1.3cm}
$a_1 \sim a_2$ \quad und \quad $b_1 \sim b_2$
\\[0.2cm]
gilt.
Da die \"{a}quivalenz-Relation $\sim$ mit der Operation ``$+$'' vertr\"{a}glich ist, folgt daraus
\\[0.2cm]
\hspace*{1.3cm}
$a_1 + b_1 \sim  a_2 + b_2$ 
\\[0.2cm]
und damit gilt wieder nach Satz \ref{satz:aequivalenz-klassen} 
\\[0.2cm]
\hspace*{1.3cm}
$[a_1 + b_1]_\sim = [a_2 + b_2]_\sim$. 
\\[0.2cm]
Genauso l\"{a}sst sich zeigen, dass auch die Multiplikation $\cdot$ auf $R/\!\sim$ wohldefiniert ist.
Mit den obigen  Definitionen von $+$ und $\cdot$ haben wir nun eine Struktur 
$\mathcal{R}/\!\sim\; = \langle R/\!\sim, [0]_\sim, [1]_\sim, +, \cdot \rangle$ geschaffen, die als der
\emph{Faktor-Ring} $\mathcal{R}$ modulo $\sim$ bezeichnet wird. 
Der n\"{a}chste Satz zeigt, dass es sich bei dieser Struktur tats\"{a}chlich um einen Ring handelt.

\begin{Satz}[Faktor-Ring] \label{satz:faktor_ring} 
  Ist $\mathcal{R} = \langle R, 0, 1, +, \cdot \rangle$ ein kommutativer Ring mit Eins und ist
  $\sim$ eine Kongruenz-Relation auf diesem Ring, so ist die Struktur 
  $\mathcal{R}/\!\!\sim\; = \langle R/\!\sim, [0]_\sim, [1]_\sim, +, \cdot \rangle$
  mit den oben f\"{u}r \"{a}quivalenz-Klassen definierten Operationen $+$ und $\cdot$ ein kommutativer 
  Ring mit Eins. 
\end{Satz}

\proof
Wir weisen die Eigenschaften, die einen Ring auszeichnen, einzeln nach.
\begin{enumerate}
\item F\"{u}r die Operation ``$+$'' gilt das Assoziativ-Gesetz, denn f\"{u}r alle 
      $[a]_\sim, [b]_\sim, [c]_\sim \in R/\!\sim$ gilt
      \\[0.2cm]
      \hspace*{1.3cm}
      $
      \begin{array}[t]{lcl}
            [a]_\sim + \bigl([b]_\sim + [c]_\sim\bigr) 
      & = & [a]_\sim + [b + c]_\sim                     \\[0.2cm]
      & = & [a + (b + c)]_\sim                          \\[0.2cm]
      & = & [(a + b) + c]_\sim                          \\[0.2cm]
      & = & [a + b]_\sim + [c]_\sim                     \\[0.2cm]
      & = & \bigl([a]_\sim + [b]_\sim\bigr) + [c]_\sim. 
      \end{array}
      $
\item F\"{u}r die Operation ``$+$'' gilt das Kommutativ-Gesetz, denn f\"{u}r alle $[a]_\sim, [b]_\sim \in R/\!\sim$ gilt
      \\[0.2cm]
      \hspace*{1.3cm}
      $[a]_\sim + [b]_\sim = [a + b]_\sim = [b + a]_\sim = [b]_\sim + [a]_\sim$.
\item $[0]_\sim$ ist das neutrale Element bez\"{u}glich der Addition, denn f\"{u}r alle $[a]_\sim \in R/\!\sim$ gilt
      \\[0.2cm]
      \hspace*{1.3cm}
      $[a]_\sim + [0]_\sim = [a + 0]_\sim = [a]_\sim$.
\item Ist $[a]_\sim \in R/\!\sim$ und bezeichnet $-a$ das additive Inverse von $a$ in $R$, so ist das
      additive Inverse von  $[a]_\sim$ durch 
      die \"{a}quivalenz-Klasse $[-a]_\sim$ gegeben, denn wir haben
      \\[0.2cm]
      \hspace*{1.3cm}
      $[a]_\sim + [-a]_\sim = [a + -a]_\sim = [0]_\sim$.
\item Die Nachweise, dass auch f\"{u}r den  Operator ``$\cdot$'' dass Assoziativ- und das Kommutativ-Gesetz 
      sind v\"{o}llig analog zu den entsprechenden Beweisen f\"{u}r den Operator ``$+$''.
      Ebenso ist der Nachweis, dass die \"{a}quivalenz-Klasse $[1]_\sim$ das neutrale Element bez\"{u}glich des Operators 
      ``$\cdot$'' ist, analog zu dem Nachweis, dass die \"{a}quivalenz-Klasse $[0]_\sim$ das neutrale Element
      bez\"{u}glich des Operators ``$+$'' ist.
\item Als letztes weisen wir die G\"{u}ltigkeit des Distributiv-Gesetzes nach.
      Es seien  $[a]_\sim, [b]_\sim, [c]_\sim$ beliebige \"{a}quivalenz-Klassen aus $R/\!\sim$. Dann gilt
      \\[0.2cm]
      \hspace*{1.3cm}
      $
      \begin{array}[t]{lcl}
            [a]_\sim \cdot \bigl([b]_\sim + [c]_\sim\bigr) 
      & = & [a]_\sim \cdot [b + c]_\sim                        \\[0.2cm]
      & = & [a \cdot (b + c)]_\sim                             \\[0.2cm]
      & = & [a \cdot b + a \cdot c]_\sim                       \\[0.2cm]
      & = & [a \cdot b]_\sim + [a \cdot c]_\sim                \\[0.2cm]
      & = & [a]_\sim \cdot [b]_\sim + [a]_\sim \cdot [c]_\sim. 
      \end{array}
      $
\end{enumerate}
Damit ist gezeigt, dass $\mathcal{R}/\!\sim$ ein kommutativer Ring mit Eins ist. \qed

\exercise
Es sei $\mathcal{K} = \langle K, 0_K, 1_K, +, \cdot \rangle$ ein K\"{o}rper und die Menge $K$ sei endlich.
Dann k\"{o}nnen wir eine Funktion 
\\[0.2cm]
\hspace*{1.3cm}
$\mathtt{count}: \mathbb{N} \rightarrow K$
\\[0.2cm]
definieren, welche die nat\"{u}rlichen Zahlen $\mathbb{N}$ in den K\"{o}rper $K$ abbildet.  Die Definition
von $\mathtt{count}$ erfolgt durch Induktion:
\begin{enumerate}
\item[I.A.:] $n = 0$:

             $\mathtt{count}(0) := 0_K$.

             Beachten Sie hier, dass die $0$, die auf der linken Seite dieser Gleichung auftritt,
             eine nat\"{u}rliche Zahl ist, w\"{a}hrend die $0_K$ auf der rechten Seite dieser Gleichung
             das neutrale Element der Addition in dem K\"{o}rper $\mathcal{K}$ bezeichnet.
\item[I.S.:] $n \mapsto n + 1$.

             $\mathtt{count}(n+1) := \mathtt{count}(n) + 1_K$.

             Auch hier bezeichnet die $1$ auf der linken Seite der Definition eine nat\"{u}rliche Zahl,
             w\"{a}hrend die $1_K$ auf der rechten Seite dieser Gleichung das neutrale Element der
             Multiplikation in dem K\"{o}rper $\mathcal{K}$ bezeichnet.
\end{enumerate}
Falls die Menge $K$ endlich ist, definieren wir die \emph{Charakteristik}
$\mathtt{char}(\mathcal{K})$ des K\"{o}rpers $\mathcal{K}$ als die kleinste nat\"{u}rliche Zahl $k$, f\"{u}r die
$\mathtt{count}(k) = 0_K$ gilt:
\\[0.2cm]
\hspace*{1.3cm}
$\mathtt{char}(\mathcal{K}) = \min\bigl(\{ k \in \mathbb{N} \mid \mathtt{count}(k) = 0_K \}\bigr)$.
\\[0.2cm]
Zeigen Sie, dass f\"{u}r einen endlichen K\"{o}rpers $\mathcal{K}$ das Folgende gilt:
\begin{enumerate}
\item[(a)] Die Menge $\bigl\{ k \in \mathbb{N} \mid \mathtt{count}(k) = 0_K \bigr \}$
           ist nicht leer, die Charakteristik eines endlichen K\"{o}rpers ist also wohldefiniert.

           \noindent
           \textbf{Hinweis}:
           Da die Menge $K$ endlich ist, ist auch die Menge
           \\[0.2cm]
           \hspace*{1.3cm}
           $\{ \mathtt{count}(n) \mid n \in \mathbb{N} \}$
           \\[0.2cm]
           endlich.  Also muss es $n, m \in \mathbb{N}$ geben, so dass
           \\[0.2cm]
           \hspace*{1.3cm}
           $n \not= m$, \quad aber \quad $\mathtt{count}(n) = \mathtt{count}(m)$
           \\[0.2cm]
           gilt.
\item[(b)] Die Charakteristik $\mathtt{char}(\mathcal{K})$ ist eine Primzahl. \eox
\end{enumerate}


\exercise
\renewcommand{\labelenumi}{(\alph{enumi})}
\begin{enumerate}
\item Zeigen Sie, dass es einen K\"{o}rper $K$ mit drei Elementen gibt.
\item Zeigen Sie, dass es einen K\"{o}rper $K$ mit vier Elementen gibt. \exend
\end{enumerate}
\renewcommand{\labelenumi}{\arabic{enumi}.}



\section{Konstruktion des Quotienten-K\"{o}rpers$^*$}
Betrachten wir einen Integrit\"{a}ts-Ring $\mathcal{R} = \langle R, 0, 1, +, \cdot \rangle$, der selbst noch
kein K\"{o}rper ist, so k\"{o}nnen wir uns fragen, in welchen F\"{a}llen es m\"{o}glich ist, aus diesem
Ring einen K\"{o}rper zu konstruieren.  Wir versuchen bei einer solchen Konstruktion \"{a}hnlich
vorzugehen wie bei der Konstruktion der rationalen Zahlen $\mathbb{Q}$ aus den ganzen
Zahlen $\mathbb{Z}$.  Es sei also ein Integrit\"{a}ts-Ring $\mathcal{R} = \langle R, 0, 1, +, \cdot \rangle$
gegeben.  Dann definieren wir zun\"{a}chst die Menge
\\[0.2cm]
\hspace*{1.3cm}
$Q := \bigl\{ \pair(x,y) \in R \times R \mid y \not= 0 \bigr\}$
\\[0.2cm]
der formalen Br\"{u}che.  Weiter definieren wir eine Relation 
\\[0.2cm]
\hspace*{1.3cm}
$\sim \;\subseteq Q \times Q$
\\[0.2cm]
auf $Q$ indem wir festsetzen, dass f\"{u}r alle $\pair(x_1,y_1), \pair(x_2,y_2) \in Q$ das Folgende gilt:
\\[0.2cm]
\hspace*{1.3cm}
$\pair(x_1, y_1) \;\sim\; \pair(x_2, y_2) \df x_1 \cdot y_2 = x_2 \cdot y_1$.

\begin{Satz}
Die oben definierte Relation  $\sim$ ist eine \"{a}quivalenz-Relation auf der Menge $Q$.
\end{Satz}

\proof
Wir m\"{u}ssen zeigen, dass die Relation reflexiv, symmetrisch und transitiv ist.
\begin{enumerate}
\item Reflexivit\"{a}t: F\"{u}r alle Paare $\pair(x,y) \in Q$ gilt nach Definition der Relation $\sim$:
      \\[0.2cm]
      \hspace*{1.3cm}
      $
      \begin{array}[t]{cll}
                      & \pair(x,y) \sim \pair(x,y) \\[0.2cm]
      \Leftrightarrow & x \cdot y = x \cdot y.
      \end{array}
      $
      \\[0.2cm]
      Da die letzte Gleichung offensichtlich wahr ist, ist die Reflexivit\"{a}t nachgewiesen. \checkmark
\item Symmetrie: Wir m\"{u}ssen zeigen, dass
      \\[0.2cm]
      \hspace*{1.3cm}
      $\pair(x_1, y_1) \sim \pair(x_2, y_2) \rightarrow \pair(x_2, y_2) \sim \pair(x_1, y_1)$
      \\[0.2cm]
      gilt.  Wir nehmem also an, dass $\pair(x_1, y_1) \sim \pair(x_2, y_2)$ gilt, und zeigen, dass 
      daraus 
      \\[0.2cm]
      \hspace*{1.3cm}
      $\pair(x_2, y_2) \sim \pair(x_1, y_1)$ 
      \\[0.2cm]
      folgt.  Die Annahme 
      \\[0.2cm]
      \hspace*{1.3cm}
      $\pair(x_1, y_1) \sim \pair(x_2, y_2)$
      \\[0.2cm]
      ist nach Definition von $\sim$ \"{a}quivalent zu der Gleichung
      \\[0.2cm]
      \hspace*{1.3cm}
      $x_1 \cdot y_2 = x_2 \cdot y_1$.
      \\[0.2cm]
      Diese Gleichung drehen wir um und erhalten
      \\[0.2cm]
      \hspace*{1.3cm}
      $x_2 \cdot y_1 = x_1 \cdot y_2$.
      \\[0.2cm]
      Nach Definition der Relation $\sim$ gilt dann
      \\[0.2cm]
      \hspace*{1.3cm}
      $\pair(x_2, y_2) \sim \pair(x_1, y_1)$
      \\[0.2cm]
      und das war zu zeigen. $\checkmark$
\item Transitivit\"{a}t: Wir m\"{u}ssen zeigen, dass
      \\[0.2cm]
      \hspace*{1.3cm}
      $\pair(x_1, y_1) \sim \pair(x_2, y_2) \;\wedge\; \pair(x_2, y_2) \sim \pair(x_3, y_3)
       \;\rightarrow\; \pair(x_1, y_1) \sim \pair(x_3, y_3)
      $
      \\[0.2cm]
      gilt.  Wir nehmen also an, dass
      \\[0.2cm]
      \hspace*{1.3cm}
      $\pair(x_1, y_1) \sim \pair(x_2, y_2)$  \quad und \quad
      $\pair(x_2, y_2) \sim \pair(x_3, y_3)$
      \\[0.2cm]
      gilt und zeigen, dass daraus $\pair(x_1,y_1) \sim \pair(x_3, y_3)$ folgt.  Nach Definition von $\sim$
      folgt aus unserer Annahme, dass
      \\[0.2cm]
      \hspace*{1.3cm}
      $x_1 \cdot y_2 = x_2 \cdot y_1$  \quad und \quad
      $x_2 \cdot y_3 = x_3 \cdot y_2$
      \\[0.2cm]
      gilt.  Wir multiplizieren die erste dieser beiden Gleichungen mit $y_3$ und die zweite Gleichung mit 
      $y_1$.  Dann erhalten wir die Gleichungen
      \\[0.2cm]
      \hspace*{1.3cm}
      $x_1 \cdot y_2 \cdot y_3 = x_2 \cdot y_1 \cdot y_3$  \quad und \quad
      $x_2 \cdot y_3 \cdot y_1 = x_3 \cdot y_2 \cdot y_1$
      \\[0.2cm]
      Da f\"{u}r den Operator ``$\cdot$'' das Kommutativ-Gesetzes gilt, k\"{o}nnen wir diese Gleichungen auch in der 
      Form 
      \\[0.2cm]
      \hspace*{1.3cm}
      $x_1 \cdot y_3 \cdot y_2 = x_2 \cdot y_3 \cdot y_1$  \quad und \quad
      $x_2 \cdot y_3 \cdot y_1 = x_3 \cdot y_1 \cdot y_2$
      \\[0.2cm]
      schreiben.  Setzen wir diese Gleichungen zusammen, so sehen wir, dass 
      \\[0.2cm]
      \hspace*{1.3cm}
      $x_1 \cdot y_3 \cdot y_2 = x_3 \cdot y_1 \cdot y_2$
      \\[0.2cm]
      gilt.
      Da der betrachtete Ring nullteilerfrei ist und wir nach Definition von $Q$ wissen, 
      dass $y_2 \not= 0$ ist, k\"{o}nnen  wir hier die Streichungs-Regel benutzen und
      $y_2$ aus der letzten Gleichung herausk\"{u}rzen.  Dann erhalten wir
      \\[0.2cm]
      \hspace*{1.3cm}
      $x_1 \cdot y_3 = x_3 \cdot y_1$. 
      \\[0.2cm]
      Nach Definition der Relation $\sim$ haben wir jetzt
      \\[0.2cm]
      \hspace*{1.3cm}
      $\pair(x_1, y_1) \sim \pair(x_3, y_3)$
      \\[0.2cm]
      und das war zu zeigen. \checkmark       \qed

      \textbf{Bemerkung}: W\"{u}rden wir in der Definition 
      $Q := \bigl\{ \pair(x, y) \in R \times R \mid y \not= 0 \bigr\}$
      die Bedingung $y \not= 0$ weglassen, so w\"{u}rde 
      \\[0.2cm]
      \hspace*{1.3cm}
      $\pair(x, y) \sim \pair(0, 0)$ \quad f\"{u}r alle $x,y \in R$
      \\[0.2cm]
      gelten und damit w\"{a}re dann die Relation $\sim$ nicht mehr transitiv.
\end{enumerate}
Auf der Menge $Q$ definieren wir jetzt Operatoren ``$+$'' und ``$\cdot$''.
Den Operator $+: Q \times Q \rightarrow Q$  definieren wir durch die Festlegung
\\[0.2cm]
\hspace*{1.3cm}
$\pair(x,y) + \pair(u,v) := \pair(x \cdot v + u \cdot y, y \cdot v)$.
\\[0.2cm]
Motiviert ist diese Definition durch die Addition von Br\"{u}chen, bei der wir die beteiligten Br\"{u}che
zun\"{a}chst auf den Hauptnenner bringen:
\\[0.2cm]
\hspace*{1.3cm}
$\bruch{a}{b} + \bruch{c}{d} = \bruch{a \cdot d + c \cdot b}{b \cdot d}$.
\\[0.2cm]
Die \"{a}quivalenz-Relation $\sim$ erzeugt auf der Menge $Q$ der formalen Br\"{u}che den Quotienten-Raum
$Q/\!\sim$.
Unser Ziel ist es, auf diesem Quotienten-Raum 
eine Ringstruktur zu definieren.  Damit dies m\"{o}glich ist zeigen wir, dass die oben definierte
Funktion $+$ mit der auf $Q$ definierten \"{a}quivalenz-Relationen $\sim$ 
vertr\"{a}glich ist.  Es gelte also
\\[0.2cm]
\hspace*{1.3cm}
$\pair(x_1,y_1) \sim \pair(x_2,y_2)$ \quad und \quad
$\pair(u_1,v_1) \sim \pair(u_2,v_2)$.
\\[0.2cm]
Nach Definition der \"{a}quivalenz-Relation $\sim$ hei\3t das
\\[0.2cm]
\hspace*{1.3cm}
$x_1 \cdot y_2 = x_2 \cdot y_1$ \quad und \quad
$u_1 \cdot v_2 = u_2 \cdot v_1$.
\\[0.2cm]
Zu zeigen ist dann
\\[0.2cm]
\hspace*{1.3cm}
$\pair(x_1 \cdot v_1 + u_1 \cdot y_1,\, y_1 \cdot v_1) \sim 
 \pair(x_2 \cdot v_2 + u_2 \cdot y_2,\, y_2 \cdot v_2)$.
\\[0.2cm]
Nach Definition der \"{a}quivalenz-Relation $\sim$ ist dies \"{a}quivalent zu der Gleichung
\\[0.2cm]
\hspace*{1.3cm}
$(x_1 \cdot v_1 + u_1 \cdot y_1) \cdot y_2 \cdot v_2 = 
 (x_2 \cdot v_2 + u_2 \cdot y_2) \cdot y_1 \cdot v_1
$.
\\[0.2cm]
Multiplizieren wir dies mittels des Distributiv-Gesetzes aus und benutzen wir weiter das
Kommutativ-Gesetz f\"{u}r die Multiplikation, so ist die letzte Gleichung \"{a}quivalent zu
\\[0.2cm]
\hspace*{1.3cm}
$x_1 \cdot y_2 \cdot v_1 \cdot v_2 + u_1 \cdot v_2 \cdot y_1 \cdot y_2 = 
 x_2 \cdot y_1 \cdot v_1 \cdot v_2 + u_2 \cdot v_1 \cdot y_1 \cdot y_2
$.
\\[0.2cm]
Formen wir die linke Seite dieser Gleichung durch Verwendung der Voraussetzungen 
$x_1 \cdot y_2 = x_2 \cdot y_1$  und $u_1 \cdot v_2 = u_2 \cdot v_1$ um, so erhalten wir
die offensichtlich wahre Gleichung
\\[0.2cm]
\hspace*{1.3cm}
$x_2 \cdot y_1 \cdot v_1 \cdot v_2 + u_2 \cdot v_1 \cdot y_1 \cdot y_2 = 
 x_2 \cdot y_1 \cdot v_1 \cdot v_2 + u_2 \cdot v_1 \cdot y_1 \cdot y_2
$.
\\[0.2cm]
Damit haben wir die Vertr\"{a}glichkeit des oben definierten Operators ``$+$'' nachgewiesen.
Folglich kann die oben definierte Funktion $+$ auf den Quotienten-Raum $Q/\!\sim$ durch die Festlegung
\\[0.2cm]
\hspace*{1.3cm}
$\bigl[\pair(x,y)\bigr]_\sim + \bigl[\pair(u,v)\bigr]_\sim := 
 \bigl[\pair(x \cdot v + u \cdot y, y \cdot v)\bigr]_\sim
$
\\[0.2cm]
fortgesetzt werden.  Die \"{a}quivalenz-Klasse
$\bigl[\pair(0,1)\bigr]_\sim$
ist bez\"{u}glich der Operation ``$+$'' das neutrale Element, denn es gilt
\\[0.2cm]
\hspace*{1.3cm}
$\bigl[\pair(0,1)\bigr]_\sim + \bigl[\pair(x,y)\bigr]_\sim = 
 \bigl[\pair(0 \cdot y + x \cdot 1, 1 \cdot y)\bigr]_\sim = 
 \bigl[\pair(x, y)\bigr]_\sim
$.
\\[0.2cm]
Es gilt weiter
\\[0.2cm]
\hspace*{1.3cm}
$\bigl[\pair(0,1)\bigr]_\sim = \bigl[\pair(0,y)\bigr]_\sim$ \quad f\"{u}r alle $y \not= 0$,
\\[0.2cm]
denn wir haben
\\[0.2cm]
\hspace*{1.3cm}
$0 \cdot y = 0 \cdot 1$.
\\[0.2cm]
Das bez\"{u}glich der Operation ``$+$'' zu $\bigl[\pair(x,y)\bigr]_\sim$ inverse Element ist
$\bigl[\pair(-x,y)\bigr]_\sim$, denn es gilt
\\[0.2cm]
\hspace*{1.3cm}
$
\begin{array}[t]{lcl}
      \bigl[\pair(-x,y)\bigr]_\sim + \bigl[\pair(x,y)\bigr]_\sim 
& = & \bigl[\pair(-x \cdot y + x \cdot y,\, y \cdot y)\bigr]_\sim      \\[0.2cm]
& = & \bigl[\pair((-x + x) \cdot y,\, y \cdot y)\bigr]_\sim      \\[0.2cm]
& = & \bigl[\pair(0,\, y \cdot y)\bigr]_\sim      \\[0.2cm]
& = & \bigl[\pair(0,\, 1) \bigr]_\sim.
\end{array}
$
\vspace*{0.2cm}

\noindent
Als n\"{a}chstes definieren wir auf der Menge $Q$ den Operator $\cdot: Q \times Q \rightarrow Q$ wie folgt:
\\[0.2cm]
\hspace*{1.3cm}
$\pair(x,y) \cdot \pair(u,v) := \pair(x \cdot u, y \cdot v)$.
\\[0.2cm]
Auch dies wird durch die Analogie f\"{u}r Br\"{u}che motiviert, denn f\"{u}r Br\"{u}che gilt
\\[0.2cm]
\hspace*{1.3cm}
$\bruch{a}{b} \cdot \bruch{c}{d} = \bruch{a \cdot c}{b \cdot d}$.
\\[0.2cm]
Nun zeigen wir, dass die Operation ``$\cdot$'' mit der \"{a}quivalenz-Relation $\sim$ vertr\"{a}glich ist.
Es gelte also
\\[0.2cm]
\hspace*{1.3cm}
$\pair(x_1,y_1) \sim \pair(x_2,y_2)$ \quad und \quad
$\pair(u_1,v_1) \sim \pair(u_2,v_2)$.
\\[0.2cm]
Nach Definition der \"{a}quivalenz-Relation $\sim$ folgt daraus
\\[0.2cm]
\hspace*{1.3cm}
$x_1 \cdot y_2 = x_2 \cdot y_1$ \quad und \quad
$u_1 \cdot v_2 = u_2 \cdot v_1$.
\\[0.2cm]
Zu zeigen ist 
\\[0.2cm]
\hspace*{1.3cm}
$\pair(x_1 \cdot u_1,\, y_1 \cdot v_1) \sim  \pair(x_2 \cdot u_2,\, y_2 \cdot v_2)$.
\\[0.2cm]
Dies ist nach Definition der Relation $\sim$ \"{a}quivalent zu
\\[0.2cm]
\hspace*{1.3cm}
$x_1 \cdot u_1 \cdot y_2 \cdot v_2 = x_2 \cdot u_2 \cdot y_1 \cdot v_1$.
\\[0.2cm]
Diese Gleichung erhalten wir aber sofort, wenn wir die beiden Gleichungen
$x_1 \cdot y_2 = x_2 \cdot y_1$ und $u_1 \cdot v_2 = u_2 \cdot v_1$ mit einander multiplizieren.
Folglich kann der Operator ``$\cdot$'' durch die Definition
\\[0.2cm]
\hspace*{1.3cm}
$\bigl[\pair(x,y)\bigr]_\sim \cdot \bigl[\pair(u,v)\bigr]_\sim := 
 \bigl[\pair(x \cdot u, y \cdot v)\bigr]_\sim
$
\\[0.2cm]
auf den Quotienten-Raum $Q/\!\sim$ fortgesetzt werden.
Das bez\"{u}glich der Operation ``$\cdot$'' neutrale Element ist $\bigl[\pair(1,\,1)\bigr]_\sim$, denn es gilt
\\[0.2cm]
\hspace*{1.3cm}
$
\begin{array}[t]{lcl}
      \bigl[\pair(1,\,1)\bigr]_\sim \cdot  \bigl[\pair(x,\,y)\bigr]_\sim
& = & \bigl[\pair(1 \cdot x,\,1 \cdot y)\bigr]_\sim                      \\[0.2cm]
& = & \bigl[\pair(x,\,y)\bigr]_\sim.
\end{array}
$
\\[0.2cm]
Das bez\"{u}glich der Operation ``$\cdot$'' zu $\bigl[\pair(x,y)\bigr]_\sim$ inverse Element ist nur definiert,
falls
\\[0.2cm]
\hspace*{1.3cm}
 $\bigl[\pair(x,y)\bigr]_\sim \not= \bigl[\pair(0,1)\bigr]_\sim$ 
\\[0.2cm]
ist.  Wir formen diese Ungleichung um:
\\[0.2cm]
\hspace*{1.3cm}
$
\begin{array}[t]{lrcl}
                & \bigl[\pair(x,y)\bigr]_\sim & \not= & \bigl[\pair(0,1)\bigr]_\sim  \\[0.2cm]
\Leftrightarrow & \pair(x,y) & \not\sim & \pair(0,1)                                 \\[0.2cm]
\Leftrightarrow & x \cdot 1  & \not=    & 0 \cdot y                                  \\[0.2cm]
\Leftrightarrow & x          & \not=    & 0. 
\end{array}
$
\\[0.2cm]
Damit sehen wir, dass wir zu dem Ausdruck $\bigl[\pair(x,y)\bigr]_\sim$ nur dann ein bez\"{u}glich der Operation
``$\cdot$'' inverses Element angeben m\"{u}ssen, wenn $x \not= 0$ ist.  Wir behaupten, dass f\"{u}r $x \not= 0$
das Element
\\[0.2cm]
\hspace*{1.3cm}
$\bigl[\pair(y,x)\bigr]_\sim$
\\[0.2cm]
zu $\bigl[\pair(x,y)\bigr]_\sim$ invers ist, denn es gilt:
\\[0.2cm]
\hspace*{1.3cm}
$
\begin{array}[t]{lcl}
      \bigl[\pair(y,x)\bigr]_\sim \cdot \bigl[\pair(x,y)\bigr]_\sim 
& = & \bigl[\pair(y \cdot x,x \cdot y)\bigr]_\sim                   \\[0.2cm]
& = & \bigl[\pair(x \cdot y,x \cdot y)\bigr]_\sim                   \\[0.2cm]
& = & \bigl[\pair(1, 1)\bigr]_\sim                   \\[0.2cm]
\end{array}
$
\\[0.2cm]
denn offenbar gilt $\pair(x \cdot y,x \cdot y) \sim \pair(1,1)$.
\vspace*{0.2cm}

Um zu zeigen, dass die Struktur 
\\[0.2cm]
\hspace*{1.3cm}
$\mathcal{R}/\!\sim :=
\Bigl\langle Q/\!\sim, \bigl[\pair(0,1)\bigr]_\sim, \bigl[\pair(1,1)\bigr]_\sim, +, \cdot \Bigr\rangle$
\\[0.2cm]
mit den oben definierten Operationen ``$+$'' und ``$\cdot$'' ein K\"{o}rper ist, bleibt nachzuweisen,
dass f\"{u}r die Operatoren ``$+$'' und ``$\cdot$'' jeweils das Assoziativ-Gesetz und das Kommutativ-Gesetz gilt. 
Zus\"{a}tzlich muss das Distributiv-Gesetz nachgewiesen werden.
\begin{enumerate}
\item Der Operator ``$+$'' ist in $Q$ assoziativ, denn f\"{u}r beliebige
      Paare $\pair(x_1,y_1),\pair(x_2,y_2),\pair(x_3,y_3) \in Q$ gilt:
      \\[0.2cm]
      \hspace*{1.3cm}
      $
      \begin{array}[t]{cl}
        & \bigl(\pair(x_1, y_1) + \pair(x_2, y_2)\bigr) + \pair(x_3, y_3)                      \\[0.2cm]
      = & \bigr(\pair(x_1 \cdot y_2 + x_2 \cdot y_1, y_1 \cdot y_2)\bigr) + \pair(x_3, y_3)    \\[0.2cm]
      = & \pair((x_1 \cdot y_2 + x_2 \cdot y_1) \cdot y_3 + x_3 \cdot y_1 \cdot y_2,\,
                y_1 \cdot y_2 \cdot y_3)                                                       \\[0.2cm]
      = & \pair(x_1 \cdot y_2 \cdot y_3 + x_2 \cdot y_1 \cdot y_3 + x_3 \cdot y_1 \cdot y_2,\, 
                y_1 \cdot y_2 \cdot y_3)                                                       \\[0.2cm]
      \end{array}
      $
      \\[0.2cm]
      Auf der anderen Seite haben wir
      \\[0.2cm]
      \hspace*{1.3cm}
      $
      \begin{array}[t]{cl}
        & \pair(x_1, y_1) + \bigl(\pair(x_2, y_2) + \pair(x_3, y_3)\bigr)                      \\[0.2cm]
      = & \pair(x_1, y_1) + \bigl(\pair(x_2 \cdot y_3 + x_3 \cdot y_2,\, y_2 \cdot y_3)\bigr)  \\[0.2cm]
      = & \pair(x_1 \cdot y_2 \cdot y_3 + (x_2 \cdot y_3 + x_3 \cdot y_2) \cdot y_1,\,
                y_1 \cdot y_2 \cdot y_3)                                                       \\[0.2cm]
      = & \pair(x_1 \cdot y_2 \cdot y_3 + x_2 \cdot y_1 \cdot y_3 + x_3 \cdot y_1 \cdot y_2,\, 
                y_1 \cdot y_2 \cdot y_3)                                                       \\[0.2cm]
      \end{array}
      $
      \\[0.2cm]
      Da dies mit dem oben abgeleiteten Ergebnis \"{u}bereinstimmt, haben wir die G\"{u}ltigkeit des 
      Assoziativ-Gesetzes  nachgewiesen.
\item Der Operator ``$+$'' ist in $Q$ kommutativ, denn f\"{u}r beliebige
      Paare $\pair(x_1,y_1),\pair(x_2,y_2) \in Q$ gilt:
      \\[0.2cm]
      \hspace*{1.3cm}
      $
      \begin{array}[t]{cl}
        & \pair(x_1, y_1) + \pair(x_2, y_2)               \\[0.2cm]
      = & \pair(x_1 \cdot y_2 + x_2 \cdot y_1, y_1 \cdot y_2)   \\[0.2cm]
      = & \pair(x_2 \cdot y_1 + x_1 \cdot y_2, y_2 \cdot y_1)   \\[0.2cm]
      = & \pair(x_2, y_2) + \pair(x_1, y_1).               \\[0.2cm]
      \end{array}
      $
      \\[0.2cm]
      Genau wie oben folgt nun, dass das Kommutativ-Gesetz auch in $Q/\!\sim$ gilt.
\item Den Nachweis der Assoziativit\"{a}t und der Kommutativit\"{a}t des Multiplikations-Operators
      \"{u}berlasse ich Ihnen zur \"{u}bung.

      \exercise
      Zeigen Sie, dass in $Q/\!\sim$ f\"{u}r die Multiplikation sowohl das Assoziativ-Gesetz als auch das
      Kommutativ-Gesetz gilt.
\item Zum Nachweis des Distributiv-Gesetzes in $Q/\!\sim$ 
      zeigen wir, dass f\"{u}r alle Paare \\
      $\pair(x_1,y_1),\, \pair(x_2,y_2),\, \pair(x_3,y_3) \in Q$
      folgendes gilt:
      \\[0.2cm]
      \hspace*{0.3cm}
      $
      \bpr(x_1,y_1) \cdot \Bigl(\bpr(x_2,y_2) + \bpr(x_3,y_3)\Bigr) =
      \bpr(x_1,y_1) \cdot \bpr(x_2,y_2) + \bpr(x_1,y_1) \cdot \bpr(x_3,y_3)
      $.
      \\[0.2cm]
      Wir werten die linke und rechte Seite dieser Gleichung getrennt aus und beginnen mit der
      linken Seite. 
      \\[0.2cm]
      \hspace*{1.3cm}
      $
      \begin{array}[t]{cl}
        & \bpr(x_1,y_1) \cdot \Bigl(\bpr(x_2,y_2) + \bpr(x_3,y_3)\Bigr)                     \\[0.2cm]
      = & \bpr(x_1,y_1) \cdot \bpr(x_2 \cdot y_3 + x_3 \cdot y_2,\,y_2 \cdot y_3)           \\[0.2cm]
      = & \bpr(x_1 \cdot \kla x_2 \cdot y_3 + x_3 \cdot y_2\klz,\,y_1 \cdot y_2 \cdot y_3)  \\[0.2cm]
      = & \bpr(x_1 \cdot x_2 \cdot y_3 + x_1 \cdot x_3 \cdot y_2,\,y_1 \cdot y_2 \cdot y_3) \\[0.2cm]
      \end{array}
      $
      \\[0.2cm]
      Wir werten nun die rechte Seite aus.
      \\[0.2cm]
      \hspace*{1.3cm}
      $
      \begin{array}[t]{cl}
      & \bpr(x_1,y_1) \cdot \bpr(x_2,y_2) + \bpr(x_1,y_1) \cdot \bpr(x_3,y_3)        \\[0.2cm]
      = & \bpr(x_1 \cdot x_2,\, y_1 \cdot y_2) + \bpr(x_1 \cdot x_3,\,y_1 \cdot y_3)   \\[0.2cm]
      = & \bpr(x_1 \cdot x_2 \cdot y_1 \cdot y_3 + x_1 \cdot x_3 \cdot y_1 \cdot y_2,\, 
      y_1 \cdot y_2 \cdot y_1 \cdot y_3). 
      \end{array}
      $
      \\[0.2cm]
      Allgemein haben wir bereits gesehen, dass f\"{u}r $c \not= 0$, $c \in R$ und beliebige $a,b \in R$ die Gleichung
      \\[0.2cm]
      \hspace*{1.3cm}
      $\bpr(a,b) = \bpr(a \cdot c,\,b \cdot c)$
      \\[0.2cm]
      gilt.  Wenden wir diese Gleichung  auf die oben f\"{u}r die linke und rechte Seite des
      Distributiv-Gesetzes erzielten Ergebnisse an, so sehen wir, dass beide Seiten gleich sind.
      Damit ist die G\"{u}ltigkeit des Distributiv-Gesetzes in $Q/\!\sim$ nachgewiesen.  
\end{enumerate}
Damit haben wir nun gezeigt, dass die Struktur
\\[0.2cm]
\hspace*{1.3cm}
$\textsl{Quot}(\mathcal{R}) := \langle Q/\!\sim, \bpr(0,1), \bpr(1,1), +, \cdot \rangle$ 
\\[0.2cm]
ein K\"{o}rper ist.  Dieser K\"{o}rper wird als der von $\mathcal{R}$ erzeugte \emph{Quotienten-K\"{o}rper} bezeichnet.


\section{Ideale und Faktor-Ringe$^*$}
Der im Folgenden definierte Begriff des \emph{Ideals} hat in der Theorie der Ringe eine \"{a}hnliche
Stellung wie der Begriff der Untergruppe in der Theorie der Gruppen.
\begin{Definition}[Ideal]
Es sei $\mathcal{R} = \langle R, 0, 1, +, \cdot \rangle$ ein kommutatives Ring mit Eins.
Eine Teilmenge $I \subseteq R$ ist ein \emph{Ideal in $\mathcal{R}$} falls folgendes gilt:
\begin{enumerate}
\item $I \leq R$,

      die Struktur $I$ ist also eine Untergruppe der Gruppe $\langle R, 0, + \rangle$.
\item $\forall a \in I: \forall b \in R: b \cdot a \in I$,

      f\"{u}r alle Elemente $a$ aus dem Ideal $I$ ist das Produkt mit einem beliebigen Element
      $b$ aus dem Ring $R$ wieder ein Element aus dem Ideal. \eox
\end{enumerate}
\end{Definition}

\remark
An dieser Stelle sollten Sie sich noch einmal die Definition einer Untergruppe ins
Ged\"{a}chtnis rufen:  Es gilt $I \leq R$ genau
dann, wenn folgende Bedingungen erf\"{u}llt sind:
\begin{enumerate}
\item $0 \in I$,
\item $a,b \in I \rightarrow a + b \in I$,
\item $a \in I \rightarrow -a \in I$.
\end{enumerate}
Beachten Sie au\3erdem, dass in der Formel 
\\[0.2cm]
\hspace*{1.3cm}
$\forall a \in I: \forall b \in R: b \cdot a \in I$,
\\[0.2cm]
der zweite All-Quantor nicht nur \"{u}ber die Elemente aus $I$ l\"{a}uft, sondern \"{u}ber alle Elemente
von $R$.  \eox

\examples
\begin{enumerate}
\item Die Menge alle geraden Zahlen
      \\[0.2cm]
      \hspace*{1.3cm}
      $2 \mathbb{Z} = \{ 2 \cdot x \mid x \in \mathbb{Z} \}$
      \\[0.2cm]
      ist ein Ideal in dem Ring $\langle \mathbb{Z}, 0, 1, +, \cdot \rangle$ der ganzen
      Zahlen, denn wir haben
      \begin{enumerate}
      \item $0 \in 2\mathbb{Z}$, da $0 = 2 \cdot 0$ ist und somit ist $0$ eine gerade Zahl.
      \item Sind $a$, $b$ gerade Zahlen, so gibt es $x,y \in \mathbb{Z}$ mit $a = 2 \cdot x$ und 
            $b = 2 \cdot y$.  Daraus folgt
            \\[0.2cm]
            \hspace*{1.3cm}
            $a + b = 2 \cdot x + 2 \cdot y = 2 \cdot (x + y)$
            \\[0.2cm]
            und damit ist auch $a+b$ eine gerade Zahl.
      \item Ist $a \in 2\mathbb{Z}$, so gibt es $x \in \mathbb{Z}$ mit $a = 2 \cdot x$.  Dann gilt
            \\[0.2cm]
            \hspace*{1.3cm}
            $-a = - 2 \cdot x = 2 \cdot (-x)$
            \\[0.2cm]
            und damit ist auch $-a$ eine gerade Zahl.
      \item Ist $a$ eine gerade Zahl und ist $b \in \mathbb{Z}$, so gibt es zun\"{a}chst eine
            Zahl $x \in \mathbb{Z}$ mit $a = 2 \cdot x$.  Daraus folgt
            \\[0.2cm]
            \hspace*{1.3cm}
            $a \cdot b = (2 \cdot x) \cdot b = 2 \cdot (x \cdot b)$
            \\[0.2cm]
            und das ist offenbar wieder eine gerade Zahl.
      \end{enumerate}
\item Das letzte Beispiel l\"{a}sst sich verallgemeinern: Es sei $k \in \mathbb{Z}$. Dann ist die Menge 
      \\[0.2cm]
      \hspace*{1.3cm}
      $k\mathbb{Z} := \{ a \cdot k \mid a \in \mathbb{Z} \}$
      \\[0.2cm]
      der Vielfachen von $k$  ein Ideal in dem Ring $\langle \mathbb{Z}, 0, 1, +, \cdot \rangle$.
      Der Nachweis ist anlog zu dem oben gef\"{u}hrten Nachweis, dass $2\mathbb{Z}$ ein Ideal
      in dem Ring der ganzen Zahlen ist.
\item Wir verallgemeinern das letzte Beispiel f\"{u}r beliebige kommutative Ringe mit Eins.
      Es sei also $\mathcal{R} = \langle R, 0, 1, +, \cdot \rangle$ ein kommutativer Ring
      mit Eins und es sei $a \in R$.  Dann definieren wir die Menge
      \\
      \hspace*{1.3cm}
      $\textsl{gen}(k) := \{ k \cdot x \mid x \in R \}$
      \\[0.2cm]
      aller Vielfachen von $k$ in $R$.  Wir zeigen, dass diese Menge ein Ideal in
      $\mathcal{R}$ ist.
      \begin{enumerate}
      \item $0 \in \textsl{gen}(k)$, da $0 = k \cdot 0$ gilt. 
      \item Sind $a, b \in \textsl{gen}(k)$, so gibt es $x,y \in R$ mit $a = k \cdot x$ und 
            $b = k \cdot y$.  Daraus folgt
            \\[0.2cm]
            \hspace*{1.3cm}
            $a + b = k \cdot x + k \cdot y = k \cdot (x + y)$
            \\[0.2cm]
            und folglich gilt $a+b \in \textsl{gen}(k)$.
      \item Ist $a \in \textsl{gen}(k)$, so gibt es ein $x \in R$ mit $a = k \cdot x$.  Dann gilt
            \\[0.2cm]
            \hspace*{1.3cm}
            $-a = - (k \cdot x) = k \cdot (-x) \in \textsl{gen}(a)$.
      \item Ist $a \in \textsl{gen}(k)$ und ist $b \in \mathbb{Z}$, so gibt es zun\"{a}chst ein
            $x \in R$ mit $a = k \cdot x$.  Daraus folgt
            \\[0.2cm]
            \hspace*{1.3cm}
            $a \cdot b = (k \cdot x) \cdot b = k \cdot (x \cdot b) \in \textsl{gen}(k)$.
      \end{enumerate}
      Die Menge $\textsl{gen}(a)$ wird das von $a$ \emph{erzeugte Ideal} genannt.
      Ideale dieser Form werden in der Literatur als \emph{Haupt-Ideale} bezeichnet.
\item Wieder sei $\mathcal{R} = \langle R, 0, 1, +, \cdot \rangle$ ein kommutativer Ring mit Eins.  Dann
      sind die Mengen $\{0\}$ und $R$ offenbar wieder Ideale von $R$.  Wir nennen  die Menge $\{0\}$
      das Null-Ideal und $R$ das Eins-Ideal.  Diese beiden Ideale werden auch als die \emph{trivialen}
      Ideale bezeichnet.
      \eox
\end{enumerate}


Mit Hilfe von Idealen l\"{a}sst sich auf einem Ring eine Kongruenz-Relation erzeugen.  Ist $I$ ein Ideal auf
dem kommutativen Ring mit Eins $\mathcal{R} = \langle R, 0, 1, +, \cdot \rangle$, so definieren wir eine Relation $\sim_I$ auf
$R$ durch die Forderung
\\[0.2cm]
\hspace*{1.3cm}
$a \sim_I b \df a - b \in I$.
\\[0.2cm]
Wir zeigen, dass die Relation $\sim_I$ eine Kongruenz-Relation auf $R$ ist.
\begin{enumerate}
\item $\sim_I$ ist reflexiv auf $R$, denn f\"{u}r alle $x \in R$ gilt
      \\[0.2cm]
      \hspace*{1.3cm}
      $
      \begin{array}[t]{cl}
                      & x \sim_I x   \\[0.2cm] 
      \Leftrightarrow & x - x \in I  \\[0.2cm] 
      \Leftrightarrow & 0 \in I  
      \end{array}
      $
      \\[0.2cm]
      Da ein Ideal insbesondere eine Untergruppe ist, gilt $0 \in I$ und damit ist $x \sim_I x$
      gezeigt. \checkmark
\item Wir zeigen: $\sim_I$ ist symmetrisch.  Sei $x \sim_I y$ gegeben.  Nach Definition der
      Relation $\sim_I$ folgt  
      \\[0.2cm]
      \hspace*{1.3cm}
      $x - y \in I$.
      \\[0.2cm]
      Da eine Untergruppe bez\"{u}glich der Bildung des additiven Inversen abgeschlossen ist, gilt dann auch
      \\[0.2cm]
      \hspace*{1.3cm}
      $-(x - y) = y - x \in I$.
      \\[0.2cm]
      Wieder nach Definition der Relation $\sim_I$ hei\3t das 
      \\[0.2cm]
      \hspace*{1.3cm}
      $y \sim_I x$.  
      \checkmark
\item Wir zeigen: $\sim_I$ ist transitiv.  Es gelte
      \\[0.2cm]
      \hspace*{1.3cm}
      $x \sim_I y$ \quad und \quad $y \sim_I z$.
      \\[0.2cm]
      Nach Definition der Relation $\sim_I$ folgt daraus
      \\[0.2cm]
      \hspace*{1.3cm}
      $x - y \in I$ \quad und \quad $y - z \in I$.
      \\[0.2cm]
      Da Ideale unter Addition abgeschlossen sind, folgt daraus
      \\[0.2cm]
      \hspace*{1.3cm}
      $x - z = (x - y) + (y - z) \in I$.
      \\[0.2cm]
      Nach Definition der Relation $\sim_I$ hei\3t das 
      \\[0.2cm]
      \hspace*{1.3cm}
      $x \sim_I z$.  
      \checkmark
\item Wir zeigen: $\sim_I$ ist mit der Addition auf dem Ring $\mathcal{R}$ vertr\"{a}glich.  Es sei
      also
      \\[0.2cm]
      \hspace*{1.3cm}
      $x_1 \sim_I x_2$ \quad und \quad $y_1 \sim_I y_2$ 
      \\[0.2cm]
      gegeben.  Zu zeigen ist, dass dann auch
      \\[0.2cm]
      \hspace*{1.3cm}
      $x_1 + y_1 \sim_I x_2 + y_2$
      \\[0.2cm]
      gilt.  Aus den Voraussetzungen $x_1 \sim_I x_2$  und $y_1 \sim_I y_2$ folgt nach Definition
      der Relation $\sim_I$, dass
      \\[0.2cm]
      \hspace*{1.3cm}
      $x_1 - x_2 \in I$ \quad und \quad $y_1 - y_2 \in I$ 
      \\[0.2cm]
      gilt.  Addieren wir diese Gleichungen und ber\"{u}cksichtigen, dass das Ideal $I$ unter Addition
      abgeschlossen ist, so erhalten wir
      \\[0.2cm]
      \hspace*{1.3cm}
      $(x_1 - x_2) + (y_1 - y_2) \in I$.
      \\[0.2cm]
      Wegen $(x_1 - x_2) + (y_1 - y_2) = (x_1 + y_1) - (x_2 + y_2)$ folgt daraus
      \\[0.2cm]
      \hspace*{1.3cm}
      $(x_1 + y_1) - (x_2 + y_2) \in I$
      \\[0.2cm]
      und nach Definition der Relation $\sim_I$ hei\3t das
      \\[0.2cm]
      \hspace*{1.3cm}
      $x_1 + y_1 \sim_I x_2 + y_2$. \checkmark
\item Wir zeigen: $\sim_I$ ist mit der Multiplikation auf dem Ring $\mathcal{R}$ vertr\"{a}glich.
      Es sei also wieder
      \\[0.2cm]
      \hspace*{1.3cm}
      $x_1 \sim_I x_2$ \quad und \quad $y_1 \sim_I y_2$ 
      \\[0.2cm]
      gegeben.  Diesmal ist zu zeigen, dass daraus
      \\[0.2cm]
      \hspace*{1.3cm}
      $x_1 \cdot y_1 \sim_I x_2 \cdot y_2$
      \\[0.2cm]
      folgt.  Aus den Voraussetzungen $x_1 \sim_I x_2$  und $y_1 \sim_I y_2$ folgt nach Definition
      der Relation $\sim_I$ zun\"{a}chst, dass
      \\[0.2cm]
      \hspace*{1.3cm}
      $x_1 - x_2 \in I$ \quad und \quad $y_1 - y_2 \in I$ 
      \\[0.2cm]
      gilt.  Da ein Ideal unter Multiplikation mit beliebigen Elementen des Rings abgeschlossen ist,
      folgt daraus, dass auch
      \\[0.2cm]
      \hspace*{1.3cm}
      $(x_1 - x_2) \cdot y_2 \in I$ \quad und \quad $x_1 \cdot (y_1 - y_2) \in I$ 
      \\[0.2cm]
      gilt. Addieren wir diese Gleichungen und ber\"{u}cksichtigen, dass das Ideal $I$ unter Addition
      abgeschlossen ist, so erhalten wir
      \\[0.2cm]
      \hspace*{1.3cm}
      $(x_1 - x_2) \cdot y_2 + x_1 \cdot (y_1 - y_2) \in I$.
      \\[0.2cm]
      Nun gilt
      \\[0.2cm]
      \hspace*{1.3cm}
      $
      \begin{array}[t]{lcl}
            (x_1 - x_2) \cdot y_2 + x_1 \cdot (y_1 - y_2) 
      & = & x_1 \cdot y_2 - x_2 \cdot y_2 + x_1 \cdot y_1 - x_1 \cdot y_2 \\[0.2cm]
      & = & x_1 \cdot y_1 - x_2 \cdot y_2
      \end{array}
      $
      \\[0.2cm]
      Also haben wir
      \\[0.2cm]
      \hspace*{1.3cm}
      $x_1 \cdot y_1 - x_2 \cdot y_2 \in I$
      \\[0.2cm] 
      gezeigt. Nach Definition der Relation $\sim_I$ ist das \"{a}quivalent zu
      \\[0.2cm]
      \hspace*{1.3cm}
      $x_1 \cdot y_1 \sim_I x_2 \cdot y_2$. 
      \\[0.2cm]
      und das war zu zeigen. \checkmark 
\end{enumerate}

Ist $\mathcal{R} = \langle R, 0, 1, +, \cdot \rangle$ ein kommutativer Ring mit Eins und ist $I$ ein
Ideal dieses Rings, so haben wir gerade gezeigt, dass die von diesem Ideal erzeugte Relation $\sim_I$
eine Kongruenz-Relation auf $\mathcal{R}$ ist.  Nach dem Satz \"{u}ber Faktor-Ringe
(das war Satz \ref{satz:faktor_ring} auf Seite \pageref{satz:faktor_ring}) folgt nun, dass die Struktur
\\[0.2cm]
\hspace*{1.3cm}
 $\mathcal{R}/I := \langle R/\!\sim_I, [0]_{\sim_I}, [1]_{\sim_I}, +, \cdot \rangle$
\\[0.2cm]
ein Ring ist.  In bestimmten F\"{a}llen ist diese Struktur sogar ein K\"{o}rper.  Das werden wir jetzt n\"{a}her
untersuchen.

\begin{Definition}[maximales Ideal]
  Es sei $\mathcal{R} = \langle R, 0, 1, +, \cdot \rangle$ ein kommutativer Ring mit Eins.  Ein Ideal
  $I$ von $\mathcal{R}$ mit $I \not= R$ ist ein \emph{maximales Ideal} genau dann, wenn f\"{u}r jedes andere Ideal
  $J$ von $\mathcal{R}$ gilt:
  \\[0.2cm]
  \hspace*{1.3cm}
  $I \subseteq J \rightarrow J = I \vee J = R$.
  \\[0.2cm]
  Das Ideal ist also maximal, wenn es zwischen dem Ideal $I$ und dem Eins-Ideal $R$ keine 
  weiteren Ideale gibt.
  \eox
\end{Definition}

Der n\"{a}chste Satz zeigt uns, in welchen F\"{a}llen wir mit Hilfe eines Ideals einen K\"{o}rper konstruieren
k\"{o}nnen. 
\begin{Satz}[Faktor-Ringe maximaler Ideale sind K\"{o}rper] \lb
  Es $\mathcal{R} = \langle R, 0, 1, +, \cdot \rangle$ ein kommutativer Ring mit Eins und
  $I$ sei ein maximales Ideal in $\mathcal{R}$.  Dann ist der Faktor-Ring
  \\[0.2cm]
  \hspace*{1.3cm}
  $\mathcal{R}/I := \langle R/\!\sim_I, [0]_{\sim_I}, [1]_{\sim_I}, +, \cdot \rangle$
  \\[0.2cm]
  ein K\"{o}rper.
\end{Satz}

\proof  
Es ist zu zeigen, dass es f\"{u}r jede \"{a}quivalenz-Klasse $[a]_{\sim_I} \not= [0]_{\sim_I}$ ein
multiplikatives Inverses, also eine \"{a}quivalenz-Klasse $[b]_{\sim_I}$ existiert, so dass
\\[0.2cm]
\hspace*{1.3cm}
$[a]_{\sim_I} \cdot [b]_{\sim_I} = [1]_{\sim_I}$
\\[0.2cm]
gilt.  Nach unserer Definition des Multiplikations-Operators ``$\cdot$'' auf $R/\!\!\sim_I$ ist diese Gleichung
\"{a}quivalent zu
\\[0.2cm]
\hspace*{1.3cm}
$[a \cdot b]_{\sim_I} = [1]_{\sim_I}$
\\[0.2cm] 
und nach dem Satz \"{u}ber die Charakterisierung der \"{a}quivalenz-Klassen (\ref{satz:aequivalenz-klassen} auf
Seite \pageref{satz:aequivalenz-klassen})  ist diese Gleichung genau dann erf\"{u}llt, wenn
\\[0.2cm]
\hspace*{1.3cm}
$a \cdot b \sim_I 1$
\\[0.2cm]
gilt.  Nach Definition der Aquivalenz-Relation $\sim_I$ k\"{o}nnen wir diese Bedingung als
\\[0.2cm]
\hspace*{1.3cm}
$a \cdot b - 1 \in I$
\\[0.2cm]
schreiben.  Genauso sehen wir, dass die Bedingung $[a]_{\sim_I} \not= [0]_{\sim_I}$ zu $a - 0 \not\in I$ \"{a}quivalent ist.
Wir m\"{u}ssen also f\"{u}r alle $a \in R$ mit $a \not\in I$ ein $b \in R$ finden, so dass $a \cdot b - 1 \in I$
gilt.
\\[0.2cm]
\hspace*{1.3cm}
zu zeigen: \quad 
$\forall a \in R:\bigl(a \not \in I \rightarrow \exists b \in R: a \cdot b - 1 \in I\bigr)$
\hspace*{\fill} $(*)$
\\[0.2cm]
Wir definieren eine Menge $J$ als
\\[0.2cm]
\hspace*{1.3cm}
$J := \bigl\{ a \cdot x + y \mid x \in R \wedge y \in I \bigr\}$.
\\[0.2cm]
Wir zeigen, dass $J$ ein Ideal des Rings $\mathcal{R}$ ist.
\begin{enumerate}
\item Wir zeigen, dass $0 \in J$ ist.
 
      Da $I$ eine Ideal ist, gilt $0 \in I$.  Setzen wir in der Definition von
      $x := 0$ und $y := 0$, was wegen $0 \in I$ m\"{o}glich ist, so erhalten wir
      \\[0.2cm]
      \hspace*{1.3cm}
      $a \cdot 0 + 0 \in J$, \quad also \quad $0 \in J$.
      \checkmark
\item Wir zeigen, dass $J$ abgeschlossen ist unter Addition.

      Es gelte $a \cdot x_1 + y_1 \in J$ und $a \cdot x_2 + y_2 \in J$ und es seien
      $x_1,x_2 \in R$ und $y_1,y_2 \in I$.  Ofensichtlich ist dann auch $x_1 + x_2 \in R$ und
      da $I$ unter Addition abgeschlossen ist, folgt $y_1 + y_2 \in I$.  Dann haben wir
      \\[0.2cm]
      \hspace*{1.3cm}
      $(a \cdot x_1 + y_1) + (a \cdot x_2 + y_2) = a \cdot (x_1 + x_2) + (y_1 + y_2) \in J$.
      \checkmark
\item Wir zeigen, dass $J$ mit jeder Zahl $z$ auch das zugeh\"{o}rige additive Inverse $-z$ enth\"{a}lt.

      Es gelte $a \cdot x + y \in J$, wobei
      $x \in R$ und $y \in I$ gelte.  Offensichtlich ist dann auch $-x \in R$ und
      da mit $y$ auch $-y$ ein Element von $I$ ist, haben wir
      \\[0.2cm]
      \hspace*{1.3cm}
      $-(a \cdot x + y) = a \cdot (-x) + (-y) \in J$.
      \checkmark
\item Wir zeigen, dass $J$ unter Multiplikation mit beliebigen Elementen des Rings abgeschlossen ist.

      Es gelte $a \cdot x + y \in J$ mit $x \in R$ und $y \in J$.  Weiter sei $k \in R$.
      Dann gilt auch $k \cdot y \in I$, denn $I$ ist ja ein Ideal. Offensichtlich gilt $k \cdot x \in R$.
      Also haben wir
      \\[0.2cm]
      \hspace*{1.3cm}
      $k \cdot (a \cdot x + y) = a \cdot (k \cdot x) + (k \cdot y) \in J$.
      \checkmark
\end{enumerate}
Damit ist gezeigt, dass $J$ ein Ideal des Rings $\mathcal{R}$ ist.
Offenbar ist $J$ eine Obermenge von $I$, denn f\"{u}r alle $y \in I$ gilt
\\[0.2cm]
\hspace*{1.3cm}
$y = a \cdot 0 + y \in J$, \quad also $I \subseteq J$.
\\[0.2cm]
Als n\"{a}chstes bemerken wir, dass das Ideal $J$ von dem Ideal $I$ verschieden ist, denn es gilt
\\[0.2cm]
\hspace*{1.3cm}
$a = a \cdot 1 + 0 \in J$, \quad aber \quad $a \not\in I$.
\\[0.2cm]
Nun ist die Voraussetzung, dass das Ideal $I$ maximal ist.  Da $J \not= I$ aber $I \subseteq J$ ist,
kann jetzt nur noch $J = R$ gelten.  Wegen $1 \in R$ folgt also $1 \in J$.  Damit gibt es 
ein $x \in R$ und ein $y \in I$, so dass
\\[0.2cm]
\hspace*{1.3cm}
$1 = a \cdot x + y$
\\[0.2cm]
gilt.  Aus  $y \in I$ folgt $-y \in I$ und damit haben wir
\\[0.2cm]
\hspace*{1.3cm}
$a \cdot x - 1 = -y \in I$.
\\[0.2cm]
Setzen wir $b := x$, so haben wir damit die Formel $(*)$ nachgewiesen. \qed
\pagebreak

\exercise
Es seien $a,b \in \mathbb{N}$ mit $0 < a < b$ und die Zahlen $a, b$ seien teilerfremd, es gelte also
$\mathtt{ggt}(a,b) = 1$.  Zeigen Sie durch Induktion \"{u}ber $b \in \mathbb{N}$, dass 
\\[0.2cm]
\hspace*{1.3cm}
$\exists k,l \in \mathbb{Z}: k \cdot a + l \cdot b = 1$
\\[0.2cm]
gilt.  
\vspace*{0.2cm}

\noindent
\textbf{Hinweis}: Im Induktions-Schritt ist es nicht sinnvoll, von $b$ auf $b+1$ zu schlie\3en.  Statt dessen
sollten Sie im Induktions-Schritt versuchen, die Behauptung f\"{u}r $b$ aus der Tatsache zu folgern,
dass die Behauptung f\"{u}r sowohl f\"{u}r $b-a$ als auch f\"{u}r $a$ gilt. \exend

\remark
Beim Beweis einer Behauptung, die f\"{u}r nat\"{u}rliche Zahlen gezeigt werden soll, hilft es manchmal, die
Behauptung zun\"{a}chst f\"{u}r kleine Zahlen zu \"{u}berpr\"{u}fen, denn das liefert oft eine Idee f\"{u}r den allgemeinen
Fall. 
\eox

\exercise
Wir betrachten den Ring der ganzen Zahlen $\mathbb{Z}$.  Es sei $p \in \mathbb{Z}$ und wir
definieren die Menge $p\mathbb{Z}$ als
\\[0.2cm]
\hspace*{1.3cm}
$p\mathbb{Z} := \{ k \cdot p \mid k \in \mathbb{Z} \}$.
\\[0.2cm]
Zeigen Sie, dass die Menge $p\mathbb{Z}$ genau dann ein maximales Ideal in dem Ring $\mathbb{Z}$
ist, wenn $p$ eine Primzahl ist.
\vspace*{0.2cm}

\noindent
\textbf{Hinweis}: Um zu zeigen, dass f\"{u}r eine Primzahl $p$ die Menge $p\mathbb{Z}$ ein maximales
Ideal ist, ist es sinnvoll anzunehmen, dass es ein Ideal $I$
gibt, so dass
\\[0.2cm]
\hspace*{1.3cm}
$p\mathbb{Z} \subseteq I$ \quad und \quad $I \not= p\mathbb{Z}$
\\[0.2cm]
gilt.  Sie m\"{u}ssen zeigen, dass in diesem Fall $I = \mathbb{Z}$ ist und dazu reicht es zu zeigen,
dass $1 \in I$ ist.  Um diesen Nachweis zu f\"{u}hren, betrachten Sie die
kleinste nat\"{u}rliche Zahl $r \in I \backslash p\mathbb{Z}$ und wenden anschlie\3end auf $r$ und $p$
den in der vorhergehenden Aufgabe gezeigten Satz an. \exend 

\exercise
Zeigen Sie, dass der Faktor-Ring $\mathbb{Z}_p := \mathbb{Z}/p\mathbb{Z}$ genau dann ein K\"{o}rper ist,
wenn $p$ eine Primzahl ist.
\eox

%%% Local Variables: 
%%% mode: latex
%%% TeX-master: "lineare-algebra"
%%% End: 
