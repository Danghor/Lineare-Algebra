\documentclass{article}
\usepackage{german}
\usepackage{fancyvrb}
\usepackage[latin1]{inputenc}

\renewcommand{\labelenumi}{(\alph{enumi})}
\renewcommand{\labelenumii}{\arabic{enumii}.}

\usepackage{a4wide}
\usepackage{amssymb}
\usepackage{epsfig}

\setlength{\textwidth}{15cm}

\newcommand{\bruch}[2]{\displaystyle\frac{#1}{#2}}
\def\pair(#1,#2){\langle #1, #2 \rangle}
\newcommand{\punkte}[1]{\hspace*{\fill} \mbox{(#1 Punkte)}}
\renewcommand{\labelenumi}{(\alph{enumi})}
\renewcommand{\labelenumii}{\arabic{enumii}.}

\newcounter{aufgabe}

\newcommand{\exercise}{\vspace*{0.3cm}
\stepcounter{aufgabe}

\noindent
\textbf{Aufgabe \arabic{aufgabe}}: }

\newcommand{\solution}{\vspace*{0.3cm}

\noindent
\textbf{L�sung:}\ }


\begin{document}

\noindent
{\large  Aufgaben zur Vorlesung  ``{\sl Mathematik \texttt{I}}''}
\vspace{0.5cm}
\exercise
F�r zwei Mengen $A$ und $B$ werde die \emph{symmetrische Differenz} $A \oplus B$ durch die Gleichung
\\[0.2cm]
\hspace*{1.3cm}
$A \oplus B := (A \cup B) \backslash (A \cap B)$
\\[0.2cm]
definiert.  \underline{Beweisen} Sie die folgenden Eigenschaften der symmetrischen Differenz:
\begin{enumerate}
\item $(A \oplus B) \oplus C = A \oplus (B \oplus C)$.
\item $A \oplus B = A \oplus C \rightarrow B = C$.
\end{enumerate}
Beweisen oder widerlegen Sie die folgenden Gleichungen:
\begin{enumerate}
\setcounter{enumi}{2}
\item $A \cap (B \oplus C) = (A \cap B) \oplus (A \cap C)$.
\item $A \cup (B \oplus C) = (A \cup B) \oplus (A \cup C)$.
\end{enumerate}

\exercise
Beweisen Sie die Gleichung
\\[0.2cm]
\hspace*{1.3cm}
$\sum\limits_{i=1}^n i \cdot (i+1) = \frac{1}{3} \cdot n \cdot (n+1) \cdot (n+2)$.
\\[0.2cm]


\exercise
Es sei $M$ eine beliebige Menge.  Beweisen Sie die folgende Behauptung: 
\\[0.2cm]
\hspace*{1.3cm}
Wenn eine Relation $R \subseteq M \times M$
transitiv ist, dann ist auch die Relation $R \circ R$ transitiv.


\exercise
Welche Bedingungen m�ssen die Zahlen $\alpha$, $\beta$, $\gamma$ und $\delta$ erf�llen,
damit das Gleichungssystem 
\\[0.2cm]
\hspace*{1.3cm}
$
\begin{array}[c]{lcl}
  \alpha \cdot x + \beta  \cdot y & = & b_1  \\[0.2cm]
  \gamma \cdot x + \delta \cdot y & = & b_2  
\end{array}
$
\\[0.2cm]
in den Unbekannten $x$ und $y$ f�r beliebige Zahlen $b_1$ und $b_2$  eine eindeutige L�sung hat?

\exercise
Bestimmen Sie alle komplexe Zahlen $z$, f�r die die Gleichung $z^3 = -1$ gilt.
Machen Sie dazu den Ansatz $z = a + b \cdot i$ mit $a,b \in \mathbb{R}$ und bestimmen Sie $a$ und $b$.

\exercise
L�sen Sie die Rekurrenz-Gleichung \\[0.2cm]
\hspace*{1.3cm} $a_{n+2} = 3 \cdot a_{n+1} -2 \cdot a_n$ \\[0.2cm]
f�r die Anfangs-Bedingungen $a_0 = 0$ und $a_1 = 1$.

\exercise
Es sei $\mathcal{G} = \langle G, e, \circ \rangle$ eine kommutative Gruppe und es sei $a \in G$.
Beweisen oder widerlegen Sie, dass die Menge
\\[0.2cm]
\hspace*{1.3cm}
$U_a := \{ a \circ g \mid g \in G \}$
\\[0.2cm]
eine Untergruppe von $\mathcal{G}$ ist!

\exercise
F�r welche Werte von $p$ und $q$ hat die Gleichung
\\[0.2cm]
\hspace*{1.3cm}
$x^3 - p \cdot x - q = 0$
\\[0.2cm]
genau eine reelle L�sung?

\pagebreak
\exercise 
Es sei $G := \mathbb{Q}\backslash \{1\}$.  Wir definieren auf $G$ eine Operation
\\[0.2cm]
\hspace*{1.3cm} $\circ: G \times G \rightarrow G$ 
\\[0.2cm]
durch die Festlegung
\\[0.2cm]
\hspace*{1.3cm}
$a \circ b := a + b - a \cdot b$.
\\[0.2cm]
Zeigen Sie, dass die Struktur $\langle G, 0, \circ \rangle$ eine Gruppe ist.



\exercise Es sei $\langle G, e, \cdot \rangle$ eine kommutative Gruppe, f�r welche die Menge $G$
endlich ist.  Weiter sei $n := \textsl{card}(G)$, $G$ hat also die Form
\\[0.2cm]
\hspace*{1.3cm} $G = \{ a_1, a_2, \cdots, a_n \}$
\\[0.2cm]
Zeigen Sie, dass
\\[0.2cm]
\hspace*{1.3cm} $g^n = e$ \quad f�r alle $g \in G$ gilt.
\\[0.2cm]
\textbf{Hinweis 1}: Betrachten Sie die beiden Produkte
\\[0.2cm]
\hspace*{1.3cm} 
$p_1 := a_1 \cdot a_2 \cdot {\dots} \cdot a_n$ \quad und \quad
$p_2 := (a_1 \cdot g) \cdot (a_2 \cdot g) \cdot {\dots} \cdot (a_n \cdot g)$ 
\\[0.2cm]
und zeigen Sie, dass die beiden Produkte gleich sind.  �berlegen Sie sich dazu, wie die
Menge der Faktoren des Produkts $p_2$ aus der  Menge der Faktoren des Produkts $p_1$
hervorgeht.
\vspace{0.2cm}

\noindent
\textbf{Hinweis 2}: In der Vorlesung wurde gezeigt, dass f�r eine endliche Menge $M$ jede
injektive Funktion $f:M \rightarrow M$ auch surjektiv ist.


\exercise
Zeigen Sie mit vollst�ndiger Induktion:
\\[0.2cm]
\hspace*{1.3cm}
$\forall n \in \mathbb{N}: 
 \Bigl( n \geq 8 \rightarrow \exists a, b \in \mathbb{N}: 3 \cdot a + 5 \cdot b = n\Bigr)
$.

\exercise
Es seien $\langle G, 1, \cdot \rangle$ und $\langle F, 0, + \rangle$ zwei kommutative Gruppen.
Weiter sei eine Abbildung
\\[0.2cm]
\hspace*{1.3cm}
$\varphi : G \rightarrow F$ 
\\[0.2cm]
gegeben, die folgende Eigenschaften hat:
\\[0.2cm]
\hspace*{1.3cm}
$\varphi(1) = 0$, \quad $\varphi(a^{-1}) = -\varphi(a)$, 
\quad und \quad $\varphi(a \cdot b) := \varphi(a) + \varphi(b)$. 
\\[0.2cm]
Weiter sei $U$ eine Untergruppe von $F$.
Wir definieren auf $G$ eine Relation $=_\varphi$ durch die Festlegung
\\[0.2cm]
\hspace*{1.3cm}
$x =_\varphi y$ \quad g.d.w. \quad $\varphi(x) - \varphi(y) \in U$.
\\[0.2cm]
Zeigen Sie, dass die Relation $=_\varphi$ eine �quivalenz-Relation auf $G$ ist, die mit
der auf $G$ definierten Operation $\cdot$ vertr�glich ist.

\exercise
Bestimmen Sie alle Eigenwerte und Eigenvektoren der Matrix
\\[0.2cm]
\hspace*{1.3cm}
$A = \left(
  \begin{array}{lll}
    1 & 1 & 1 \\
    1 & 1 & 0 \\
    1 & 0 & 1
  \end{array}
\right)
$.
\\[0.2cm]
\end{document}

%%% Local Variables: 
%%% mode: latex
%%% TeX-master: t
%%% End: 
