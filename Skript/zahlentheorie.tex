\chapter{Zahlentheorie*}
In diesem Kapitel besch\"{a}ftigen wir uns mit den ganzen Zahlen.  Am Ende des Kapitels werden
wir ausreichend Theorie entwickelt haben, um die Funktionsweise des
RSA-Verschl\"{u}sselungs-Algorithmus verstehen zu k\"{o}nnen.  
Wir beginnen unsere \"{U}berlegungen damit, dass wir den Begriff der Teilbarkeit von Zahlen
analysieren und uns ein wenig mit modularer Arithmetik besch\"{a}ftigen.

\section{Teilbarkeit und modulare Arithmetik}

\begin{Definition}[Teiler]
  Es seien $a,b \in \mathbb{N}_0$.  Dann ist $a$ ein {\emph{\color{blue}Teiler}} von $b$, wenn es ein
  $c \in \mathbb{N}_0$ gibt, so dass $a \cdot c = b$ gilt.  In diesem Fall schreiben wir $a \teilt b$.
  Formal k\"{o}nnen wir die Teilbarkeitsrelation also wie folgt definieren:
  \\[0.2cm]
  \hspace*{1.3cm}
  $a \teilt b \df \exists c \in \mathbb{N}_0: b = a \cdot c$.
\end{Definition}

\remark
Offenbar gilt $1 \teilt n$ f\"{u}r alle $n \in \mathbb{N}_0$, denn f\"{u}r alle $n \in \mathbb{N}_0$ gilt 
$n = 1 \cdot n$.   Aus der Gleichung $0 = n \cdot 0$ folgt analog, dass $n \teilt 0$ f\"{u}r alle 
$n \in \mathbb{N}_0$ gilt.  Schlie\3lich zeigt die Gleichung $n = n \cdot 1$, dass $n \teilt n$
f\"{u}r alle  $n \in \mathbb{N}_0$ gilt.
\vspace*{0.3cm}

F\"{u}r eine nat\"{u}rliche Zahl $a$ bezeichnen wir die Menge aller Teiler von $a$ mit
$\texttt{teiler}(a)$.  Es gilt also
\[ \texttt{teiler}(a) = \{ q \in \mathbb{N}_0 \mid \exists k \in \mathbb{N}_0: k \cdot q = a \}. \]
Die Menge aller {\emph{\color{blue}gemeinsamen Teiler}} zweier nat\"{u}rlicher Zahlen
$a$ und $b$ bezeichnen wir mit $\texttt{gt(a,b)}$.  Es gilt
\[ \texttt{gt}(a,b) = \texttt{teiler}(a) \cap \texttt{teiler}(b). \]
Der {\emph{\color{blue}gr\"{o}\3te gemeinsame Teiler}} von $a$ und $b$ ist als das Maximum dieser Menge definiert,
es gilt also 
\[ \texttt{ggt}(a,b) = \max\bigl(\texttt{gt}(a,b)\bigr). \]

\begin{Satz}[Division mit Rest]
  \label{satz:division_mit_rest}
  Sind $a$ und $b$ nat\"{u}rliche Zahlen, so gibt es eindeutig bestimmte ganze Zahlen 
  $q$ und $r$, so dass 
  \\[0.2cm]
  \hspace*{1.3cm}
  $a = q \cdot b + r$ \quad mit $r < b$
  \\[0.2cm]
  gilt.  Wir nennen dann $q$ den {\emph{\color{blue}Ganzzahl-Quotienten}} und $r$ den {\emph{\color{blue}Rest}} der ganzzahligen
  Division von $a$ durch $b$.  Als Operator f\"{u}r die Ganzzahl-Division verwenden wir ``$\div$''
  und f\"{u}r die Bildung des Rests verwenden wir den Operator ``\texttt{\%}''.
  Damit gilt 
  \\[0.2cm]
  \hspace*{1.3cm}
  $q = a \div b$ \quad und \quad $r = a \modulo b$.
\end{Satz}  

\proof
Zun\"{a}chst zeigen wir die Existenz der Zahlen $q$ und $r$ mit den oben behaupteten Eigenschaften.
Dazu definieren wir eine Menge $M$ von ganzen Zahlen wie folgt:
\\[0.2cm]
\hspace*{1.3cm}
$M := \{ p \in \mathbb{N}_0 \mid p \cdot b \leq a \}$.
\\[0.2cm]
Diese Menge ist nicht leer, es gilt $0 \in M$, da $0 \cdot b \leq a$ ist.
Au\3erdem k\"{o}nnen wir sehen, dass $(a+1) \not\in M$ ist, denn
\\[0.2cm]
\hspace*{1.3cm}
$(a+1) \cdot b = a \cdot b + b > a \cdot b \geq a$, \quad denn $b \geq 1$.
\\[0.2cm]
Damit ist auch klar, dass alle Zahlen, die gr\"{o}\3er als $a$ sind, keine Elemente
von $M$ sein k\"{o}nnen.  Insgesamt wissen wir jetzt, dass die Menge nicht leer ist und dass alle
Elemente von $M$ kleiner-gleich $a$ sind.  Folglich muss die Menge $M$ ein Maximum haben.
Wir definieren
\\[0.2cm]
\hspace*{1.3cm}
$q := \max(M)$ \quad und \quad $r := a - q \cdot b$.
\\[0.2cm]
Wegen $q \in M$ wissen wir, dass
\\[0.2cm]
\hspace*{1.3cm}
$a \geq q \cdot b$
\\[0.2cm]
gilt, woraus wir $r \geq 0$ schlie\3en k\"{o}nnen und damit ist $r \in \mathbb{N}_0$.

Da wir $q$ als Maximum der Menge $M$ definiert haben, wissen wir weiter, dass
die Zahl $q + 1$ kein Element der Menge $M$ sein kann, denn sonst w\"{a}re $q$ nicht das Maximum.
Also muss
\\[0.2cm]
\hspace*{1.3cm}
$(q+1) \cdot b > a$
\\[0.2cm]
gelten.  Wir formen diese Ungleichung wie folgt um:
\\[0.2cm]
\hspace*{1.3cm}
$
\begin{array}[t]{lrcll}
                & (q + 1) \cdot b & > & a                  \\[0.1cm]   
\Leftrightarrow & q \cdot b + b   & > & a                  \\[0.1cm]   
\Leftrightarrow & b               & > & a - q \cdot b      \\[0.1cm]   
\Leftrightarrow & b               & > & r & \mbox{nach Definition von $r$}. \\[0.1cm]   
\end{array}
$
\\[0.2cm]
Also haben wir jetzt die zweite Behauptung $r < b$ gezeigt.
Aus der Definition von $r$ als $r = a - q \cdot b$ folgt sofort, dass
\\[0.2cm]
\hspace*{1.3cm}
$a = q \cdot b + r$ 
\\[0.2cm]
gilt.  Damit haben $q$ und $r$ die behaupteten Eigenschaften.
\vspace*{0.2cm}

Als n\"{a}chstes zeigen wir, dass $q$ und $r$ eindeutig bestimmt sind.  Dazu nehmen wir an,
dass zu den gegebenen Werten von $a$ und $b$ vier Zahlen $q_1$, $q_2$, $r_1$ und $r_2$ mit den
Eigenschaften 
\\[0.2cm]
\hspace*{1.3cm}
$a = q_1 \cdot b + r_1$, \quad
$a = q_2 \cdot b + r_2$, \quad 
$r_1 < b$, \quad und \quad $r_2 < b$
\\[0.2cm]
existieren.  Wir m\"{u}ssen zeigen, dass dann $q_1 = q_2$ und $r_1 = r_2$ folgt.  Aus den beiden
Gleichungen folgt zun\"{a}chst
\\[0.2cm]
\hspace*{1.3cm}
$q_1 \cdot b + r_1 = q_2 \cdot b + r_2$.
\\[0.2cm]
Diese Gleichung k\"{o}nnen wir zu
\begin{equation}
  \label{eq:division1}
  (q_1 - q_2) \cdot b = r_2 - r_1  
\end{equation}
umstellen.  Wir k\"{o}nnen  ohne Beschr\"{a}nkung der Allgemeinheit annehmen, dass $r_2 \geq r_1$
ist, denn andernfalls k\"{o}nnen wir die Zahlen $r_1$ und $q_1$ mit den Zahlen $r_2$ und $q_2$
vertauschen.  Dann zeigt Gleichung (\ref{eq:division1}), dass $b$ ein Teiler von $r_2 - r_1$ ist.
Wegen $b > r_2 \geq r_1$ wissen wir, dass
\\[0.2cm]
\hspace*{1.3cm}
$0 \leq r_2 - r_1 < b$
\\[0.2cm]
gilt.  Soll nun $b$ ein Teiler von $r_2 - r_1$ sein, so muss $r_2 - r_1 = 0$ also $r_2 = r_1$
gelten.  Daraus folgt dann
\\
\hspace*{1.3cm}
$(q_1 - q_2) \cdot b = 0$  
\\[0.2cm]
und wegen $b \not= 0$ muss auch $q_1 = q_2$ gelten. 
\qed

\exercise
Wie m\"{u}ssen wir den obigen Satz \"{a}ndern, damit er auch dann noch gilt, wenn  $a \in \mathbb{Z}$ ist?
Formulieren Sie die ge\"{a}nderte Version des Satzes und beweisen Sie Ihre Version des Satzes!

\remark
In vielen Programmier-Sprachen ist die ganzzahlige Division so implementiert, dass die Gleichung 
\\[0.2cm]
\hspace*{1.3cm}
$(a \div q) \cdot q + a \modulo q = a$
\\[0.2cm]
f\"{u}r $a < 0$ im Allgemeinen \underline{nicht} gilt! \eox



Mit dem letzten Satz k\"{o}nnen wir die Menge der Teiler einer nat\"{u}rlichen Zahl $a$ auch wie folgt definieren:
\\[0.2cm]
\hspace*{1.3cm}
$\texttt{teiler}(a) = \{ q \in \mathbb{N}_0 \mid a \modulo q = 0 \}$.
\vspace*{0.3cm}


\noindent
Wir erinnern an dieser Stelle an die Definition der \"{A}quivalenz-Relationen $\approx_n$, die wir f\"{u}r 
$n \in \mathbb{N}$ im Abschnitt \ref{section:aequivalenz_relation} durch die Formel
\\[0.2cm]
\hspace*{1.3cm}
 $\approx_n \;:=\; \{ \pair(x,y) \in \mathbb{Z}^2 \mid \exists k \in \mathbb{Z} \colon k \cdot n = x - y \}$
\\[0.2cm]
definiert hatten.  Der n\"{a}chste Satz zeigt, dass sich diese Relation auch etwas anders charakterisieren
l\"{a}sst. 

\begin{Satz}
F\"{u}r $a,b \in \mathbb{N}$ und $n \in \mathbb{N}$ mit $n > 0$ gilt
\\[0.2cm]
\hspace*{1.3cm}
$a \approx_n b \;\leftrightarrow\; a \modulo n = b \modulo n$.
\end{Satz}
  
\proof
Wir zerlegen den Beweis in zwei Teile:
\begin{enumerate}
\item ``$\Rightarrow$'':
      Aus $a \approx_n b$ folgt nach Definition der Relation $\approx_n$, dass es ein $h \in \mathbb{Z}$ gibt mit
      \\[0.2cm]
      \hspace*{1.3cm}
      $a - b = h \cdot n$.
      \\[0.2cm]
      Definieren wir $l := b \div n$, so ist $l \in \mathbb{N}$ und es gilt
      \\[0.2cm]
      \hspace*{1.3cm}
      $b = l \cdot n + b \modulo n$.
      \\[0.2cm]
      Setzen wir dies in die Gleichung f\"{u}r $a - b$ ein, so erhalten wir
      \\[0.2cm]
      \hspace*{1.3cm}
      $a - (l \cdot n + b \modulo n) = h \cdot n$,
      \\[0.2cm]
      was wir zu
      \\[0.2cm]
      \hspace*{1.3cm}
      $a = (h + l) \cdot l + b \modulo n$,
      \\[0.2cm]
      umstellen k\"{o}nnen.  Aus dieser Gleichung folgt wegen der im Satz von der Division mit Rest 
      (Satz \ref{satz:division_mit_rest})
      gemachten Eindeutigkeits-Aussage, dass
      \\[0.2cm]
      \hspace*{1.3cm}
      $a \modulo n = b \modulo n$
      \\[0.2cm]
      gilt. \checkmark
\item ``$\Leftarrow$'':
      Es sei nun 
      \\[0.2cm]
      \hspace*{1.3cm}
      $a \modulo n = b \modulo n$ 
      \\[0.2cm]
      vorausgesetzt. Nach Definition des Modulo-Operators gibt es nat\"{u}rliche Zahlen
      $k,l \in \mathbb{N}$, so dass 
      \\[0.2cm]
      \hspace*{1.3cm}
      $a \modulo n = a - k \cdot n$ \quad und \quad
      $b \modulo n = b - l \cdot n$ 
      \\[0.2cm]
      gilt, so dass wir insgesamt
      \\[0.2cm]
      \hspace*{1.3cm}
      $a - k \cdot n = b - l \cdot n$
      \\[0.2cm]
      haben.   Daraus folgt
      \\[0.2cm]
      \hspace*{1.3cm}
      $a - b = (k - l) \cdot n$,
      \\[0.2cm]
      so dass $n$ ein Teiler von $(a- b)$ ist und das hei\3t $a \approx_n b$. \checkmark \qed
\end{enumerate}

\begin{Satz}
  Die Relation $\approx_n$ ist eine Kongruenz-Relation.
\end{Satz}

\proof
Es gilt 
\\[0.2cm]
\hspace*{1.3cm}
$
\begin{array}[t]{cl}
                & x \approx_n y                                \\[0.2cm]
\Leftrightarrow & \exists k \in \mathbb{Z}: x - y = k \cdot n  \\[0.2cm]
\Leftrightarrow & x - y \in n\mathbb{Z}                        \\[0.2cm]
\Leftrightarrow & x \sim_{n\mathbb{Z}} y 
\end{array}
$
\\[0.2cm]
Damit sehen wir, dass die Relation $\approx_n$ mit der von dem Ideal $n\mathbb{Z}$
erzeugten Kongruenz-Relation $\sim_{n\mathbb{Z}}$ \"{u}bereinstimmt und folglich
eine Kongruenz-Relation ist.  \qed

%In dem Abschnitt \"{u}ber \"{A}quivalenz-Relationen (Abschnitt \ref{section:aequivalenz_relation})
%hatten wir bereits gezeigt, dass die Relation $\approx_n$ eine \"{A}quivalenz-Relation ist.
%Es bleibt daher nur zu zeigen, dass die Relation $\approx_n$ mit der Addition und der
%Multiplikation vertr\"{a}glich ist.
%\begin{enumerate}
%\item Wir zeigen als erstes, dass die Relation $\approx_n$ mit der Addition vertr\"{a}glich
%      ist.  Dazu ist zu zeigen, dass f\"{u}r alle $a_1, a_2, b_1, b_2 \in \mathbb{Z}$
%      \\[0.2cm]
%      \hspace*{1.3cm} 
%      $a_1 \approx_n b_1 \wedge a_2 \approx_n b_2 \rightarrow a_1 + a_2 \approx_n b_1 + b_2$.
%      \\[0.2cm]
%      gilt.  Wir nehmen also an, dass $a_1 \approx_n a_2$ und $b_1 \approx_n b_2$ gilt und
%      m\"{u}ssen zeigen, dass dann auch $a_1 + a_2 \approx_n b_1 + b_2$ richtig ist.  Nach
%      Definition der Relation $\approx_n$ folgt aus den Voraussetzungen $a_1 \approx_n a_2$ und
%      $b_1 \approx_n b_2$, dass es Zahlen $k,l \in \mathbb{Z}$ gibt, so dass
%      \\[0.2cm]
%      \hspace*{1.3cm} 
%      $a_1 - a_2 = k \cdot n$ \quad und \quad $b_1 - b_2 = l \cdot n$ 
%      \\[0.2cm]
%      gilt.  Addieren wir diese beiden Gleichungen, so erhalten wir
%      \\[0.2cm]
%      \hspace*{1.3cm}
%      $(a_1 - a_2) + (b_1 - b_2) = k \cdot n + l \cdot n$.
%      \\[0.2cm]
%      Diese Gleichung k\"{o}nnen wir zu
%      \\[0.2cm]
%      \hspace*{1.3cm}
%      $(a_1 + b_1) - (a_2 + b_2) = (k + l) \cdot n$
%      \\[0.2cm]
%      umstellen und nach Definition der Relation $\approx_n$ haben wir damit
%      \\[0.2cm]
%      \hspace*{1.3cm}
%      $a_1 + b_1 \approx_n a_2 + b_2$
%      \\[0.2cm]
%      gezeigt. \checkmark
%\item Wir zeigen nun, dass die Relation $\approx_n$ mit der Multiplikation vertr\"{a}glich
%      ist.  Dazu ist zu zeigen, dass f\"{u}r alle $a_1, a_2, b_1, b_2 \in \mathbb{Z}$
%      \\[0.2cm]
%      \hspace*{1.3cm} 
%      $a_1 \approx_n b_1 \wedge a_2 \approx_n b_2 \rightarrow a_1 \cdot a_2 \approx_n b_1 \cdot b_2$.
%      \\[0.2cm]
%      gilt.  Wir nehmen also an, dass $a_1 \approx_n a_2$ und $b_1 \approx_n b_2$ gilt und
%      m\"{u}ssen zeigen, dass dann auch $a_1 \cdot a_2 \approx_n b_1 \cdot b_2$ richtig ist.  Nach
%      Definition der Relation $\approx_n$ folgt aus den Voraussetzungen $a_1 \approx_n a_2$ und
%      $b_1 \approx_n b_2$, dass es Zahlen $k,l \in \mathbb{Z}$ gibt, so dass
%      \\[0.2cm]
%      \hspace*{1.3cm} 
%      $a_1 - a_2 = k \cdot n$ \quad und \quad $b_1 - b_2 = l \cdot n$ 
%      \\[0.2cm]
%      gilt.  Nun rechnen wir wie folgt
%      \\[0.2cm]
%      \hspace*{1.3cm}
%      $
%      \begin{array}[t]{cl}
%        & a_1 \cdot b_1 - a_2 \cdot b_2                                 \\[0.2cm]
%      = & a_1 \cdot b_1 - a_1 \cdot b_2 + a_1 \cdot b_2 - a_2 \cdot b_2 \\[0.2cm]
%      = & a_1 \cdot (b_1 - b_2) + (a_1 - a_2) \cdot b_2                 \\[0.2cm]
%      = & a_1 \cdot l \cdot n + k \cdot n \cdot b_2                     \\[0.2cm]
%      = & (a_1 \cdot l + k \cdot b_2) \cdot n.
%      \end{array}
%      $
%      \\[0.2cm]
%      Folglich ist die Differenz $a_1 \cdot b_1 - a_2 \cdot b_2$ ein Vielfaches von $n$ und
%      nach Definition der Relation $\approx_n$ haben wir damit
%      \\[0.2cm]
%      \hspace*{1.3cm}
%      $a_1 \cdot b_1 \approx_n a_2 \cdot b_2$
%      \\[0.2cm]
%      gezeigt. \checkmark \qed
%\end{enumerate}


Wir erinnern an dieser Stelle daran, dass wir im letzten Kapitel f\"{u}r nat\"{u}rliche Zahlen $k$
die Menge $k\mathbb{Z}$ aller Vielfachen von $k$ als
\\[0.2cm]
\hspace*{1.3cm}
$k\mathbb{Z} = \{ k \cdot z \mid z \in \mathbb{Z} \}$
\\[0.2cm]
definiert haben.  Au\3erdem hatten wir gezeigt, dass diese Mengen Ideale sind.
Der n\"{a}chste Satz zeigt, dass alle Ideale in dem Ring $\mathbb{Z}$ der ganzen Zahlen
diese Form haben.

\begin{Satz}[$\mathbb{Z}$ ist ein Haupt-Ideal-Ring] \lb
Ist $I \subseteq \mathbb{Z}$ ein Ideal, so gibt es eine nat\"{u}rliche Zahl $k$, so dass $I = k\mathbb{Z}$
gilt.
\end{Satz}

\proof
Wir betrachten zwei F\"{a}lle:  Entweder ist $I = \{0\}$ oder nicht.
\begin{enumerate}
\item Fall: $I = \{ 0 \}$.

      Wegen $\{ 0 \} = 0\mathbb{Z}$ ist die Behauptung in diesem Fall offensichtlich
      wahr.
\item Fall: $I \not= \{ 0 \}$.

      Dann gibt es ein $l \in I$ mit $l \not= 0$.  Da $I$ ein Ideal ist, liegt mit $l$ auch $-l$ in dem
      Ideal $I$.  Eine dieser beiden Zahlen ist positiv.  Daher ist die Menge
      \\[0.2cm]
      \hspace*{1.3cm}
      $M := \{ x \in I \mid x > 0 \}$
      \\[0.2cm]
      nicht leer und hat folglich ein Minimum $k = \min(M)$, f\"{u}r welches offenbar
      \\[0.2cm]
      \hspace*{1.3cm}
      $k \in I$ \quad und \quad $k > 0$
      \\[0.2cm]
      gilt.  Wir behaupten, dass
      \\[0.2cm]
      \hspace*{1.3cm}
      $I = k\mathbb{Z}$
      \\[0.2cm]
      gilt.  Sei also $y \in I$.  Wir teilen $y$ durch $k$ und  nach dem Satz
      \"{u}ber ganzzahlige Division mit Rest finden wir dann Zahlen $q \in \mathbb{Z}$ und $r \in \mathbb{N}$ mit
      \\[0.2cm]
      \hspace*{1.3cm}
      $y = q \cdot k + r$ \quad und \quad $0 \leq r < k$.
      \\[0.2cm]
      Aus der ersten Gleichung folgt
      \\[0.2cm]
      \hspace*{1.3cm}
      $r = y + (-q) \cdot k$.
      \\[0.2cm]
      Da nun sowohl $y \in I$ als auch $k \in I$ gilt und Ideale sowohl unter Multiplikation mit
      beliebigen Ring-Elementen als auch unter Addition abgeschlossen sind, folgt
      \\[0.2cm]
      \hspace*{1.3cm}
      $r \in I$.
      \\[0.2cm]
      Nun ist einerseits $k = \min\bigl(\{x \in I \mid x > 0 \}\bigr)$, andererseits ist $r < k$.
      Das geht beides zusammen nur, wenn
      \\[0.2cm]
      \hspace*{1.3cm}
      $r = 0$
      \\[0.2cm]
      ist.  Damit haben wir dann aber
      \\[0.2cm]
      \hspace*{1.3cm}
      $y = q \cdot k$
      \\[0.2cm]
      gezeigt, woraus sofort
      \\[0.2cm]
      \hspace*{1.3cm}
      $y \in k\mathbb{Z}$
      \\[0.2cm]
      folgt.  Da $y$ bei diesen Betrachtungen ein beliebiges Element der Menge $I$ war,
      zeigt diese \"{U}berlegung insgesamt, dass $I \subseteq k\mathbb{Z}$ gilt.  Aus der
      Tatsache, dass $k \in I$ ist, folgt andererseits, dass $k\mathbb{Z} \subseteq I$
      gilt, so dass wir insgesamt
      \\[0.2cm]
      \hspace*{1.3cm}
      $I = k\mathbb{Z}$
      \\[0.2cm]
      gezeigt haben.  \qed
\end{enumerate}

\remark
Wir erinnern an dieser Stelle daran, dass wir f\"{u}r einen Ring 
$\mathcal{R} = \langle R, 0, 1, +, \cdot \rangle$ und ein
Ring-Element $k \in R$ die Ideale der Form
\\[0.2cm]
\hspace*{1.3cm}
$\textsl{gen}(k) = \{ k \cdot x \mid x \in R \}$
\\[0.2cm]
als \emph{Haupt-Ideale} bezeichnet haben.  Der letzte Satz zeigt also, dass alle Ideale des Rings
der ganzen Zahlen Haupt-Ideale sind.  Einen Ring mit der Eigenschaft, dass alle Ideale
bereits Haupt-Ideale sind, bezeichnen wir als {\emph{\color{blue}Haupt-Ideal-Ring}}.  Der letzte Satz
zeigt daher, dass der Ring der ganzen Zahlen ein Haupt-Ideal-Ring ist. \eoxs

\begin{Lemma}
  F\"{u}r $u,v \in \mathbb{N}$ gilt
  \\[0.2cm]
  \hspace*{1.3cm}
  $u\mathbb{Z} \subseteq v\mathbb{Z} \;\Leftrightarrow\; v \teilt u$.
\end{Lemma}

\proof
Wir zerlegen den Beweis der \"{A}quivalenz der beiden Aussagen in den Beweis der beiden Implikationen.
\begin{enumerate}
\item ``$\Rightarrow$'': Wegen $u = u \cdot 1$ gilt  
      \\[0.2cm]
      \hspace*{1.3cm}
      $u \in u\mathbb{Z}$
      \\[0.2cm]
      und aus der Voraussetzung $u\mathbb{Z} \subseteq v\mathbb{Z}$ folgt dann
      \\[0.2cm]
      \hspace*{1.3cm}
      $u \in v\mathbb{Z}$.
      \\[0.2cm]
      Nach Definition der Menge $v\mathbb{Z}$ gibt es nun ein $k \in \mathbb{Z}$, so dass
      \\[0.2cm]
      \hspace*{1.3cm}
      $u = v \cdot k$
      \\[0.2cm]
      gilt.  Nach der Definition der Teilbarkeit haben wir also
      \\[0.2cm]
      \hspace*{1.3cm}
      $v \teilt u$.
\item ``$\Leftarrow$'':  Es sei jetzt $v \teilt u$ vorausgesetzt.  Dann gibt es ein $k \in \mathbb{N}$
      mit 
      \\[0.2cm]
      \hspace*{1.3cm}
      $u = v \cdot k$.
      \\[0.2cm]
      Sei weiter $a \in u\mathbb{Z}$ beliebig.  Nach Definition der Menge $u\mathbb{Z}$ gibt es also
      ein $x \in \mathbb{Z}$ mit
      \\[0.2cm]
      \hspace*{1.3cm}
      $a = u \cdot x$.
      \\[0.2cm]
      Ersetzen wir in dieser Gleichung $u$ durch $v \cdot k$, so erhalten wir
      \\[0.2cm]
      \hspace*{1.3cm}
      $a = v \cdot (k \cdot x)$
      \\[0.2cm]
      und daraus folgt sofort
      \\[0.2cm]
      \hspace*{1.3cm}
      $a \in v\mathbb{Z}$,
      \\[0.2cm]
      so dass wir insgesamt $u\mathbb{Z} \subseteq v\mathbb{Z}$ gezeigt haben.
\end{enumerate}

\begin{Satz}
  Es sei $p$ eine Primzahl.  Dann ist das Ideal $p\mathbb{Z}$ ein maximales Ideal.
\end{Satz}

\proof
Es sei $J \subseteq \mathbb{Z}$ ein Ideal, f\"{u}r das
\\[0.2cm]
\hspace*{1.3cm}
$p\mathbb{Z} \subseteq J$ 
\\[0.2cm]
gilt.  Wir m\"{u}ssen zeigen, dass dann $J = p\mathbb{Z}$ oder $J = \mathbb{Z}$ gilt.  Da $\mathbb{Z}$ ein
Haupt-Ideal-Ring ist, gibt es ein $q \in \mathbb{Z}$ mit
\\[0.2cm]
\hspace*{1.3cm}
$J = q\mathbb{Z}$.
\\[0.2cm]
Damit haben wir 
\\[0.2cm]
\hspace*{1.3cm}
$p\mathbb{Z} \subseteq q\mathbb{Z}$
\\[0.2cm]
und nach dem letzten Lemma folgt daraus
\\[0.2cm]
\hspace*{1.3cm}
$q \teilt p$.
\\[0.2cm]
Da $p$ eine Primzahl ist, gibt es nur zwei Zahlen, die Teiler von $p$ sind: Die Zahl $1$ und die Zahl $p$.
Wir haben also
\\[0.2cm]
\hspace*{1.3cm}
$q = 1$ \quad oder \quad $q = p$,
\\[0.2cm]
woraus
\\[0.2cm]
\hspace*{1.3cm}
$J = 1\mathbb{Z} = \mathbb{Z}$ \quad oder \quad $J = p\mathbb{Z}$
\\[0.2cm]
folgt und damit ist das Ideal $p\mathbb{Z}$ ein maximales Ideal.  \qed

\begin{Korollar}
Falls $p$ eine Primzahl ist, dann ist $\mathbb{Z}_p := \mathbb{Z}/p\mathbb{Z}$ ein K\"{o}rper.
\end{Korollar}

\proof
Im letzten Kapitel haben wir gezeigt, dass f\"{u}r einen Ring $R$ und ein maximales Ideal
$I \subseteq R$ der Faktor-Ring $R/I$ ein K\"{o}rper ist.  Wir haben gerade gesehen, dass f\"{u}r eine Primzahl
$p$ das Ideal $p\mathbb{Z}$ maximal ist.  Diese beiden Tatsachen ergeben zusammen die Behauptung. \qed

\remark
Bisher hatten alle K\"{o}rper, die wir kennengelernt haben, unendlich viele Elemente.
Der letzte Satz zeigt uns, dass es auch endliche K\"{o}rper gibt, denn die Menge $\mathbb{Z}_p$ hat
die Form
\\[0.2cm]
\hspace*{1.3cm}
$\mathbb{Z}_p = \mathbb{Z} / p\mathbb{Z} = \bigl\{0 + \mathbb{Z}, 1 + \mathbb{Z}, \cdots, (p-1) + \mathbb{Z} \bigr\}$
\\[0.2cm]
und ist offenbar endlich. \eoxs


\begin{Satz}[Lemma von B\'{e}zout, (\'{E}tienne B\'{e}zout, 1730--1783)]
\label{satz:bezout}
  Es seien $a,b \in \mathbb{N}_0$.  Dann existieren $x,y \in \mathbb{Z}$, so dass
  \\[0.2cm]
  \hspace*{1.3cm}
  $\textsl{ggt}(a,b) = x \cdot a + y \cdot b$
  \\[0.2cm]
  gilt.  Der gr\"{o}\3te gemeinsame Teiler zweier nat\"{u}rlicher Zahlen $a$ und $b$ 
  l\"{a}sst sich also immer als ganzzahlige Linear-Kombination von $a$ und $b$ schreiben.
\end{Satz}

\proof
Wir definieren die Menge $I$ wie folgt:
\\[0.2cm]
\hspace*{1.3cm}
$I := \bigl\{ x \cdot a + y \cdot b \mid x,y \in \mathbb{Z} \bigr\}$.
\\[0.2cm]
Wir zeigen, dass $I$ ein Ideal ist:
\begin{enumerate}
\item $0 = 0 \cdot a + 0 \cdot b \in I$.
\item $I$ ist abgeschlossen unter Bildung des additiven Inversen:  Sei
      $u = x \cdot a + y \cdot b \in I$.  Dann folgt sofort
      \\[0.2cm]
      \hspace*{1.3cm}
      $-u = (-x) \cdot a + (-y) \cdot b \in I$.
\item $I$ ist abgeschlossen unter Addition: Seien  
      $u = x_1 \cdot a + y_1 \cdot b \in I$ und $v = x_2 \cdot a + y_2 \cdot b \in I$.
      dann folgt
      \\[0.2cm]
      \hspace*{1.3cm}
      $u + v = (x_1 + x_2) \cdot a + (y_1 + y_2) \cdot b \in I$.
\item $I$ ist abgeschlossen unter Multiplikation mit beliebigen ganzen Zahlen.  
      Sei $u = x \cdot a + y \cdot b \in I$ und $z \in \mathbb{Z}$.  Dann gilt
      \\[0.2cm]
      \hspace*{1.3cm}
      $z \cdot u = (z \cdot x) \cdot a + (z \cdot y) \cdot b \in I$.
\end{enumerate}
Da der Ring der ganzen Zahlen ein Haupt-Ideal-Ring ist, gibt es also eine Zahl $d \in \mathbb{N}$, 
so dass
\\[0.2cm]
\hspace*{1.3cm}
$I = d\mathbb{Z}$
\\[0.2cm]
gilt.  Setzen wir in der Definition von $I$ wahlweise $y = 0$ oder $x = 0$ ein, so sehen wir, dass
\\[0.2cm]
\hspace*{1.3cm}
$a\mathbb{Z} \subseteq I$ \quad und \quad $b\mathbb{Z} \subseteq I$
\\[0.2cm]
gilt, woraus nun
\\[0.2cm]
\hspace*{1.3cm}
$a\mathbb{Z} \subseteq d\mathbb{Z}$ \quad und \quad $b\mathbb{Z} \subseteq d\mathbb{Z}$,
\\[0.2cm]
folgt.  Nach dem Lemma, das wir gerade bewiesen haben, folgt daraus
\\[0.2cm]
\hspace*{1.3cm}
$d \teilt a$ \quad und \quad $d \teilt b$.
\\[0.2cm]
Damit ist $d$ ein gemeinsamer Teiler von $a$ und $b$.  Wir zeigen, dass $d$ sogar der gr\"{o}\3te gemeinsame
Teiler von $a$ und $b$ ist.  Dazu betrachten wir einen beliebigen anderen gemeinsamen Teiler $e$ von $a$ und
$b$: 
\\[0.2cm]
\hspace*{1.3cm}
$e \teilt a$ \quad und \quad $e \teilt b$.
\\[0.2cm]
Nach dem letzten Lemma folgt daraus
\\[0.2cm]
\hspace*{1.3cm}
$a\mathbb{Z} \subseteq e\mathbb{Z}$ \quad und \quad $b\mathbb{Z} \subseteq e\mathbb{Z}$.
\\[0.2cm]
Da wir das Ideal $I$ als $\bigl\{ x \cdot a + y \cdot b \mid x,y \in \mathbb{Z} \bigr\}$ definiert hatten,
k\"{o}nnen wir nun sehen, dass 
\\[0.2cm]
\hspace*{1.3cm}
$I \subseteq e\mathbb{Z}$
\\[0.2cm]
gilt, denn f\"{u}r $x,y \in \mathbb{Z}$ haben wir einerseits $x \cdot a \in a\mathbb{Z} \subseteq e\mathbb{Z}$
und andererseits $y \cdot b \in b\mathbb{Z} \subseteq e\mathbb{Z}$, so dass aufgrund der Abgeschlossenheit
des Ideals $e\mathbb{Z}$ unter Addition insgesamt
\\[0.2cm]
\hspace*{1.3cm}
$a \cdot x + b \cdot y \in e\mathbb{Z}$
\\[0.2cm]
gilt.  Setzen wir in der Beziehung $I \subseteq e\mathbb{Z}$ f\"{u}r $I$ den Ausdruck $d\mathbb{Z}$ ein, haben
wir also 
\\[0.2cm]
\hspace*{1.3cm}
$d\mathbb{Z} \subseteq e\mathbb{Z}$
\\[0.2cm]
gezeigt, was nach dem letzten Lemma zu
\\[0.2cm]
\hspace*{1.3cm}
$e \teilt d$
\\[0.2cm]
\"{a}quivalent ist.  Damit haben wir insgesamt gezeigt, dass
\\[0.2cm]
\hspace*{1.3cm}
 $d = \textsl{ggt}(a,b)$ 
\\[0.2cm]
gilt.  Wegen $I = d\mathbb{Z}$ und $d \in d\mathbb{Z}$ folgt also 
\\[0.2cm]
\hspace*{1.3cm}
$\textsl{ggt}(a,b) \in \bigl\{ x \cdot a + y \cdot b \mid x,y \in \mathbb{Z} \bigr\}$.
\\[0.2cm]
Damit gibt es dann $x,y \in \mathbb{Z}$, so dass
\\[0.2cm]
\hspace*{1.3cm}
$\textsl{ggt}(a,b) = x \cdot a + y \cdot b$
\\[0.2cm]
gilt.  \qed

%Wir betrachten zun\"{a}chst den Spezialfall, dass die Zahlen $a$ und $b$ teilerfremd sind und damit
%$\textsl{ggt}(a,b) = 1$ gilt.  In diesem Fall sind also ganze Zahlen $x$ und $y$ zu finden, so dass
%\begin{equation}
%\label{eq:bezout1}
%   1 = x \cdot a + y \cdot b
%\end{equation}
%gilt.  Wir definieren die Menge $M$ als die Menge aller positiven Linear-Kombination von $a$ und $b$:
%\\[0.2cm]
%\hspace*{1.3cm}
%$M := \bigl\{ x \cdot a + y \cdot b \mid x,y \in \mathbb{Z} \wedge x \cdot a + y \cdot b > 0 \bigr\}$.
%\\[0.2cm]
%Es ist offensichtlich, dass die Menge $M$ nicht leer ist, denn f\"{u}r $x = 1$ und $y = 1$ gilt sicher
%\\[0.2cm]
%\hspace*{1.3cm}
%$1 \cdot a + 1 \cdot b = a + b \geq 0$, \quad denn $a,b \in \mathbb{N}$.
%\\[0.2cm]
%Wir definieren $d$ als das Minimum von $M$:
%\\[0.2cm]
%\hspace*{1.3cm}
%$d = \min (M)$.
%\\[0.2cm]
%Dann gibt es ganze Zahlen $x_d$ und $y_d$, so dass
%\begin{equation}
%\label{eq:bezout2}
%d = x_d \cdot a + y_d \cdot b  
%\end{equation}
%gilt.  Wir zeigen nun, dass $d$ ein Teiler von $a$ ist.  Dazu dividieren wir $a$ durch $d$ und erhalten
%nat\"{u}rliche Zahlen $q$ und $r$, so dass
%\begin{equation} 
%\label{eq:bezout3}
%a = q \cdot d + r \quad \mbox{und} \quad r < d
%\end{equation}
%gilt.  Wir stellen die Gleichung f\"{u}r $a$ nach $r$ um und haben 
%\\[0.2cm]
%\hspace*{1.3cm}
%$r = 1 \cdot a + (-q) \cdot d$.
%\\[0.2cm]
%Setzen wir hier die Gleichung (\ref{eq:bezout2}) f\"{u}r $d$ ein, so erhalten wir
%\\[0.2cm]
%\hspace*{1.3cm}
%$r = 1 \cdot a + (-q) \cdot (x_d \cdot a + y_d \cdot b)$,
%\\[0.2cm]
%was wir zu
%\\[0.2cm]
%\hspace*{1.3cm}
%$r = (1 -q \cdot x_d) \cdot a + (-q) \cdot y_d \cdot b$
%\\[0.2cm]
%umformen k\"{o}nnen.  Da $r$ eine nat\"{u}rliche Zahl ist, gilt sicher $r \geq 0$.  W\"{a}re nun $r \not= 0$, so
%w\"{a}re $r$ ein Element der Menge $M$, dass allerdings wegen der Ungleichung in (\ref{eq:bezout3})
%kleiner als $d$ ist.  Auf der anderen Seite
%hatten wir $d$ aber als Minimum der Menge $M$ definiert, es kann also in $M$ keine kleinere Zahl
%geben.  Folglich muss $r= 0$ sein und aus Gleichung (\ref{eq:bezout3}) folgt nun
%\\[0.2cm]
%\hspace*{1.3cm}
%$a = q \cdot d$.
%\\[0.2cm]
%Damit ist $d$ aber ein Teiler von $a$.  Genauso wir wir gezeigt haben, dass $d$ ein Teiler von $a$
%ist, k\"{o}nnen wir auch sehen, dass $d$ ein Teiler von $b$ ist, denn die Definition von $d$ ist
%symmetrisch bez\"{u}glich $a$ und $b$.  Daraus folgt nun aber, dass $d$ ein gemeinsamer Teiler von $a$
%und $b$ ist.  Da wir $\textsl{ggt}(a,b) = 1$ vorausgesetzt haben, k\"{o}nnen wir schlie\3en, dass $d = 1$
%ist.  Damit schreibt sich Gleichung (\ref{eq:bezout2}) als
%\\[0.2cm]
%\hspace*{1.3cm}
%$1 = x_d \cdot a + y_d \cdot b$
%\\[0.2cm]
%und $x_d$ und $y_d$ sind die Zahlen, die wir gesucht haben.
%\vspace*{0.2cm}

%Um den Beweis abzuschlie\3en betrachten wir nun den Fall, in dem $\textsl{ggt}(a,b) \not=1$ ist.
%Dann k\"{o}nnen wir sowohl $a$ als auch $b$ durch ihren gemeinsamen Teiler $\textsl{ggt}(a,b)$ teilen.
%Wir definieren nun
%\\[0.2cm]
%\hspace*{1.3cm}
%$a' := \bruch{a}{\textsl{ggt}(a,b)}$  \quad und \quad
%$b' := \bruch{b}{\textsl{ggt}(a,b)}$.
%\\[0.2cm]
%Dann gilt offenbar $\textsl{ggt}(a', b') = 1$ und nach dem, was wir bereits bewiesen haben, gibt es
%dann ganze Zahlen $x$ und $y$, so dass
%\\[0.2cm]
%\hspace*{1.3cm}
%$1 = x \cdot a' + y \cdot b'$
%\\[0.2cm]
%gilt.  Setzen wir in dieser Gleichung die Definition von $a'$ und $b'$ ein, so erhalten wir
%\\[0.2cm]
%\hspace*{1.3cm}
%$1 = x \cdot \bruch{a}{\textsl{ggt}(a,b)} + y \cdot \bruch{b}{\textsl{ggt}(a,b)}$.
%\\[0.2cm]
% Multiplizieren wir diese Gleichung mit $\textsl{ggt}(a,b)$, so finden wir
%\\[0.2cm]
%\hspace*{1.3cm}
%$\textsl{ggt}(a,b) = x \cdot a + y \cdot b$.
%\\[0.2cm]
%Das ist aber genau das Ergebnis, das wir zeigen wollten.
%\qed

\remark
Der Beweis des letzten Satzes war nicht konstruktiv.  Wir werden im n\"{a}chsten Abschnitt ein
Verfahren angeben, mit dessen Hilfe wir die Zahlen $x$ und $y$, f\"{u}r 
$x \cdot a + y \cdot b = \textsl{ggt}(a, b)$ gilt, auch tats\"{a}chlich berechnen k\"{o}nnen.


\section{Der Euklidische Algorithmus}
Wir pr\"{a}sentieren nun einen Algorithmus zur Berechnung des gr\"{o}\3ten gemeinsamen 
Teilers zweier nat\"{u}rlicher Zahlen $x$ und $y$.
 Abbildung \ref{fig:ggt.stlx} auf Seite \pageref{fig:ggt.stlx} zeigt eine
\textsl{SetlX}-Funktion, die f\"{u}r gegebene positive nat\"{u}rliche Zahlen $x$ und $y$ den gr\"{o}\3ten gemeinsamen  Teiler 
$\textsl{ggt}(x,y)$ berechnet.  Diese Funktion implementiert den 
\href{https://en.wikipedia.org/wiki/Euclidean_algorithm}{\emph{Euklidischen Algorithmus}} zur
Berechnung des gr\"{o}\3ten gemeinsamen Teilers. 


\begin{figure}[!ht]
\centering
\begin{Verbatim}[ frame         = lines, 
                  framesep      = 0.3cm, 
                  labelposition = bottomline,
                  numbers       = left,
                  numbersep     = -0.2cm,
                  xleftmargin   = 0.8cm,
                  xrightmargin  = 0.8cm,
                ]
    // Precondition: x > 0 and y > 0.
    ggtS := procedure(x, y) {
        if (x < y) {
            return ggt(x, y - x);
        }
        if (y < x) {
            return ggt(x - y, y); 
        }
        // We must have x = y at this point.
        return x;
    };
\end{Verbatim}  
\vspace*{-0.3cm}
\caption{Der Euklidische Algorithmus zur Berechnung des gr\"{o}\3ten gemeinsamen Teilers.}
\label{fig:ggt.stlx}
\end{figure} %$

Um die Korrektheit des Euklidischen Algorithmus zu beweisen, ben\"{o}tigen wir das folgende Lemma.

\begin{Lemma} \label{lemma:modulo1}
Sind $x, y \in \mathbb{Z}$, so gilt f\"{u}r alle $n \in \mathbb{N}_1$
\[ x \modulo n = 0 \wedge y \modulo n = 0 \;\leftrightarrow\; 
   (x + y) \modulo n = 0 \wedge y \modulo n = 0. 
\]  
\end{Lemma}
\textbf{Beweis}:  Die Formel
\\[0.2cm]
\hspace*{1.3cm}
$p \wedge q \;\leftrightarrow\; r \wedge q$
\\[0.2cm]
ist aussagenlogisch \"{a}quivalent zu der Formel
\\[0.2cm]
\hspace*{1.3cm}
$q \rightarrow (p \leftrightarrow r)$
\\[0.2cm]
Daher reicht es, wenn wir
\\[0.2cm]
\hspace*{1.3cm}
$y \modulo n = 0 \rightarrow \bigl(x \modulo n = 0 \;\leftrightarrow\; (x + y) \modulo n = 0\bigr)$ 
\\[0.2cm]
nachweisen.  Unter Benutzung der Relation $\approx_n$ und bei weiterer Ber\"{u}cksichtigung
der Tatsache, dass 
\\[0.2cm]
\hspace*{1.3cm}
 $a \approx_n b \;\leftrightarrow\; a \modulo n = b \modulo n$ 
\\[0.2cm]
gilt, k\"{o}nnen wir diese Formel auch als
\\[0.2cm]
\hspace*{1.3cm}
$y \approx_n 0 \rightarrow \bigl(x \approx_n 0 \;\leftrightarrow\; (x + y) \approx_n 0\bigr)$ 
\\[0.2cm]
schreiben.  Diese Formel folgt aber aus der schon fr\"{u}her bewiesenen Tatsache, dass
$\approx_n$ eine Kongruenz-Relation ist.  F\"{u}r die Richtung ``$\rightarrow$'' ist das
unmittelbar klar und f\"{u}r die Richtung ``$\leftarrow$'' ist nur zu bemerken, dass aus
\\[0.2cm]
\hspace*{1.3cm}
$(x + y) \approx_n 0$ \quad und \quad $y \approx_n 0$, 
\\[0.2cm]
aus der Vertr\"{a}glichkeit der Relation $\approx_n$ mit der Addition selbstverst\"{a}ndlich auch
die Vertr\"{a}glichkeit mit der Subtraktion folgt, so dass
\\[0.2cm]
\hspace*{1.3cm}
$x = (x + y) - y \approx_n 0 - 0 = 0$, \quad also $x \approx_n 0$
\\[0.2cm]
folgt. \qed


\begin{Korollar} \label{korollar:ggt}
Sind $x$ und $y$ positive nat\"{u}rliche Zahlen, so gilt
\[ \texttt{ggt}(x + y, y) = \texttt{ggt}(x,y). \]  
\end{Korollar}

\noindent
\textbf{Beweis}: Das vorige Lemma zeigt, dass die Menge der gemeinsamen Teiler
der beiden Paare $\langle x, y \rangle$ und $\langle x + y, y \rangle$ identisch sind,
dass also 
\[ \texttt{gt}(x,y) = \texttt{gt}(x + y, y) \]
gilt.  Wegen
\[ \texttt{ggt}(x,y) = \max\bigl(\texttt{gt}(x,y)\bigr) = \max\bigl(\texttt{gt}(x+y,y)\bigr)
   = \texttt{ggt}(x+y,y)
\]
folgt die Behauptung. \hspace*{\fill} $\Box$


\begin{Satz}[Korrektheit des Euklidischen Algorithmus] \label{satz:euklid_korrekt}
Der Aufruf $\texttt{ggtS}(x, y)$ des in Abbildung \ref{fig:ggt.stlx} gezeigten Algorithmus berechnet 
f\"{u}r zwei positive nat\"{u}rliche Zahlen $x$ und $y$ den gr\"{o}\3ten gemeinsamen Teiler von $x$ und $y$:
\\[0.2cm]
\hspace*{1.3cm}
$\forall x, y \in \mathbb{N}: \mathtt{ggtS}(x, y) = \textsl{ggt}(x,y)$.
\end{Satz}

\proof
Wir f\"{u}hren den Beweis der Behauptung durch \emph{Wertverlaufs-Induktion}.
\begin{enumerate}
\item Induktions-Anfang: 

      Die Berechnung bricht genau dann ab, wenn $x = y$ ist.  In diesem Fall
      wird als Ergebnis $x$ zur\"{u}ckgegeben, wir haben also
      \\[0.2cm]
      \hspace*{1.3cm}
      $\mathtt{ggtS}(x, y) = x$
      \\[0.2cm]
      Andererseits gilt dann 
      \\[0.2cm]
      \hspace*{1.3cm}
      $\textsl{ggt}(x,y) = \textsl{ggt}(x,x) = x$,
      \\[0.2cm]
      denn $x$ ist offenbar der gr\"{o}\3te gemeinsame Teiler von $x$ und $x$.
\item Induktions-Schritt:  Hier gibt es zwei F\"{a}lle zu betrachten, $x < y$ und $y < x$.
      \begin{enumerate}
      \item $x < y$:
            In diesem Fall sagt die Induktions-Voraussetzung, dass das Ergebnis des rekursiven Aufrufs 
            der Funktion $\texttt{ggtS}()$ korrekt ist, wir d\"{u}rfen also voraussetzen, dass
            \\[0.2cm]
            \hspace*{1.3cm}
            $\texttt{ggtS}(x, y - x) = \textsl{ggt}(x, y - x)$
            \\[0.2cm]
            gilt.  Zu zeigen ist
            \\[0.2cm]
            \hspace*{1.3cm}
            $\texttt{ggtS}(x, y) = \textsl{ggt}(x, y)$.
            \\[0.2cm]
            Der Nachweis verl\"{a}uft wie folgt:
            \\[0.2cm]
            \hspace*{1.3cm}
            $
            \begin{array}[t]{lcll}
             \texttt{ggtS}(x, y)    & = & \mathtt{ggtS}(x, y - x)               &    
                                          \mbox{(nach Definition von $\texttt{ggtS}(x,y)$)} 
                                          \\[0.1cm]
                                    & \stackrel{IV}{=} &
                                          \textsl{ggt}(x, y - x)                    \\[0.1cm]
                                    & = & \textsl{ggt}(y - x, x) &
                                          \mbox{(denn $\textsl{ggt}(a,b) = \textsl{ggt}(b,a)$)} \\[0.1cm]
                                    & = & \textsl{ggt}\bigl((y - x) + x, x\bigr) &
                                          \mbox{(nach Korollar \ref{korollar:ggt})} \\[0.1cm]
                                    & = & \textsl{ggt}(y, x)                        \\[0.1cm]
                                    & = & \textsl{ggt}(x, y)                        
            \end{array}
            $
            \\[0.2cm]
            und das war zu zeigen.
      \item $y < x$:  
            Dieser Fall ist analog zu dem vorhergehenden Fall und wird daher nicht weiter
            ausgef\"{u}hrt.  
      \end{enumerate}
\end{enumerate}
Um nachzuweisen, dass das in Abbildung \ref{fig:ggt.stlx} gezeigte Programm tats\"{a}chlich
funktioniert, m\"{u}ssen wir noch zeigen, dass es in jedem Fall terminiert.  Dies folgt aber sofort
daraus, dass die Summe der Argumente $x + y$ bei jedem rekursiven Aufruf kleiner wird: 
\begin{enumerate}
\item Falls $x < y$ ist, haben wir f\"{u}r die Summe der Argumente des  rekursiven Aufrufs
      \\[0.2cm]
      \hspace*{1.3cm}
      $x + (y - x) = y < x + y$ \quad falls $x > 0$ ist.
\item Falls $y < x$ ist, haben wir f\"{u}r die Summe der Argumente des  rekursiven Aufrufs
      \\[0.2cm]
      \hspace*{1.3cm}
      $(x - y) + y = x < x + y$ \quad falls $y > 0$ ist.
\end{enumerate}
Falls nun $x$ und $y$ beim ersten Aufruf von $0$ verschieden sind, so werden die Summen bei jedem
Aufruf kleiner, denn es ist auch sichergestellt, dass bei keinem rekursiven Aufruf eines der
Argumente von $\textsl{ggt}()$ den Wert $0$ annimmt: Falls $x < y$ ist, ist $y - x > 0$ und
wenn $y < x$ ist, dann ist $x - y > 0$ und im Fall $x = y$ bricht die Rekursion ab.
\qed
\vspace*{0.3cm}


\begin{figure}[!ht]
\centering
\begin{Verbatim}[ frame         = lines, 
                  framesep      = 0.3cm, 
                  labelposition = bottomline,
                  numbers       = left,
                  numbersep     = -0.2cm,
                  xleftmargin   = 0.8cm,
                  xrightmargin  = 0.8cm,
                ]
    ggtS2 := procedure(x, y) {
        if (y == 0) {
            return x;
        }
        return ggt(y, x % y); 
    };
\end{Verbatim}
\vspace*{-0.3cm}
\caption{Der verbesserte Euklidische Algorithmus.}
\label{fig:ggt2.stlx}
\end{figure}

Der in Abbildung \ref{fig:ggt.stlx} gezeigte Algorithmus ist nicht sehr effizient. 
Abbildung \ref{fig:ggt2.stlx} zeigt eine verbesserte Version.  Um die Korrektheit der verbesserten
Version beweisen zu k\"{o}nnen, ben\"{o}tigen wir einen weiteren Hilfssatz.

\begin{Lemma} F\"{u}r $x,y \in \mathbb{Z}$, $n \in \mathbb{N}_1$ und $k \in \mathbb{N}$ gilt
\\[0.2cm]
\hspace*{1.3cm}
  $x \modulo n = 0 \wedge y \modulo n = 0 \;\leftrightarrow\; 
   (x - k \cdot y) \modulo n = 0 \wedge y \modulo n = 0
  $.
\end{Lemma}

\proof
Ber\"{u}cksichtigen wir, dass beispielsweise die Gleichung $x \modulo n = 0$ \"{a}quivalent zu 
$x \approx_n 0$ ist, so k\"{o}nnen wir die obige Behauptung auch in der Form
\\[0.2cm]
\hspace*{1.3cm}
$y \approx_n 0 \rightarrow \bigl(x \approx_n 0 \leftrightarrow (x - k \cdot y) \approx_n 0\bigr)$
\\[0.2cm]
schreiben.  Diese Behauptung folgt aber aus der
Tatsache, dass die Relation $\approx_n$ eine Kongruenz-Relation ist. \qed


\begin{Korollar} \label{korollar:ggt2}
F\"{u}r $x,y \in \mathbb{Z}$ und $y \not= 0$ gilt
\\[0.2cm]
\hspace*{1.3cm}
$\textsl{ggt}(x,y) = \textsl{ggt}(y, x \modulo y)$.
\end{Korollar}

\proof
Nach dem Satz \"{u}ber die Division mit Rest gibt es eine Zahl $k \in \mathbb{Z}$, so dass
\\[0.2cm]
\hspace*{1.3cm}
$x = k \cdot y + x \modulo y$
\\[0.2cm]
gilt.  Diese Gleichung formen wir zu
\\[0.2cm]
\hspace*{1.3cm}
$x \modulo y = x - k \cdot y$
\\[0.2cm]
um.  Dann haben wir f\"{u}r beliebige $n \in \mathbb{N}$ die folgende Kette von \"{A}quivalenzen:
\\[0.2cm]
\hspace*{1.3cm}
$
\begin{array}[t]{lll}
                & (x \modulo y)   \modulo n = 0 \wedge y \modulo n = 0   \\[0.2cm]
\Leftrightarrow & (x - k \cdot y) \modulo n = 0 \wedge y \modulo n = 0   \\[0.2cm]
\Leftrightarrow & x \modulo n = 0 \wedge y \modulo n = 0               &
                  \mbox{nach der letzten Aufgabe}                        
\end{array}
$
\\[0.2cm]
Damit sehen wir aber, dass die Zahlen $x \modulo n$ und $y$ dieselben gemeinsamen Teiler haben wie
die Zahlen $x$ und $y$
\\[0.2cm]
\hspace*{1.3cm}
$\textsl{gt}(x \modulo n, y) = \textsl{gt}(x,y)$.
\\[0.2cm]
Daraus folgt sofort
\\[0.2cm]
\hspace*{1.3cm}
$\textsl{ggt}(x \modulo n, y) = \textsl{ggt}(x,y)$
\\[0.2cm]
und wegen $\textsl{ggt}(a,b) = \textsl{ggt}(b,a)$ ist das die Behauptung. \qed

\begin{Satz}[Korrektheit des verbesserten Euklidischen Algorithmus] \hspace*{\fill} \\
F\"{u}r die in Abbildung \ref{fig:ggt2.stlx} gezeigte Funktion $\textsl{ggtS2}()$ gilt
\\[0.2cm]
\hspace*{1.3cm}
$\texttt{ggtS2}(x,y) = \textsl{ggt}(x,y)$ \quad f\"{u}r $x,y \in \mathbb{N}$.
\end{Satz}

\proof
Wir f\"{u}hren den Beweis wieder durch eine Wertverlaufs-Induktion.
\begin{enumerate}
\item Induktions-Anfang: $y = 0$.  In diesem Fall gilt
      \\[0.2cm]
      \hspace*{1.3cm}
      $\mathtt{ggtS2}(x, 0) = x = \textsl{ggt}(x, 0)$.
\item Induktions-Schritt: Falls $y \not= 0$ ist, haben wir
      \\[0.2cm]
      \hspace*{1.3cm}
      $
      \begin{array}[t]{lcl}
        \mathtt{ggtS2}(x, y) & = & \mathtt{ggtS2}(y, x \modulo y)              \\[0.2cm]
                             & \stackrel{IV}{=} & \textsl{ggt}(y, x \modulo y) \\[0.2cm]
                             & = & \textsl{ggt}(x, y),
      \end{array}
      $
      \\[0.2cm]
      wobei wir im letzten Schritt das Korollar \ref{korollar:ggt2} benutzt haben. 
\end{enumerate}
Es ist noch zu zeigen, dass ein Aufruf der Prozedur $\texttt{ggtS2}(x, y)$ f\"{u}r beliebige
$x,y \in \mathbb{N}$ terminiert.  Wir k\"{o}nnen ohne Beschr\"{a}nkung der Allgemeinheit anehmen,
dass $x \geq y$ ist, denn falls $x < y$ ist, dann gilt $x \modulo y = x$ und damit werden
dann bei dem ersten rekursiven Aufruf $\mathtt{ggt}(y, x \modulo y)$ die Argumente $x$ und
$y$ vertauscht.

Sei also jetzt $y \leq x$.  Wir zeigen, dass diese Ungleichung dann auch bei jedem
rekursiven Aufruf bestehen bleibt, denn es gilt
\\[0.2cm]
\hspace*{1.3cm}
$x \modulo y < y$.
\\[0.2cm]
Weiter sehen wir, dass unter der Voraussetzung $y \leq x$ die Summe der Argumente bei
jedem rekursiven Aufruf kleiner wird,
denn wenn $y \leq x$ ist, haben wir
\\[0.2cm]
\hspace*{1.3cm}
$y + x \modulo y < y + x$, \quad da $x \modulo y < y \leq x$ ist.
\\[0.2cm]
Da die Summe zweier nat\"{u}rlicher Zahlen nur endlich oft verkleinert werden kann, 
terminiert der Algorithmus.  
\qed


\begin{figure}[!ht]
\centering
\begin{Verbatim}[ frame         = lines, 
                  framesep      = 0.3cm, 
                  firstnumber   = 1,
                  labelposition = bottomline,
                  numbers       = left,
                  numbersep     = -0.2cm,
                  xleftmargin   = 0.8cm,
                  xrightmargin  = 0.8cm,
                ]
    eggt := procedure(x, y) {
        if (y == 0) {
            return [ 1, 0 ];
        }
        q := x / y;
        r := x % y;
        [ s, t ] := eggt(y, r);
        return [ t, s - q * t ]; 
    };
\end{Verbatim}
\vspace*{-0.3cm}
\caption{Der erweiterte Euklidische Algorithmus.}
\label{fig:eggt.stlx}
\end{figure}

Der Euklidische Algorithmus kann so erweitert werden, dass f\"{u}r gegebene Zahlen $x,y \in \mathbb{N}$
zwei Zahlen $\alpha,\beta \in \mathbb{Z}$ berechnet werden, so dass 
\\[0.2cm]
\hspace*{1.3cm}
$\alpha \cdot x + \beta \cdot y = \textsl{ggt}(x, y)$
\\[0.2cm]
gilt.  Abbildung \ref{fig:eggt.stlx} zeigt eine entsprechende Erweiterung.

\begin{Satz}[Korrektheit des erweiterten Euklidischen Algorithmus] \hspace*{\fill} \\ 
  Die in Abbildung \ref{fig:eggt.stlx} gezeigte Funktion $\mathtt{eggt}$ erf\"{u}llt
  folgende Spezifikation:
  \\[0.2cm]
  \hspace*{1.3cm}
  $\forall x, y \in \mathbb{N}_0: \bigl(
   \mathtt{eggt}(x, y) = [ \alpha, \beta ] \;\Rightarrow\; 
   \alpha \cdot x + \beta \cdot y = \mathtt{ggt}(x,y)\bigr)
  $.
\end{Satz}

\proof
Wir f\"{u}hren den Beweis durch Wertverlaufs-Induktion.
\begin{enumerate}
\item Induktions-Anfang: $y = 0$.

      Es gilt $\mathtt{eggt}(x, 0) = [ 1, 0 ]$.  Also haben wir $\alpha = 1$ und $\beta = 0$. 
      Offensichtlich gilt 
      \\[0.2cm]
      \hspace*{1.3cm}
      $\alpha \cdot x + \beta \cdot y = 1 \cdot x + 0 \cdot y = x = \textsl{ggt}(x, 0)$.
      \\[0.2cm]
      und damit ist die Behauptung in diesem Fall gezeigt. 
\item Induktions-Schritt: $y \not= 0$.

      Nach Induktions-Voraussetzung wissen wir, dass f\"{u}r den rekursiven Aufruf $\textsl{eggt}(y, r)$
      die Gleichung
      \begin{equation}
        \label{eq:eggt1}
        s \cdot y + t \cdot r = \textsl{ggt}(y, r)
      \end{equation}
      richtig ist.  Nach dem Programm gilt $r = x \modulo y$ und nach der Definition
      des Modulo-Operators ist $x \modulo y = x - (x \div y) \cdot y$.
      Setzen wir dies in Gleichung (\ref{eq:eggt1}) ein, so erhalten wir
      \begin{equation}
        \label{eq:eggt2}
        s \cdot y + t \cdot \bigl(x - (x \div y) \cdot y\bigr) = \textsl{ggt}(y, x \modulo y).        
      \end{equation}
      Stellen wir die linke Seite dieser Gleichung um und ber\"{u}cksichtigen weiter, dass nach 
      Korollar \ref{korollar:ggt2} $\textsl{ggt}(y, x \modulo y) = \textsl{ggt}(x, y)$ gilt,
      so vereinfacht sich Gleichung (\ref{eq:eggt2}) zu
      \\[0.2cm]
      \hspace*{1.3cm}
      $t \cdot x + \bigl(s - (x \div y)\cdot t\bigr) \cdot y = \textsl{ggt}(x, y)$.        
      \\[0.2cm]
      In der Funktion $\mathtt{eggt}()$ ist $q$ als $x / y$ definiert, wobei dort der Operator
      ``\texttt{/}''  aber
      f\"{u}r die ganzzahlige Division mit Rest steht, so dass tats\"{a}chlich $q = x \div y$ gilt.  Damit
      haben wir 
      \\[0.2cm]
      \hspace*{1.3cm}
      $t \cdot x + (s - q \cdot t) \cdot y = \textsl{ggt}(x, y)$
      \\[0.2cm]
      und das ist wegen $\alpha = t$ und $\beta = s - q \cdot t$ die Behauptung. 
\end{enumerate}
Der Nachweis der Terminierung ist derselbe wie bei der Funktion $\textsl{ggtS2}()$ und
wird daher nicht noch einmal angegeben. \qed


\section{Der Fundamentalsatz der Arithmetik}
\begin{Satz}[Lemma von Euler] \lb
Es seien $a$ und $b$ nat\"{u}rliche Zahlen und $p$ sei eine Primzahl.  Dann gilt
\\[0.2cm]
\hspace*{1.3cm}
$p \mid a \cdot b \;\Rightarrow\; p \mid a  \vee p \mid b$.
\\[0.2cm]
In Worten: Wenn $p$ das Produkt zweier Zahlen teilt, dann muss $p$ eine der beiden Zahlen
des Produkts teilen.  
\end{Satz}

\proof
Wenn $p$ das Produkt $a \cdot b$ teilt, dann gibt es eine nat\"{u}rliche Zahl $c$, so dass
\begin{equation}
  \label{eq:leuklid1}
  c \cdot p = a \cdot b
\end{equation}
ist.  Wir nehmen an, dass $p$ kein Teiler von $a$ ist.  Dann m\"{u}ssen wir zeigen, dass $p$ ein
Teiler von $b$ ist.  Wenn $p$ kein Teiler von $a$ ist, dann folgt aus der Tatsache, dass
$p$ eine Primzahl ist, dass
\\[0.2cm]
\hspace*{1.3cm}
$\textsl{ggt}(p, a) = 1$
\\[0.2cm]
gilt.  Nach dem Lemma von Bez\'out (Satz \ref{satz:bezout}) gibt es also ganze Zahlen $x$
und $y$, so dass 
\\[0.2cm]
\hspace*{1.3cm}
$1 = x \cdot p + y \cdot a$
\\[0.2cm]
gilt.  Wir multiplizieren diese Gleichung mit $b$ und erhalten
\\[0.2cm]
\hspace*{1.3cm}
$b = x \cdot p \cdot b + y \cdot a \cdot b$.
\\[0.2cm]
An dieser Stelle nutzen wir aus, dass nach Gleichung (\ref{eq:leuklid1}) 
$a \cdot b = c \cdot p$ gilt und formen die obige Gleichung f\"{u}r $b$ wie folgt um:
\\[0.2cm]
\hspace*{1.3cm}
$b = x \cdot p \cdot b + y \cdot c \cdot p$.
\\[0.2cm]
Klammern wir hier $p$ aus, so haben wir
\\[0.2cm]
\hspace*{1.3cm}
$b = (x \cdot b + y \cdot c) \cdot p$.
\\[0.2cm]
und daraus sehen wir, dass $p$ ein Teiler von $b$ ist, was zu zeigen war. \qed
\vspace*{0.3cm}

Eine \emph{Primfaktor-Zerlegung} einer nat\"{u}rlichen Zahl $n$ ist ein Produkt der Form
\\[0.2cm]
\hspace*{1.3cm}
$p_1 \cdot p_2 \cdot\; \dots \;\cdot p_k$,
\\[0.2cm]
wobei alle Faktoren $p_1$, $\cdots$, $p_k$ Primzahlen sind.  Beispielsweise ist
\\[0.2cm]
\hspace*{1.3cm}
$2 \cdot 3 \cdot 3 \cdot 5$
\\[0.2cm]
eine Primfaktor-Zerlegung der Zahl 90.  \"{u}blicherweise fassen wir dabei noch gleiche Faktoren
zusammen, in dem oberen Beispiel w\"{u}rden wir also
\\[0.2cm]
\hspace*{1.3cm}
$90 = 2 \cdot 3^2 \cdot 5$
\\[0.2cm]
schreiben.  Sind $p_1$, $\cdots$, $p_k$ verschiedene Primzahlen, die der Gr\"{o}\3e nach angeordnet sind,
gilt also
\\[0.2cm]
\hspace*{1.3cm}
$p_1 < p_2 < \cdots < p_{i} < p_{i+1} < \cdots < p_k$,
\\[0.2cm]
und sind $e_1$, $\cdots$, $e_k$ positive nat\"{u}rliche Zahl, so nennen wir einen Ausdruck der Form
\\[0.2cm]
\hspace*{1.3cm}
$\prod\limits_{i=1}^k  p_i^{e_i} = p_1^{e_1} \cdot \;\dots\; \cdot p_k^{e_k}$
\\[0.2cm]
eine \emph{kanonische Primfaktor-Zerlegung}.  Die Tatsache, dass es f\"{u}r jede nat\"{u}rliche Zahl, die
gr\"{o}\3er als $1$ ist, eine kanonische Primfaktor-Zerlegung gibt, die dar\"{u}ber hinaus noch eindeutig
ist,  ist ein wesentliches Ergebnis der elementaren Zahlentheorie.


\begin{Theorem}[Fundamentalsatz der Arithmetik]
\label{theorem:fundamental-arithmetik} \lb
  Es sei $n$ eine nat\"{u}rliche Zahl gr\"{o}\3er als $1$.  Dann l\"{a}sst sich $n$ auf genau eine Weise in der Form
  \\[0.2cm]
  \hspace*{1.3cm}
  $n = p_1^{e_1} \cdot \;\dots\; \cdot p_k^{e_k}$ \quad mit Primzahlen $p_1 < p_2 < \cdots < p_k$
  \\[0.2cm]
  und positiven ganzzahligen Exponenten $e_1, \cdots, e_n$ schreiben.
\end{Theorem}

\proof
Wir zeigen zun\"{a}chst die Existenz einer Primfaktor-Zerlegung f\"{u}r jede nat\"{u}rliche Zahl $n$ gr\"{o}\3er als
1.  Wir f\"{u}hren diesen Nachweis durch Induktion nach $n$.
\begin{enumerate}
\item[I.A.] $n=2$:

      Da $2$ eine Primzahl ist, k\"{o}nnen wir
      \\[0.2cm]
      \hspace*{1.3cm}
      $n = 2^1$
      \\[0.2cm]
      schreiben und haben damit eine kanonische Primfaktor-Zerlegung gefunden.
\item[I.S.] $2,\cdots,n\!-\!1 \mapsto n$: 

      Wir f\"{u}hren eine Fallunterscheidung durch.
      \begin{enumerate}
      \item Fall: $n$ ist eine Primzahl.

            Dann ist $n = n^1$  bereits eine kanonische Primfaktor-Zerlegung von $n$.
      \item Fall: $n$ ist keine Primzahl.  

            Dann gibt es nat\"{u}rliche Zahlen $a$ und $b$ mit
            \\[0.2cm]
            \hspace*{1.3cm}
            $n = a \cdot b$ \quad und $a > 1$ und $b > 1$,
            \\[0.2cm]
            denn wenn es keine solche Zerlegung g\"{a}be, w\"{a}re $n$ eine Primzahl. Damit ist klar, dass
            sowohl $a < n$ als auch $b < n$ gilt.  Nach Induktions-Voraussetzung gibt es also
            Primfaktor-Zerlegungen f\"{u}r $a$ und $b$:
            \\[0.2cm]
            \hspace*{1.3cm}
             $a = p_1^{e_1} \cdot \;\dots\; \cdot p_k^{e_k}$ \quad und \quad
             $b = q_1^{f_1} \cdot \;\dots\; \cdot q_l^{f_l}$.
            \\[0.2cm]
            Multiplizieren wir diese Primfaktor-Zerlegungen, sortieren die Faktoren geeignet und
            fassen wir dann noch Faktoren mit der gleichen Basis zusammen, so erhalten wir offenbar
            eine Primfaktor-Zerlegung von $a \cdot b$ und damit von $n$.
      \end{enumerate}
\end{enumerate}
Um den Beweis abzuschlie\3en zeigen wir, dass die Primfaktor-Zerlegung eindeutig sein muss.
Diesen Nachweis f\"{u}hren wir indirekt und nehmen an, dass $n$ die kleinste nat\"{u}rliche Zahl ist, die
zwei verschiedene Primfaktor-Zerlegung hat, beispielsweise die beiden Zerlegungen
\begin{equation}
  \label{eq:hauptsatz-arithmetik1}
n = p_1^{e_1} \cdot \;\dots\; \cdot p_k^{e_k} \quad \mbox{und} \quad
n = q_1^{f_1} \cdot \;\dots\; \cdot q_l^{f_l}.  
\end{equation}
Zun\"{a}chst stellen wir fest, dass dann die Mengen
\\[0.2cm]
\hspace*{1.3cm}
$\{p_1, \cdots, p_k\}$ \quad und \quad
$\{q_1, \cdots, q_l\}$ 
\\[0.2cm]
disjunkt sein m\"{u}ssen, denn wenn beispielsweise $p_i = q_j$ w\"{a}re, k\"{o}nnten wir die
Primfaktor-Zerlegung durch $p_i$ teilen und h\"{a}tten dann
\\[0.2cm]
\hspace*{1.3cm}
$p_1^{e_1} \cdot \;\dots\; \cdot p_i^{e_i-1} \cdot \;\dots\; \cdot p_k^{e_k} = n/p_i = n/q_j
 = q_1^{f_1} \cdot \;\dots\; \cdot q_j^{f_j-1} \cdot \;\dots\; \cdot q_l^{f_l}$.
\\[0.2cm]
Damit h\"{a}tte auch die Zahl $n/p_i$, die offenbar kleiner als $n$ ist, zwei verschiedene
Primfaktor-Zerlegungen, was im Widerspruch zu 
der Annahme steht, dass $n$ die kleinste Zahl mit zwei verschiedenen Primfaktor-Zerlegungen ist.
Wir sehen also, dass die Primfaktoren $p_1$, $\cdots$, $p_k$ und \\ $q_1$, $\cdots$, $q_l$ 
voneinander verschieden sein m\"{u}ssen.  Nun benutzen wir das Lemma von Euklid:  
Aus 
\\[0.2cm]
\hspace*{1.3cm}
$p_1^{e_1} \cdot \;\dots\; \cdot p_k^{e_k} = q_1^{f_1} \cdot \;\dots\; \cdot q_l^{f_l}$
\\[0.2cm]
folgt zun\"{a}chst, dass $p_1$ ein Teiler von dem Produkt $q_1^{f_1} \cdot \;\dots\; \cdot q_l^{f_l}$ ist.
Nach dem Lemma von Euklid folgt nun, dass $p_1$ entweder $q_1^{f_1}$ oder 
$q_2^{f_2} \cdot \;\dots\; \cdot q_l^{f_l}$ teilt.  Da $p_1$ von $q_1$ verschieden ist, kann $p_1$ kein
Teiler von $q_1^{f_1}$ sein.  Durch Iteration dieses Arguments sehen wir, dass $p_1$ auch kein
Teiler von $q_2^{f_2}$, $\cdots$, $q_{l-1}^{f_{l-1}}$ ist.  Schlie\3lich bleibt als einzige
M\"{o}glichkeit,  dass $p_1$ ein Teiler von
 $q_l^{f_l}$ ist, was aber wegen $p_1 \not= q_l$ ebenfalls unm\"{o}glich ist.
Damit haben wir einen Widerspruch zu der Annahme, dass $n$
zwei verschiedene Primfaktor-Zerlegungen besitzt und der Beweis ist abgeschlossen.
\qed

\section{Die Eulersche $\varphi$-Funktion}
Es sei $n \in \mathbb{N}$ mit $n > 0$ gegeben.  Die \emph{multiplikative Gruppe} $\mathbb{Z}_n^*$
ist durch
\\[0.2cm]
\hspace*{1.3cm}
$\zns{n} := \{ x \in \zn{n} \mid \exists y \in \zn{n} : x \cdot y \approx_n 1 \}$
\\[0.2cm]
definiert.  Die Menge $\zns{n}$ enth\"{a}lt also genau die Zahlen aus $\zn{n}$, die
bez\"{u}glich der Multiplikation ein Inverses modulo $n$ haben.  Beispielsweise gilt
\\[0.2cm]
\hspace*{1.3cm}
$\mathbb{Z}_5^* = \{ 1, 2, 3, 4 \}$,
\\[0.2cm]
denn alle Zahlen der Menge $\{ 1, 2, 3, 4 \}$ haben ein Inverses bez\"{u}glich der Multiplikation modulo
$5$.
\begin{enumerate}
\item Wir haben $1 \cdot 1 \approx_5 1$, also ist $1$ das multiplikative Inverse modulo 5 von $1$.
\item Wir haben $2 \cdot 3 = 6 \approx_5 1$, also ist $3$ das multiplikative Inverse modulo 5 von $2$.
\item Wir haben $3 \cdot 2 = 6 \approx_5 1$, also ist $2$ das multiplikative Inverse modulo 5 von $3$.
\item Wir haben $4 \cdot 4 = 16 \approx_5 1$, also ist $4$ das multiplikative Inverse modulo 5 von $4$.
\end{enumerate}
Auf der anderen Seite haben wir
\\[0.2cm]
\hspace*{1.3cm}
$\mathbb{Z}_4^* = \{ 1, 3 \}$,
\\[0.2cm]
denn $3 \cdot 3 = 9 \approx_4 1$, so dass die Zahl $3$ das multiplikative Inverse modulo 4 von $3$
ist, aber die Zahl $2$ hat kein multiplikative Inverses modulo 4, denn wir haben $2 \cdot 2 = 4 \approx_4 0$.
Generell kann eine Zahl $x$, f\"{u}r die es ein $y \not\approx_n 0$ mit
\\[0.2cm]
\hspace*{1.3cm}
$x \cdot y \approx_n 0$ 
\\[0.2cm]
gibt, kein multiplikatives Inverses haben, denn falls $z$ ein solches Inverses w\"{a}re, so k\"{o}nnten wir
die obige Gleichung einfach von links mit $z$ multiplizieren und h\"{a}tten dann
\\[0.2cm]
\hspace*{1.3cm}
$z \cdot x \cdot y \approx_n z \cdot 0$,
\\[0.2cm]
woraus wegen $z \cdot x \approx_n 1$ sofort $y \approx_n 0$ folgen w\"{u}rde, was im Widerspruch zu der
Voraussetzung $y \not\approx_n 0$ steht.

\remark
Wir haben oben von der \emph{multiplikativen Gruppe} $\zns{n}$ gesprochen.  Wenn wir von
einer Gruppe sprechen, dann meinen wir damit genau genommen nicht nur die Menge $\zns{n}$
sondern die Struktur 
\\[0.2cm]
\hspace*{1.3cm}
$\langle \zns{n}, 1, \cdot_n \rangle$,
\\[0.2cm]
wobei die Funktion $\cdot_n : \zns{n} \times \zns{n} \rightarrow \zns{n}$ durch 
\\[0.2cm]
\hspace*{1.3cm}
$x \cdot_n y := (x \cdot y) \modulo n$
\\[0.2cm]
definiert ist.  Dass die Menge $\zns{n}$ mit der so definierten Multiplikation tats\"{a}chlich
zu einer Gruppe wird, folgt letztlich aus der Vertr\"{a}glichkeit der Relation $\approx_n$ mit der
gew\"{o}hnlichen Multiplikation.  Die Details \"{u}berlasse ich Ihnen in der folgenden Aufgabe.

\exercise
Zeigen Sie, dass die oben definierte Struktur $\langle \zns{n}, 1, \cdot_n \rangle$
eine Gruppe ist.


\begin{Definition}[Eulersche $\varphi$-Funktion] F\"{u}r alle nat\"{u}rlichen Zahlen $n > 1$ definieren wir
\\[0.2cm]
\hspace*{1.3cm}
$\varphi(n) := \textsl{card}\bigl(\zns{n}\bigr)$.
\\[0.2cm]
Um sp\"{a}tere Definitionen zu vereinfachen,  setzen wir au\3erdem $\varphi(1) := 1$. \qed
\end{Definition}

\begin{Satz}[Existenz von multiplikativen Inversen modulo $n$] 
  \label{satz:multiplikatives-inverses}
  \hspace*{\fill} \\
  Es sei $n \in \mathbb{N}$ mit $n \geq 1$.
  Eine Zahl $a \in \zn{n}$ hat genau dann ein multiplikatives Inverses modulo $n$, wenn
  $\textsl{ggt}(a, n) = 1$ gilt.
\end{Satz}

\noindent
\textbf{Beweis}: Wir zerlegen den Beweis in zwei Teile.
\begin{enumerate}
\item ``$\Rightarrow$'':  Wir nehmen an, dass $a$ ein multiplikatives Inverses hat und zeigen,
      dass daraus $\textsl{ggt}(a,n) = 1$ folgt.

      Bezeichnen wir das multiplikative Inverse modulo $n$ von $a$ mit $b$, so gilt
      \\[0.2cm]
      \hspace*{1.3cm}
      $b \cdot a \approx_n 1$
      \\[0.2cm]
      Nach Definition der Relation $\approx_n$ gibt es dann eine nat\"{u}rliche Zahl $k$, so dass
      \\[0.2cm]
      \hspace*{1.3cm}
      $b \cdot a = 1 + k \cdot n$
      \\[0.2cm]
      gilt.  Daraus folgt sofort
      \begin{equation}
        \label{eq:ba-minus-kn}
      b \cdot a - k \cdot n = 1.         
      \end{equation}
      Sei nun $d$ ein gemeinsamer Teiler von $a$ und $n$.  Dann ist $d$ offenbar auch ein
      gemeinsamer Teiler von $b \cdot a$ und $k \cdot n$ und weil allgemein gilt, dass ein
      gemeinsamer Teiler zweier Zahlen $x$ und $y$ auch ein Teiler der Differenz $x - y$ ist, 
      k\"{o}nnen wir folgern, dass $d$ auch ein Teiler von $b \cdot a - k \cdot n$ ist:
      \\[0.2cm]
      \hspace*{1.3cm}
      $d \teilt b \cdot a - k \cdot n$.
      \\[0.2cm]
      Aus Gleichung (\ref{eq:ba-minus-kn}) folgt nun, dass $d$ auch ein Teiler von $1$ ist.
      Damit haben wir gezeigt, dass $a$ und $n$ nur den gemeinsamen Teiler $1$ haben:
      \\[0.2cm]
      \hspace*{1.3cm}
      $\textsl{ggt}(a, n) = 1$.
\item ``$\Leftarrow$'': Jetzt nehmen wir an, dass $\textsl{ggt}(a, n) = 1$ ist und zeigen, dass
      $a$ dann ein multiplikatives Inverses modulo $n$ besitzt.

      Sei also $\textsl{ggt}(a, n) = 1$.  Nach dem Lemma von Bez\'out (Satz \ref{satz:bezout})
      gibt es also ganze Zahlen $x$ und $y$, so dass 
      \\[0.2cm]
      \hspace*{1.3cm}
      $x \cdot a + y \cdot n = 1$.
      \\[0.2cm]
      gilt. Stellen wir diese Gleichung um, so erhalten wir
      \\[0.2cm]
      \hspace*{1.3cm}
      $x \cdot a = 1 - y \cdot n \approx_n 1$, \quad also $x \cdot a \approx_n 1$.
      \\[0.2cm]
      Damit ist $x$ das multiplikative Inverse von $a$ modulo $n$. \qed
\end{enumerate}

\begin{Korollar} \label{korollar:phi}
F\"{u}r alle nat\"{u}rlichen Zahlen $n > 1$ gilt
\\[0.2cm]
\hspace*{1.3cm}
$\varphi(n) = \textsl{card}\bigl(\{x \in \zn{n} \mid \textsl{ggt}(x, n) = 1 \}\bigr)$. 
\end{Korollar}


\noindent
Als Konsequenz des letzten Satzes k\"{o}nnen wir nun die Eulersche $\varphi$-Funktion f\"{u}r Potenzen von
Primzahlen berechnen.

\begin{Satz}[Berechnung der $\varphi$-Funktion f\"{u}r Primzahl-Potenzen]
\label{satz:phi_primzahl} 
Es sei $p$ eine Primzahl und $n$ eine positive nat\"{u}rliche Zahl.  Dann gilt
\\[0.2cm]
\hspace*{1.3cm}
$\varphi\bigl(p^n\bigr) = p^{n-1} \cdot (p - 1)$.  
\end{Satz}

\proof
Nach Satz \ref{satz:multiplikatives-inverses} m\"{u}ssen wir z\"{a}hlen, welche Zahlen in der Menge
\\[0.2cm]
\hspace*{1.3cm}
$\mathbb{Z}_{p^n} = \{0, 1, \cdots, p^n-1\}$ 
\\[0.2cm]
zu der Zahl $p^{n}$ teilerfremd sind, denn es gilt
\\[0.2cm]
\hspace*{1.3cm}
$\varphi\bigl(p^n\bigr) = 
 \textsl{card}\bigl(\{ x \in \mathbb{Z}_{p^n} \mid \textsl{ggt}(x, p^n) = 1 \}\bigr)
$.
\\[0.2cm]
Wir definieren daher die Menge $A$ als
\\[0.2cm]
\hspace*{1.3cm}
$A := \{ x \in \mathbb{Z}_{p^n} \mid \textsl{ggt}(x, p^n) = 1 \}$.
\
\\[0.2cm]
Weiter ist es n\"{u}tzlich, das Komplement dieser Menge bez\"{u}glich $\mathbb{Z}_{p^n}$ zu betrachten. Daher
definieren wir 
\\[-0.2cm]
\hspace*{1.3cm}
$
\begin{array}[t]{lrl}
B & := & \mathbb{Z}_{p^n} \backslash A \\[0.1cm]
  &  = & \mathbb{Z}_{p^n} \backslash \{ x \in \mathbb{Z}_{p^n} \mid \textsl{ggt}(x, p^n) = 1 \} \\[0.1cm]
  &  = & \{ x \in \mathbb{Z}_{p^n} \mid \textsl{ggt}(x, p^n) > 1 \}.
\end{array}
$
\\[0.2cm]
Die Menge $B$ enth\"{a}lt also die Zahlen aus $\mathbb{Z}_{p^n}$, die mit $p^n$ einen gemeinsamen Teiler
haben.  Da $p$ eine Primzahl ist, enth\"{a}lt die Menge $B$ folglich genau die Vielfachen  der Primzahl
$p$, die kleiner als $p^n$ sind.  Daher k\"{o}nnen wir $B$ wie folgt schreiben:
\\[0.2cm]
\hspace*{1.3cm}
$B := \bigl\{ y \cdot p \mid y \in \{ 0, 1, \cdots, p^{n-1} - 1\} \bigr\}$.
\\[0.2cm]
Offenbar gilt
\\[0.2cm]
\hspace*{1.3cm}
$\textsl{card}(B) = \textsl{card}\bigl(\{ 0, 1, \cdots, p^{n-1} - 1\}\bigr) = p^{n-1}$.
\\[0.2cm]
Andererseits folgt aus der Gleichung $A = \mathbb{Z}_{p^n} \backslash B$ sofort
\\[0.2cm]
\hspace*{1.3cm}
$\varphi\bigl(p^n\bigr) = \textsl{card}(A) = \textsl{card}\bigl(\mathbb{Z}_{p^n}\bigr) - \textsl{card}(B)
                  = p^n - p^{n-1} = p^{n-1} \cdot (p - 1)$.
\\[0.2cm]
Damit ist der Beweis abgeschlossen. \qed
\vspace*{0.3cm}

Um  das Produkt $\varphi(p \cdot q)$ f\"{u}r zwei verschiedene Primzahlen $p$ und $q$ berechnen zu
k\"{o}nnen, ben\"{o}tigen wir den folgenden Satz.

\begin{Satz}[Chinesischer Restesatz, 1.~Teil] \hspace*{\fill} \\
Es seien $m,n \in \mathbb{N}$ nat\"{u}rliche Zahlen gr\"{o}\3er als $1$ und es gelte $\textsl{ggt}(m,n) =1$.
Weiter gelte $a \in \zn{m}$ und $b \in \zn{n}$. 
Dann gibt es genau eine Zahl $x \in \zn{m \cdot n}$, so dass
\\[0.2cm]
\hspace*{1.3cm}
$x \approx_m a$ \quad und \quad $x \approx_n b$
\\[0.2cm]
gilt.
\end{Satz}

\proof
Wir zerlegen den Beweis in zwei Teile.  
Zun\"{a}chst zeigen wir, dass tats\"{a}chlich ein $x \in \zn{m \cdot n}$ existiert, das die beiden
Gleichungen $x \approx_m a$ und  $x \approx_n b$ erf\"{u}llt sind.  Anschlie\3end zeigen wir,
dass dieses $x$ eindeutig bestimmt ist.
\begin{enumerate}
\item Aus der Voraussetzung,
      dass $\textsl{ggt}(m,n) = 1$ folgt nach dem Satz \"{u}ber das multiplikative Inverse modulo
      $n$ (Satz \ref{satz:multiplikatives-inverses}), dass die Zahl $m$ ein multiplikatives
      Inverses modulo $n$ und die Zahl $n$ ein multiplikatives
      Inverses modulo $m$ hat.  Bezeichnen wir diese Inversen mit $u$ bzw.~$v$, so gilt also
      \\[0.2cm]
      \hspace*{1.3cm}
      $m \cdot u \approx_n 1$ \quad und \quad $n \cdot v \approx_m 1$.
      \\[0.2cm]
      Wir definieren nun
      \\[0.2cm]
      \hspace*{1.3cm}
      $x := (a \cdot n \cdot v + b \cdot m \cdot u) \modulo (m \cdot n)$.
      \\[0.2cm]
      Nach dem Satz \"{u}ber die Division mit Rest hat $x$ dann die Form
      \\[0.2cm]
      \hspace*{1.3cm}
      $x = a \cdot n \cdot v + b \cdot m \cdot u - k \cdot (m \cdot n)$
      \\[0.2cm]
      mit einem geeigneten $k$.  Nach Definition von $x$ ist klar, dass 
      $x \in \mathbb{Z}_{m \cdot n}$ ist.  Einerseits folgt aus der Vertr\"{a}glichkeit der Relation
      $\approx_m$ mit Addition und Multiplikation und der Tatsache, dass
      \\[0.2cm]
      \hspace*{1.3cm}
      $m \modulo m = 0$, \quad also $m \approx_m 0$
      \\[0.2cm]
      ist, dass auch
      \\[0.2cm]
      \hspace*{1.3cm}
      $b \cdot m \cdot u - k \cdot (m \cdot n) \approx_m 0$
      \\[0.2cm]
      gilt.  Andererseits folgt aus $n \cdot v \approx_m  1$, dass
      \\[0.2cm]
      \hspace*{1.3cm}
      $a \cdot n \cdot v \approx_m a$ 
      \\[0.2cm]
      gilt, so dass wir insgesamt
      \\[0.2cm]
      \hspace*{1.3cm}
      $x = a \cdot n \cdot v + b \cdot m \cdot u - k \cdot (m \cdot n) \approx_m a$
      \\[0.2cm]
      haben.  Analog sehen wir, dass
      \\[0.2cm]
      \hspace*{1.3cm}
      $a \cdot n \cdot v - k \cdot (m \cdot n) \approx_n 0$
      \\[0.2cm]
      gilt.  Weiter folgt aus $m \cdot u \approx_n 1$, dass
      \\[0.2cm]
      \hspace*{1.3cm}
      $b \cdot m \cdot u \approx_m b$ 
      \\[0.2cm]
      gilt, so dass wir au\3erdem
      \\[0.2cm]
      \hspace*{1.3cm}
      $x = b \cdot m \cdot u + a \cdot n \cdot v - k \cdot (m \cdot n) \approx_n b$
      \\[0.2cm]
      haben. 
\item Es bleibt die Eindeutigkeit von $x$ zu zeigen.  Wir nehmen dazu an, dass f\"{u}r 
      $x_1, x_2\in \zn{m \cdot n}$ sowohl
      \\[0.2cm]
      \hspace*{1.3cm}
      $x_1 \approx_m a$ und $x_1 \approx_n b$, \quad als auch \quad
      $x_2 \approx_m a$ und $x_2 \approx_n b$
      \\[0.2cm]
      gelte.  O.B.d.A.~gelte weiter $x_1 \leq x_2$.  Wir wollen zeigen, dass dann $x_1 = x_2$ gelten muss.
      Aus $x_1 \approx_m a$ und $x_2 \approx_m a$ folgt $x_1 \approx_m x_2$.  Also gibt es
      eine Zahl $k \in \mathbb{N}$, so 
      dass 
      \begin{equation}
        \label{eq:china1}
        x_2 = x_1 + k \cdot m  
      \end{equation}
      gilt.  Aus $x_1 \approx_n b$ und $x_2 \approx_n b$ folgt $x_1 \approx_n x_2$.  Also gibt es eine Zahl 
      $l \in \mathbb{N}$, so dass 
      \begin{equation}
        \label{eq:china2}
        x_2 = x_1 + l \cdot n  
      \end{equation}
      gilt.  Aus den Gleichungen  (\ref{eq:china1}) und (\ref{eq:china2}) folgt dann
      \\[0.2cm]
      \hspace*{1.3cm}
      $k \cdot m = x_2 - x_1 = l \cdot n$.
      \\[0.2cm]
      Da $m$ und $n$ teilerfremd sind, folgt daraus, dass $m$ ein Teiler von $l$ ist.  Es gibt also eine
      nat\"{u}rliche Zahl $i$, so dass $l = i \cdot m$ ist.  Damit haben wir dann insgesamt
      \\[0.2cm]
      \hspace*{1.3cm}
      $x_2 - x_1 = i \cdot m \cdot n$.
      \\[0.2cm]
      Da andererseits sowohl $x_2$ als auch $x_1$ Elemente von $\zn{m \cdot n}$ sind, muss
      \\[0.2cm]
      \hspace*{1.3cm}
      $x_2 - x_1 < m \cdot n$
      \\[0.2cm]
      sein.  Da $i$ eine nat\"{u}rliche Zahl ist, geht das nur, wenn $i = 0$ ist.  Wir haben also
      \\[0.2cm]
      \hspace*{1.3cm}
      $x_2 - x_1 = 0 \cdot m \cdot n = 0$
      \\[0.2cm]
      und folglich gilt $x_2 = x_1$. \qed
\end{enumerate}


\begin{Korollar}[Chinesischer Restesatz, 2.~Teil] \label{satz:china2} \lb
Sind $m,n \in \mathbb{N}$ mit $\textsl{ggt}(m,n) =1$ und definieren
 wir die Funktion 
\\[0.2cm]
\hspace*{1.3cm}
$\pi : \zn{m \cdot n} \rightarrow \zn{m} \times \zn{n}$ \quad durch \quad
$\pi(x) := \langle x \modulo m, x \modulo n \rangle$,
\\[0.2cm]
so ist Funktion $\pi$ bijektiv.  
\end{Korollar}

\proof
Wir zeigen Injektivit\"{a}t
und Surjektivit\"{a}t der Funktion getrennt.
\begin{enumerate}
\item Injektivit\"{a}t:  Es seien $x_1,x_2 \in \zn{m \cdot n}$ und es gelte $\pi(x_1) = \pi(x_2)$.  Nach
      Definition der Funktion $\pi$ gilt dann
      \\[0.2cm]
      \hspace*{1.3cm}
      $x_1 \modulo m = x_2 \modulo m$ \quad und \quad 
      $x_1 \modulo n = x_2 \modulo n$.
      \\[0.2cm]
      Wir definieren $a := x_1 \modulo m$ und $b := x_1 \modulo n$ und haben dann sowohl
      \\[0.2cm]
      \hspace*{1.3cm}
      $x_1 \approx_m a$ \quad und \quad $x_1 \approx_n b$
      \\[0.2cm]
      als auch
      \\[0.2cm]
      \hspace*{1.3cm}
      $x_2 \approx_m a$ \quad und \quad $x_2 \approx_n b$.
      \\[0.2cm]
      Nach dem Chinesischen Restesatz
      gibt es aber nur genau ein $x \in \zn{m \cdot n}$, welches die beiden Gleichungen
      \\[0.2cm]
      \hspace*{1.3cm}
      $x \approx_m a$ \quad und \quad $x \approx_n b$.
      \\[0.2cm]
      gleichzeitig erf\"{u}llt.  Folglich muss $x_1 = x_2$ sein.
\item Surjektivit\"{a}t: Nun sei $\langle a, b \rangle \in \zn{m} \times \zn{n}$ gegeben.  Wir m\"{u}ssen
      zeigen, dass es ein $x \in \zn{m \cdot n}$ gibt, so dass $\pi(x) = \langle a, b \rangle$
      gilt.  Nach dem Chinesischen Restesatz existiert ein $x \in \zn{m \cdot n}$, so dass
      \\[-0.2cm]
      \hspace*{1.3cm}
      $x \approx_m a$ \quad und \quad $x \approx_n b$
      \\[0.2cm]
      gilt.  Wegen $a \in \zn{m}$ und $b \in \zn{n}$ gilt $a \modulo m = a$ und $b \modulo n = b$
      und daher k\"{o}nnen wir die beiden Gleichungen auch 
      in der Form
      \\[0.2cm]
      \hspace*{1.3cm}
      $x \modulo m = a$ \quad und \quad $x \modulo n = b$
      \\[0.2cm]
      schreiben.  Damit gilt
      \\[0.2cm]
      \hspace*{1.3cm}
      $\pi(x) = \langle x \modulo m, x \modulo n \rangle = \langle a, b \rangle$
      \\[0.2cm]
      und der Beweis ist abgeschlossen.  \qed
\end{enumerate}

\exercise
Versuchen Sie den Chinesischen Restesatz so zu verallgemeinern, dass er f\"{u}r eine beliebige
Liste $[m_1, m_2, \cdots, m_k]$ von paarweise teilerfremden Zahlen gilt und beweisen
Sie den verallgemeinerten Satz.

\exercise
Implementieren Sie ein Programm, das mit Hilfe des Chinesischen Restesatzes Systeme von
Kongruenzen der Form
\\[0.2cm]
\hspace*{1.3cm}
$x \modulo m_1 = a_1$, 
$x \modulo m_2 = a_2$, $\cdots$,
$x \modulo m_k = a_k$
\\[0.2cm]
l\"{o}sen kann und l\"{o}sen Sie mit diesem Programm das folgende R\"{a}tsel.
\vspace*{0.2cm}

\begin{minipage}[t]{0.9\linewidth}
  A girl was carrying a basket of eggs, and a man riding a horse hit
  the basket and broke all the eggs. Wishing to pay for the damage, he
  asked the girl how many eggs there were. The girl said she did not
  know, but she remembered that when she counted them by twos, there
  was one left over; when she counted them by threes, there were two
  left over; when she counted them by fours, there were three left
  over; when she counted them by fives, there were four left; and when
  she counted them by sixes, there were five left over. Finally, when
  she counted them by sevens, there were none left over. `Well,' said
  the man, `I can tell you how many you had.' What was his answer?
\end{minipage}

\begin{Satz} \label{satz:china3}
  Es seien $m$ und $n$ positive nat\"{u}rliche Zahlen.
  Dann gilt f\"{u}r alle positiven nat\"{u}rlichen Zahlen $x$ die \"{A}quivalenz
  \\[0.2cm]
  \hspace*{1.3cm}
  $\textsl{ggt}(x, m \cdot n) = 1 \quad \Leftrightarrow \quad
   \textsl{ggt}(x, m) = 1 \;\wedge\; \textsl{ggt}(x, n) = 1.
  $
\end{Satz}

\proof
Dies folgt aus dem Fundamentalsatz der Arithmetik und dem Lemma von Euler: Ist $p$ ein
Primfaktor von $x$, so teilt $p$ das Produkt $m \cdot n$ genau dann, wenn es einen der
Faktoren teilt. \qed



\begin{Satz}[Produkt-Regel zur Berechnung der $\varphi$-Funktion] \lb
  Es seien $m$ und $n$ nat\"{u}rliche Zahlen gr\"{o}\3er als $1$ und es gelte $\textsl{ggt}(m,n) = 1$.
  Dann gilt
  \\[0.2cm]
  \hspace*{1.3cm}
  $\varphi(m \cdot n) = \varphi(m) \cdot \varphi(n)$.
\end{Satz}

\proof
Nach Definition der Eulerschen $\varphi$-Funktion m\"{u}ssen wir zeigen, dass unter den gegebenen
Voraussetzungen 
\\[0.2cm]
\hspace*{1.3cm}
$\textsl{card}(\zns{m \cdot n}) = \textsl{card}(\zns{m}) \cdot \textsl{card}(\zns{n})$
\\[0.2cm]
gilt.  Nach dem 2.~Teil des Chinesischen Restesatzes (Korollar \ref{satz:china2}) ist die Funktion
\\[0.2cm]
\hspace*{1.3cm}
$\pi : \zn{m \cdot n} \rightarrow \zn{m} \times \zn{n}$ \quad mit
$\pi(x) := \langle x \modulo m, x \modulo n \rangle$
\\[0.2cm]
eine Bijektion vom $\zn{m \cdot n}$ in das kartesische Produkt $\zn{m} \times \zn{n}$.  Offenbar gilt
\\[0.2cm]
\hspace*{1.3cm}
$\zns{m} \subseteq \zn{m}$, \quad
$\zns{n} \subseteq \zn{n}$, \quad und \quad
$\zns{m \cdot n} \subseteq \zn{m \cdot n}$.
\\[0.2cm]
Au\3erdem haben wir die folgende Kette von Schlussfolgerungen:
\\[0.2cm]
\hspace*{1.3cm}
$
\begin{array}[t]{lll}
            & x \in \zns{m \cdot n}  \\[0.2cm]
\Rightarrow & \textsl{ggt}(x, m \cdot n) = 1 
            & \mbox{nach Definition von $\zns{m \cdot n}$}                              \\[0.2cm]
\Rightarrow & \textsl{ggt}(x, m) = 1 \wedge \textsl{ggt}(x, n) = 1 
            & \mbox{Satz \ref{satz:china3}}                                             \\[0.2cm]
\Rightarrow & \textsl{ggt}(x \modulo m, m) = 1 \wedge \textsl{ggt}(x \modulo n, n) = 1 
              \\[0.2cm]
\Rightarrow & x \modulo m \in \zns{m} \quad \mbox{und} \quad x \modulo n \in \zns{n}  \\[0.2cm]
\Rightarrow & \langle x \modulo m, x \modulo n \rangle \in \zns{m} \times \zns{n}  \\[0.2cm]
\end{array}
$
\\[0.2cm]
Dies zeigt, dass die Funktion $\pi$ die Menge $\zns{m \cdot n}$ in das kartesische Produkt
$\zns{m} \times \zns{n}$ abbildet.  Haben wir umgekehrt ein Paar $\langle a, b \rangle \in \zns{m} \times \zns{n}$ gegeben, 
so zeigt zun\"{a}chst der Chinesische Restesatz, dass es ein $x \in \zn{m \cdot n}$ gibt, f\"{u}r das 
\\[0.2cm]
\hspace*{1.3cm}
$x \modulo m = a$ \quad und \quad $x \modulo n = b$ ist.
\\[0.2cm]
Weiter haben wir dann die folgende Kette von Schlussfolgerungen:
\\[0.2cm]
\hspace*{1.3cm}
$
\begin{array}[t]{cl}
            & a \in \zns{m} \wedge b \in \zns{n}                                        \\[0.2cm]
\Rightarrow & \textsl{ggt}(a, m) = 1 \wedge \textsl{ggt}(b, n) = 1                      \\[0.2cm]
\Rightarrow & \textsl{ggt}(x \modulo m, m) = 1 \wedge \textsl{ggt}(x \modulo n, n) = 1  \\[0.2cm]
\Rightarrow & \textsl{ggt}(x, m) = 1 \wedge \textsl{ggt}(x, n) = 1                      \\[0.2cm]
\Rightarrow & \textsl{ggt}(x, m \cdot n) = 1                                            \\[0.2cm]
\Rightarrow & x \in \zns{m \cdot n}                                                     
\end{array}
$
\\[0.2cm]
Dies zeigt, dass die Einschr\"{a}nkung der Funktion $\pi$ auf die Menge $\zns{m \cdot n}$ eine
surjektive Abbildung auf das kartesische Produkt $\zns{m} \times \zns{n}$ ist.  Da wir weiterhin
wissen, dass die Funktion $\pi$ injektiv ist, m\"{u}ssen die Mengen $\zns{m \cdot n}$ und 
 $\zns{m} \times \zns{n}$ die gleiche Anzahl von Elementen haben:
\\[0.2cm]
\hspace*{1.3cm}
$\textsl{card}(\zns{m \cdot n}) 
  = \textsl{card}(\zns{m} \times \zns{n}) 
  = \textsl{card}(\zns{m}) \cdot \textsl{card}(\zns{n}) 
$
\\[0.2cm]
Also gilt $\varphi(m \cdot n) = \varphi(m) \cdot \varphi(n)$. 
\qed


\section{Die S\"{a}tze von Fermat und Euler}
Der folgende Satz von Pierre de Fermat (160? - 1665) bildet die Grundlage verschiedener kryptografischer
Verfahren. 

\begin{Satz}[Kleiner Satz von Fermat] \hspace*{\fill} \\ 
Es sei $p$ eine Primzahl.  Dann gilt f\"{u}r jede Zahl $k \in \zns{p}$
\\[0.2cm]
\hspace*{1.3cm}
$k^{p-1} \approx_p 1$.  
\end{Satz}

\proof
Wir erinnern zun\"{a}chst an die Definition der multiplikativen Gruppe $\zns{p}$ als
\\[0.2cm]
\hspace*{1.3cm}
$\zns{p} := \{ x \in \zn{p} \mid \exists y \in \zn{p} : x \cdot y \approx_p 1 \}$.
\\[0.2cm]
Wir wissen nach Satz \ref{satz:phi_primzahl}, dass 
\\[0.2cm]
\hspace*{1.3cm}
$\textsl{card}(\zns{p}) = \varphi(p) = p - 1$
\\[0.2cm]
gilt.  Da die $0$ sicher kein Inverses bez\"{u}glich der Multiplikation modulo $p$
haben, m\"{u}ssen alle Zahlen aus der Menge $\{1, \cdots, p-1\}$ ein multiplikatives Inverses haben und
es gilt
\\[0.2cm]
\hspace*{1.3cm}
$\zns{p} = \{ 1, \cdots, p - 1\}$.
\\[0.2cm]
Diese Behauptung h\"{a}tten wir alternativ auch aus dem Satz \ref{satz:multiplikatives-inverses} folgern
k\"{o}nnen, denn f\"{u}r alle $x \in \{ 1, \cdots, p-1\}$ gilt $\textsl{ggt}(x, p) = 1$.

Als n\"{a}chstes definieren wir  eine Funktion
\\[0.2cm]
\hspace*{1.3cm}
$f: \zns{p} \rightarrow \zns{p}$ \quad durch \quad $f(l) = (k \cdot l) \modulo p$.
\\[0.2cm]
Zun\"{a}chst m\"{u}ssen wir zeigen, dass f\"{u}r alle $l \in \zns{p}$ tats\"{a}chlich 
\\[0.2cm]
\hspace*{1.3cm}
$f(l) \in \zns{p}$
\\[0.2cm]
gilt.  Dazu ist zu zeigen, dass $(k \cdot l) \modulo p \not= 0$ gilt, denn sonst h\"{a}tte $k \cdot l$
kein multiplikatives Inverses.  Falls $k \cdot l \approx_p 0$ w\"{a}re, dann w\"{a}re $p$ ein Teiler von $k \cdot l$.
Da $p$ eine Primzahl ist, m\"{u}sste $p$ dann entweder $k$ oder $l$ teilen, was wegen $k,l < p$ nicht
m\"{o}glich ist.

Als n\"{a}chstes zeigen wir, dass die Funktion $f$ injektiv ist.  Seien also $l_1, l_2 \in \zns{p}$
gegeben, so dass
\\[0.2cm]
\hspace*{1.3cm}
$f(l_1) = f(l_2)$
\\[0.2cm]
gilt.  Nach Definition der Funktion $f$ bedeutet dies
\\[0.2cm]
\hspace*{1.3cm}
$(k \cdot l_1) \modulo p = (k \cdot l_2) \modulo p$, 
\\[0.2cm]
was wir auch k\"{u}rzer als
\\[0.2cm]
\hspace*{1.3cm}
$k \cdot l_1 \approx_p k \cdot l_2$, 
\\[0.2cm]
schreiben k\"{o}nnen.  Da $k \in \zns{p}$ ist, gibt es ein multiplikatives Inverses $h$ zu $k$, f\"{u}r das
$h \cdot k \approx_p 1$ gilt.  Multiplizieren wir daher die obige Gleichung mit $h$, so erhalten wir
\\[0.2cm]
\hspace*{1.3cm}
$h \cdot k \cdot l_1 \approx_p h \cdot k \cdot l_2$, 
\\[0.2cm] 
woraus sofort
\\[0.2cm]
\hspace*{1.3cm}
$l_1 \approx_p l_2$ 
\\[0.2cm]
folgt.  Da sowohl $l_1$ als auch $l_2$ Elemente der Menge $\zns{p}$ sind, bedeutet dies $l_1 = l_2$ und
damit ist die Injektivit\"{a}t der Funktion $f$ gezeigt.
\vspace*{0.3cm}

Nun folgt eine Zwischen\"{u}berlegung, die wir gleich ben\"{o}tigen.
Ist allgemein $f: A \rightarrow B$ eine injektive Funktion, f\"{u}r die $n := \textsl{card}(A) = \textsl{card}(B)$ 
ist, so muss $f$ auch surjektiv sein, was wir anschaulich wie folgt einsehen k\"{o}nnen:
Wenn wir $n$ verschiedene Murmeln (die Elemente von $A$) auf $n$ Schubladen (die Elemente von $B$)
verteilen m\"{u}ssen und wir (wegen der Injektivit\"{a}t von $f$) niemals zwei Murmeln in dieselbe
Schublade legen d\"{u}rfen, dann m\"{u}ssen wir tats\"{a}chlich in jede Schublade eine Murmel legen und
letzteres hei\3t, dass $f$ surjektiv sein muss.
\vspace*{0.3cm}

Wir wenden nun die Zwischen\"{u}berlegung an:
Da $f$ eine Funktion von $\zns{p}$ nach $\zns{p}$ ist und trivialerweise $\textsl{card}(\zns{p})=\textsl{card}(\zns{p})$
gilt, k\"{o}nnen wir aus der Injektivit\"{a}t von $f$ auf die Surjektivit\"{a}t von $f$ schlie\3en.
\vspace*{0.2cm}

Der Schl\"{u}ssel des Beweises liegt in der Betrachtung des folgenden Produkts:
\\[0.2cm]
\hspace*{1.3cm}
$P := \prod\limits_{i=1}^{p-1} f(i) = f(1) \cdot f(2) \cdot \;\dots\; \cdot f(p-1)$.
\\[0.2cm]
Aufgrund der Tatsache, dass die Funktion $f:\zns{p} \rightarrow \zns{p}$ surjektiv ist, wissen  wir, dass
\\[0.2cm]
\hspace*{1.3cm}
$f(\zns{p}) = \zns{p}$
\\[0.2cm]
gilt. Schreiben  wir die Mengen auf beiden Seiten dieser Gleichung hin, so erhalten wir die Gleichung
\\[0.2cm]
\hspace*{1.3cm}
$\bigl\{ f(1), f(2), \cdots, f(p-1) \bigr\} = \bigl\{ 1, 2, \cdots, p-1 \bigr\}$.
\\[0.2cm]
Damit k\"{o}nnen wir das oben definierte Produkt $P$ auch anders schreiben, es gilt
\\[0.2cm]
\hspace*{1.3cm}
$f(1) \cdot f(2) \cdot \;\dots\; \cdot f(p-1) = 1 \cdot 2 \cdot \;\dots\; \cdot (p-1)$,
\\[0.2cm]
denn auf beiden Seiten haben wir alle Elemente der Menge $\zns{p}$ aufmultipliziert, lediglich die
Reihenfolge ist eine andere.  Setzen wir hier die Definition der Funktion $f$ ein, so
folgt zun\"{a}chst
\\[0.2cm]
\hspace*{1.3cm}
$\bigl((k \cdot 1) \modulo p\bigr) \cdot \bigl((k \cdot 2) \modulo p\bigr) \cdot \;\dots\; \cdot 
 \bigl((k \cdot (p-1)) \modulo p\bigr) =
  1 \cdot 2 \cdot \;\dots\; \cdot (p-1)$.
\\[0.2cm]
Da offenbar $(k \cdot i) \modulo p \approx_p k \cdot i$ gilt, folgt daraus
\\[0.2cm]
\hspace*{1.3cm}
$(k \cdot 1) \cdot (k \cdot 2) \cdot \;\dots\; \cdot \bigl(k\cdot(p-1)) \approx_p 
  1 \cdot 2 \cdot \;\dots\; \cdot (p-1)$.
\\[0.2cm]
Ordnen wir die Terme auf der linken Seite dieser Gleichung um, so folgt
\\[0.2cm]
\hspace*{1.3cm}
$k^{p-1} \cdot 1 \cdot \cdot 2 \cdot \;\dots\; \cdot \cdot(p-1) \approx_p 
  1 \cdot 2 \cdot \;\dots\; \cdot (p-1)
$.
\\[0.2cm]
Da die Zahlen $1$, $2$, $\cdots$, $p_1$ modulo $p$ ein multiplikatives Inverses haben, k\"{o}nnen diese Zahlen
auf beiden Seiten der Gleichung herausgek\"{u}rzt werden und wir erhalten
\\[0.2cm]
\hspace*{1.3cm}
$k^{p-1} \approx_p 1$.
\\[0.2cm]
Das war gerade die Behauptung. \qed

\begin{Korollar}
  Es sei $p$ eine Primzahl.  F\"{u}r alle $k \in \zn{p}$ gilt dann $k^p \approx_p k$.
\end{Korollar}

\proof
Falls $k \in \zns{p}$ ist, folgt die Behauptung, indem wir die Gleichung $k^{p-1} \approx_p 1$
mit $k$ multiplizieren.  Andernfalls gilt $k= 0$ und offenbar gilt $0^p = 0$. \qed
\vspace*{0.3cm}

Der kleine Satz von Fermat $a^{p-1} \approx_p 1$ l\"{a}sst sich auf den Fall, dass $p$ keine Primzahl ist,
verallgemeinern.  Es ist dann lediglich zu fordern, dass die Zahlen $a$ und $n$ teilerfremd sind und an
Stelle des Exponenten $p - 1$ tritt nun die $\varphi$-Funktion.   Diese Verallgemeinerung wurde von Leonhard
Euler (1707 -- 1783) gefunden.


\begin{Satz}[Satz von Euler] \hspace*{\fill} \\ 
Es sei $n \in \mathbb{N}$ und  $a \in \zns{n}$. Dann gilt
\\[0.2cm]
\hspace*{1.3cm}
$a^{\varphi(n)} \approx_n 1$.  
\end{Satz}

\proof
Wir gehen aus von der Definition von
\\[0.2cm]
\hspace*{1.3cm}
$\zns{n} := \{ x \in \zn{n} \mid \exists y \in \zn{n} : x \cdot y \approx_p 1 \}$
\\[0.2cm]
als der Menge aller der Zahlen, die ein multiplikatives Inverses bez\"{u}glich der Multiplikation modulo $n$
haben.  Wir erinnern au\3erdem daran, dass
\\[0.2cm]
\hspace*{1.3cm}
$\zns{n} := \{ x \in \zn{n} \mid \textsl{ggt}(x, n) = 1 \}$
\\[0.2cm]
gilt.  Nach Definition der $\varphi$-Funktion gilt
\\[0.2cm]
\hspace*{1.3cm}
$\textsl{card}(\zns{n}) = \varphi(n)$.
\\[0.2cm]
gilt.  Analog zum Beweis des Satzes von Fermat definieren wir eine Funktion
\\[0.2cm]
\hspace*{1.3cm}
$f: \zns{n} \rightarrow \zns{n}$ \quad durch \quad $f(l) = (a \cdot l) \modulo n$.
\\[0.2cm]
Zun\"{a}chst m\"{u}ssen wir zeigen, dass f\"{u}r alle $l \in \zns{n}$ tats\"{a}chlich 
\\[0.2cm]
\hspace*{1.3cm}
$f(l) \in \zns{n}$
\\[0.2cm]
gilt.  Dazu ist zu zeigen, dass $(a \cdot l) \modulo n \in \zns{n}$ ist.  
Dies folgt aber sofort aus Satz \ref{satz:china3}, denn
wegen $l \in \zns{n}$ und $a \in \zns{n}$ wissen wir, dass $\textsl{ggt}(l,n) = 1$ und
$\textsl{ggt}(a,n) = 1$ ist und nach Satz \ref{satz:china3} folgt dann auch 
$\textsl{ggt}(a \cdot l, n) = 1$, woraus $\textsl{ggt}\bigl((a \cdot l) \modulo n, n\bigr) = 1$
folgt und letzteres ist zu $(a \cdot l) \modulo n \in \zns{n}$ \"{a}quivalent.


Als n\"{a}chstes zeigen wir, dass die Funktion $f$ injektiv ist.  Seien also $l_1, l_2 \in \zns{n}$
gegeben, so dass
\\[0.2cm]
\hspace*{1.3cm}
$f(l_1) = f(l_2)$
\\[0.2cm]
gilt.  Nach Definition der Funktion $f$ bedeutet dies
\\[0.2cm]
\hspace*{1.3cm}
$(a \cdot l_1) \modulo n = (a \cdot l_2) \modulo n$, 
\\[0.2cm]
was wir auch k\"{u}rzer als
\\[0.2cm]
\hspace*{1.3cm}
$a \cdot l_1 \approx_n a \cdot l_2$, 
\\[0.2cm]
schreiben k\"{o}nnen.  Da $a \in \zns{n}$ ist, gibt es ein multiplikatives Inverses $b$ zu $a$, f\"{u}r das
$b \cdot a \approx_n 1$ gilt.  Multiplizieren wir daher die obige Gleichung mit $b$, so erhalten wir
\\[0.2cm]
\hspace*{1.3cm}
$b \cdot a \cdot l_1 \approx_n b \cdot a \cdot l_2$, 
\\[0.2cm] 
woraus wegen $b \cdot a \approx_n 1$ sofort
\\[0.2cm]
\hspace*{1.3cm}
$l_1 \approx_n l_2$ 
\\[0.2cm]
folgt.  Da sowohl $l_1$ als auch $l_2$ Elemente der Menge $\zns{n}$ sind, folgt $l_1 = l_2$ und
damit ist die Injektivit\"{a}t der Funktion $f$ gezeigt.
\vspace*{0.3cm}


Genau wie im Beweis des kleinen Satzes von Fermat folgern wir nun aus der Injektivit\"{a}t der Funktion $f$, dass
$f$ auch surjektiv sein muss und betrachten das folgende Produkt:
\\[0.2cm]
\hspace*{1.3cm}
$P := \prod\limits_{i \in \zns{n}} f(i)$.
\\[0.2cm]
Aufgrund der Tatsache, dass die Funktion $f:\zns{n} \rightarrow \zns{n}$ surjektiv ist, wissen  wir, dass
\\[0.2cm]
\hspace*{1.3cm}
$f(\zns{n}) = \zns{n}$
\\[0.2cm]
gilt. Daher k\"{o}nnen wir $P$ auch einfacher berechnen, es gilt
\\[0.2cm]
\hspace*{1.3cm}
$P = \prod\limits_{i \in \zns{n}} i$,
\\[0.2cm] 
die beiden Darstellungen von $P$ unterscheiden sich nur in der Reihenfolge der Faktoren.
Damit haben wir
\\[0.2cm]
\hspace*{1.3cm}
$\prod\limits_{i \in \zns{n}} f(i) = \prod\limits_{i \in \zns{n}} i$.
\\[0.2cm]
Auf der linken Seite setzen wir nun die Definition von $f$ ein und haben dann
\\[0.2cm]
\hspace*{1.3cm}
$\prod\limits_{i \in \zns{n}} (a \cdot i) \modulo n = \prod\limits_{i \in \zns{n}} i$,
\\[0.2cm]
woraus
\\[0.2cm]
\hspace*{1.3cm}
$a^{\textsl{card}(\zns{n})} \cdot \prod\limits_{i \in \zns{n}} i \approx_n \prod\limits_{i \in \zns{n}} i$
\\[0.2cm]
folgt.  K\"{u}rzen wir nun das Produkt $\prod\limits_{i \in \zns{n}} i$ auf beiden Seiten dieser Gleichung 
weg und ber\"{u}cksichtigen, dass $\textsl{card}(\zns{n}) = \varphi(n)$ ist, so haben wir die Gleichung
\\[0.2cm]
\hspace*{1.3cm}
$a^{\varphi(n)} \approx_n 1$
\\[0.2cm]
bewiesen.  \qed

\section{Der RSA-Algorithmus}
In diesem Abschnitt sehen wir, wozu die $\varphi$-Funktion n\"{u}tzlich ist:  Wir pr\"{a}sentieren den
Algorithmus von Rivest, Shamir und Adleman \cite{rivest:78} (kurz: RSA-Algorithmus), der zur Erstellung
digitaler Signaturen verwendet werden kann.

Der RSA-Algorithmus beginnt damit, dass wir zwei \emph{gro\3e} Primzahlen $p$ und $q$ mit $p \not= q$
erzeugen.  \emph{Gro\3} hei\3t in diesem Zusammenhang, dass zur Darstellung der beiden Zahlen $p$ und $q$
jeweils mehrere hundert Stellen ben\"{o}tigt werden.  Anschlie\3end bilden wir das Produkt
\\[0.2cm]
\hspace*{1.3cm}
$n := p \cdot q$.
\\[0.2cm]
Das Produkt $n$ machen wir \"{o}ffentlich bekannt, aber die beiden Primzahlen $p$ und $q$ bleiben geheim.
Weiter berechnen wir
\\[0.2cm]
\hspace*{1.3cm}
$\varphi(n) = \varphi(p \cdot q) = (p-1) \cdot (q-1)$
\\[0.2cm]
und suchen eine nat\"{u}rliche Zahl $e < (p-1) \cdot (q-1)$, so dass 
\\[0.2cm]
\hspace*{1.3cm}
$\textsl{ggt}\bigl(e, (p-1) \cdot (q-1)\bigr) = 1$
\\[0.2cm]
gilt.  Die Zahl $e$ wird wieder \"{o}ffentlich bekannt gemacht.
Aufgrund der Tatsache, dass die beiden Zahlen $e$ und $(p-1) \cdot (q-1)$
teilerfremd sind, gilt $e \in \zns{(p-1) \cdot (q-1)}$ und damit hat die Zahl $e$ ein
multiplikatives Inverses $d$ modulo  $(p-1)\cdot(q-1)$, es gilt also
\\[0.2cm]
\hspace*{1.3cm}
$d \cdot e \approx_{(p-1)\cdot(q-1)} 1$.
\\[0.2cm]
Wir erinnern an dieser Stelle daran, dass die Zahl $d$ mit Hilfe des erweiterten Euklid'schen Algorithmus
berechnet werden kann, denn da $\textsl{ggt}\bigl(e, (p-1)\cdot(q-1)\bigr) = 1$ ist, liefert der
Euklid'sche Algorithmus Zahlen $\alpha$ und $\beta$, f\"{u}r die
\\[0.2cm]
\hspace*{1.3cm}
$\alpha \cdot e + \beta \cdot (p-1) \cdot (q-1) = 1$
\\[0.2cm]
gilt.  Definieren wir $d := \alpha \modulo \bigl((p-1) \cdot (q-1)\bigr)$, so sehen wir, dass in der Tat
\\[0.2cm]
\hspace*{1.3cm}
$d \cdot e \approx_{(p-1)\cdot(q-1)} 1$.
\\[0.2cm]
gilt.  Die Zahl $d$ bleibt geheim.
Wegen der letzten Gleichung gibt es ein $k \in \mathbb{N}$, so dass
\\[0.2cm]
\hspace*{1.3cm}
$d \cdot e = 1 + k \cdot (p-1) \cdot (q-1)$
\\[0.2cm]
gilt.  
Wir definieren eine \emph{Verschl\"{u}sselungs-Funktion}
\\[0.2cm]
\hspace*{1.3cm}
$f: \zn{n} \rightarrow \zn{n}$ \quad durch \quad $f(x) := x^e \modulo n$.
\\[0.2cm]
Weiter definieren wir eine Funktion
\\[0.2cm]
\hspace*{1.3cm}
$g: \zn{n} \rightarrow \zn{n}$ \quad durch \quad $g(x) := x^d \modulo n$.
\\[0.2cm]
Wir behaupten, dass f\"{u}r alle $x$, die kleiner als $m := \min(p,q)$ sind, 
\\[0.2cm]
\hspace*{1.3cm}
$g\bigl(f(x)\bigr) = x$ 
\\[0.2cm]
gilt.  Dies rechnen wir wie folgt nach:
\\[0.2cm]
\hspace*{1.3cm}
$
\begin{array}[t]{lcl}
  g\bigl(f(x)\bigr) & = & g\bigl(x^e \modulo n\bigr)                      \\[0.2cm]
                    & = & \bigl(x^e \modulo n\bigr)^d \modulo n           \\[0.2cm]
                    & = & x^{e \cdot d} \modulo n                         \\[0.2cm]
                    & = & x^{1 + k \cdot (p-1) \cdot (q-1)} \modulo n     \\[0.2cm]
                    & = & x \cdot x^{k \cdot (p-1) \cdot (q-1)} \modulo n 
\end{array}
$
\\[0.2cm]
Um den Beweis abzuschlie\3en, zeigen wir, dass
\\[0.2cm]
\hspace*{1.3cm}
$x^{k \cdot (p-1) \cdot (q-1)} \modulo n = 1$
\\[0.2cm]
ist.   Da $x < \textsl{min}(p,q)$ gilt und $n = p \cdot q$ ist, haben wir $\textsl{ggt}(x,n) = 1$.
Daher gilt nach dem Satz von Euler 
\\[0.2cm]
\hspace*{1.3cm}
$x^{\varphi(n)} \approx_n 1$.
\\[0.2cm]
Da $n = p \cdot q$ ist und da $p$ und $q$ als verschiedene Primzahlen sicher teilerfremd sind, wissen
wir, dass
\\[0.2cm]
\hspace*{1.3cm}
$\varphi(n) = \varphi(p \cdot q) = \varphi(p) \cdot \varphi(q) = (p-1) \cdot (q-1)$
\\[0.2cm]
gilt.  Damit folgt aus dem Satz von Euler, dass
\begin{equation}
  \label{eq:rsa3}
  x^{(p-1) \cdot (q-1)} \approx_n 1
\end{equation}
gilt, woraus sofort 
\\[0.2cm]
\hspace*{1.3cm}
$x^{k \cdot (p-1) \cdot (q-1)} \approx_n 1$
\\[0.2cm]
folgt.  Diese Gleichung k\"{o}nnen wir auch als
\\[0.2cm]
\hspace*{1.3cm}
$x^{k \cdot (p-1) \cdot (q-1)} \modulo n = 1$
\\[0.2cm]
schreiben.  Multiplizieren wir diese Gleichung mit $x$ und ber\"{u}cksichtigen, dass $x \modulo n = x$, denn
$x < n$, so erhalten wir
\\[0.2cm]
\hspace*{1.3cm}
$g\bigl(f(x)\bigr) = x \cdot x^{k \cdot (p-1) \cdot (q-1)} \modulo n = x$
\\[0.2cm]
und damit kann $g(x)$ tats\"{a}chlich als \emph{Entschl\"{u}sselungs-Funktion} benutzt werden um aus dem 
kodierten Wert $f(x)$ den urspr\"{u}nglichen Wert $x$ zur\"{u}ckzurechnen.
\vspace*{0.3cm}

Der RSA-Algorithmus funktioniert nun wie folgt: 
\begin{enumerate}
\item Zun\"{a}chst wird die zu verschl\"{u}sselnde Nachricht in einzelne Bl\"{o}cke aufgeteilt, die
      jeweils durch Zahlen $x$ kodiert werden k\"{o}nnen, die kleiner als $p$ und $q$ sind.
\item Jede solche Zahl $x$ wird nun zu dem Wert $x^e \modulo n$ verschl\"{u}sselt:
      \\[0.2cm]
      \hspace*{1.3cm}
      $x \mapsto x^e \modulo n$.
\item Der Empf\"{a}nger der Nachricht kann aus dem verschl\"{u}sselten Wert $y = x^e \modulo n$ die
      urspr\"{u}ngliche Botschaft $x$ wieder entschl\"{u}sseln, indem er die Transformation
      \\[0.2cm]
      \hspace*{1.3cm}
      $y \mapsto y^d \modulo n$
      \\[0.2cm]
      durchf\"{u}hrt, denn wir hatten ja oben gezeigt, dass
      \\[0.2cm]
      \hspace*{1.3cm}
      $(x^e \modulo n)^d \modulo n = x$
      \\[0.2cm]
      ist.  Dazu muss er allerdings den Wert von $d$ kennen.  Dieser Wert von $d$ ist der geheime Schl\"{u}ssel.
\end{enumerate}
In der Praxis ist es so, dass die Werte von $n$ und $e$ ver\"{o}ffentlicht werden, der Wert
von $d$ bleibt geheim.   Um den Wert von $d$ zu berechnen, muss das Produkt 
$(p-1) \cdot (q-1)$ berechnet werden, was \"{u}brigens gerade $\varphi(n)$ ist.  Nun ist
bisher kein Algorithmus bekannt, mit dem ein Zahl $n$ effizient in Primfaktoren zerlegt
werden kann.  Daher kann die Zahl $d$ nur mit sehr hohen Aufwand bestimmt werden.
Folglich kann, da $n$ und $e$ \"{o}ffentlich bekannt sind, jeder eine Nachricht verschl\"{u}sseln,
aber nur derjenige, der auch $d$ kennt, kann die Nachricht wieder entschl\"{u}sseln.

Der RSA-Algorithmus kann auch zum digitalen Signieren eingesetzt werden.  Dazu bleibt $e$ geheim und
$d$ wird \"{o}ffentlich gemacht. Eine Nachricht $x$ wird dann als $f(x)$ verschl\"{u}sselt.  Diese Nachricht kann
jeder durch Anwendung der Funktion $x \mapsto g(x)$ wieder entschl\"{u}sseln, um aber eine gegebene Nachricht
$x$ als $f(x)$ zu verschl\"{u}sseln, bedarf es der Kenntnis von $e$.

%%% local Variables: 
%%% mode: latex
%%% TeX-master: "lineare-algebra"
%%% End: 
