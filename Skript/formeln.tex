\chapter{Pr\"adikatenlogische Formeln}
Der Begriff der \emph{pr\"adikatenlogischen Formel} wird in dieser Vorlesung eine zentrale
Rolle spielen.  Wir werden pr\"adikatenlogische Formeln als \emph{Abk\"urzungen} definieren.  
Zun\"achst motivieren wir die Verwendung solcher Formeln.

\section{Warum Formeln}
Betrachten wir einmal den folgenden mathematischen Text: 
\begin{center}
\begin{minipage}{14cm}
{\em 
  Addieren wir zwei Zahlen und bilden dann das Quadrat dieser Summe, so ist das Ergebnis dasselbe,
  wie wenn wir zun\"achst beide Zahlen einzeln quadrieren, diese Quadrate aufsummieren 
  und dazu noch das Produkt der beiden Zahlen zweifach hinzu addieren.
}
\end{minipage}
\end{center}
Der mathematische Satz, der hier ausgedr\"uckt wird, ist Ihnen aus der Schule bekannt,
es handelt sich um den ersten Binomischen Satz.  Um dies zu sehen, f\"uhren wir f\"ur die
in dem Text genannten zwei Zahlen die Variablen $a$ und $b$ ein und \"ubersetzen dann die 
in dem obigen Text auftretenden Teils\"atze in Terme.  Die folgende Tabelle zeigt diesen Prozess: \\[0.3cm]
\hspace*{1.3cm} 
\begin{tabular}{ll}
  \emph{Addieren wir zwei Zahlen} & $a+b$ \\
  \emph{bilden das Quadrat dieser Summe} & $(a+b)^2$ \\
  \emph{beide Zahlen einzeln quadrieren} & $a^2$, $b^2$ \\
  \emph{diese Quadrate aufsummieren} & $a^2 + b^2$ \\
  \emph{das Produkt der beiden Zahlen \ldots} & $a\cdot b$ \\
  \emph{\ldots zweifach hinzu addieren} & $a^2 + b^2 + 2\cdot a \cdot b $ \\
  \emph{} & $ $ \\
\end{tabular} \\
Insgesamt finden wir so, dass der obige Text zu der folgenden Formel \"aquivalent ist: \\[0.2cm]
\hspace*{1.3cm} $(a+b)^2 = a^2 + b^2 + 2\cdot a\cdot b$. \\[0.2cm]
F\"ur den mathematisch Ge\"ubten ist diese Formel offensichtlich leichter zu verstehen ist als
der oben angegebene Text.  Aber die Darstellung von mathematischen Zusammenh\"angen durch
Formeln bietet neben der verbesserten Lesbarkeit noch zwei weitere Vorteile:
\begin{enumerate}
\item Formeln sind \emph{manipulierbar}, d.h.~wir k\"onnen mit Formeln \emph{rechnen}.
      Au\3erdem lassen Formeln sich aufgrund ihrer vergleichsweise einfachen Struktur auch 
      mit Hilfe von Programmen bearbeiten und analysieren.  Beim heutigen Stand der Technik ist
      es hingegen nicht m\"oglich, nat\"urlichsprachlichen Text mit dem Rechner vollst\"andig
      zu analysieren und zu verstehen.
\item Dar\"uber hinaus l\"asst sich die Bedeutung von Formeln mathematisch definieren und
      steht damit zweifelsfrei fest.  Eine solche mathematische Definition der Bedeutung
      ist f\"ur nat\"urlichsprachlichen Text so nicht m\"oglich, da
      nat\"urlichsprachlicher Text oft mehrdeutig ist und die genaue Bedeutung nur
      aus dem Zusammenhang hervorgeht.
\end{enumerate}

\section{Formeln als Kurzschreibweise}
Nach dieser kurzen Motivation f\"uhren wir zun\"achst Formeln als Abk\"urzungen ein und stellen
der Reihe nach die Ingredienzen vor, die wir zum Aufbau einer Formel ben\"otigen. 
\begin{enumerate}
\item \emph{Variablen}

      Variablen dienen uns als Namen f\"ur verschieden Objekte.  Oben haben wir beispielsweise 
      f\"ur die beiden zu addierenden Zahlen die Variablen $a$ und $b$ eingef\"uhrt.  Die Idee bei
      der Einf\"uhrung einer Variable ist, dass diese ein Objekt bezeichnet, dessen Identit\"at 
      noch nicht feststeht.
\item \emph{Konstanten}

      Konstanten bezeichnen  Objekte, deren Identit\"at schon feststeht. 
      In der Mathematik werden beispielsweise Zahlen wie $1$ oder $\pi$ als Konstanten verwendet.
      W\"urden wir Aussagen \"uber den biblischen Stammbaum als Formeln darstellen, so w\"urden
      wir  \texttt{Adam} und \texttt{Eva} als Konstanten verwenden.

      \begin{center}
      \begin{minipage}{13.4cm}  
        { \footnotesize
          \setlength{\baselineskip}{9pt} 
          Dieses letzte Beispiel mag Sie vielleicht verwundern, weil Sie davon
          ausgehen, dass Formeln nur dazu benutzt werden, mathematische oder allenfalls
          technische Zusammenh\"ange zu beschreiben.  Der logische Apparat ist aber
          keineswegs auf eine Anwendung in diesen Bereichen beschr\"ankt.  Gerade auch
          Sachverhalte aus dem t\"aglichen Leben lassen sich mit Hilfe von Formeln pr\"azise
          beschreiben.  Das ist auch notwendig, denn wir wollen ja sp\"ater unsere Formeln
          zur Analyse von Programmen benutzen und diese Programme werden sich durchaus
          auch mit der L\"osung von Problemen besch\"aftigen, die ihren Ursprung au\3erhalb der
          Technik haben. \par} 
      \end{minipage}
      \end{center}

      Variablen und Konstanten werden zusammenfassend auch als \emph{atomare Terme}
      bezeichnet.  Das Attribut \emph{atomar} bezieht sich hierbei auf die Tatsache,
      dass diese Terme sich nicht weiter in Bestandteile zerlegen lassen.  Im Gegensatz
      dazu stehen die \emph{zusammengesetzten Terme}.  Dies sind Terme, die mit Hilfe von 
      Funktions-Zeichen aus anderen Termen aufgebaut werden.

\item \emph{Funktions-Zeichen}

      Funktions-Zeichen benutzen wir, um aus Variablen und Konstanten neue Ausdr\"ucke aufzubauen,
      die wiederum Objekte bezeichnen.  In dem obigen Beispiel haben wir das Funktions-Zeichen
      ``$+$'' benutzt und mit diesem Funktions-Zeichen aus den Variablen $a$ und $b$ den Ausdruck $a+b$
      gebildet.  Allgemein nennen wir Ausdr\"ucke, die sich aus Variablen, Konstanten und Funktions-Zeichen 
      bilden lassen, \emph{Terme}.  

      Das Funktions-Zeichen ``$+$'' ist zweistellig, aber nat\"urlich gibt es auch einstellige und
      mehrstellige Funktions-Zeichen. Ein Beispiel aus der Mathematik f\"ur ein einstelliges Funktions-Zeichen ist 
      das Zeichen ``$\sqrt{\rule{0pt}{9pt}\quad}$''.  Ein weiteres Beispiel
      ist durch das Zeichen ``$\mathtt{\sin}$'' gegeben, dass in der Mathematik f\"ur die
      Sinus-Funktion verwendet wird.

      Allgemein gilt: Ist $f$ ein $n$-stelliges Funktions-Zeichen und sind 
       $t_1, \cdots, t_n$ Terme,
      so kann mit Hilfe des Funktions-Zeichen $f$  daraus der neue Term \\[0.2cm]
      \hspace*{1.3cm} $f(t_1,\cdots,t_n)$ \\[0.2cm]
      gebildet werden.  Diese Schreibweise, bei der zun\"achst das Funktions-Zeichen
      gefolgt von einer \"offnenden Klammer angegeben wird und anschlie\3end die Argumente
      der Funktion durch Kommata getrennt aufgelistet werden, gefolgt von einer
      schlie\3enden Klammer, ist der ``Normalfall''.  Diese Notation wird auch als \emph{Pr\"afix-Notation}
      bezeichnet. Bei einigen zweistelligen Funktions-Zeichen hat es sich aber eingeb\"urgert, diese in einer
      \emph{Infix-Notation} darzustellen, d.h.~solche Funktions-Zeichen werden zwischen
      die Terme geschrieben. In der Mathematik liefern die  Funktions-Zeichen ``$+$'',
      ``$-$'', ``$\cdot $'' und ``$/$'' hierf\"ur Beispiele.  Schlie\3lich gibt es noch Funktions-Zeichen,
      die auf ihr Argument folgen.  Ein Beispiel daf\"ur ist das Zeichen ``\texttt{!}'' zur
      Bezeichnung der Fakult\"at\footnote{
      F\"ur eine positive nat\"urliche Zahl $n$ ist die \emph{Fakult\"at} von $n$ als das Produkt aller
      nat\"urlichen Zahlen von $1$ bis $n$ definiert.  Die Fakult\"at von $n$ wird mit $n!$ bezeichnet, 
      es gilt also $n! = 1 \cdot 2 \cdot 3 \cdot \dots \cdot (n-1) \cdot n$.}
      denn f\"ur die Fakult\"at einer Zahl $n$ hat sich in der Mathematik
      die Schreibweise ``$n!$'' eingeb\"urgert.  Eine solche Notation wird als \emph{Postfix-Notation}
      bezeichnet.

      Benutzen wir Funktions-Zeichen nicht nur in der Pr\"afix-Notation, sondern auch als Infix- oder
      Postfix-Operatoren, so m\"ussen wir zus\"atzlich die \emph{Bindungsst\"arke} und
      \emph{Assoziativit\"at} dieser Operatoren festlegen.  Beispielsweise ist die Bindungsst\"arke des
      Operators ``$\cdot$'' in der \"ublichen Verwendung als Infix-Operator h\"oher als die
      Bindungsst\"arke des Operators ``$+$''.  Daher wird der Ausdruck
      \\[0.2cm]
      \hspace*{1.3cm}
      $1 + 2 \cdot 3$ \quad implizit in der Form \quad $1 + (2 \cdot 3)$
      \\[0.2cm]
      geklammert.  Weiter ist der Operator ``\texttt{-}'' links-assoziativ.  Daher wird der Ausdruck
      \\[0.2cm]
      \hspace*{1.3cm}
      $1 - 2 - 3$ \quad in der Form \quad $(1 - 2) - 3$
      \\[0.2cm]
      berechnet.
\item \emph{Pr\"adikate}

      Pr\"adikate stellen zwischen verschiedenen Objekten eine Beziehung her.  Ein wichtiges Pr\"adikat
      ist das Gleichheits-Pr\"adikat, dass durch das Gleichheits-Zeichen ``$=$'' dargestellt
      wird. Setzen wir zwei Terme $t_1$ und $t_2$ durch das Gleichheits-Zeichen in Beziehung,
      so erhalten wir die \emph{Formel} $t_1 = t_2$.

      Genau wie Funktions-Zeichen auch hat jedes Pr\"adikat eine vorgegebene \emph{Stelligkeit}.
      Diese gibt an, wie viele Objekte durch das Pr\"adikat in Relation gesetzt werden.  Im Falle des
      Gleichheits-Zeichens ist die Stelligkeit 2, aber es gibt auch Pr\"adikate mit anderen Stelligkeiten.
      Zum Beispiel k\"onnten wir ein Pr\"adikat ``\texttt{istQuadrat}'' definieren, dass f\"ur nat\"urliche
      Zahlen ausdr\"uckt, dass diese Zahl eine Quadrat-Zahl ist.  Ein solches Pr\"adikat w\"are dann
      einstellig.

      Ist allgemein $p$ ein $n$-stelliges Pr\"adikats-Zeichen und sind die
      Ausdr\"ucke $t_1, \cdots, t_n$
      Terme, so kann aus diesen Bestandteilen die \emph{Formel} \\[0.2cm]
      \hspace*{1.3cm} $p(t_1,\cdots,t_n)$ \\[0.2cm]
      gebildet werden.  Formeln von dieser Bauart bezeichnen wir auch als \emph{atomare Formel}, denn
      sie ist zwar aus Termen, nicht jedoch aus anderen Formeln zusammengesetzt.

      Genau wie bei zweistelligen  Funktions-Zeichen hat sich auch bei zweistelligen
      Pr\"adikats-Zeichen eine \emph{Infix-Notation} eingeb\"urgert.
      Das Pr\"adikats-Zeichen ``$=$'' liefert ein Beispiel hierf\"ur, denn wir schreiben
      ``$a=b$'' statt ``$=(a,b)$''.  Andere Pr\"adikats-Zeichen, f\"ur die sich eine
      Infix-Notation eingeb\"urgert hat, sind die Pr\"adikats-Zeichen ``$<$'', 
      ``$\leq$'', ``$>$'' und ``$\geq$'', die zum Vergleich von Zahlen benutzt werden. 
\item \emph{Junktoren} 

      Junktoren werden dazu benutzt, Formeln mit einander in Beziehung zu setzen.  Der einfachste Junktor
      ist das ``\emph{und}''. Haben wir zwei Formeln $F_1$ und $F_2$ und wollen ausdr\"ucken, dass sowohl
      $F_1$ als auch $F_2$ g\"ultig ist, so schreiben wir \\[0.2cm]
      \hspace*{1.3cm} $F_1 \wedge F_2$ \\[0.2cm]
      und lesen dies als ``$F_1$ \emph{und} $F_2$''.  Die nachfolgende Tabelle listet alle
      Junktoren auf, die wir verwenden werden: \\[0.2cm]
      \hspace*{1.3cm} 
      \begin{tabular}{|l|l|}
      \hline
      Junktor & Bedeutung \\
      \hline
      \hline
        $\neg F$ & nicht $F$ \\
      \hline
        $F_1 \wedge F_2$ & $F_1$ und $F_2$ \\
      \hline
        $F_1 \vee F_2$ & $F_1$ oder $F_2$ \\
      \hline
        $F_1 \rightarrow F_2$ & wenn $F_1$, dann $F_2$ \\
      \hline
        $F_1 \leftrightarrow F_2$ &  $F_1$ genau dann, wenn $F_2$ \\
      \hline
      \end{tabular}

      Hier ist noch zu bemerken, dass es bei komplexeren Formeln zur Vermeidung von Mehrdeutigkeiten
      notwendig ist, diese geeignet zu klammern.  Bezeichnen beispielsweise
      $P$, $Q$ und $R$ atomare Formeln,
      so k\"onnen wir unter Zuhilfenahme von Klammern daraus  die folgenden Formeln bilden: \\[0.2cm]
      \hspace*{1.3cm}  $P \rightarrow (Q \vee R)$ \quad und \quad $(P \rightarrow Q) \vee R$. \\[0.2cm]
      Umgangssprachlich w\"urden beide Formeln wie folgt interpretiert: 
      \begin{center}
      \begin{minipage}{12cm}
        \textsl{Aus $P$ folgt $Q$ oder $R$.}
      \end{minipage}
      \end{center}
      Die mathematische Schreibweise ist hier im Gegensatz zu der umgangssprachlichen Formulierung
      eindeutig.

      Die Verwendung von vielen Klammern vermindert die Lesbarkeit einer Formel.  Um
      Klammern einsparen zu k\"onnen, vereinbaren wir daher \"ahnliche Bindungsregeln, wie wir
      sie aus der Schulmathematik kennen.  Dort wurde vereinbart, dass ``$+$'' und ``$-$'' schw\"acher
      binden als  ``$\cdot $'' und ``$/$'' und damit ist gemeint, dass \\[0.2cm]
      \hspace*{1.3cm} $x + y \cdot z$  \quad als \quad $x + (y \cdot z)$ \\[0.2cm]
      interpretiert wird.  \"Ahnlich vereinbaren wir hier, dass ``$\neg$'' st\"arker bindet als ``$\wedge$''
      und ``$\vee$'' und dass diese beiden Operatoren st\"arker binden als
      ``$\rightarrow$''.  Schlie\3lich bindet der Operator ``$\leftrightarrow$'' 
      schw\"acher als alle anderen Operatoren.  Mit diesen Vereinbarungen lautet die Formel \\[0.2cm]
      \hspace*{1.3cm} $P \wedge Q \rightarrow R \leftrightarrow \neg R \rightarrow \neg P \vee \neg Q$ \\[0.2cm]
      dann in einer vollst\"andig geklammerten Schreibweise \\[0.2cm]
      \hspace*{1.3cm}  
      $\bigl((P \wedge Q) \rightarrow R\bigr) \leftrightarrow \bigl(\,(\neg R) \rightarrow ((\neg P) \vee (\neg Q))\,\bigr)$. 
      
     \item \emph{Quantoren}
      geben an, in welcher Weise eine Variable in einer Formel verwendet wird. Wir kennen zwei
      Quantoren, den All-Quantor ``$\forall$'' und den Existenz-Quantor ``$\exists$''.  Eine Formel der Form\\[0.2cm]
      \hspace*{1.3cm} $\forall x: F$ \\[0.2cm]
      lesen wir als ``\emph{f\"ur alle $x$ gilt $F$}'' und eine Formel der Form \\[0.2cm]
      \hspace*{1.3cm} $\exists x: F$ \\[0.2cm]
      wird als ``\emph{es gibt ein $x$, so dass $F$ gilt}'' gelesen.  In dieser Vorlesung
      werden wir \"ublicherweise \emph{qualifizierte Quantoren} verwenden.  Die Qualifizierung
      gibt dabei an, in welchem Bereich die durch die Variablen bezeichneten Objekte liegen m\"ussen.
      Im Falle des All-Quantors schreiben wir dann \\[0.2cm]
      \hspace*{1.3cm} $\forall x \in M: F$ \\[0.2cm]
      und lesen dies als ``\emph{f\"ur alle $x$ aus $M$ gilt $F$}''.  Hierbei bezeichnet $M$
      eine Menge. Dies ist nur eine abk\"urzende Schreibweise,
      die wir wie folgt definieren k\"onnen: \\[0.2cm]
      \hspace*{1.3cm} $\forall x \in M: F \stackrel{de\!f}{\Longleftrightarrow} \forall x\colon (x\in M \rightarrow F)$ 

      Entsprechend
      lautet die Notation f\"ur den Existenz-Quantor \\[0.2cm]
      \hspace*{1.3cm}  $\exists x \in M: F$ \\[0.2cm]
      und das wird dann  als ``\emph{es gibt ein $x$ aus $M$, so dass $F$ gilt}'' gelesen.  
      Formal l\"asst sich das als \\[0.2cm]
      \hspace*{1.3cm} $\exists x \in M: F \stackrel{de\!f}{\Longleftrightarrow} \exists x\colon (x\in M \wedge F)$ \\[0.2cm]
      definieren.       Wir verdeutlichen die Schreibweisen durch ein Beispiel.  Die Formel \\[0.2cm]
      \hspace*{1.3cm} $\forall x \in \mathbb{R}: \exists n \in \mathbb{N} : n > x$ \\[0.2cm]
      lesen wir wie folgt:
      \begin{center}
        {\em
        \begin{minipage}{12cm}
          F\"ur alle $x$ aus $\mathbb{R}$ gilt: Es gibt ein $n$ aus $\mathbb{N}$, so dass
          $n$ gr\"o\3er als $x$ ist.
        \end{minipage}
        }
      \end{center}
      Hier steht $\mathbb{R}$ f\"ur die reellen Zahlen und $\mathbb{N}$ bezeichnet die nat\"urlichen Zahlen.
      Die obige Formel dr\"uckt also aus, dass es zu jeder reellen Zahl $x$ eine nat\"urlich Zahl $n$ gibt, 
      so dass $n$ gr\"o\3er als $x$ ist.

      Treten in einer Formel Quantoren und Junktoren gemischt auf, so stellt sich die
      Frage, was st\"arker bindet.  Wir vereinbaren, dass Quantoren st\"arker binden als
      Junktoren.  In der folgenden Formel sind die Klammern also notwendig: \\[0.2cm]
      \hspace*{1.3cm} 
      $\forall x \colon \bigl(p(x) \wedge q(x)\bigr)$. 
\end{enumerate}

\section{Beispiele f\"ur Terme und Formeln}
Um die Konzepte ``Term'' und ``Formel'' zu verdeutlichen, geben wir im Folgenden einige
Beispiele an.  Wir w\"ahlen ein Beispiel aus dem t\"aglichen Leben und geben Terme und Formeln an, die sich mit
Verwandtschaftsbeziehungen besch\"aftigen.  Wir beginnen damit, dass wir die Konstanten,
Variablen, Funktions-Zeichen und Pr\"adikats-Zeichen festlegen.
\begin{enumerate}
\item Als \emph{Konstanten} verwenden wir die W\"orter \\[0.2cm]
      \hspace*{1.3cm} ``\texttt{adam}'', ``\texttt{eva}'', ``\texttt{kain}'', ``\texttt{abel}'' und
      ``\texttt{lisa}''.
\item Als \emph{Variablen} verwenden wir die Buchstaben \\[0.2cm]
      \hspace*{1.3cm} ``$x$'', ``$y$'' und ``$z$''.
\item Als \emph{Funktions-Zeichen} verwenden wir die W\"orter \\[0.2cm]
      \hspace*{1.3cm} ``\texttt{vater}'' und ``\texttt{mutter}''. \\[0.2cm]
      Diese beiden Funktions-Zeichen sind einstellig. 
\item Als \emph{Pr\"adikats-Zeichen} verwenden wir die W\"orter \\[0.2cm]
      \hspace*{1.3cm} ``\texttt{bruder}'', ``\texttt{schwester}'', 
      ``\texttt{m\"annlich}'' und ``\texttt{weiblich}''. \\[0.2cm]
      Hier sind die Pr\"adikats-Zeichen ``\texttt{m\"annlich}'' und ``\texttt{weiblich}'' einstellig,
      w\"ahrend ``\texttt{bruder}'' und ``\texttt{schwester}'' zweistellig sind. 
      Als weiteres zweistelliges Pr\"adikats-Zeichen verwenden wir das Gleichheits-Zeichen ``$=$''.
\end{enumerate}
Eine solche Ansammlung von Konstanten,
Variablen, Funktions-Zeichen und Pr\"adikats-Zeichen bezeichnen wir auch als
\emph{Signatur}.  Wir geben zun\"achst einige Terme an, die sich mit dieser Signatur
bilden lassen:
\begin{enumerate}
\item ``\texttt{kain}'' ist ein Term, denn ``\texttt{kain}'' ist eine Konstante.
\item ``$\mathtt{vater}(\mathtt{kain})$'' ist ein Term, denn ``\texttt{kain}''
      ist ein Term und ``\texttt{vater}'' ist ein einstelliges Funktions-Zeichen.
\item ``$\mathtt{mutter}\bigl(\mathtt{vater}(\mathtt{kain})\bigr)$'' ist ein Term, denn ``$\mathtt{vater}(\mathtt{kain})$'' ist
      ein Term und ``\texttt{mutter}'' ist ein einstelliges Funktions-Zeichen,
\item ``$\texttt{m\"annlich}(\mathtt{kain})$'' ist eine Formel, denn
      ``\texttt{kain}'' ist ein Term und
      ``\texttt{m\"annlich}'' ist ein einstelliges Pr\"adikats-Zeichen.
\item ``$\texttt{m\"annlich}(\mathtt{lisa})$'' ist ebenfalls eine Formel, denn
      ``\texttt{lisa}'' ist ein Term. 

      Dieses Beispiel zeigt, dass Formeln durchaus auch falsch sein k\"onnen.  Die Frage, ob eine
      Formel wahr oder falsch ist, wird in diesem Kapitel noch nicht untersucht.  Stattdessen geht
      es in diesem Kapitel nur darum aufzuzeigen, wie Formeln gebildet werden.

      Die bisher gezeigten Formeln sind alle atomar.  Wir geben nun Beispiele f\"ur zusammengesetzte 
      Formeln.
\item Die Formel
      \\[0.2cm]
      \hspace*{1.3cm}
      $\mathtt{vater}(x) = \mathtt{vater}(y) \wedge \mathtt{mutter}(x) = \mathtt{mutter}(y)
         \rightarrow       \mathtt{bruder}(x,y) \vee \mathtt{schwester}(x,y)$
      \\[0.2cm]
      ist eine Formel, die aus den beiden Formeln \\[0.2cm]
      \hspace*{1.3cm}  $\mathtt{vater}(x) = \mathtt{vater}(y) \wedge \mathtt{mutter}(x) =
      \mathtt{mutter}(y)$ \quad 
      \\[0.0cm]
      und 
      \\[0.0cm]
      \hspace*{1.3cm}  $\mathtt{bruder}(x,y) \vee \mathtt{schwester}(x,y)$ 
      \\[0.2cm]
      durch Verwendung des Operators ``$\rightarrow$'' erzeugt wird.  
\item ``$\forall x\colon \forall y\colon \bigl(\mathtt{bruder}(x,y) \vee \mathtt{schwester}(x,y)\bigr)$''   ist eine Formel.
\end{enumerate}
Die Formel Nr.~7 ist intuitiv gesehen falsch. 
Auch die Formel Nr.~6 ist falsch, wenn wir davon ausgehen, dass niemand sein eigener
Bruder ist.
Um die Begriffe ``\emph{wahr}'' und ``\emph{falsch}'' f\"ur Formeln streng definieren zu k\"onnen,
ist es notwendig, die \emph{Interpretation} der verwendeten Signatur festzulegen. 
Anschaulich gesehen definiert eine \emph{Interpretation} die Bedeutung der \emph{Symbole},
also der Konstanten, Funktions- und Pr\"adikats-Zeichen, aus denen die Signatur besteht.
Exakt kann der Begriff aber erst angegeben werden, wenn Hilfsmittel aus der Mengenlehre
zur Verf\"ugung stehen.  Die Mengenlehre ist der Gegenstand des folgenden Kapitels.

\exercise
Nehmen Sie an, dass Sie nichts weiter \"uber Frau M\"uller wissen.  Ist eine der folgenden Aussagen
wahrscheinlicher als alle anderen Aussagen oder ist das nicht der Fall?  Begr\"unden Sie Ihre Aussage!
\begin{enumerate}[(a)]
\item Frau M\"uller spielt Klavier und arbeitet in einer Bank.
\item Frau M\"uller ist katholisch und arbeitet in einer Bank.
\item Frau M\"uller ist evangelisch und arbeitet in einer Bank.
\item Frau M\"uller hat zwei Kinder und arbeitet in einer Bank.
\item Frau M\"uller arbeitet in einer Bank.
\item Frau M\"uller arbeitet in der Commerzbank. \eox
\end{enumerate}

\exercise
Nehmen Sie an, dass Sie nichts weiter \"uber Frau Meyer wissen.  Ist eine der folgenden Aussagen
wahrscheinlicher als alle anderen Aussagen oder ist das nicht der Fall?  Begr\"unden Sie Ihre Aussage!
\begin{enumerate}[(a)]
\item Frau Meyer spielt Klavier und arbeitet in einer Bank.
\item Frau Meyer ist katholisch oder Frau Meyer arbeitet in einer Bank.
\item Frau Meyer ist katholisch.
\item Frau Meyer hat zwei Kinder und arbeitet in einer Bank.
\item Frau Meyer arbeitet in einer Bank.
\item Frau Meyer arbeitet in der Commerzbank. \eox
\end{enumerate}


\exercise
\begin{enumerate}[(a)]
\item Das Pr\"adikat $\textsl{isPrime}(n)$ soll f\"ur eine nat\"urliche Zahl $n \in \mathbb{N}$ genau dann
      wahr sein, wenn $n$ eine \href{https://de.wikipedia.org/wiki/Primzahl}{Primzahl} ist.  
      Dabei ist eine Primzahl als eine nat\"{u}rliche Zahl definiert, die genau zwei verschiedene Teiler hat.
      Definieren Sie dieses Pr\"adikat mit Hilfe einer
      geeigneten Formel.  Sie d\"urfen bei der Erstellung der Formel das Funktions-Zeichen 
      ``\texttt{\%}'' 
      benutzen.  F\"ur zwei nat\"urlichen Zahlen $a$ und $b$ berechnet $a \;\texttt{\%}\; b$ den Rest, der bei der
      Division von $a$ durch $b$ \"ubrig bleibt.  Beispielsweise gilt $13 \;\texttt{\%}\; 5 = 3$, denn
      $13 = 2 \cdot 5 + 3$.
      % isPrime(n) <=> forall x:(n % x = 0 => x = 1 || x = n)
\item Formalisieren Sie die Aussage 
      \\[0.2cm]
      \hspace*{1.3cm}
      ``Es gibt keine gr\"o\3te Primzahl.'' 
      \\[0.2cm]
      als pr\"adikatenlogische Formel.  Benutzen Sie dabei das in Teil (a) dieser Aufgabe definierte
      Pr\"adikatszeichen \textsl{isPrime}.  \eox
      % $\forall n \in \mathbb{N}: \exists p \in \mathbb{N}:\bigl(\textsl{isPrime}(p) \wedge n < p\bigr)$.
\end{enumerate}

\exercise
Der Hilbert'sche Existenz-Quantor ``$\exists!x$'' wird als ``es gibt genau ein $x$'' gelesen.
Falls $\phi(x)$ eine Formel ist, in der die Variable $x$ vorkommt, dann wird also der Ausdruck
\\[0.2cm]
\hspace*{1.3cm}
$\exists! x\!:\!\phi(x)$ \quad als \quad ``Es gibt genau ein $x$, so dass $\phi(x)$ gilt.''
\\[0.2cm]
gelesen.
Geben Sie eine Formel an, die zu der Formel $\exists! x\!:\!\phi(x)$ \"aquivalent ist, in der aber nur die
gew\"ohnlichen Quantoren $\forall$ und $\exists$ verwendet werden.  Zeigen Sie also, wie sich der
Hilbert'sche Existenz-Quantor definieren l\"asst.
\eox
%$\exists!x: \phi(x) \Leftrightarrow \exists x:\Bigl(\phi(x) \wedge \forall y:\bigl(\phi(y)\rightarrow x = y\bigr)\Bigr)$

%%% Local Variables: 
%%% mode: latex
%%% TeX-master: "lineare-algebra"
%%% End: 
