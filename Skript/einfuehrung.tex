\chapter{Einf\"{u}hrung}
Das vorliegende Skript ist die Grundlage der Mathematik-Vorlesung des ersten Semesters.
Einige Kapitel und Abschnitte in diesem Skript sind mit einem Stern ``$^*$'' markiert.  
Der dort vorgestellte Stoff ist f\"{u}r die Klausur nicht relevant.  Das Skript enth\"{a}lt diese Abschnitte
um eine der vielen Anwendungen der Mathematik, das \href{https://de.wikipedia.org/wiki/RSA-Kryptosystem}{RSA-Kryptosystem}, pr\"{a}sentieren zu k\"{o}nnen.  

\section{Motivation}
Bevor wir uns mit dem eigentlichen Stoff dieser Vorlesung, der Mathematik, befassen, m\"{o}chte ich
begr\"{u}nden, warum Sie als zuk\"{u}nftige Informatiker sich mit der Mathematik befassen m\"{u}ssen.  
\begin{enumerate}
\item Historisch ist die Informatik als Teilgebiet der Mathematik entstanden.  Dies wird schon an
      dem Namen  ``\emph{Informatik}''  deutlich, denn dieses Wort ist aus den beiden W\"{o}rtern
      ``\emph{Information}'' und ``\emph{Mathematik}'' gebildet worden ist, was  
      durch die Gleichung 
      \\[0.2cm]
      \hspace*{1.3cm}
      \href{https://de.wikipedia.org/wiki/Informatik#Etymologie}{$\texttt{Informatik} = \texttt{Information} + \texttt{Mathematik}$}
      \\[0.2cm]
      symbolisiert wird.  Aufgrund der Tatsache, dass die Informatik aus der Mathematik entstanden
      ist, bedient sich die Informatik an vielen Stellen
      mathematischer Sprech- und Denkweisen.  Um diese verstehen zu k\"{o}nnen, ist eine 
      Vertrautheit mit der formalen mathematischen Denkweise unabdingbar.
\item Mathematik schult das abstrakte Denken und genau das wird in der Informatik ebenfalls
      ben\"{o}tigt.  Ein komplexes Software-System, das von hunderten von Programmierern \"{u}ber Jahre
      hinweg entwickelt wird, ist nur durch die Einf\"{u}hrung geeigneter Abstraktionen beherrschbar.
      Die F\"{a}higkeit, abstrakt denken zu k\"{o}nnen, ist genau das, was einen Mathematiker auszeichnet.
      Eine M\"{o}glichkeit,  diese F\"{a}higkeit zu erwerben besteht darin, sich mit den abstrakten
      Gedankengeb\"{a}uden, die in der Mathematik konstruiert werden, auseinander zu setzen.
\item Es gibt eine Vielzahl von mathematischen Methoden, die unmittelbar in der Informatik
      angewendet werden.  In dieser Vorlesung zeigen wir exemplarisch eine Reihe von Problemen, die
      ihren Ursprung in der Informatik haben, deren L\"{o}sung aber mathematische Methoden erfordert:
      \begin{enumerate}
      \item \href{https://en.wikipedia.org/wiki/Recurrence_relation}{\emph{Rekurrenz-Gleichungen}}
            sind Gleichungen, durch die Folgen von Zahlen definiert werden.  Beispielsweise k\"{o}nnen die
            \href{https://de.wikipedia.org/wiki/Fibonacci-Folge}{\emph{Fibonacci-Zahlen}} durch die
            Rekurrenz-Gleichung 
            \\[0.2cm]
            \hspace*{1.3cm}
            $a_{n+2} = a_{n+1} + a_n$  \quad und die Anfangs-Bedingungen $a_0 = 0$ und $a_1 = 1$
            \\[0.2cm]
            definiert werden.  Wir k\"{o}nnen mit der oberen Rekurrenz-Gleichung sukzessiv die verschiedenen
            Werte der Folge $(a_n)_n$ berechnen und finden 
            \\[0.2cm]
            \hspace*{1.3cm}
            $a_0 = 0$, $a_1 = 1$, $a_2 = 1$, $a_3 = 2$, $a_4 = 3$, $a_5 = 5$, $a_6 = 8$, $a_7 = 13$, $\cdots$.
            \\[0.2cm]
            Wir werden im Kapitel \ref{chapter:eigenwerte} sehen, dass es eine geschlossene Formel zur Berechnung der Fibonacci-Zahlen
            gibt, es gilt
            \\[0.2cm]
            \hspace*{1.3cm}
            $\ds a_n = \frac{1}{\sqrt{5}} \cdot 
            \left( 
                  \biggl(\frac{1 + \sqrt{5}}{2}\biggr)^n - \biggl(\frac{1 - \sqrt{5}}{2}\biggr)^n 
            \right)
            $.
            \\[0.2cm]
            Sie werden im Laufe der ersten beiden Semester verschiedene Verfahren kennen lernen, mit
            denen sich f\"{u}r in der Praxis auftretende Rekurrenz-Gleichungen geschlossene Formeln
            finden lassen.  Solche Verfahren sind wichtig bei der Analyse der Komplexit\"{a}t von
            Algorithmen, denn die Berechnung der Laufzeit rekursiver Algorithmen f\"{u}hrt auf
            Rekurrenz-Gleichungen.   
      \item Elementare Zahlentheorie bildet die Grundlage moderner kryptografischer Verfahren.
            Konkret beinhaltet dieses Skript eine Beschreibung des RSA-Algorithmus zur asymmetrischen
            Verschl\"{u}sselung.  Allerdings werden wir das Kapitel zur Zahlen-Theorie, in dem der 
            RSA-Algorithmus beschrieben wird, aus Zeitgr\"{u}nden nicht besprechen k\"{o}nnen, es handelt
            sich bei dem Kapitel also nur um Zusatzstoff, welchen Sie sich bei Bedarf selbst aneignen
            k\"{o}nnen.  Ich habe das Kapitel hier deswegen eingef\"{u}gt, damit Sie unmittelbar anhand
            eines konkreten Beispiels sehen k\"{o}nnen, welche Bedeutung die Mathematik in der
            Informatik hat.
      \item \href{https://de.wikipedia.org/wiki/Maschinelles_Lernen}{Maschinelles Lernen} ist ein Teilgebiet 
            der k\"{u}nstlichen Intelligenz und geh\"{o}rt 
            zu den Gebieten der Informatik, die zurzeit am st\"{a}rksten expandieren.  Viele der im Bereich
            des maschinellen Lernens angewendeten Methoden stammen aus der numerischen Mathematik
            und der Statistik.
      \end{enumerate}
      Die Liste der mathematischen Algorithmen, die in der Praxis eingesetzt werden, k\"{o}nnte leicht
      \"{u}ber mehrere Seiten fortgesetzt werden.  Nat\"{u}rlich k\"{o}nnen im Rahmen eines Bachelor-Studiums
      nicht alle mathematischen Verfahren, die in der Informatik eine Anwendung finden, auch
      tats\"{a}chlich diskutiert werden.  Das Ziel kann nur sein, Ihnen ausreichend mathematische
      F\"{a}higkeiten zu vermitteln, so dass Sie sp\"{a}ter in der Lage sind, sich die
      mathematischen Verfahren, die sie  im Beruf tats\"{a}chlich ben\"{o}tigen, selbstst\"{a}ndig anzueignen.
\item Mathematik schult das \emph{exakte Denken}.  Wie wichtig dieses ist, m\"{o}chte ich mit den
      folgenden Beispielen verdeutlichen:  
      \begin{enumerate}
      \item Am 9.~Juni 1996 st\"{u}rzte die Rakete Ariane 5 auf ihrem Jungfernflug ab.
            Ursache war eine Kette von Software-Fehlern:  Ein Sensor im Navigations-System
            der Ariane 5 misst die horizontale Neigung und speichert diese zun\"{a}chst als Gleitkomma-Zahl
            mit einer Genauigkeit von 64 Bit ab.  Sp\"{a}ter wird dieser Wert dann in eine 
            16 Bit Festkomma-Zahl konvertiert.
            Bei dieser Konvertierung trat ein \"{U}berlauf ein, da die zu konvertierende Zahl
            zu gro\3 war, um als 16 Bit Festkomma-Zahl dargestellt werden zu k\"{o}nnen.
            In der Folge gab das Navigations-System auf dem Datenbus, der dieses System mit
            der Steuerungs-Einheit verbindet, eine Fehlermeldung aus.
            Die Daten dieser Fehlermeldung wurden von der Steuerungs-Einheit als Flugdaten 
            interpretiert.  Die Steuer-Einheit leitete daraufhin eine Korrektur des
            Fluges ein, die dazu f\"{u}hrte, dass die Rakete auseinanderbrach und die 
            automatische Selbstzerst\"{o}rung eingeleitet werden musste.
            Die Rakete war mit 4 Satelliten beladen. Der wirtschaftliche Schaden, der durch den Verlust dieser
            Satelliten entstanden ist, lag bei mehreren 100 Millionen Dollar.
            
            Ein vollst\"{a}ndiger Bericht \"{u}ber die Ursache des Absturzes des Ariane 5 findet sich
            im Internet unter der Adresse \\[0.2cm]
            \hspace*{0.5cm} 
            \href{http://www.ima.umn.edu/~arnold/disasters/ariane5rep.html}{\texttt{http://www.ima.umn.edu/\symbol{126}arnold/disasters/ariane5rep.html}}.
      \item Die Therac 25 ist ein medizinisches Bestrahlungs-Ger\"{a}t, das durch 
            Software kontrolliert wird.  Durch  Fehler in dieser Software erhielten 1985
            mindestens 6 Patienten eine \"{U}berdosis an Strahlung.  Drei dieser Patienten sind an den
            Folgen dieser \"{U}berdosierung gestorben. 

            Einen detaillierten Bericht \"{u}ber diese Unf\"{a}lle finden Sie unter \\[0.2cm]
            \hspace*{0.5cm} 
            \href{http://courses.cs.vt.edu/~cs3604/lib/Therac_25/Therac\_1.html}{\texttt{http://courses.cs.vt.edu/\symbol{126}cs3604/lib/Therac\_25/Therac\_1.html}}.
      \item Im ersten Golfkrieg konnte eine irakische \textsl{Scud} Rakete von dem \textsl{Patriot}
            Flugabwehrsystem aufgrund eines Programmier-Fehlers in der Kontrollsoftware des Flugabwehrsystems
            nicht abgefangen werden.  28 Soldaten verloren dadurch ihr Leben, 100 weitere wurden
            verletzt. \\[0.2cm]
            \hspace*{0.5cm} 
            \href{http://www.ima.umn.edu/~arnold/disasters/patriot.html}{\texttt{http://www.ima.umn.edu/\symbol{126}arnold/disasters/patriot.html}}.
      \item Im Internet finden Sie auf der Seite der
            \href{https://de.wikipedia.org/wiki/Institute_of_Electrical_and_Electronics_Engineers}{\textsc{Ieee}}\footnote{
              Die Abk\"{u}rzung \textsc{Ieee} steht f\"{u}r \emph{Institute of Electrical and Electronics Engineers}.}
            den Artikel   
            \\[0.2cm]
            \hspace*{0.5cm}
            \href{http://spectrum.ieee.org/static/the-staggering-impact-of-it-systems-gone-wrong}{Staggering Impact of IT Systems Gone Wrong}.
            \\[0.2cm]
            In diesem Artikel werden die Kosten, die durch fehlgeschlagene IT-Projekte entstanden sind, f\"{u}r eine Reihe von Projekten aufgelistet.
      \end{enumerate}
      Diese Beispiele zeigen, dass bei der Konstruktion von IT-Systemen mit gro\3er Sorgfalt
      und Pr\"{a}zision gearbeitet werden sollte.  Die Erstellung von IT-Systemen muss auf einer 
      wissenschaftlich fundierten Basis erfolgen, denn nur dann ist es m\"{o}glich, die Korrektheit
      solcher Systeme zu \emph{verifizieren}, also mathematisch zu beweisen.
      Diese oben geforderte wissenschaftliche Basis f\"{u}r die Entwicklung von IT-Systemen ist die Informatik, 
      und diese hat ihre Wurzeln sowohl in der Mengenlehre als auch in der mathematischen
      Logik.  Diese beiden Gebiete werden uns daher im ersten Semester
      des Informatik-Studiums besch\"{a}ftigen.  Obwohl sowohl die Logik als auch die Mengenlehre zur
      Mathematik geh\"{o}ren, werden wir uns in dieser Mathematik-Vorlesung nur mit der Mengenlehre und
      der linearen Algebra besch\"{a}ftigen.  Die Behandlung der Logik erfolgt dann im Rahmen der
      Informatik-Vorlesung. 
\end{enumerate}

% Schlie\3lich gibt es f\"{u}r Sie noch einen sehr gewichtigen Grund, sich intensiv mit Logik und
% Mengenlehre zu besch\"{a}ftigen, den ich Ihnen nicht verschweigen m\"{o}chte:  Es handelt sich
% dabei um die Klausur am Ende des ersten Semesters!  

\section{\"{U}berblick}
Ich m\"{o}chte Ihnen zum Abschluss dieser Einf\"{u}hrung noch einen \"{U}berblick \"{u}ber all die Themen geben, die
ich im Rahmen der Vorlesung behandeln werde.  
\begin{enumerate}
\item Mathematische Formeln dienen der Abk\"{u}rzung.  Sie werden aus den \emph{Junktoren} 
      \begin{enumerate}
      \item $\wedge$ (``\emph{und}''),
      \item $\vee$ (``\emph{oder}''),
      \item $\neg$ (``\emph{nicht}''),
      \item $\rightarrow$ (``\emph{wenn $\cdots$, dann}'') und
      \item $\leftrightarrow$ (``\emph{genau dann, wenn}'') 
      \end{enumerate}
      sowie den \emph{Quantoren}
      \begin{enumerate}
      \item $\forall$ (``\emph{f\"{u}r alle}'') und
      \item $\exists$ (``\emph{es gibt}'') 
      \end{enumerate}
      aufgebaut.
      Wir werden  Junktoren und Quantoren zun\"{a}chst
      als reine Abk\"{u}rzungen einf\"{u}hren.  Im Rahmen der Logik-Vorlesung werden wir die Bedeutung
      und Verwendung von Junktoren und Quantoren weiter untersuchen.
\item Beweis-Prinzipien

      In der Informatik ben\"{o}tigen wir im Wesentlichen vier Arten von Beweisen:
      \begin{enumerate}
      \item Ein \emph{direkter Beweis} folgert eine zu beweisende Aussage mit Hilfe elementarer 
            logischer Schl\"{u}sse und algebraischer Umformungen.  Diese Art von Beweisen kennen Sie
            bereits aus der Schule.
      \item Bei einem \emph{Beweis durch Fallunterscheidung} teilen wir den Beweis dadurch auf, dass
            wir alle in einer bestimmten Situation m\"{o}glichen F\"{a}lle untersuchen und zeigen,
            dass die zu beweisende Aussage in jedem der F\"{a}lle wahr ist.  Beispielsweise k\"{o}nnen wir
            mit Hilfe einer Fallunterscheidung zeigen, dass die Zahl $n \cdot (n+1)$ f\"{u}r jede
            nat\"{u}rliche Zahl $n$ gerade ist.
      \item Ein \emph{indirekter Beweis} hat das Ziel zu zeigen, dass eine bestimmte Aussage $A$
            falsch ist.  Bei einem indirekten Beweis nehmen wir an, dass $A$ doch gilt und
            leiten aus dieser Annahme einen Widerspruch her.  Dieser Widerspruch zeigt uns dann,
            dass die Annahme $A$ nicht wahr sein kann.

            Beispielsweise werden wir mit Hilfe eines indirekten Beweises zeigen, dass 
            $\sqrt{2}$ keine rationale Zahl ist.  In der Vorlesung zur Logik zeigen wir dann, dass das
            \emph{Halte-Problem} unl\"{o}sbar ist:  Es ist nicht m\"{o}glich, eine Funktion \texttt{stops}
            zu implementieren, so dass f\"{u}r eine gegebene einstellige Funktion $f$ und eine Eingabe $x$
            der Aufruf
            \\[0.2cm]
            \hspace*{1.3cm}
            $\texttt{stops}(f,x)$
            \\[0.2cm]
            genau dann als Ergebnis \texttt{true} zur\"{u}ck liefert, wenn der Aufruf der Funktion $f$
            mit der Eingabe $x$ terminiert.
      \item Ein induktiver Beweis hat das Ziel, eine Aussage f\"{u}r alle nat\"{u}rliche Zahlen zu beweisen.
            Beispielsweise werden wir zeigen, dass die Summenformel
            \\[0.2cm]
            \hspace*{1.3cm}
            $\ds\sum\limits_{i=1}^n i = \frac{1}{2} \cdot n \cdot (n+1)$ 
            \\[0.2cm]
            f\"{u}r alle nat\"{u}rlichen Zahlen $n \in \mathbb{N}$ gilt.  Summenformeln dieser Art treten
            beispielsweise bei der Berechnung der Rechenzeit von Programmen auf.
      \end{enumerate}
\item Mengenlehre

      Die Mengenlehre bildet die Grundlage der modernen Mathematik.  Die meisten Lehrb\"{u}cher und
      Ver\"{o}ffentlichungen bedienen sich der Begriffsbildungen und Schreibweisen der Mengenlehre.
      Daher ist eine solide Grundlage an dieser Stelle f\"{u}r das weitere Studium unabdingbar.
\item Grundlagen der Algebra

      Wir besprechen \emph{Gruppen}, \emph{Ringe} und \emph{K\"{o}rper}.  Diese abstrakten Konzepte
      verallgemeinern  die Rechenregeln, die Sie von den reellen Zahlen kennen.  Sie bilden dar\"{u}ber
      hinaus die Grundlage f\"{u}r die lineare Algebra.
\item Zahlentheorie$^*$
  
      Dieses Skript enth\"{a}lt ein \underline{o}p\underline{tionales} Kapitel, in dem wir uns mit der 
      elementaren Zahlentheorie auseinandersetzen.  Die Zahlentheorie ist die Grundlage
      von vielen modernen Verschl\"{u}sselungs-Algorithmen.
\item Komplexe Zahlen

      Aus der Schule wissen Sie, dass die Gleichung
      \\[0.2cm]
      \hspace*{1.3cm}
      $x^2 = -1$
      \\[0.2cm]
      f\"{u}r $x \in \mathbb{R}$ keine L\"{o}sung hat.  Wir werden die Menge der reellen Zahlen $\mathbb{R}$
      zur Menge der komplexen Zahlen $\mathbb{C}$ erweitern und zeigen, dass jede quadratische
      Gleichung eine L\"{o}sung in der Menge der komplexen Zahlen hat.
\item Lineare Vektor-R\"{a}ume

      Die Theorie der \emph{linearen Vektor-R\"{a}ume} ist unter anderem die Grundlage f\"{u}r das L\"{o}sen von linearen
      Gleichungs-Systemen, linearen Rekurrenz-Gleichungen und linearen Differential-Gleichungen.
      Bevor wir uns also mit konkreten Algorithmen zur L\"{o}sung von Gleichungs-Systemen besch\"{a}ftigen
      k\"{o}nnen, gilt es also die Theorie der linearen Vektor-R\"{a}ume zu verstehen.
\item Lineare Gleichungs-Systeme

      Lineare Gleichungs-Systeme treten sowohl in der Informatik als auch in vielen anderen Gebieten auf. 
      Wir zeigen, wie sich solche Gleichungs-Systeme algorithmisch l\"{o}sen lassen.
\item Eigenwerte und Eigenvektoren

      Ist $A$ eine \emph{Matrix},  $\vec{x}$ ein Vektor, $\lambda$ eine Zahl und gilt dar\"{u}ber hinaus
      \\[0.2cm]
      \hspace*{1.3cm}
      $A \cdot \vec{x} = \lambda \cdot \vec{x}$
      \\[0.2cm]
      so nennen wir $\vec{x}$ ein Eigenvektor von der Matrix $A$ zum Eigenwert $\lambda$.
      
      Sie brauchen an dieser Stelle keine Angst haben: Im Laufe der Vorlesung werden den Begriff der
      \emph{Matrix} definieren  und die Frage, wie die Multiplikation $A \cdot \vec{x}$ der Matrix $A$ mit
      dem Vektor $\vec{x}$ definiert ist, wird ebenfalls noch gekl\"{a}rt.  Weiter werden wir sehen,
      wie Eigenvektoren berechnet werden k\"{o}nnen.
\item Rekurrenz-Gleichungen

      Die Analyse der Komplexit\"{a}t rekursiver Prozeduren f\"{u}hrt auf Rekurrenz-Gleichungen.
      Wir werden Verfahren entwickeln, mit denen sich solche Rekurrenz-Gleichungen l\"{o}sen lassen.
\end{enumerate}
\remark
Ich gehe davon aus,  dass das  Skript eine Reihe von Tippfehlern und m\"{o}glicherweise auch anderen
Fehlern enth\"{a}lt.  Ich m\"{o}chte Sie darum bitten, mir solche Fehler per Email unter der Adresse 
\\[0.2cm]
\hspace*{1.3cm}
\texttt{karl.stroetmann\symbol{64}dhbw-mannheim.de}
\\[0.2cm]
mitzuteilen.  Falls Sie sich mit \texttt{git} und \href{https://github.com}{\texttt{https://github.com}}
auskennen, d\"{u}rfen Sie mir auch gerne einen \texttt{pull}-Request schicken. 

\section{Literaturhinweise}
Zum Schluss dieser Einf\"{u}hrung m\"{o}chte ich noch einige Hinweise auf die Literatur geben.  Dabei m\"{o}chte
ich zwei B\"{u}cher besonders hervorheben:
\begin{enumerate}
\item Das Buch ``\emph{Set Theory and Related Topics}'' von Seymour Lipschutz \cite{lipschutz:1998} enth\"{a}lt den Stoff, der
      in diesem Skript in dem Kapitel \"{u}ber Mengenlehre abgehandelt wird.
\item Das Buch ``\emph{Linear Algebra}'' von Seymour Lipschutz und Marc Lipson \cite{lipschutz:2012}
      enth\"{a}lt den Stoff zur eigentlichen linearen Algebra.
\end{enumerate}
Beide B\"{u}cher enthalten eine gro\3e Anzahl von Aufgaben mit L\"{o}sungen, was gerade f\"{u}r den Anf\"{a}nger
wichtig ist.  Dar\"{u}ber hinaus sind die B\"{u}cher
sehr preiswert.  

Vor etwa 30 Jahren habe ich selbst die lineare Algebra aus den B\"{u}chern von Gerd Fischer
\cite{fischer:2008} und Hans-Joachim Kowalsky \cite{kowalsky:2003} gelernt, an denen ich mich auch
jetzt wieder orientiert habe.  Zus\"{a}tzlich habe ich in der Zwischenzeit das Buch 
``\emph{Linear Algebra Done Right}'' von Sheldon Axler \cite{axler:1997} gelesen, das sehr gut geschrieben ist
und einen alternativen Zugang zur linearen Algebra bietet, bei dem die Theorie der Determinanten
allerdings in den Hintergrund ger\"{a}t.  Der fachkundige Leser wird bei der Lekt\"{u}re dieses Skripts
unschwer Parallelen zu den oben zitierten Werken erkennen.   Demjenigen Leser, der sich mehr Wissen
aneignen m\"{o}chte als das, was in dem engen zeitlichen Rahmen dieser Vorlesung vermittelt werden kann,
m\"{o}chte ich auf die oben genannte Literatur verweisen, wobei mir pers\"{o}nlich die Darstellung des Buchs
von Sheldon Axler am besten gef\"{a}llt. 

In den vergangenen Jahren habe ich festgestellt, dass viele Studenten zum Teil erhebliche L\"{u}cken im Bereich der
Schulmathematik haben.  Diesen Studenten m\"{o}chte ich empfehlen, diese L\"{u}cken mit Hilfe der 
\href{https://www.khanacademy.org}{Khan Academy} zu schlie\3en.  

%%% Local Variables: 
%%% mode: latex
%%% TeX-master: "lineare-algebra"
%%% End: 
