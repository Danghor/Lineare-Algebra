\chapter{Mathematische Grundlagen} 
Die erste Informatik-Vorlesung legt die Grundlagen, die f\"{u}r das weitere Studium
der Informatik ben\"{o}tigt werden.  Bei diesen Grundlagen handelt es sich im Wesentlichen
um die \emph{mathematische Logik} und die \emph{Mengenlehre}.  
Als Anwendung dieser Grundlagen werden zwei Programmiersprachen vorgestellt:
\begin{enumerate}
\item \textsc{Setl} (\underline{set} \underline{l}anguage) ist eine mengenbasierte
      Programmiersprache, in der dem Programmierer 
      die Operationen der Mengenlehre zur Verf\"{u}gung gestellt werden.
\item \textsl{Prolog} (\underline{pro}gramming in \underline{log}ic) basiert
      auf pr\"{a}dikatenlogischen Konzepten.
\end{enumerate}
Da insbesondere die mathematische Logik und die
Mengenlehre sehr abstrakte Gebiete sind, bereiten sie erfahrungsgem\"{a}\3 vielen Studenten
Schwierigkeiten.  Der Umgang mit dieser, auf den ersten Blick trocken anmutenden Materie f\"{a}llt
auch deshalb schwer, weil zun\"{a}chst noch gar nicht klar ist, wozu Logik und Mengenlehre in der
Informatik \"{u}berhaupt ben\"{o}tigt werden.  Aus diesem Grunde m\"{o}chte ich an den Anfang dieser
Vorlesung eine Motivation stellen. Diese Motivation soll Ihnen zeigen, dass es zwingend
notwendig ist, den Entwurf von Software und Hardware auf eine solide wissenschaftliche
Grundlage zu stellen. 

\section{Motivation und \"{u}berblick}
Wir beginnen mit der Feststellung, dass informationstechnische Systeme 
(im Folgenden kurz als IT-Systeme bezeichnet) zu den komplexesten Systemen geh\"{o}ren, die
die Menschheit je entwickelt hat.  Das l\"{a}sst sich schon an dem Aufwand erkennen,
der bei der Erstellung von IT-Systemen anf\"{a}llt.  So sind im Bereich der Telekommunikations-Industrie
IT-Projekte, bei denen mehr als 1000 Entwickler \"{u}ber mehrere Jahre zusammenarbeiten,
die Regel.  Es ist offensichtlich, dass ein Scheitern solcher Projekte mit enormen
 Kosten verbunden ist.  Einige Beispiele m\"{o}gen dies verdeutlichen.
\begin{enumerate}
\item Am 9.~Juni 1996 st\"{u}rzte die Rakete Ariane 5 auf ihrem Jungfernflug ab.
      Ursache war ein Kette von Software-Fehlern:  Ein Sensor im Navigations-System
      der Ariane 5 misst die horizontale Neigung und speichert diese zun\"{a}chst als Gleitkomma-Zahl
      mit einer Genauigkeit von 64 Bit ab.  Sp\"{a}ter wird dieser Wert dann in eine 
      16 Bit Festkomma-Zahl konvertiert.
      Bei dieser Konvertierung trat ein \"{u}berlauf ein, da die zu konvertierende Zahl
      zu gro\3 war, um als 16 Bit Festkomma-Zahl dargestellt werden zu k\"{o}nnen.
      In der Folge gab das Navigations-System auf dem Datenbus, der dieses System mit
      der Steuerungs-Einheit verbindet, eine Fehlermeldung aus.
       Die Daten dieser Fehlermeldung wurden von der Steuerungs-Einheit als Flugdaten 
      interpretiert.  Die Steuer-Einheit leitete daraufhin eine Korrektur des
      Fluges ein, die dazu f\"{u}hrte, dass die Rakete auseinander brach und die 
      automatische Selbstzerst\"{o}rung eingeleitet werden musste.
      Die Rakete war mit 4 Satelliten beladen. Der wirtschaftliche Schaden, der durch den Verlust dieser
      Satelliten entstanden ist, lag bei mehreren 100 Millionen Dollar.
      
      Ein vollst\"{a}ndiger Bericht \"{u}ber die Ursache des Absturzes des Ariane 5 findet sich
      im Internet unter der Adresse \\[0.1cm]
      \hspace*{1.3cm} \texttt{http://www.ima.umn.edu/\symbol{126}arnold/disasters/ariane5rep.html}
\item Die Therac 25 ist ein medizinisches Bestrahlungs-Ger\"{a}t, das durch 
      Software kontrolliert wird.  Durch  Fehler in dieser Software erhielten 1985
      mindestens 6 Patienten eine \"{u}berdosis an Strahlung.  Drei dieser Patienten sind an den Folgen dieser 
      \"{u}berdosierung gestorben. 

      Einen detaillierten Bericht \"{u}ber diese Unf\"{a}lle finden Sie unter \\[0.1cm]
      \hspace*{1.3cm} \texttt{http://courses.cs.vt.edu/\symbol{126}cs3604/lib/Therac\_25/Therac\_1.html}      
\item Im ersten Golfkrieg konnte eine irakische \textsl{Scud} Rakete von dem \textsl{Patriot} Flugabwehrsystem
      aufgrund eines Programmier-Fehlers in der Kontrollsoftware des Flugabwehrsystems
      nicht abgefangen werden.  28 Soldaten verloren dadurch ihr Leben, 100 weitere wurden
      verletzt. \\[0.1cm]
      \hspace*{1.3cm} \texttt{http://www.ima.umn.edu/\symbol{126}arnold/disasters/patriot.html}
\item Im Internet finden Sie unter \\[0.1cm]
      \hspace*{1.3cm}
      \href{http://www.computerworld.com/article/2515483/enterprise-applications/epic-failures--11-infamous-software-bugs.html}{\texttt{http://www.computerworld.com/\\
              article/2515483/enterprise-applications/epic-failures--11-infamous-software-bugs.html}}
      \\[0.1cm]
      eine Auflistung von schweren Unf\"{a}llen, die auf Software-Fehler zur\"{u}ckgef\"{u}hrt werden konnten.
\end{enumerate}
Diese Beispiele zeigen, dass bei der Konstruktion von IT-Systemen mit gro\3er Sorgfalt
und Pr\"{a}zision gearbeitet werden sollte.  Die Erstellung von IT-Systemen muss auf einer 
wissenschaftlich fundierten Basis erfolgen, denn nur dann ist es m\"{o}glich, die Korrekheit
solcher Systeme zu \emph{verifizieren}, also mathematisch zu beweisen.
Diese oben geforderte wissenschaftliche Basis f\"{u}r die Entwicklung von IT-Systemen ist die Informatik, 
und diese hat ihre Wurzeln sowohl in der Mengenlehre als auch in der mathematischen
Logik.  Diese beiden Gebiete werden uns daher im ersten Semester
des Informatik-Studiums besch\"{a}ftigen.

Sowohl die Mengenlehre als auch die Logik haben unmittelbare praktische Anwendungen in der
Informatik.
\begin{enumerate}
\item Die Mengenlehre und die damit verbundene Theorie der Relationen bietet die Grundlage
      der Theorie der relationalen Datenbanken.  Au\3erdem basiert die Programmier-Sprache \textsc{Setl}
      auf der Mengenlehre.
\item Die Programmier-Sprache \textsl{Prolog}, die vorwiegend im Bereich der KI
      (K\"{u}nstliche Intelligenz) eingesetzt wird, setzt Konzepte der mathematischen
      Logik um.  
\end{enumerate}
Neben den unmittelbaren Anwendungen von Logik und Mengenlehre hat die Besch\"{a}ftigung mit
diesen beiden Gebiete aber noch eine andere, sehr wichtige Funktion:
Ohne die Einf\"{u}hrung geeigneter Abstraktionen sind komplexe Systeme nicht beherrschbar.
Kein Mensch ist in der Lage, alle Details eines Software-Systems, dass aus mehreren
$100\,000$ Programm-Zeilen besteht, zu verstehen.   Die einzige Chance um ein solches
System zu beherrschen besteht in der Einf\"{u}hrung geeigneter Abstraktionen.
Daher geh\"{o}rt ein \"{u}berdurchschnittliches Abstraktionsverm\"{o}gen zu den wichtigsten Werkzeugen
eines Informatikers.  Die Besch\"{a}figung mit Logik und Mengenlehre trainiert gerade dieses
abstrakte Denkverm\"{o}gen. 

% Schlie\3lich gibt es f\"{u}r Sie noch einen sehr gewichtigen Grund, sich intensiv mit Logik und
% Mengenlehre zu besch\"{a}ftigen, den ich Ihnen nicht verschweigen m\"{o}chte:  Es handelt sich
% dabei um die Klausur am Ende des ersten Semesters!  

Aus meiner Erfahrung wei\3 ich, dass einige der Studenten sich unter dem Thema Informatik etwas Anderes
vorgestellt haben als die Diskussion abstrakter Konzepte.  F\"{u}r diese Studenten ist die Beherrschung
einer Programmiersprache und einer dazugeh\"{o}rigen Programmierumgebung das Wesentliche der Informatik.
Nat\"{u}rlich ist die Beherrschung einer Programmiersprache f\"{u}r einen Informatiker unabdingbar.  Sie
sollten sich allerdings dar\"{u}ber im klaren sein, dass das damit verbundene Wissen sehr verg\"{a}nglich
ist, denn niemand kann heute sagen, in welcher Programmiersprache in 10 Jahren programmiert werden wird.
Im Gegensatz dazu sind die mathematischen Grundlagen der Informatik wesentlich best\"{a}ndiger.


\paragraph{\"{u}berblick \"{u}ber den Inhalt der Vorlesung:} 
Im Rest dieses Kapitels werden wir zun\"{a}chst den Begriff der 
\emph{pr\"{a}dikatenlogischen Formel} auf einer informellen Ebene einf\"{u}hren.  Auf dieser Ebene
werden wir pr\"{a}dikatenlogische Formeln zun\"{a}chst nur als Abk\"{u}rzungen vorstellen:
Die Sprache der Pr\"{a}dikaten-Logik bietet uns einen Weg, komplizierte
Zusammenh\"{a}nge pr\"{a}gnanter und k\"{u}rzer darzustellen, als dies mit den Mitteln der nat\"{u}rlichen
Sprache m\"{o}glich ist.  Um allerdings die Bedeutung pr\"{a}dikatenlogischer Formeln 
mathematisch pr\"{a}zisieren zu k\"{o}nnen, ben\"{o}tigen wir einige Grundbegriffe aus der
Mengenlehre, mit der wir uns im Rest des ersten Kapitels besch\"{a}ftigen.

Die Begriffs-Bildungen der Mengenlehre sind nicht sehr kompliziert, daf\"{u}r aber umso
abstrakter.  Um diese Begriffs-Bildungen konkreter werden zu lassen und dar\"{u}ber hinaus den
Studenten ein Gef\"{u}hl f\"{u}r die N\"{u}tzlichkeit der Mengenlehre zu geben, stellen wir im zweiten
Kapitel die Sprache \textsc{Setl2} vor.  Dies ist eine Programmier-Sprache, die auf der
Mengenlehre aufgebaut ist.  Neben den klassischen Datentypen wie Zahlen und Strings gibt
es hier als Datentypen zus\"{a}tzlich Mengen.  Dadurch ist es in \textsc{Setl2} m\"{o}glich,
Algorithmen in der Sprache der Mengenlehre zu formulieren.  Solche Algorithmen sind zwar
meistens nicht so effizient wie Implementierungen in einer klassischen
Programmier-Sprache, aber daf\"{u}r in der Regel wesentlich klarer (und damit schneller zu
implementieren) als beispielsweise ein entsprechendes \textsl{C}-Programm.  
Zus\"{a}tzlich hat
\textsc{Setl2} eine konzeptuelle \"{a}hnlichkeit mit Datenbank-Abfrage-Sprachen wie
beispielsweise \textsl{SQL}, so dass sich eine Vertrautheit mit den Konzepten dieser
Sprache auch sp\"{a}ter noch als n\"{u}tzlich erweist.

In dem dritten Kapitel widmen wir uns der \emph{Aussagen-Logik}.  Diese kann als ein Teil der
Pr\"{a}dikaten-Logik aufgefasst werden. Die Handhabung aussagenlogischer Formeln ist einfacher als die
Handhabung pr\"{a}dikatenlogischer Formeln.  Daher bietet sich die Aussagen-Logik gewissermassen als
Trainings-Objekt an um mit den Methoden der Logik vertraut zu werden.  Die Aussagen-Logik hat
gegen\"{u}ber der Pr\"{a}dikaten-Logik noch einen weiteren Vorteil: Sie ist \emph{entscheidbar}, d.h.~wir
k\"{o}nnen ein Programm schreiben, dass als Eingabe eine aussagenlogische Formel verarbeitet und welches
dann entscheidet, ob diese Formel g\"{u}ltig ist.  Ein solches Programm existiert f\"{u}r beliebige Formeln
der Pr\"{a}dikaten-Logik nicht.  Dar\"{u}ber hinaus gibt es in der Praxis eine Reihe von Problemen, die
bereits mit Hilfe der Aussagenlogik gel\"{o}st werden k\"{o}nnen.  Beispielsweise l\"{a}sst sich die Frage nach der
Korrektheit kombinatorischer digitaler Schaltungen auf die Entscheidbarkeit einer aussagenlogischen
Formel zur\"{u}ckf\"{u}hren.

Im vierten Kapitel behandeln wir die Pr\"{a}dikatenlogik und analysieren den Begriff
des pr\"{a}dikatenlogischen Beweises mit Hilfe eines \emph{Kalk\"{u}ls}.  Ein
\emph{Kalk\"{u}l} ist dabei ein formales Verfahren, einen mathematischen Beweis zu f\"{u}hren.
Ein solches Verfahren l\"{a}sst sich programmieren.  Der im vierten Kaptel vorgestellte
\emph{Resolutions-Kalk\"{u}l]} bildet  die Grundlage f\"{u}r die
Programmier-Sprache \textsl{Prolog}, deren Grundz\"{u}ge wir im f\"{u}nften Kapitel
skizzieren.  

Zum Schluss m\"{o}chte ich hier noch ein Paar Worte zum Gebrauch von neuer und alter
Rechtschreibung und der Verwendung von Spell-Checkern in diesem Skript sagen.
Dieses Skript wurde unter Verwendung strenger marktwirtschaftlicher Kriterien
erstellt.  Im Klartext hei\3t das: Zeit ist Geld und als Dozent an der DHBW hat man
weder das eine noch das andere.  Daher ist es sehr wichtig zu wissen, wo eine
zus\"{a}tzliche Investition von Zeit noch einen f\"{u}r die Studenten n\"{u}tzlichen Effekt
bringt und wo dies nicht der Fall ist.  Ich habe mich an aktuellen
Forschungs-Ergebnissen zum Nutzen der Rechtschreibung orientiert. Diese zeigen,
dass es nicht wichtig ist, in welcher Reihenfolge die Bcushatebn in eniem Wrot
setehn, das eniizge was wihtcig ist, ist dass der esrte und der ltzete Bcusthabe
an der rcihitgen Psoiiton sthet. Der Rset knan ein ttolaer B\"{o}ldisnn sien,
trtodzem knan man ihn onhe Porbelme lseen. Das ist so, wiel wir nciht jdeen
Buhctsaben eniezln lseen, snoedrn das Wrot als gseatmes.  Wie sie sheen, ist das
tastc\"{a}hilch der Flal. $\displaystyle\smiley$




%%% Local Variables: 
%%% mode: latex
%%% TeX-master: "lineare-algebra"
%%% ispell-local-dictionary: "deutsch8"
%%% End: 
