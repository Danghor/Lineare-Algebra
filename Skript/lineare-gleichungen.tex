\chapter{Lineare Gleichungs-Systeme}
Wir wollen uns in diesem Kapitel mit der L\"{o}sung \emph{linearer Gleichungs-Systeme} besch\"{a}ftigen.
Ein \emph{Gleichungs-System} ist dabei einfach eine Menge von Gleichungen, in denen verschiedene
Variablen auftreten.  Bezeichnen wir die Variablen mit $x_1$, $\cdots$, $x_m$ und liegen die Gleichungen
in der Form
\\[0.2cm]
\hspace*{1.3cm}
$\sum\limits_{i=1}^n a_{i,j} \cdot x_j = b_i$ \quad f\"{u}r alle $i=1,\cdots,m$
\\[0.2cm]
vor, wobei die Zahlen $a_{i,j}$ f\"{u}r $i = 1, \cdots,n$ und $j = 1, \cdots, m$ und $b_i$ f\"{u}r 
$i = 1, \cdots, n$ entweder reell oder komplex sind, so nennen wir das Gleichungs-System \emph{linear}.
Anstatt in der obigen kompakten Notation verwenden wir zu Darstellung eines solchen Gleichungs-Systems
auch die folgende \emph{Matrix-Schreibweise}:
\\[0.2cm]
\hspace*{1.3cm}
$\left(
  \begin{array}[c]{ll c l}
    a_{1,1} & a_{1,2} & \cdots & a_{1,m} \\[0.2cm]
    a_{2,1} & a_{2,2} & \cdots & a_{2,m} \\[0.2cm]
    \vdots  & \vdots  & \ddots & \vdots  \\[0.2cm]
    a_{n,1} & a_{n,2} & \cdots & a_{n,m} 
  \end{array}
 \right) \cdot \left(
   \begin{array}[c]{l}
     x_1 \\[0.2cm]
     x_2 \\[0.2cm]
     \vdots \\[0.2cm]
     x_n 
   \end{array}\right) = \left(
   \begin{array}[c]{l}
     b_1 \\[0.2cm]
     b_2 \\[0.2cm]
     \vdots \\[0.2cm]
     b_n      
   \end{array}
   \right)
$
\\[0.2cm]
Hier haben wir die Koeffizienten $a_{i,j}$ der Variablen $x_j$ zu der $n \times m$-\emph{Matrix}
\\[0.2cm]
\hspace*{1.3cm}
$A := \left(
  \begin{array}[c]{ll c l}
    a_{1,1} & a_{1,2} & \cdots & a_{1,m} \\[0.2cm]
    a_{2,1} & a_{2,2} & \cdots & a_{2,m} \\[0.2cm]
    \vdots  & \vdots  & \ddots & \vdots  \\[0.2cm]
    a_{n,1} & a_{n,2} & \cdots & a_{n,m} 
  \end{array}
 \right)
$
\\[0.2cm]
zusammengefasst.  Eine Matrix ist dabei nichts anderes als ein rechteckiges Schema, in dem die doppelt 
indizierte Koeffizienten $a_{i,j}$ \"{u}bersichtlich dargestellt werden k\"{o}nnen.  $n$ bezeichnet die Anzahl
der Zeilen der Matrix, w\"{a}hrend $m$ die Anzahl der Spalten angibt.  Gleichzeitig haben wir oben  die
Variablen $x_1, x_2, \cdots, x_m$ und die Konstanten  
$b_1, b_2, \cdots, b_n$ als \emph{Vektoren} notiert.  Definieren wir
\\[0.2cm]
\hspace*{1.3cm}
$\vec{x} = \left(
  \begin{array}[c]{l}
     x_1    \\
     \vdots \\
     x_m       
  \end{array}
 \right)
$ \quad und \quad
$\vec{b} = \left(
  \begin{array}[c]{l}
     b_1    \\
     \vdots \\
     b_n       
  \end{array}
 \right)
$
\\[0.2cm]
so k\"{o}nnen wir das oben gegebene System von linearen Gleichungen auch kurz in der Form
\\[0.2cm]
\hspace*{1.3cm}
$A \cdot \vec{x} = \vec{y}$
\\[0.2cm]
schreiben, wenn wir vereinbaren, dass die Multiplikation ``$\cdot$'' zwischen einer Matrix
\\[0.2cm]
\hspace*{1.3cm}
$A := \left(
  \begin{array}[c]{ll c l}
    a_{1,1} & a_{1,2} & \cdots & a_{1,m} \\[0.2cm]
    a_{2,1} & a_{2,2} & \cdots & a_{2,m} \\[0.2cm]
    \vdots  & \vdots  & \ddots & \vdots  \\[0.2cm]
    a_{n,1} & a_{n,2} & \cdots & a_{n,m} 
  \end{array}
 \right)
$
\quad und einem Vektor \quad 
$\vec{x} = \left(
  \begin{array}[c]{l}
     x_1    \\[0.2cm]
     x_2    \\[0.2cm]
     \vdots \\[0.2cm]
     x_m       
  \end{array}
 \right)
$ 
\\[0.2cm]
als der Vektor
\\[0.2cm]
\hspace*{1.3cm}
$\vec{x} = \left(
  \begin{array}[c]{l}
     \sum\limits_{j=1}^m a_{1,j} \cdot x_j    \\[0.4cm]
     \sum\limits_{j=1}^m a_{2,j} \cdot x_j    \\
     \quad\vdots                                   \\[0.2cm]
     \sum\limits_{j=1}^m a_{n,j} \cdot x_j    \\[0.4cm]
  \end{array}
 \right)
$ 
\\[0.2cm]
definiert wird.  Betrachten wir zur Verdeutlichung ein Beispiel:  Die drei Gleichungen
\begin{equation}
  \label{eq:gauss0} 
\begin{array}[c]{lllll}
  3 \cdot x + 5 \cdot y + 3 \cdot z = 4, & \quad & \Rm{1} \\[0.2cm]
  4 \cdot x + 3 \cdot y + 2 \cdot z = 3, & \quad & \Rm{2} \\[0.2cm]
  2 \cdot x + 2 \cdot y + 1 \cdot z = 1, & \quad & \Rm{3}
\end{array}
\end{equation}
die wir zur sp\"{a}teren Referenzierung mit r\"{o}mischen Zahlen nummeriert haben,
k\"{o}nnen in Matrix-Schreibweise als
\\[0.2cm]
\hspace*{1.3cm}
$
\left(
\begin{array}[c]{lll}
  3  & 5 & 3  \\[0.1cm]
  4  & 3 & 2  \\[0.1cm]
  2  & 2 & 1  
\end{array}
\right) \cdot \left(
  \begin{array}[c]{lll}
    x \\[0.1cm] y \\[0.1cm] z
  \end{array}
\right) = \left(
  \begin{array}[c]{lll}
    4 \\[0.1cm] 3 \\[0.1cm] 1
  \end{array}
\right)
$
\\[0.2cm]
geschrieben werden.  
Zur L\"{o}sung eines solchen Gleichungs-Systems
werden wir zwei verschiedene Verfahren diskutieren:  In diesem Kapitel betrachten wir das 
\emph{Gau\3'sche Eliminations-Verfahren}\footnote{
  Das \emph{Gau\3'sche Eliminations-Verfahren} wurde zwar nach Carl Friedrich Gau\3 benannt,
  es war aber bereits lange vor Gau\3 bekannt.  Eine schriftliche Beschreibung des Verfahrens  findet
  sich beispielsweise bereits bei Issac Newton, aber das Verfahren war schon deutlich vor Newton 
  mathematisches Allgemeinwissen.},
sp\"{a}ter werden wir sehen, wie sich lineare Gleichungs-Systeme mit Hilfe von
\emph{Determinanten} l\"{o}sen lassen.   

\section{Das Gau\3'sche Eliminations-Verfahren}
Wir demonstrieren das Gau\3'sche Eliminations-Verfahren zun\"{a}chst an dem in (\ref{eq:gauss0}) gezeigten
Beispiel.  Der Einfachheit halber benutzen wir hier noch keine Matrix-Schreibweise.
Um die in (\ref{eq:gauss0}) angegebenen Gleichungen zu l\"{o}sen, versuchen wir im ersten Schritt, die
Variable $x$ aus der zweiten und der dritten Gleichung zu eliminieren.  Um $x$ aus der zweiten Gleichung
loszuwerden, multiplizieren wir die erste Gleichung mit $-\frac{4}{3}$ und addieren die resultierende
Gleichung zu der zweiten Gleichung.  Wegen $-\frac{4}{3} \cdot 3 \cdot x + 4 \cdot x = 0$ f\"{a}llt die
Variable $x$ dann aus der zweiten Gleichung heraus und wir erhalten statt der zweiten Gleichung die Gleichung
\\[0.2cm]
\hspace*{1.3cm}
$-\frac{4}{3} \cdot 5 \cdot y + 3 \cdot y -\frac{4}{3} \cdot 3 \cdot z + 2 \cdot z = -\frac{4}{3} \cdot 4 + 3$,
\\[0.2cm]
die wir noch zu der Gleichung
\\[0.2cm]
\hspace*{1.3cm}
$-\frac{11}{3} \cdot y - 2 \cdot z = -\frac{7}{3}$
\\[0.2cm]
vereinfachen.  Um die Variable $x$ aus der dritten Gleichung zu eliminieren, multiplizieren wir die
erste Gleichung mit $-\frac{2}{3}$ und addieren die so entstandene Gleichung zu der dritten Gleichung.  
Wieder f\"{a}llt der Term mit der Variablen $x$ weg und wir erhalten nun an Stelle der dritten Gleichung die
Gleichung 
\\[0.2cm]
\hspace*{1.3cm}
$(-\frac{2}{3} \cdot 5 + 2) \cdot y + (-\frac{2}{3} \cdot 3 + 1) \cdot z = -\frac{2}{3} \cdot 4 + 1$,
\\[0.2cm]
die wir zu
\\[0.2cm]
\hspace*{1.3cm}
$-\frac{4}{3} \cdot y + (-1) \cdot z = -\frac{5}{3}$,
\\[0.2cm]
vereinfachen.  Insgesamt haben wir damit das urspr\"{u}ngliche Gleichungs-System zu dem \"{a}quivalenten
Gleichungs-System 
\begin{equation}
  \label{eq:gauss1} 
\begin{array}[c]{rcrcrcr}
  3 \cdot x & + & 5 \cdot y             & + &    3 \cdot z & = &           4  \\[0.2cm]
            &   & -\frac{11}{3} \cdot y & + & (-2) \cdot z & = & -\frac{7}{3} \\[0.2cm]
            &   & -\frac{4}{3}  \cdot y & + & (-1) \cdot z & = & -\frac{5}{3}
\end{array}
\end{equation}
umgeformt.  Um aus der letzten Gleichung die Variable $y$ zu entfernen, multiplizieren wir die zweite
Gleichung mit
\\[0.2cm]
\hspace*{1.3cm}
$\displaystyle -\frac{-\frac{4}{3}}{-\frac{11}{3}} = -\frac{4}{11}$
\\[0.2cm]
und addieren diese Gleichung zu der letzten Gleichung unseres Gleichungs-Systems.  Damit erhalten wir
dann an Stelle der letzten Gleichung die neue Gleichung
\\[0.2cm]
\hspace*{1.3cm}
$\left(-\frac{4}{11} \cdot (-2) + (-1)\right) \cdot z = 
 \left(-\frac{4}{11}\right) \cdot \left(-\frac{7}{3}\right) - \frac{5}{3}
$,
\\[0.2cm]
die wir zu
\\[0.2cm]
\hspace*{1.3cm}
$\left(-\frac{3}{11}\right) \cdot z = -\frac{9}{11}$
\\[0.2cm]
vereinfachen.  Insgesamt haben wir unser urspr\"{u}ngliches Gleichungs-System jetzt zu dem Gleichungs-System
\begin{equation}
  \label{eq:gauss2} 
\begin{array}[c]{rcrcrcr}
  3 \cdot x & + & 5 \cdot y             & + &    3 \cdot z  & = &           4  \\[0.2cm]
            &   & -\frac{11}{3} \cdot y & + & (-2) \cdot z  & = & -\frac{7}{3} \\[0.2cm]
            &   &                       &   & -\frac{3}{11} \cdot z & = & -\frac{9}{11}
\end{array}
\end{equation}
umgeformt.  Dieses Gleichungs-System hat, wie man sagt, \emph{obere Dreiecks-Form}, denn
unterhalb der Diagonalen haben alle Eintr\"{a}ge der Matrix den Wert $0$.
Wir k\"{o}nnen es jetzt durch
\emph{R\"{u}ckw\"{a}rts-Substitution} l\"{o}sen, indem wir zun\"{a}chst die letzte Gleichung nach $z$ aufl\"{o}sen, diesen
Wert f\"{u}r $z$ dann in die zweite Gleichung einsetzten, die zweite Gleichung nach $y$ aufl\"{o}sen und weiter
die Werte f\"{u}r $y$ und $z$ in der ersten Gleichung einsetzten, so dass wir schlie\3lich den Wert von $x$
bestimmen k\"{o}nnen.  Die Aufl\"{o}sung der letzten Gleichung nach $z$ liefert $z = 3$.  Setzen wir diesen Wert
in der zweiten Gleichung ein, so erhalten wir
\\[0.2cm]
\hspace*{1.3cm} $-\frac{11}{3} \cdot y - 6 = -\frac{7}{3}$,
\\[0.2cm]
was sich zu
\\[0.2cm]
\hspace*{1.3cm} $-\frac{11}{3} \cdot y = \frac{11}{3}$,
\\[0.2cm]
vereinfacht, woraus sofort $y = -1$ folgt.  Setzen wir nun die Werte von $x$ und $y$ in der ersten
Gleichung ein, so erhalten wir
\\[0.2cm]
\hspace*{1.3cm} $3 \cdot x - 5 + 9 = 4$,
\\[0.2cm]
woraus $x = 0$ folgt. Damit lautet die L\"{o}sung des urspr\"{u}nglichen Gleichungs-Systems
\\[0.2cm]
\hspace*{1.3cm} 
$x = 0$, \quad $y = -1$, \quad $z = 3$.
\pagebreak

\exercise
Bestimmen  Sie die L\"{o}sung des folgenden Gleichungs-Systems mit Hilfe des Gau\3'schen
Eliminations-Verfahrens:
\begin{equation*}
\begin{array}[c]{l}
  2 \cdot x + 1 \cdot y + 3 \cdot z = 2, \\[0.2cm]
  1 \cdot x + 3 \cdot y + 2 \cdot z = 0, \\[0.2cm]
  1 \cdot x + 2 \cdot y + 1 \cdot z = 0.
\end{array}
\end{equation*}
\textbf{Hinweis}: Die L\"{o}sungen sind keine ganzen Zahlen.
\vspace*{0.3cm}

Wir beschreiben nun, wie ein lineares Gleichungs-System der Form
\\[0.2cm]
\hspace*{1.3cm}
$\sum\limits_{j=1}^m a_{i,j} \cdot x_j = b_i$
\\[0.2cm]
f\"{u}r eine  gegebenen Matrix 
\\[0.2cm]
\hspace*{1.3cm}
$A := \left(
  \begin{array}[c]{ll c l}
    a_{1,1} & a_{1,2} & \cdots & a_{1,m} \\[0.2cm]
    a_{2,1} & a_{2,2} & \cdots & a_{2,m} \\[0.2cm]
    \vdots  & \vdots  & \ddots & \vdots  \\[0.2cm]
    a_{n,1} & a_{n,2} & \cdots & a_{n,m} 
  \end{array}
 \right)
$
\quad und einen gegebenen Vektor \quad
$\vec{b} = \left(
  \begin{array}[c]{l}
     b_1    \\
     \vdots \\[0.1cm]
     b_n       
  \end{array}
 \right)
$
\\[0.2cm]
gel\"{o}st werden kann, wobei wir uns auf den Spezialfall $m=n$ beschr\"{a}nken wollen.
\begin{enumerate}
\item Im ersten Schritt eliminieren wir die Variable $x_1$ aus der 2-ten, 3-ten, $\cdots$, $n$-ten
      Gleichung.  Um die Variable $x_1$ aus der $i$-ten Gleichung zu eliminieren, multiplizieren 
      wir die erste Gleichung mit dem Faktor
      \\[0.2cm]
      \hspace*{1.3cm}
      $-\bruch{a_{i,1}}{a_{1,1}}$
      \\[0.2cm]
      und addieren die so multiplizierte erste Gleichung zu der $i$-ten Gleichung.  Wegen
      \\[0.2cm]
      \hspace*{1.3cm}
      $-\bruch{a_{i,1}}{a_{1,1}} \cdot a_{1,1} \cdot x_1 + a_{i,1} \cdot x_1 = 0$
      \\[0.2cm]
      enth\"{a}lt die resultierende Gleichung die Variable $x_1$ nicht mehr.  Die neue $i$-te Gleichung
      hat dann die Form
      \\[0.2cm]
      \hspace*{1.3cm}
      $\bigl(a_{i,2} - \bruch{a_{i,1}}{a_{1,1}} \cdot a_{1,1}\bigr) \cdot x_2 + \cdots + 
       \bigl(a_{i,n} - \bruch{a_{i,1}}{a_{1,1}} \cdot a_{1,n}\bigr) \cdot x_n =
       b_{i} - \bruch{a_{i,1}}{a_{1,1}} \cdot b_{1}
      $,
      \\[0.2cm]
      was sich unter Verwendung der Summen-Schreibweise auch kompakter in der Form
      \\[0.2cm]
      \hspace*{1.3cm}
      $\sum\limits_{j=2}^n \bigl(a_{i,j} - \bruch{a_{i,1}}{a_{1,1}} \cdot a_{1,j}\bigr) \cdot x_j =
       b_{i} - \bruch{a_{i,1}}{a_{1,1}} \cdot b_{1}
      $
      \\[0.2cm]
      schreiben l\"{a}sst.

      \textbf{Bemerkung}: An dieser Stelle fragen Sie sich vermutlich, was passiert, wenn
      $a_{1,1} = 0$ ist, denn dann ist der Ausdruck $\frac{a_{i,1}}{a_{1,1}}$ offenbar undefiniert.
      In so einem Fall vertauschen wir einfach die erste Gleichung mit einer anderen Gleichung,
      f\"{u}r die der Koeffizient der Variablen $x_1$ von $0$ verschieden ist.  In der Praxis hat es sich
      bew\"{a}hrt, immer die Gleichung als erste zu nehmen, f\"{u}r die der Koeffizient der Variablen $x_1$ den
      gr\"{o}\3ten Betrag hat, denn dadurch fallen die bei  einer numerischen Rechnung zwangsl\"{a}ufig
      auftretenden  Rundungsfehler weniger schwer ins Gewicht als wenn wir die
      Reihenfolge der Gleichungen beliebig w\"{a}hlen.   Diese Verfahren wird als
      \emph{partielle Pivotisierung} bezeichnet.
\item Im $k$-ten Schritt nehmen wir an, dass wir die Variablen $x_1, \cdots, x_{k-1}$ bereits aus 
      der $k$-ten, $(k+1)$-ten, $\cdots$, $n$-Gleichung entfernt haben und wollen nun die 
      Variable $x_k$ aus der $(k+1)$-ten bis $n$-ten Gleichung entfernen. 
      Um die Variable $x_k$ aus der $i$-ten Gleichung ($i \in \{ k+1,\cdots, n\}$)
      zu entfernen multiplizieren wir die $k$-te Gleichung mit dem Faktor
      \\[0.2cm]
      \hspace*{1.3cm}
      $-\bruch{a_{i,k}}{a_{k,k}}$
      \\[0.2cm]
      und addieren die so multiplizierte $k$-te Gleichung zu der $i$-ten Gleichung.  Wegen
      \\[0.2cm]
      \hspace*{1.3cm}
      $-\bruch{a_{i,k}}{a_{k,k}} \cdot a_{k,k} \cdot x_k + a_{i,k} \cdot x_k = 0$
      \\[0.2cm]
      enth\"{a}lt die resultierende Gleichung die Variable $x_k$ nicht mehr.  Die Variablen
      $x_1$, $\cdots$, $x_{k-1}$ wurden aus der $i$-ten Gleichung bereits vorher 
      eliminiert, so dass die neue $i$-te Gleichung
      dann die Form
      \\[0.2cm]
      \hspace*{1.3cm}
      $\sum\limits_{j=k+1}^n \bigl(a_{i,j} - \bruch{a_{i,k}}{a_{k,k}} \cdot a_{k,j}\bigr) \cdot x_j =
       b_{i} - \bruch{a_{i,k}}{a_{k,k}} \cdot b_{k}
      $
      \\[0.2cm]
      hat.   Aus Gr\"{u}nden der numerischen Stabilit\"{a}t kann wieder eine partielle Pivotisierung durchgef\"{u}hrt
      werden.  Wir wollen im Folgenden voraussetzen, dass dies immer m\"{o}glich ist, das hei\3t wir setzen
      voraus, dass es immer eine Gleichung unter den Gleichungen mit den Nummern $k,\cdots,n$ gibt,
      in denen die Variable $x_k$ auch tats\"{a}chlich auftritt, so dass also immer f\"{u}r mindestens ein
      $i \in \{ k, \cdots, n\}$ der Koeffizient $a_{i,k} \not= 0$ ist.  
      Diese Voraussetzung ist \"{a}quivalent zu der Forderung, dass das urspr\"{u}ngliche Gleichungs-System
      eindeutig l\"{o}sbar ist.
\item Im letzten Schritt k\"{o}nnen wir voraussetzen, dass das Gleichungs-System eine obere Dreiecks-Form
      hat.  Es bleiben nun noch $n$ Teilschritte zur Berechnung der Variablen $x_1,\cdots,x_n$.
      \begin{enumerate}
      \item Im ersten Teilschritt l\"{o}sen wir die $n$-te Gleichung nach $x_n$ auf.
            Die $n$-te Gleichung hat die Form
            \\[0.2cm]
            \hspace*{1.3cm}
            $c_{n,n} \cdot x_n = d_n$,
            \\[0.2cm] 
            wobei wir die Koeffizienten $c_{n,n}$ und $d_n$ im zweiten Schritt berechnet haben.
            Die L\"{o}sung dieser Gleichung ist dann offenbar
            \\[0.2cm]
            \hspace*{1.3cm}
            $x_{n} = \bruch{d_n}{c_{n,n}}$.
            \\[0.2cm]
            Sollte nun $c_{n,n} = 0$ gelten, so ist das Gleichungs-System nicht eindeutig l\"{o}sbar.
      \item Im $i$-ten Teilschritt k\"{o}nnen wir voraussetzen, dass wir die Variablen
            \\[0.2cm]
            \hspace*{1.3cm}
            $x_n$, $x_{n-1}$, $\cdots$, $x_{n-(i-2)}$ 
            \\[0.2cm]
            bereits berechnet haben.  Ziel ist nun die Bestimmung der Variablen $x_{n-(i-1)}$ mit Hilfe
            der $\bigl(n-(i-1)\bigr)$-ten Gleichung.  Definieren wir zur Vereinfachung der Notation
            $k = n - (i-1)$, so hat die $k$-te Gleichung die Form
            \\[0.2cm]
            \hspace*{1.3cm}
            $\sum\limits_{j=k}^n c_{k,j} \cdot x_j = d_k$.
            \\[0.2cm]
            Da nun die Variablen $x_{k+1},\cdots,x_n$ bereits bekannt sind, k\"{o}nnen wir diese Gleichung
            nach $x_k$ aufl\"{o}sen und erhalten
            \\[0.2cm]
            \hspace*{1.3cm}
            $x_k = \bruch{1}{c_{k,k}} \Bigl(d_k - \sum\limits_{j=k+1}^n c_{k,j} \cdot x_j\Bigr)$.
            \\[0.2cm]
            Sollte hier $c_{k,k} = 0$ gelten, so ist das Gleichungs-System nicht eindeutig l\"{o}sbar.
      \end{enumerate}
\end{enumerate}


\begin{figure}[!ht]
\centering
\begin{Verbatim}[ frame         = lines, 
                  framesep      = 0.3cm, 
                  firstnumber   = 1,
                  labelposition = bottomline,
                  numbers       = left,
                  numbersep     = -0.2cm,
                  xleftmargin   = 0.8cm,
                  xrightmargin  = 0.8cm,
                ]
    solve := procedure(a, b) {
        [ a, b ] := eliminate(a, b);
        x := solveTriangular(a, b);
        return x;
    };
    eliminate := procedure(a, b) {
        n := #a;    // number of equations
        pivot := procedure(a, n, i) {
            r := i;  // index of row containing maximal element
            for (j in [i+1 .. n]) {
                if (abs(a(j)(i)) > abs(a(r)(i))) {
                    r := j;
                }
            }
            return r;
        };
        for (i in [1 .. n]) {
            r := pivot(a, n, i);
            [ u, v ] := [ a(r), a(i) ];   
            a(r) := v;
            a(i) := u;
            [ u, v ] := [ b(i), b(r) ];
            b(i) := v;
            b(r) := u;
            for (j in [i+1 .. n]) {
                f := 1.0 * a(j)(i) / a(i)(i);
                a(j)(i) := 0;
                for (k in [i+1 .. n]) {
                    a(j)(k) -= f * a(i)(k);
                }
                b(j) -= f * b(i);
            }    
        }
        return [ a, b ];
    };
    solveTriangular := procedure(a, b) {
        x := [];
        n := #a;    // number of equations
        i := n;     // index to equation
        for (i in [n, n-1 .. 1]) {
            r := b(i);
            r -= +/ { a(i)(k) * x(k) : k in [i+1 .. n] };
            x(i) := 1.0 * r / a(i)(i);
        }
        return x;
    };
\end{Verbatim}
\vspace*{-0.3cm}
\caption{Implementierung des Gau\3'schen Eliminations-Verfahrens in \textsl{SetlX}.}
\label{fig:gauss.stlx}
\end{figure}

\noindent
Abbildung \ref{fig:gauss.stlx} zeigt eine einfache Implementierung des Gau\3'schen Algorithmus, welche die
oben ausgef\"{u}hrten \"{u}berlegungen umsetzt.  Wir diskutieren das Programm nun im Detail.
\begin{enumerate}
\item Die Funktion $\textsl{solve}(a, b)$ erh\"{a}lt als erstes Argument $a$ die Matrix und
      als zweites Argument $b$ die rechte Seite des linearen Gleichungs-Systems
      \\[0.2cm]
      \hspace*{1.3cm}
      $a \cdot \vec{x} = b$
      \\[0.2cm]
      Zun\"{a}chst bringen wir dieses Gleichungs-System in Zeile 2 mit Hilfe der Funktion \textsl{eliminate} auf
      eine obere Dreiecks-Form.  Die Funktion \textsl{solveTriangular} l\"{o}st dieses System dann durch
      R\"{u}ckw\"{a}rts-Substitution.
\item Die Funktion $\textsl{eliminate}(a, b)$ hat die Aufgabe, das Gleichungs-System $a \cdot \vec{x} = b$
      in eine obere Dreiecks-Form zu \"{u}berf\"{u}hren.  Wir gehen davon aus, dass die Matrix $a$ quadratisch ist. 
      Dann ist das in Zeile 7 bestimmte $n$ sowohl die Anzahl der Zeilen der Matrix als auch die Anzahl
      der Variablen.
\item In den Zeilen 8 bis 16 definieren wir die lokale Funktion $\textsl{pivot}(a, n, i)$.  Diese Funktion 
      hat die Aufgabe, diejenige Zeile $r$ in der Matrix $a$ zu finden, f\"{u}r die der Wert
      \\[0.2cm]
      \hspace*{1.3cm}
      $|a_{j,i}|$ \quad f\"{u}r $j \in \{i, \cdots, n \}$
      \\[0.2cm]
      maximal wird.
\item Die Schleife in Zeile 17 setzt voraus, dass die ersten $i$ Gleichungen bereits in oberer 
      Dreiecksform vorliegen und das dar\"{u}ber hinaus die Variablen $x_1, \cdots, x_{i-1}$
      bereits aus den Gleichungen $i, i+1, \cdots, n$  entfernt worden sind.
      Ziel ist es,  die Variable $x_i$ aus der $(i+1)$-sten bis zur $n$-ten
      Gleichung zu entfernen.  
      \begin{enumerate}
      \item Dazu wird  mit Hilfe des Funktions-Aufrufs $\textsl{pivot}(a,n,i)$
            bestimmt, in welcher Zeile der Betrag $a_{j,i}$ maximal ist.
      \item In den Zeilen 20 - 24 wird diese Zeile  mit der $i$-ten Zeile vertauscht.
      \item Die \texttt{for}-Schleife in Zeile 25 zieht von der $j$-ten Zeile das
            \\[0.2cm]
            \hspace*{1.3cm}
            $\bruch{a_{j,i}}{a_{i,i}}$-fache 
            \\[0.2cm]
            der $i$-ten Zeile ab.
      \end{enumerate}
      Insgesamt h\"{a}ngt nun f\"{u}r jedes $i=1,\cdots,n$ die $i$-te Gleichung nur noch von den Variablen
      $x_{i}, \cdots, x_{n}$ ab.
\item In der Prozedur \textsl{solveTriangular} wird das Gleichungs-System, das jetzt in oberer Dreiecks-Form 
      vorliegt, durch R\"{u}ckw\"{a}rts-Substitution gel\"{o}st.
\end{enumerate}

%%% Local Variables: 
%%% mode: latex
%%% TeX-master: "lineare-algebra"
%%% End: 
