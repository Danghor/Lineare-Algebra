\chapter{Naive Mengenlehre}
Die \href{https://de.wikipedia.org/wiki/Mengenlehre}{Mengenlehre} ist gegen Ende des 19-ten
Jahrhunderts aus dem Bestreben heraus entstanden, die Mathematik auf eine solide Grundlage zu
stellen.  Die Schaffung einer solchen Grundlage wurde als notwendig erachtet, da der Begriff der
Unendlichkeit den Mathematikern zunehmends Kopfzerbrechen bereitete.

Begr\"{u}ndet wurde die Mengenlehre in wesentlichen Teilen von
\href{https://de.wikipedia.org/wiki/Georg_Cantor}{Georg Cantor} (1845 -- 1918). 
Die erste Definition des Begriffs der Menge lautete etwa wie folgt \cite{cantor:1895}:
\vspace*{0.2cm}

\colorbox{red}{\framebox{\colorbox{yellow}{
\begin{minipage}{0.9\linewidth}
  \textsl{Unter einer ``Menge'' verstehen wir jede Zusammenfassung $M$ von bestimmten wohlunterschiedenen
  Objekten $x$ unserer Anschauung oder unseres Denkens zu
  einem Ganzen.}
\end{minipage}}}}
\vspace*{0.2cm}

Das Attribut ``\emph{wohldefiniert}'' dr\"{u}ckt dabei aus, dass wir f\"{u}r 
eine vorgegebene Menge $M$ und ein Objekt $x$ stets klar sein muss, ob das Objekt $x$
zu der Menge $M$ geh\"{o}rt oder nicht.  In diesem Fall sagen wir, dass $x$ ein \emph{Element} der
Menge $M$ ist und schreiben
\\[0.2cm]
\hspace*{1.3cm}
$x \in M$. 
\\[0.2cm]
Das Zeichen ``$\in$'' wird in der Mengenlehre also als zweistelliges Pr\"{a}dikats-Zeichen
gebraucht, f\"{u}r das sich eine Infix-Notation eingeb\"{u}rgert hat.
Etwas verkürzt können wir den Mengen-Begriff wie folgt definieren: 
\\[0.2cm]
\hspace*{1.3cm}
\textsl{Eine Menge ist eine \emph{wohldefinierte} Ansammlung von Elementen}.
\\[0.2cm]
Um den Begriff der \emph{wohldefinierten Ansammlung von Elementen} mathematisch zu
pr\"{a}zisieren, f\"{u}hrte Cantor das sogenannte \colorbox{yellow}{\emph{Komprehensions-Axiom}} ein.
Wir k\"{o}nnen dieses zun\"{a}chst wie folgt formalisieren: Ist $p(x)$ eine Eigenschaft, die
ein Objekt $x$ haben kann, so k\"{o}nnen wir die Menge $M$ aller Objekte, welche die
Eigenschaft $p(x)$ haben, bilden.  Wie schreiben dann \\[0.2cm]
\hspace*{1.3cm} $M = \{ x \;|\; p(x) \}$ \\[0.2cm]
und lesen dies als ``$M$ ist die Menge aller $x$, f\"{u}r die $p(x)$ gilt''.
Eine Eigenschaft $p(x)$ ist dabei nichts anderes als eine Formel, in der die Variable $x$
vorkommt.
Wir veranschaulichen das Komprehensions-Axiom durch ein Beispiel:  Es sei $\mathbb{N}$
die Menge der nat\"{u}rlichen Zahlen. Ausgehend von der Menge $\mathbb{N}$  wollen wir die
Menge der \emph{geraden Zahlen} definieren. Zun\"{a}chst m\"{u}ssen wir dazu die Eigenschaft einer
Zahl $x$,
\emph{gerade} zu sein, durch eine Formel $p(x)$ mathematisch erfassen.  Eine nat\"{u}rliche Zahl $x$ ist
genau dann gerade, wenn es eine nat\"{u}rliche Zahl $y$ gibt, so dass $x$ das Doppelte von $y$
ist.  Damit k\"{o}nnen wir die Eigenschaft $p(x)$ folgenderma\3en
definieren: \\[0.2cm]
\hspace*{1.3cm} $p(x) \;:=\; (\exists y\in \mathbb{N}: x = 2 \cdot y)$. \\[0.2cm]
Also kann die Menge der geraden Zahlen als \\[0.2cm]
\hspace*{1.3cm} $\{ x \;|\; \exists y\in \mathbb{N}: x = 2 \cdot y \}$ \\[0.2cm]
geschrieben werden.

Leider f\"{u}hrt die uneingeschr\"{a}nkte Anwendung des Komprehensions-Axiom schnell zu
Problemen.  Betrachten wir dazu die Eigenschaft einer Menge, sich \underline{nicht} selbst zu
enthalten, wir setzen also
\\[0.2cm]
\hspace*{1.3cm}
 $p(x) := \neg(x \in x)$ 
\\[0.2cm]
und definieren die Menge $R$ als \\[0.2cm]
\hspace*{1.3cm} $R := \{ x \;|\; \neg (x \in x) \}$.  \\[0.2cm]
Intuitiv w\"{u}rden wir vielleicht erwarten, dass keine Menge sich selbst enth\"{a}lt.  Wir wollen
jetzt zun\"{a}chst f\"{u}r die eben definierte Menge $R$ \"{u}berpr\"{u}fen, wie die Dinge liegen.
Dazu betrachten wir folgende �quivalenz-Umformung:
\\[0.2cm]
\hspace*{1.3cm}
$
\begin{array}{cl}
                  & R \in R                                      \\[0.2cm] 
  \Leftrightarrow & R \in \bigl\{ x \;|\; \neg (x \in x) \bigr\} \\[0.2cm] 
  \Leftrightarrow & \neg (R \in R)                     
\end{array}
$
\\[0.2cm]
Wir haben also 
\\[0.2cm]
\hspace*{1.3cm}
$R \in R \;\Leftrightarrow\; \neg(R \in R)$
\\[0.2cm]
gezeigt.  Das ist aber logisch nicht m\"{o}glich, eine Formel kann nicht genau dann wahr sein,
wenn Sie falsch ist.  Als Ausweg k\"{o}nnen wir nur feststellen, dass das vermittels \\[0.2cm]
\hspace*{1.3cm} $\{ x \mid \neg (x \in x) \}$ \\[0.2cm]
definierte Objekt keine Menge ist.
Das hei\3t dann aber, dass das Komprehensions-Axiom
zu allgemein ist.  Wir folgern, dass nicht jede  in der Form \\[0.2cm]
\hspace*{1.3cm} $M = \{ x \mid p(x) \}$ \\[0.2cm]
angegebene Menge wohldefiniert ist.  Die Konstruktion der ``Menge''
\\[0.2cm]
\hspace*{1.3cm}
$\bigl\{x \mid \neg(x \in x)\bigr\}$
\\[0.2cm]
stammt von dem britischen Logiker und Philosophen 
\href{http://de.wikipedia.org/wiki/Bertrand_Russell}{Bertrand Russell} (1872 -- 1970).  Sie wird
deswegen auch als \href{http://de.wikipedia.org/wiki/Russellsche_Antinomie}{\emph{Russell'sche Antinomie}}
bezeichnet. 

Um  Paradoxien wie die Russell'sche Antinomie  zu vermeiden, ist es erforderlich, bei der
Konstruktion von Mengen vorsichtiger vorzugehen.
Wir werden im Folgenden Konstruktions-Prinzipien f\"{u}r Mengen vorstellen,
die schw\"{a}cher sind als das Komprehensions-Axiom, die aber f\"{u}r die Praxis 
ausreichend sind.  Wir wollen dabei die dem Komprehensions-Axiom zugrunde liegende Notation 
beibehalten und Mengendefinitionen in der Form 
\\[0.2cm]
\hspace*{1.3cm}
$M = \{ x \mid p(x) \}$  
\\[0.2cm]
angeben.  Um Paradoxien wie die Russell'sche Antinomie zu vermeiden, werden wir nur bestimmte
Sonderf\"{a}lle dieser Mengendefinition zulassen.  Diese Sonderf\"{a}lle, sowie weitere M\"{o}glichkeiten
Mengen zu konstruieren, stellen wir jetzt vor.

\section{Erzeugung von Mengen durch explizites Auflisten}
Die einfachste M\"{o}glichkeit, eine Menge festzulegen, besteht in der expliziten
\emph{Auflistung} aller ihrer Elemente. Diese Elemente werden in den geschweiften
Klammern ``\texttt{\{}'' und ``\texttt{\}}'' eingefasst und durch Kommas getrennt.
Definieren wir beispielsweise \\[0.2cm]
\hspace*{1.3cm} $M := \{ 1, 2, 3 \}$, \\[0.2cm]
so haben wir damit festgelegt, dass die Menge $M$ aus den Elementen $1$, $2$ und $3$
besteht. In der Schreibweise des Komprehensions-Axioms k\"{o}nnen wir diese Menge als \\[0.2cm]
\hspace*{1.3cm} $M = \{ x \mid x = 1 \vee x = 2 \vee x = 3 \}$ \\[0.2cm]
angeben.
Als ein weiteres Beispiel f\"{u}r eine Menge, die durch explizite Aufz\"{a}hlung ihrer Elemente
angegeben werden kannn, betrachten wir die Menge der kleinen lateinischen Buchstaben, die wir wie folgt
definieren: \\[0.2cm]
\hspace*{1.3cm} 
$\{\mathtt{a}, \mathtt{b}, \mathtt{c}, \mathtt{d}, \mathtt{e},
 \mathtt{f}, \mathtt{g}, \mathtt{h}, \mathtt{i}, \mathtt{j}, \mathtt{k}, \mathtt{l},
 \mathtt{m}, \mathtt{n}, \mathtt{o}, \mathtt{p}, \mathtt{q}, \mathtt{r}, \mathtt{s},
 \mathtt{t}, \mathtt{u}, \mathtt{v}, \mathtt{w}, \mathtt{x}, \mathtt{y}, \mathtt{z\}}$.
 \\[0.2cm]
Als letztes Beispiel betrachten wir die leere Menge $\emptyset$, die durch
Aufz\"{a}hlung aller ihrer Elemente definiert werden kann:
\\[0.2cm]
\hspace*{1.3cm}
$\emptyset := \{\}$.
\\[0.2cm]
Die leere Menge enth\"{a}lt also \"{u}berhaupt keine Elemente.  Diese Menge spielt in der Mengenlehre eine
\"{a}hnliche Rolle wie die Zahl $0$ in der Zahlentheorie.

Wird eine Menge durch Auflistung ihrer Elemente definiert, so spielt die Reihenfolge, in der die
Elemente aufgelistet werden, keine Rolle.  Beispielsweise gilt
\\[0.2cm]
\hspace*{1.3cm}
$\{1,2,3\} = \{3,1,2\}$,
\\[0.2cm]
denn beide Mengen enthalten offenbar die selben Elemente.

\section{Vorgegebene unendliche Mengen von Zahlen}
Alle durch explizite Auflistung definierten Mengen haben offensichtlich nur endlich viele
Elemente.  Aus der mathematischen Praxis kennen wir aber auch Mengen mit unendlich vielen
Elementen.  Ein Beispiel ist die 
\href{http://en.wikipedia.org/wiki/Natural_number}{Menge der nat\"{u}rlichen Zahlen}, die wir mit $\mathbb{N}$
bezeichnen.  Im Gegensatz zu einigen anderen Autoren werde ich dabei die Zahl $0$ 
als nat\"{u}rliche Zahl auffassen, denn dies ist im
\href{https://en.wikipedia.org/wiki/ISO_31-11}{\textsc{Iso}-Standard 31-11} so festgelegt.\footnote{
  Die \textsc{Iso}-Norm 31-11 ist mittlerweile durch den
  \href{https://en.wikipedia.org/wiki/ISO_80000-2}{\textsc{Iso}-Standard 80000-2} abgel�st worden,
  aber die Definition der Menge $\mathbb{N}$ hat sich dabei nicht ge�ndert.
}
Mit den bisher behandelten Verfahren l\"{a}sst sich die Menge $\mathbb{N}$  nicht definieren.
Wir m\"{u}ssen daher die Existenz dieser Menge als Axiom fordern.  Genauer postulieren wir, dass es eine
Menge $\mathbb{N}$ gibt, welche die folgenden drei Eigenschaften hat:
\begin{enumerate}
\item $0 \in \mathbb{N}$.
\item Falls $n \in \mathbb{N}$ gilt, so gilt auch $n+1 \in \mathbb{N}$.
\item Au\3er den Zahlen, die auf Grund der ersten beiden Bedingungen Elemente der Menge $\mathbb{N}$ sind, enth\"{a}lt
      $\mathbb{N}$ keine weiteren Elemente.
\end{enumerate}
Anschaulich schreiben wir \\[0.2cm]
\hspace*{1.3cm} $\mathbb{N} := \{ 0, 1, 2, 3, \cdots \}$. \\[0.2cm]
Neben der Menge $\mathbb{N}$ der nat\"{u}rlichen Zahlen verwenden wir noch die folgenden
Mengen von Zahlen: 
\begin{enumerate}
\item $\mathbb{N}^*$ ist die Menge der positiven natürlichen Zahlen, es gilt also
      \\[0.2cm]
      \hspace*{1.3cm}
      $\mathbb{N}^* := \{ n \mid n \in \mathbb{N} \wedge n > 0 \}$.
\item $\mathbb{Z}$ ist die Menge der ganzen Zahlen, es gilt
      \\[0.2cm]
      \hspace*{1.3cm}
      $\mathbb{Z} := \{ 0, 1, -1, 2, -2, 3, -3, \cdots \}$ 

\item $\mathbb{Q}$ ist die Menge der rationalen Zahlen, es gilt
      \\[0.2cm]
      \hspace*{1.3cm}
      $\Bigl\{ \bruch{p}{q} \mid\, p \in \mathbb{Z} \wedge q \in \mathbb{N}^* \Bigr\}$
\item $\mathbb{R}$ ist die Menge der reellen Zahlen.

      Eine mathematisch saubere Definition der reellen Zahlen erfordert einiges an Aufwand.  Wir
      werden die Konstruktion der reellen Zahlen erst in der
      \href{https://github.com/karlstroetmann/Analysis/blob/master/Script/analysis.pdf}{Analysis-Vorlesung}
      im zweiten Semester besprechen. 
\end{enumerate}


\section{Das Auswahl-Prinzip}
Das \emph{Auswahl-Prinzip}, auch bekannt als 
\href{https://de.wikipedia.org/wiki/Aussonderungsaxiom}{\emph{Aussonderungs-Axiom}}
 ist eine Abschw\"{a}chung des Komprehensions-Axiom.  Die Idee
ist, mit Hilfe einer Eigenschaft $p$ aus einer schon vorhandenen Menge $M$ die Menge $N$ der 
Elemente $x$ \emph{auszuw\"{a}hlen}, die eine bestimmte Eigenschaft $p(x)$ besitzen: \\[0.2cm]
\hspace*{1.3cm} $N = \{ x\in M \;|\; p(x) \}$ \\[0.2cm]
In der Notation des Komprehensions-Axioms schreibt sich diese Menge als \\[0.2cm]
\hspace*{1.3cm} $N = \{ x \mid x \in M \wedge p(x) \}$. \\[0.2cm]
Im Unterschied zu dem Komprehensions-Axiom k\"{o}nnen wir uns hier nur auf die Elemente einer
bereits vorgegebenen Menge $M$ beziehen und nicht auf v\"{o}llig beliebige Objekte.
\vspace{0.2cm}

\noindent
\textbf{Beispiel}:
Die Menge der geraden Zahlen kann mit dem Auswahl-Prinzip als die Menge 
\\[0.2cm]
\hspace*{1.3cm}
 $\{ x \in \mathbb{N} \;|\; \exists y\in \mathbb{N}: x = 2 \cdot y \}$. 
\\[0.2cm]
geschrieben werden.

\section{Vereinigungs-Mengen}
Sind zwei Mengen $M$ und $N$ gegeben, so enth\"{a}lt die 
\href{https://de.wikipedia.org/wiki/Menge_(Mathematik)#Vereinigung_.28Vereinigungsmenge.29}{\emph{Vereinigung}}
von $M$ und $N$ alle Elemente, die  
 in der Menge $M$ oder in der Menge $N$ liegen.  F\"{u}r diese Vereinigung schreiben wir $M \cup N$.
Formal kann die Vereinigung als 
\\[0.2cm]
\hspace*{1.3cm} $M \cup N := \{ x \;|\; x \in M \vee x \in N \}$. 
\\[0.2cm] 
definiert werden.

\example 
Ist  $M = \{1,2,3\}$ und $N = \{2,5\}$, so gilt: \\[0.2cm]
\hspace*{1.3cm} $\{1,2,3\} \cup \{2,5\} = \{1,2,3,5\}$.  \eox


Der Begriff der Vereinigung von Mengen l\"{a}sst sich verallgemeinern.  Betrachten
wir dazu eine Menge $X$, deren Elemente selbst wieder Mengen sind. Beispielsweise ist die
Potenz-Menge, die als die Menge aller Teilmengen definiert ist,   
eine Menge von dieser Art.  Wir k\"{o}nnen dann die Vereinigung aller Mengen, die Elemente
von der Menge $X$ sind, bilden.  Diese Vereinigung schreiben wir als $\bigcup X$.  Formal
kann diese Vereinigung als
\\[0.2cm]
\hspace*{1.3cm} $\bigcup X = \{ y \;|\; \exists x \in X: y \in x \}$.
\\[0.2cm]
 definiert werden.

\example
Die Menge $X$ sei wie folgt gegeben: \\[0.2cm]
\hspace*{1.3cm}
 $X = \big\{ \{\},\, \{1,2\}, \, \{1,3,5\}, \, \{7,4\}\,\big\}$. \\[0.2cm]
Dann gilt \\[0.2cm]
\hspace*{1.3cm}
 $\bigcup X = \{ 1, 2, 3, 4, 5, 7 \}$. \eoxs

\section{Schnitt-Menge}
Sind zwei Mengen $M$ und $N$ gegeben, so definieren wir den 
\href{https://de.wikipedia.org/wiki/Menge_(Mathematik)#Schnittmenge_.28Schnitt.2C_auch_.E2.80.9EDurchschnitt.E2.80.9C.29}{\emph{Schnitt}}
von $M$ und $N$ als die Menge aller Elemente, die \underline{sowohl} in $M$ \underline{als auch} in $N$
auftreten.  Wir bezeichnen den Schnitt von $M$ und $N$ mit $M \cap N$.
Formal k\"{o}nnen wir $M \cap N$ als 
\\[0.2cm]
\hspace*{1.3cm} $M \cap N := \{ x \mid x \in M \wedge x \in N \}$.
\\[0.2cm]
definieren.

\example
Wir berechnen den Schnitt der  Mengen $M = \{ 1, 3, 5 \}$ und $N = \{ 2, 3, 5, 6 \}$.  Es gilt
\\[0.2cm]
\hspace*{1.3cm} $M \cap N = \{ 3, 5 \}$
\eoxs

\section{Differenz-Mengen}
 Sind zwei Mengen $M$ und $N$ gegeben, so bezeichnen wir die 
\href{https://de.wikipedia.org/wiki/Menge_(Mathematik)#Differenz_und_Komplement}{\emph{Differenz}}
von $M$ und $N$ als die Menge aller Elemente, die in $M$ aber nicht $N$
 auftreten.  Wir schreiben hierf\"{u}r $M \backslash N$.  Dieser Ausdruck wird als
\\[0.2cm]
\hspace*{1.3cm}
$M$ \emph{ohne} $N$
\\[0.2cm]
gelesen und kann formal als
 \\[0.2cm]
\hspace*{1.3cm} $M \backslash N := \{ x \mid x \in M \wedge x \not\in N \}$. 
 \\[0.2cm] 
definiert werden.


\example
Wir berechnen die Differenz der Mengen $M = \{ 1, 3, 5, 7 \}$ und $N = \{ 2, 3, 5, 6 \}$.  Es gilt
\\[0.2cm]
\hspace*{1.3cm} $M \backslash N = \{ 1, 7 \}$. \eoxs

\section{Bild-Mengen}
Es sei $M$ eine Menge und $f$ sei eine Funktion, die f\"{u}r alle $x$ aus $M$ definiert ist.
Dann hei\3t die Menge aller Abbilder $f(x)$ von Elementen $x$ aus der Menge $M$ das
\href{https://de.wikipedia.org/wiki/Bild_(Mathematik)}{\emph{Bild}} von $M$ unter $f$.  Wir
schreiben $f(M)$ f\"{u}r dieses Bild. 
Formal kann $f(M)$ als
 \[ f(M) := \{ y \;|\; \exists x \in M: y = f(x) \} \]
definiert werden. In der Literatur findet sich f\"{u}r die obige Menge auch die Schreibweise
\[ f(M) = \bigl\{ f(x) \;|\; x \in M \}. \]

\example
Die Menge $Q$ aller Quadrat-Zahlen kann als 
\[ Q := \{ y \mid \exists x \in \mathbb{N}: y = x^2\} \]
definiert werden.  Alternativ k\"{o}nnen wir auch 
\[ Q = \bigl\{ x^2 \mid x \in \mathbb{N} \bigr\} \]
schreiben.
\eox

\section{Potenz-Mengen}
Um den Begriff der \href{https://de.wikipedia.org/wiki/Potenzmenge}{\emph{Potenz-Menge}}
einf\"{u}hren zu k\"{o}nnen, m\"{u}ssen wir zun\"{a}chst \emph{Teilmengen} definieren.  Sind $M$ und
$N$ zwei Mengen, so hei\3t $M$ eine \emph{Teilmenge} von $N$ genau dann, wenn jedes Element der
Menge $M$ auch ein Element der Menge $N$ ist.  In diesem Fall schreiben wir $M \subseteq N$.  Formal
k\"{o}nnen wir den Begriff der Teilmenge durch die Formel
 \\[0.2cm]
\hspace*{1.3cm}
$M \subseteq N \;\stackrel{\mathrm{def}}{\Longleftrightarrow}\; \forall x: (x \in M \rightarrow x \in N)$ 
 \\[0.2cm]
definieren.

\example
Es gilt 
\\[0.2cm]
\hspace*{1.3cm}
$\{ 1, 3, 5\} \subseteq \{ 1, 2, 3, 4, 5 \}$
\\[0.2cm]
Weiter gilt f\"{u}r jede beliebige Menge $M$
\\[0.2cm]
\hspace*{1.3cm}
$\emptyset \subseteq M$. \eox


Unter der \emph{Potenz-Menge} einer Menge $M$ wollen wir nun die Menge aller Teilmengen
von $M$ verstehen.  Wir schreiben $2^M$ f\"{u}r die Potenz-Menge von $M$.  Dann gilt \\[0.2cm]
\hspace*{1.3cm} $2^M = \{ x \;|\; x \subseteq M \}$.
\vspace{0.2cm}

\example
Wir bilden  die Potenz-Menge der Menge $\{1,2,3\}$.  Es gilt: \\[0.2cm]
\hspace*{1.3cm} $2^{\{1,2,3\}} = \big\{ \{\},\, \{1\}, \, \{2\},\, \{3\},\, \{1,2\}, \, \{1,3\}, \, \{2,3\},\, \{1,2,3\}\big\}$. \\[0.2cm]
Diese Menge hat $8 = 2^3$ Elemente.  Allgemein kann durch \emph{Induktion} \"{u}ber die Anzahl der
Elemente der Menge $M$ gezeigt werden, dass die 
Potenz-Menge einer Menge $M$, die aus $m$ verschiedenen Elementen besteht, insgesamt $2^m$ 
Elemente enth\"{a}lt.  Bezeichnen wir die Anzahl der Elemente einer endlichen Menge mit
$\textsl{card}(M)$, so gilt also
\\[0.2cm]
\hspace*{1.3cm}
$\textsl{card}\left(2^M\right) = 2^{\textsl{card}(M)}$.
\\[0.2cm]
Dies erkl\"{a}rt die Schreibweise $2^M$ f\"{u}r die Potenz-Menge von $M$.  \eox


\begin{Satz}[M\"{a}chtigkeit der Potenz-Menge]
Es sei $M$ eine endliche Menge.  Dann gilt
\\[0.2cm]
\hspace*{1.3cm}
$\textsl{card}\bigl(2^M\bigr) = 2^{\textsl{card}(M)}$.
\end{Satz}

\proof
Es sei $n := \textsl{card}(M)$.  Dann hat $M$ die Form
\\[0.2cm]
\hspace*{1.3cm}
$M = \{ x_1, x_2, \cdots, x_n \}$.
\\[0.2cm]
Wir zeigen durch Induktion nach $n$, dass Folgendes gilt:
\\[0.2cm]
\hspace*{1.3cm}
$\textsl{card}\bigl(2^M\bigr) = 2^n$.
\begin{enumerate}
\item Induktions-Anfang: $n = 0$.

      Dann gilt  $M = \{\}$ und f\"{u}r die Potenz-Menge $2^M$ finden wir
      \\[0.2cm]
      \hspace*{1.3cm}
      $2^M = 2^{\{\}} = \bigl\{ \{\} \bigr\}$.
      \\[0.2cm]
      Die Potenz-Menge der leeren Menge hat also genau ein Element.  Daher gilt
      \\[0.2cm]
      \hspace*{1.3cm}
      $\textsl{card}\bigl(2^{\{\}}\bigr) = \textsl{card}\bigl(\bigl\{ \{\} \bigr\}\bigr) = 1$.
      \\[0.2cm]
      Auf der anderen Seite haben wir 
      \\[0.2cm]
      \hspace*{1.3cm}
      $2^{\textsl{card}(\{\})} = 2^0 = 1$.
\item Induktions-Schritt: $n \mapsto n + 1$.

      Wenn $\textsl{card}(M) = n+1$ ist, dann hat $M$ die Form
      \\[0.2cm]
      \hspace*{1.3cm}
      $M = \{ x_1, x_2, \cdots x_n, x_{n+1}\}$.
      \\[0.2cm]
      Es gibt zwei verschiedene Arten von Teilmengen von $M$: Solche, die $x_{n+1}$ enthalten und
      solche, die $x_{n+1}$ nicht enthalten.  Dementsprechend k\"{o}nnen wir die Potenz-Menge $2^M$ wie folgt
      aufteilen:
      \\[0.2cm]
      \hspace*{1.3cm}
      $2^M = \bigl\{ K \in 2^M \bigm| x_{n+1}     \in K \bigr\} \cup
             \bigl\{ K \in 2^M \bigm| x_{n+1} \not\in K \bigr\} 
      $
      \\[0.2cm]
      Wir bezeichnen die erste dieser Mengen mit $A$, die zweite nennen wir $B$:
      \\[0.2cm]
      \hspace*{1.3cm}
      $A := \bigl\{ K \in 2^M \bigm| x_{n+1}     \in K \bigr\}$, \quad
      $B := \bigl\{ K \in 2^M \bigm| x_{n+1} \not\in K \bigr\}$.
      \\[0.2cm]
      Offenbar sind die Mengen $A$ und $B$ \emph{disjunkt}, dass hei�t diese Mengen enthalten keine
      gemeinsamen Elemente.  Daher folgt aus der Gleichung
      \\[0.2cm]
      \hspace*{1.3cm}
      $2^M = A \cup B$,
      \\[0.2cm]
      dass die Anzahl der Elemente von $2^M$ gleich der Summe der Anzahl der Elemente in $A$ und der
      Anzahl der Elemente in $B$ ist:
      \\[0.2cm]
      \hspace*{1.3cm}
      $\textsl{card}\bigl(2^M\bigr) = \textsl{card}(A) + \textsl{card}(B)$.
      \\[0.2cm]
      Die Menge $B$ enth\"{a}lt genau die Teilmengen von $M$, die $x_{n+1}$ nicht enthalten.  Das sind dann 
      aber genau die Teilmengen der Menge $\{x_1,\cdots, x_n\}$, es gilt also
      \\[0.2cm]
      \hspace*{1.3cm}
      $B = 2^{\{x_1,\cdots, x_n\}}$.
      \\[0.2cm]
      Nach Induktions-Voraussetzung wissen wir daher, dass 
      \\[0.2cm]
      \hspace*{1.3cm}
      $\textsl{card}(B) = \textsl{card}\bigl(2^{\{x_1,\cdots, x_n\}}\bigr) \stackrel{IV}{=}
       2^{\textsl{card}(\{x_1,\cdots, x_n\})} = \displaystyle 2^n
      $
      \\[0.2cm]
      gilt.  Als n\"{a}chstes zeigen wir, dass die Menge $A$ genau so viele Elemente hat, wie die Menge $B$.
      Zu diesem Zweck konstruieren wir eine
      \href{https://de.wikipedia.org/wiki/Bijektive_Funktion}{bijektive Funktion} $f$, die jeder
      Menge $K
      \in B$ genau eine Menge $f(K) \in A$ zuordnet:
      \\[0.2cm]
      \hspace*{1.3cm}
      $f: B \rightarrow A$ \quad ist definiert durch \quad $f(K) := K \cup \{ x_{n+1} \}$.
      \\[0.2cm]
      Die Umkehrfunktion $f^{-1}: A \rightarrow B$ kann offenbar durch die Formel
      \\[0.2cm]
      \hspace*{1.3cm}
      $f^{-1}(K) := K \backslash \{ x_{n+1} \}$
      \\[0.2cm]
      definiert werden.  Damit ist aber klar, dass die Mengen $A$ und $B$ gleich viele Elemente haben:
      \\[0.2cm]
      \hspace*{1.3cm}
      $\textsl{card}(A) = \textsl{card}(B)$.
      \\[0.2cm]
      Insgesamt haben wir jetzt
      \\[0.2cm]
      \hspace*{1.3cm}
      $
      \begin{array}[t]{lcl}
        \textsl{card}\bigl(2^M\bigr) & = & \textsl{card}(A) + \textsl{card}(B) \\[0.2cm]
                                     & = & \textsl{card}(B) + \textsl{card}(B) \\[0.2cm]
                                     & = & 2 \cdot \textsl{card}(B)            \\[0.2cm]
                                     & \stackrel{IV}{=} & 2 \cdot 2^n                         \\[0.2cm]
                                     & = & 2^{n+1}.
      \end{array}
      $
      \\[0.2cm]
      Wir haben also $\textsl{card}\bigl(2^M\bigr) = 2^{n+1}$ bewiesen.  Damit ist der Induktions-Schritt
      abgeschlossen und der Beweis der Behauptung ist erbracht. \qed
\end{enumerate}

\section{Kartesische Produkte}
Um den Begriff des 
\href{https://de.wikipedia.org/wiki/Kartesisches_Produkt}{\emph{kartesischen Produktes}}
einf\"{u}hren zu k\"{o}nnen, ben\"{o}tigen wir zun\"{a}chst den Begriff des 
\href{https://de.wikipedia.org/wiki/Geordnetes_Paar}{\emph{geordneten Paares}} zweier Objekte $x$
und $y$.  Dieses wird  als 
\\[0.2cm] 
\hspace*{1.3cm}
$\langle x, y \rangle$ 
\\[0.2cm]
geschrieben.  Wir sagen, dass $x$ die \emph{erste Komponente} des Paares $\langle x, y \rangle$ ist, 
und $y$ ist die \emph{zweite Komponente}.  Zwei geordnete Paare $\langle x_1, y_1 \rangle$ und $\langle x_2, y_2 \rangle$
sind genau dann gleich, wenn sie komponentenweise gleich sind, d.h.~es gilt \\[0.2cm]
\hspace*{1.3cm} $\langle x_1, y_1 \rangle \,=\,\langle x_2, y_2 \rangle  \;\Leftrightarrow\; x_1 = x_2 \wedge y_1 = y_2$. \\[0.2cm]
Das kartesische Produkt zweier Mengen $M$ und $N$ ist nun die Menge aller geordneten
Paare, deren erste Komponente in $M$ liegt und deren zweite Komponente in $N$ liegt.
Das kartesische Produkt von $M$ und $N$ wird als $M \times N$ geschrieben, formal gilt: 
\[ M \times N := \big\{ z \mid \exists x\colon \exists y\colon\bigl(z = \langle x,y\rangle \wedge x\in M \wedge y \in N\bigr) \bigr\}. \]
Alternativ k\"{o}nnen wir auch schreiben
\[ M \times N := \big\{ \langle x,y\rangle \mid  x\in M \wedge y \in N \}. \]
\noindent
\textbf{Beispiel}:  Wir setzen $M = \{ 1, 2, 3 \}$ und $N = \{ 5, 7 \}$. Dann gilt\\[0.2cm]
\hspace*{1.3cm} 
$M \times N = \bigl\{ \pair(1,5),\pair(2,5),\pair(3,5),\pair(1,7),\pair(2,7),\pair(3,7)\bigr\}$.
\vspace{0.2cm}

Der Begriff des geordneten Paares l\"{a}sst sich leicht zum Begriff des $n$-Tupels verallgemeinern:
Ein $n$-Tupel hat die Form \\[0.2cm]
\hspace*{1.3cm} $\langle x_1, x_2, \cdots, x_n \rangle$. \\[0.2cm]
Analog kann auch der Begriff des kartesischen Produktes auf $n$ Mengen $M_1$, $\cdots$, $M_n$
verallgemeinert werden. Das sieht dann so aus: \\[0.2cm]
\hspace*{1.3cm}
$M_1 \times \cdots \times M_n =
  \big\{ z \mid \exists x_1\colon \cdots \exists x_n \colon 
    \bigl( z = \langle x_1,x_2,\cdots,x_n \rangle \wedge x_1\in M_1 \wedge \cdots \wedge x_n \in M_n\bigr) 
  \big\}
$. 
\\[0.2cm]
Ist $f$ eine Funktion, die auf $M_1 \times \cdots \times M_n$ definiert ist,
so vereinbaren wir folgende Vereinfachung der Schreibweise: \\[0.2cm]
\hspace*{1.3cm} $f(x_1, \cdots, x_n)$ steht f\"{u}r $f(\langle x_1, \cdots,
x_n\rangle)$. \\[0.2cm]
Gelegentlich werden $n$-Tupel auch als \emph{endliche Folgen} oder als
\emph{Listen} bezeichnet.  

\section{Gleichheit von Mengen}
Wir haben nun alle Verfahren, die wir zur Konstruktion von Mengen ben\"{o}tigen, vorgestellt.
Wir kl\"{a}ren jetzt die Frage, wann zwei Mengen gleich sind.  Dazu postulieren wir das folgende 
\href{https://en.wikipedia.org/wiki/Axiom_of_extensionality}{\emph{Extensionalit\"{a}ts-Axiom}}
f\"{u}r Mengen: 
\vspace*{0.2cm}

\colorbox{red}{\framebox{\colorbox{yellow}{ 
\begin{minipage}{0.75\linewidth}
 {\sl Zwei Mengen sind genau dann gleich, wenn sie dieselben Elemente besitzen.}  
\end{minipage}}}}
\vspace{0.2cm}

Mathematisch k\"{o}nnen wir diesen Sachverhalt durch die Formel
\\[0.2cm]
\hspace*{1.3cm} $M = N \;\leftrightarrow\; \forall x: (x \in M \leftrightarrow x \in N)$ 
\\[0.2cm]
ausdr\"{u}cken.  Eine wichtige Konsequenz aus diesem Axiom ist die Tatsache, dass die Reihenfolge, mit der
Elemente in einer Menge aufgelistet werden, keine Rolle spielt.  Beispielsweise gilt \\[0.2cm]
\hspace*{1.3cm} $\{1,2,3\} = \{3,2,1\}$, \\[0.2cm]
denn beide Mengen enthalten dieselben Elemente.

Falls Mengen durch explizite Aufz\"{a}hlung ihrer Elemente definiert sind, ist die Frage nach
der Gleichheit zweier Mengen trivial.  Ist eine der Mengen mit Hilfe des Auswahl-Prinzips definiert, so
kann es beliebig schwierig sein zu entscheiden, ob zwei Mengen gleich sind.  Hierzu ein
Beispiel:  Es l\"{a}sst sich zeigen, dass \\[0.2cm]
\hspace*{1.3cm} 
$\{ n \in \mathbb{N}^* \mid \exists x, y, z\in\mathbb{N}^*: x^n + y^n = z^n \} = \{1,2\}$ \\[0.2cm]
gilt.  Allerdings ist der Nachweis dieser Gleichheit sehr schwer, denn er ist \"{a}quivalent
zum Beweis der 
\href{https://en.wikipedia.org/wiki/Fermat%27s_Last_Theorem}{\emph{Fermat'schen Vermutung}}. Diese
Vermutung wurde 1637 von \href{https://de.wikipedia.org/wiki/Pierre_de_Fermat}{Pierre de Fermat}
aufgestellt und konnte erst 1994 von \href{https://de.wikipedia.org/wiki/Andrew_Wiles}{Andrew Wiles}
und \href{https://de.wikipedia.org/wiki/Richard_Taylor_(Mathematiker)}{Richard Taylor} bewiesen werden. 
Es gibt andere, \"{a}hnlich aufgebaute Mengen, wo bis heute unklar ist, welche Elemente in der
Menge liegen und welche nicht.


\section{Rechenregeln f\"{u}r das Arbeiten mit Mengen}
 Vereinigungs-Menge,  Schnitt-Menge und die Differenz zweier Mengen gen\"{u}gen Gesetzm\"{a}\3igkeiten, 
die in den folgenden Gleichungen zusammengefasst sind:
\\[0.2cm]
$\begin{array}{rlcl}
\quad 1. & M \cup \emptyset = M         & \hspace*{0.1cm} & M \cap \emptyset = \emptyset \\
2. & M \cup M = M         & \hspace*{0.1cm} & M \cap M = M          \\
3. & M \cup N = N \cup M  &  & M \cap N = N \cap M  \\
4. & (K \cup M) \cup N = K \cup (M \cup N) &  & (K \cap M) \cap N = K \cap (M \cap N) \\
5. & (K \cup M) \cap N = (K \cap N) \cup (M \cap N) &  & (K \cap M) \cup N = (K \cup N) \cap (M \cup N)  \\
6. & M \backslash \emptyset = M & & M \backslash M = \emptyset \\
7. & K \backslash (M \cup N) = (K \backslash M) \cap (K \backslash N) &&
     K \backslash (M \cap N) = (K \backslash M) \cup (K \backslash N) \\
8. & (K \cup M) \backslash N = (K \backslash N) \cup (M \backslash N) &&
     (K \cap M) \backslash N = (K \backslash N) \cap (M \backslash N) \\
9. & K \backslash (M \backslash N) = (K \backslash M) \cup (K \cap N) &&
     (K \backslash M) \backslash N = K \backslash (M \cup N) \\
10. & M \cup (N \backslash M) = M \cup N &&
      M \cap (N \backslash M) = \emptyset  \\
11. & M \cup (M \cap N) = M  &&
      M \cap (M \cup N) = M 

\end{array}$
\\[0.3cm]
Wir beweisen exemplarisch die Gleichung $K \backslash (M \cup N) = (K \backslash M) \cap (K \backslash N)$.
Um die Gleichheit zweier Mengen zu zeigen ist nachzuweisen, dass beide Mengen dieselben Elemente enthalten.
Wir haben die folgende Kette von \"{A}quivalenzen: \\[0.3cm]
\hspace*{1.3cm} $
\begin{array}{ll}
                & x \in K \backslash (M \cup N)        \\
\Leftrightarrow & x \in K \;\wedge\; \neg\; x \in M \cup N \\
\Leftrightarrow & x \in K \;\wedge\; \neg\; (x \in M \vee x \in N) \\
\Leftrightarrow & x \in K \;\wedge\;  (\neg\; x \in M) \wedge (\neg\; x \in N) \\
\Leftrightarrow & (x \in K \wedge \neg\;x \in M) \;\wedge\; (x \in K \wedge \neg\;x \in N) \\
\Leftrightarrow & (x \in K \backslash M) \;\wedge\; (x \in K \backslash N) \\
\Leftrightarrow & x \in (K \backslash M) \cap (K \backslash N). \\
\end{array}$ \\[0.3cm]
Wir haben beim dritten Schritt dieser \"{A}quivalenz-Kette ausgenutzt, dass eine
Disjunktion der Form $F \vee G$ genau dann falsch ist, wenn sowohl $F$ als auch $G$ falsch ist,
formal gilt
\\[0.2cm]
\hspace*{1.3cm}
$\neg (F \vee G) \leftrightarrow \neg F \wedge \neg G$.
\\[0.2cm]
Wir werden diese \"{A}quivalenz im Rahmen der Logik-Vorlesung noch formal beweisen.

Die \"{u}brigen der oben aufgef\"{u}hrten Gleichungen k\"{o}nnen nach dem selben Schema hergeleitet werden.

\exercise
Beweisen Sie die folgenden Gleichungen:
\begin{enumerate}[(a)]
\item $K \backslash (M \cap N) = (K \backslash M) \cup (K \backslash N)$,
\item $M \cup (M \cap N) = M$,
\item $K \backslash (M \backslash N) = (K \backslash M) \cup (K \cap N)$,
\item $(K \backslash M) \backslash N = K \backslash (M \cup N)$. \exend
\end{enumerate} 
\vspace{0.2cm}

\noindent
Zur Vereinfachung der Darstellung von Beweisen vereinbaren wir die folgende Schreibweise:
Ist $M$ eine Menge und $x$ ein Objekt, so schreiben wir $x \notin M$  f\"{u}r
die Formel $\neg\; x \in M$, formal: \\[0.2cm]
\hspace*{1.3cm} $x \notin M \;\stackrel{de\!f}{\Longleftrightarrow}\; \neg\; x \in M$.
\\[0.2cm]
Eine analoge Notation verwenden wir auch f\"{u}r das Gleichheitszeichen:
$x \not= y \;\stackrel{de\!f}{\Longleftrightarrow}\; \neg\; (x = y)$.

\section{Die Kardinalit\"at einer Menge}
In diesem Abschnitt besch\"aftigen wir uns mit der \emph{Gr\"o{\ss}e} von Mengen.  Wir werden sehen,
dass es unendlichen Mengen unterschiedlicher Gr\"o{\ss}e gibt, n\"amlich solche Mengen, bei denen wir
die Elemente aufz\"ahlen k\"onnen und solche Mengen, wo das nicht m\"oglich ist.

\begin{Definition}[Abz\"{a}hlbar] \hspace*{\fill} \\
Eine unendliche Menge $M$ hei\3t 
\href{https://en.wikipedia.org/wiki/Countable_set}{\emph{abz\"{a}hlbar}}, wenn eine 
\href{https://en.wikipedia.org/wiki/Surjective_function}{surjektive} Funktion
\\[0.2cm]
\hspace*{1.3cm}
$f: \mathbb{N} \rightarrow M$
\\[0.2cm]
existiert.  Dabei nennen wir eine Funktion $f:A \rightarrow B$  \emph{surjektiv} genau
dann, wenn es zu jedem $y \in B$ ein $x \in A$ gibt, so dass $f(x) = y$ ist:
\\[0.2cm]
\hspace*{1.3cm}
$\forall y \in B: \exists x \in A: f(x) = y$.  \eox
\end{Definition}

Die Idee bei dieser Definition ist, dass die Menge  $M$ in einem gewissen Sinne nicht mehr Elemente hat
als die Menge der nat\"{u}rlichen Zahlen, denn die Elemente k\"{o}nnen ja \"{u}ber die Funktion $f$
aufgez\"{a}hlt werden, wobei wir eventuelle Wiederholungen eines Elements zulassen wollen.

\example
Die Menge $\mathbb{Z}$ der ganzen Zahlen ist abz\"{a}hlbar, denn die Funktion
\\[0.2cm]
\hspace*{1.3cm}
$f: \mathbb{N} \rightarrow \mathbb{Z}$,
\\[0.2cm]
die durch die Fallunterscheidung
\\[0.2cm]
\hspace*{1.3cm}
$f(n) := \left\{ \begin{array}[c]{ll}
                 (n - 1) / 2 + 1  & \mbox{falls $n \modulo 2 = 1$}  \\[0.2cm]
                 - n / 2          & \mbox{falls $n \modulo 2 = 0$}
                 \end{array}
         \right.
$ 
\\[0.2cm]
definiert ist, ist surjektiv.  Um dies einzusehen, zeigen wir zun\"{a}chst, dass $f$ wohldefiniert ist.
Dazu ist zu zeigen, dass $f(n)$ tats\"{a}chlich in jedem Fall eine ganze Zahl ist.
\begin{enumerate}
\item $n \modulo 2 = 1$:  Dann ist $n$ ungerade, also ist $(n-1)$ gerade und die Division
      $(n-1)/2$ liefert eine ganze Zahl.
\item $n \modulo 2 = 0$:  In diesem Fall ist $n$ gerade.  Damit ist $n/2$ eine ganze Zahl.
\end{enumerate}
Es bleibt zu zeigen, dass $f$ surjektiv ist.  Wir m\"{u}ssen also zeigen, dass es f\"{u}r jedes 
$z \in \mathbb{Z}$ eine nat\"{u}rliche Zahl $n$ gibt, so dass $f(n) = z$ ist.  Wir f\"{u}hren diesen Nachweis
mittels einer Fall-Unterscheidung:
\begin{enumerate}
\item Fall: $z > 0$.
  
      Wir definieren $n := 2 \cdot (z - 1) + 1$.  Wegen $z >0$ gilt $n > 0$ und damit ist $n$
      tats\"{a}chlich eine nat\"{u}rliche Zahl.  Au\3erdem ist klar, dass $n$ ungerade ist. Daher gilt
      \\[0.2cm]
      \hspace*{0.3cm}
      $f(n) = (n - 1)/2 + 1 = \bigl(\bigl(2 \cdot (z - 1) + 1\bigr) - 1\bigr)/2 + 1 
            = \bigl(2 \cdot (z - 1)/2\bigr) + 1= z-1 + 1 = z
      $.
      \\[0.2cm]
      Also gilt $f(n) = z$.
\item Fall: $z \leq 0$.

      Wir definieren $n := - 2 \cdot z$.  Wegen $z \leq 0$ ist klar, dass $n$ eine gerade 
      nat\"{u}rliche Zahl ist.  Damit haben wir
      \\[0.2cm]
      \hspace*{1.3cm}
      $f(n) = -(-2 \cdot z)/2 = z$.
      \\[0.2cm]
      Also gilt ebenfalls $f(n) = z$.
\end{enumerate}
Damit ist die Surjektivit\"{a}t von $f$ gezeigt und somit ist $\mathbb{Z}$ abz\"{a}hlbar.
\qed

Den Beweis des letzten Satzes haben wir direkt gef\"{u}hrt, aber zum Nachweis des n\"{a}chsten Satzes werden wir
einen indirekten Beweis ben\"{o}tigen.  Vorab noch eine Definition.

\begin{Definition}[\"{U}berabz\"{a}hlbar] \lb
Eine unendliche Menge hei\3t
\href{https://en.wikipedia.org/wiki/Uncountable_set}{\emph{\"{u}berabz\"{a}hlbar}}, wenn sie
\underline{nicht} abz\"{a}hlbar ist. 
\end{Definition}

\begin{Satz}
  Die Potenzmenge der Menge der nat\"{u}rlichen Zahlen ist \"{u}berabz\"{a}hlbar.
\end{Satz}

\proof
Wir f\"{u}hren den Beweis indirekt und nehmen an, dass $2^{\mathbb{N}}$ abz\"{a}hlbar ist.  Dann gibt es also
eine Funktion
\\[0.2cm]
\hspace*{1.3cm}
$f: \mathbb{N} \rightarrow 2^{\mathbb{N}}$,
\\[0.2cm]
die surjektiv ist.  Wir definieren nun die Menge $C$ wie folgt:
\\[0.2cm]
\hspace*{1.3cm}
$C := \bigl\{ n \in \mathbb{N} \bigm| n \not\in f(n) \bigr\}$.
\\[0.2cm]
Offenbar ist $C$ eine Teilmenge der Menge der nat\"{u}rlichen Zahlen und damit gilt $C \in 2^{\mathbb{N}}$.
Da die Funktion $f$ nach unserer Annahme surjektiv ist, gibt es also eine nat\"{u}rliche Zahl $n_0$, so
dass 
\\[0.2cm]
\hspace*{1.3cm}
$C = f(n_0)$
\\[0.2cm]
gilt.  Wir untersuchen nun, ob $n_0 \in C$ gilt.  Dazu betrachten wir die folgende Kette von
\"{A}quivalenzen: 
\\[0.2cm]
\hspace*{1.3cm}
$
\begin{array}[t]{cl}
                & n_0 \in C                                                      \\[0.2cm]
\Leftrightarrow & n_0 \in \bigl\{ n \in \mathbb{N} \bigm| n \not\in f(n) \bigr\} \\[0.2cm]
\Leftrightarrow & n_0 \not\in f(n_0)                                             \\[0.2cm]
\Leftrightarrow & n_0 \not\in C                                               
\end{array}
$
\\[0.2cm]
Wir haben also
\\[0.2cm]
\hspace*{1.3cm}
$n_0 \in C \;\Leftrightarrow\; n_0 \not\in C$ \quad \mbox{$\color{red}{\lightning}$}
\\[0.2cm]
gezeigt und das ist ein offensichtlicher Widerspruch. \qed

\remark
Wir haben soeben gezeigt, dass es in gewisser Weise mehr Mengen von nat\"{u}rlichen Zahlen gibt, als es
nat\"{u}rliche Zahlen gibt.  In \"{a}hnlicher Weise kann gezeigt werden, dass die Menge $\mathbb{R}$ der reellen
Zahlen \"{u}berabz\"{a}hlbar ist.  

\exercise
Zeigen Sie, dass das Intervall
\\[0.2cm]
\hspace*{1.3cm}
$[0,1[ \;:= \{ x \in \mathbb{R} \mid 0 \leq x \wedge x < 1 \}$ 
\\[0.2cm]
\"{u}berabz\"{a}hlbar ist.  Nehmen Sie dazu an, dass die Zahlen $x \in [0,1[$ in der Form
\\[0.2cm]
\hspace*{1.3cm}
$ x = 0, d_1 d_2 d_3 \cdots$ \quad mit $d_i \in \{ 0, \cdots, 9 \}$  f\"{u}r alle $i \in \mathbb{N}$
\\[0.2cm]
dargestellt sind, es gilt dann also
\\[0.2cm]
\hspace*{1.3cm}
$\ds x = \sum\limits_{i=1}^\infty d_i \cdot \Bigl(\frac{1}{10}\Bigr)^{i}$.
\\[0.2cm]
Um sicher zu stellen, dass diese Darstellung eindeutig ist, fordern wir, dass diese Darstellung
nicht auf ``Periode Neun'' endet, es d\"{u}rfen also ab einem bestimmten Index $n \in \mathbb{N}$ nicht
alle Ziffern $d_i$ den Wert $9$ haben:
\\[0.2cm]
\hspace*{1.3cm}
$\neg \exists n \in \mathbb{N}: \forall i \in \mathbb{N}:\bigl( i \geq n \rightarrow c_i = 9\bigr)$.
\\[0.2cm]
F\"{u}hren Sie den Beweis indirekt und nehmen Sie an, dass es eine surjektive Funktion 
\\[0.2cm]
\hspace*{1.3cm}
$f: \mathbb{N} \rightarrow [0,1[$
\\[0.2cm]
gibt, die die Menge $[0,1[$ aufz\"{a}hlt.  Dann gibt es auch eine Funktion
\\[0.2cm]
\hspace*{1.3cm}
$g: \mathbb{N} \times \mathbb{N} \rightarrow \{0, \cdots, 9\}$
\\[0.2cm]
so dass $g(n,i)$ die $i$-te Nachkommastelle von $f(n)$ berechnet:
\\[0.2cm]
\hspace*{1.3cm}
$f(n) = 0,g(n,1) g(n,2) g(n,3) \cdots$.
\\[0.2cm]
Konstruieren Sie nun mit Hilfe dieser Funktion $g$ eine Zahl $c \in [0,1[$ in der Form
\\[0.2cm]
\hspace*{1.3cm}
$c = 0,c_1c_2c_3 \cdots$
\\[0.2cm]
so, dass sich ein Widerspruch ergibt.  Orientieren Sie sich dabei an der Konstruktion der Menge $C$ im
Beweis der \"{U}berabz\"{a}hlbarkeit von $2^\mathbb{N}$.  \exend

\pagebreak
\exercise
Zeigen Sie, dass die Menge
\\[0.2cm]
\hspace*{1.3cm}
$\mathbb{N}^\mathbb{N} := \{ f \mid f: \mathbb{N} \rightarrow \mathbb{N} \}$,
\\[0.2cm]
also die Menge aller Funktionen von $\mathbb{N}$ nach $\mathbb{N}$ \"{u}berabz\"{a}hlbar ist.
\exend

\exercise
Zeigen Sie, dass die Menge $\mathbb{Q}_+$ der positiven rationalen Zahlen abz\"{a}hlbar ist.

\hint
Ordnen Sie die positiven rationalen Zahlen im kartesischen Koordinaten-System an und versuchen Sie
nun, diese Zahlen systematisch abzuz�hlen.
\exend


%%% Local Variables: 
%%% mode: latex
%%% TeX-master: "lineare-algebra"
%%% End: 
